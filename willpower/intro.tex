\chapter*{Уступ}
\markboth{\MakeUppercase{Уступ}}{}
\addcontentsline{toc}{chapter}{Уступ}

У 2014 годзе, працуючы выкладчыкам у~мэдыцынскім унівэрсітэце, я пачаў весьці шэраг адукацыйных курсаў, датычных розных аспэктаў здароўя~--- ад харчаваньня да стрэсу. Людзі стаміліся ад мноства неправераных і сумнеўных мэтодыкаў аздараўленьня, таму з~задавальненьнем прыходзілі вучыцца быць здаровымі з~дапамогай навукі. Паступова гэтыя курсы вырасьлі ў~маю Школу здароўя, якую прайшлі ўжо тысячы людзей: мы разабралі сотні практыкаў, тысячы прыватных выпадкаў. Назапашаныя за гэтыя гады досьвед і веды я й падаю ў~сваёй кнізе. \textbf{Гэты дапаможнік~--- плён працы нашай здаровай супольнасьці, які месьціць як навуковыя веды, так і практычныя падыходы да выкарыстаньня ў~паўсядзённым жыцьці.}

Мы ўсе ведаем, што карысна, а~што шкодна. Але чаму тады мы рэгулярна робім нешта разбуральнае для сябе? Навукоўцы й філёзафы мінулых гадоў казалі пра інстынкт да жыцьця ды інстынкт да сьмерці~--- іх балянс і вызначае траекторыю нашага лёсу. Кожны з~нас мае волю (прагу) да жыцьця, жаданьне быць мацнейшым і здаравейшым.

\tipbox{Ляўрэат Нобэлеўскай прэміі, навуковец Ільля Мечнікаў выкарыстоўваў тэрмін «інстынкт жыцьця», яго ўжываў і фізыёляг Іван Паўлаў: «Усё жыцьцё ёсьць зьдзяйсьненьне адной мэты, а~менавіта ахоўваньня самога жыцьця, нястомная праца таго, што завецца агульным інстынктам жыцьця. Гэты агульны інстынкт, ці рэфлекс жыцьця, складаецца з~масы асобных рэфлексаў. Большую частку гэтых рэфлексаў уяўляюць сабой станоўча-рухальныя рэфлексы, г.~зн. рух да спрыяльных для жыцьця ўмоваў, рэфлексы, якія маюць на мэце захапіць, засвоіць гэтыя ўмовы для дадзенага арганізму».}

Кожны з~нас на ўзроўні інстынктаў імкнецца выжыць. Наша цела хоча быць здаровым, нашы цягліцы хочуць быць моцнымі, наш розум прагне быць вострым~--- і гэта цалкам натуральныя памкненьні. Страта ж волі да жыцьця, сэнсу і мэты аслабляе нас. Я хачу, каб гэтая кніга абудзіла ў~вас прагу, волю да жыцьця, да здароўя на самым глыбінным узроўні. Няхай інстынкт жыцьця дапаможа вам ня толькі займець аптымальнае здароўе, але й рэалізаваць свой патэнцыял як асобы, дамагчыся сваіх мэтаў, стаць больш моцнымі і цягавітымі як фізычна, так і разумова.


\subsection{Здароўе~--- гэта нашмат больш, чым адсутнасьць хваробаў}

З гэтай кнігі вы даведаецеся, як здароўе робіцца падмуркам даўгалецьця, прывабнасьці й шчасьця. Мы разьбяром сем ключавых рэсурсаў здароўя, асяродзьдзе, спосабы вымярэньня і ацэнкі здароўя, практычныя падыходы прымяненьня рэкамэндацый у~сваім жыцьці на сыстэмнай аснове. Бо наша здароўе~--- гэта як хата, дзе нельга абраць, што важнейшае~--- падлога, сьцены, вокны або дах. Важнае ўсё.

У падзагаловак кнігі я ўнёс словы «самавучэбнік здароўя». У выпадку, калі вы абераце іншы падзагаловак, гэты сказ трэба будзе перафармуляваць. Гэта значыць, што вы самі можаце ўкараніць у~свой лад жыцьця большасьць карысных звычак. Падумайце: кожны дзень вы ўстаяце, ідзяце на працу, бавіце час з~блізкімі і сябрамі, кладзяцеся спаць. Калі кожнае ваша звычайнае дзеяньне зрабіць хоць трохі здаравейшым, гэта прынясе вам вялікую карысьць. Зьмяняючы свае звычкі, вы мяняеце сваё жыцьцё. 

\textbf{Многія эфэктыўныя мэтады аздараўленьня практычна не запатрабуюць ад вас прыкметных затратаў часу або сродкаў: толькі пачніце~--- і вы зразумееце, як гэта проста. Сучасная навука прапануе шмат спосабаў палепшыць сваё здароўе, і мне хочацца, каб вы змаглі скарыстацца гэтымі магчымасьцямі.}

Вядома, усе людзі розныя, але тым ня менш існуюць унівэрсальныя правілы здароўя, эфэктыўныя і бясьпечныя. Я абраў іх для кнігі, грунтуючыся на навуковых зьвестках і практыцы выкарыстаньня. У кнізе няма спасылак на самі дасьледаваньні (усе яны ёсьць у~маім блогу beloveshkin.com), але я буду рады, калі кожную маю параду вы праверыце самі~--- і пераканаецеся ў~яе дзейснасьці. Вашае здароўе~--- гэта найвышэйшы прыярытэт: дасьледуйце, сумнявайцеся, тэстуйце і знаходзьце найлепшае для сябе.

У гэтай кнізе вы ня знойдзеце самага галоўнага сакрэту здароўя. Бо што можа быць самым галоўным у~летаку? У ім сотні крытычна важных дэталяў, безь якіх ён ня можа ляцець, а~наш арганізм нашмат складанейшы за лятак.

Усе здаровыя людзі здаровыя аднолькава, але хварэюць па-рознаму. У гэтай кнізе мы даведаемся пра ключавыя складнікі здароўя і як іх разьвіць у~сваім жыцьці. У ёй ня будзе набору дадаткаў з~дазоўкамі, у~ёй~--- патэрны, заканамернасьці, якія спрыяюць здароўю. Самыя розныя патэрны, ад харчаваньня да атачэньня, спалучаюцца і ўзмацняюць адзін аднаго.

Зьмяніць свае звычкі можа быць нялёгкай задачай. Калі ня ўсё атрымаецца зрабіць зь першага разу~--- гэта нармальна. Вучыцеся на сваіх памылках, рабіце высновы, вывучайце сябе. Для гэтага, апроч іншага, важна не губляць пачуцьцё гумару і ўмець пасьмяяцца зь сябе. Калі вы фармуеце сваё бачаньне будучыні, бераце пад кантроль сучаснасьць, то ясна бачыце вашыя сапраўдныя магчымасьці. Спадзяюся, гэтая кніга пасее ў~вашым розуме насеньне будучых посьпехаў і здароўя. Вядома, сама па сабе яна ня зьменіць вашае жыцьцё імгненна, але я думаю, што яна можа памяняць кірунак вашага руху. 

Прыемнага і плённага чытва!


\subsection{Падзякі}

Асаблівая падзяка жонцы Кацярыне за стварэньне ўмоваў для напісаньня кнігі. Дзякую маці Надзеі Белавешкінай за падтрымку ў~час пісьма. Рэдактарка Яна Жукава дапамагла мне разабрацца з~рукапісам і зрабіць тэкст і думкі больш яснымі. Дзякую за дапамогу ў~стварэньні ілюстрацый дызайнэрцы Галіне Пуларыі ды ілюстратарцы Кацярыне Вельчавай. За каштоўныя парады і заўвагі наконт кнігі ўдзячны Аляксею Маскалёву, Надзеі Крыжаноўскай, Дзьмітрыю Курылу, Мікалаю Белавешкіну, Алене Рыдэль. Таксама дзякуй усім маім настаўнікам, пацыентам, кліентам, чытачам блога, вучням і выпускнікам маёй Школы здароўя за пытаньні, натхненьне, якія і прывялі да стварэньня гэтай кнігі.
