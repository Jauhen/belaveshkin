\documentclass[ebook,10pt,openany]{memoir}

\usepackage{belaveshkin}
\addto\captionsbelarusian{\renewcommand{\chaptername}{Правіла}}
\renewcommand*{\cftchaptername}{Правіла }

\author{Андрусь Белавешкін}
\title{Што і калі есьці.}
\date{2020 Пераклад 2022}

\begin{document}

\CoverPic{nutrition/cover.png}

\cleartooddpage
\begin{titlingpage}
  \TitlePageFormat{Андрусь\ \ Белавешкін}
      {Што\ \ і\ \ калі\ \ есьці.}
      {Як\ \ знайсьці\ \ залатую\ \ сярэдзіну між\ \ голадам\ \ і\ \ пераяданьнем}
      {2020, Пераклад 2022}

Нашае жыцьцё – гэта ня спрынт «схуднець да лета», а маратон «быць здаровым ды энэргічным доўгія гады і перадухіліць раньняе старэньне». У харчаваньні істотная рэч – прытрымлівацца залатой сярэдзіны, улічваць навуковыя парады, традыцыйныя практыкі й асабістыя адметнасьці. Ніхто ня ведае нас лепей, чымся мы самі. Калі ж да гэтае веды дадаць разуменьне падставовых працэсаў, гэта дапаможа прымаць слушныя й здаровыя рашэньні.
Кніга Андрэя Белавешкіна, доктара, к. м. н., выкладчыка, – гэта збор гнуткіх правілаў, кожным зь якіх можна карыстацца асобна. Правілы рэжыму харчаваньня, выбару прадуктаў, а таксама псыхалёгіі харчаваньня даюць адказы на найбольш важныя пытаньні: калі есьці? што есьці? як есьці?
Увага! Інфармацыя, зьмешчаная ў кнізе, ня можа выступаць заменай кансультацыі лекара. Перад тым як ажыцьцявіць любую з рэкамэндаваных дзеяў, неабходна пракансультавацца са спэцыялістам.
здаровы лад жыцьця, рэжым харчаваньня, здаровае харчаваньне, культура харчаваньня, дыеталёгія, парады дактароў


\end{titlingpage}

\pagestyle{headings}
\setcounter{page}{3}
\setcounter{tocdepth}{0}
\tableofcontents*
\clearpage

\chapter*{Уступ}

Вытокі гэтае кнігі — у адукацыйным курсе, які я вяду для слухачоў з 2014 году. У той час, калі я яшчэ выкладаў у мэдычным унівэрсытэце, мяне папрасілі выступіць зь лекцыямі аб навуковых падставах здаровага харчаваньня, разьвеяць міты й страхі, акрэсьліць унівэрсальныя правілы харчаваньня. Спачатку гэта былі заняткі ў групах, затым навучальны анлайн-курс, які прайшлі тысячы людзей. З кожнай групай мы разьбіралі розныя аспэкты харчаваньня ды іх практычнае ўкараненьне ў абыдзённае жыцьцё. Натуральна, сытуацыя кожнага чалавека ўнікальная, аднак ёсьць і падставовыя правілы, агульныя для ўсіх. 

Працуючы з рознымі людзьмі, лекуючы розныя хваробы, я бачу, што падмурак індывідуальнай дыеты – унівэрсальныя правілы харчаваньня. Усьведамленьне і ўкараненьне ў жыцьцё гэтых правілаў дапамагае кожнаму чалавеку стварыць на іх падставе свае звычкі, якія стануць грунтам пляну харчаваньня.

У гэтай кнізе я хачу расказаць вам пра два найважнейшыя аспэкты харчаваньня — рэжым харчаваньня (калі есьці) і прадукты харчаваньня (што есьці) — і навучыць імі карыстацца ў абыдзённым жыцьці. Вядома, ёсьць яшчэ безьліч іншых пытаньняў, якія датычаць псыхалёгіі харчаваньня, харчаваньня ў час хваробаў і мноства іншых цікавых ды істотных тэмаў, аднак мы вернемся да іх пазьней. Кансультуючы кліентаў і выкладаючы на курсе здаровага харчаваньня, я склаў шэраг унівэрсальных правілаў, якія кожны чалавек можа ўкараніць у сваё жыцьцё й атрымаць ад гэтага плён. У кнізе 12 правілаў рэжыму харчаваньня і 12 правілаў выбару прадуктаў. Крытэры правілаў — навуковая інфармацыя ды эмпірычныя веды (традыцыі розных культур і народаў), кожная парада выпрабаваная тысячагадовым чалавечым досьведам і пацьверджаная навуковымі дасьледзінамі. Кожнае правіла мае розныя ступені практыкі, вы можаце пачынаць з лёгкага варыянту й рухацца ў бок складанага.

Цяпер у дыеталёгіі выходзіць безьліч размаітых кніг, але да кансэнсусу яшчэ далёка. Адныя выданьні павялічваюць колькасьць дыеталягічных мітаў, іншыя гэтыя міты разьвейваюць. Нехта спасылаецца на асабісты досьвед, хтосьці заклікае яго ігнараваць і абапірацца адно на доказную мэдыцыну. А вось трэція даводзяць, што большасьць навуковых дасьледзінаў у гэтай галіне фундуецца харчовымі вытворцамі, дый апроч іх багата хто яшчэ зацікаўлены ў лекаваньні, а не ў прафіляктыцы захворваньняў. Чацьвертыя факусуюцца толькі на асабовых адметнасьцях і падборы індывідуальнага харчаваньня, пятыя раяць кінуць усё гэта і есьці інтуітыўна.

Усё гэта нагадвае індыйскую прыпавесьць пра слана й сьляпых мудрацоў. Аднойчы ў горад прывезьлі слана, і кожны мудрэц пачаў яго абмацваць. Першы памацаў хобат і сказаў, што слон – гэта зьмяя, другі памацаў вуха і сказаў, што слон – гэта махала, трэці – нагу і вырашыў, што слон – гэта калёна, чацьверты схапіў хвост і сказаў, што слон – гэта кутас. Хто зь іх мае рацыю? Кожны кажа слушна й няслушна адначасова.

Таму варта прытрымлівацца залатой сярэдзіны, беручы пад увагу і навуковыя рады, і традыцыйнае харчаваньне, і свае асабовыя адметнасьці й перавагі. Праз гэта я й вырашыў напісаць кнігу ў выглядзе асобных разьдзелаў, кожны зь якіх зьяўляецца інструмэнтам, правілам для самастойнага прыманьня харчовых рашэньняў. Кожнае правіла гнуткае, любы чалавек можа самастойна яго карыстаць і ўкараняць у свой лад жыцьця. На жаль, шматлікія плыні ў дыеталёгіі ўяўляюць зь сябе радыкальныя скрайнасьці, але ёсьць і сярэдзінны шлях, які дазволіць вам плённа карыстацца рознымі стратэгіямі. Мы можаце атрымліваць карысьць як з калярыйнай ежы, так і з голаду, як з тлушчу, так і з вугляводаў! Прынцып сярэдзіннага, або залатога, шляху — гэта той стыль харчаваньня, які вам зручны, задавальняльны і карысны для здароўя. І гэта рэальна бяз скрайнасьцяў і фанатызму.

Кожная з 24 частак пабудаваная паводле аднаго прынцыпу: спачатку мы разьбіраем гісторыю пытаньня, даведваемся, як елі людзі раней, да чаго адаптаваны наш арганізм і што зьмянілася цяпер. Затым мы коратка разьбяромся, як менавіта гэтае харчовае правіла ўплывае на наша здароўе. Вы даведаецеся, што прадукты і рэжым харчаваньня моцна ўплываюць на наш арганізм на ўсіх узроўнях. Яны ўплываюць на нэўрахімію мозгу, спрыяючы выпрацоўцы звычак і прыхільнасьцяў, на працу гармонаў і адчувальнасьць тканін да іх, на працу імуннай сыстэмы, на актыўнасьць генаў і шматлікае іншае. Многія з гэтых эфэктаў праяўляюцца не адразу, а праз доўгія гады. У трэцяй частцы кожнага разьдзелу – кароткая фармулёўка харчовага правіла і ўзроўні яго складанасьці, а пасьля гэтага – набор парадаў і ідэй для яго больш пасьпяховага ажыцьцяўленьня. Вы можаце пачаць чытаньне зь любой часткі – бо кожнае правіла працуе, нават калі вы карыстаецеся ім асобна. Ды лепей, калі вы будзеце чытаць па адным разьдзеле ў дзень і выкарыстоўваць прачытанае на практыцы: так чытаньне кнігі зойме ў вас амаль месяц, за які вы набудзеце шмат простых і карысных харчовых звычак.

Здаровыя харчовыя звычкі – гэта ключ да даўгалецьця. Няхай зьмены будуць невялікімі, але іх моц – у штодзённым паўторы, у назапашвальным эфэкце! Бо шматлікія сучасныя хваробы, такія як сардэчна-сасудзістыя, атлусьценьне, рак, нэўрадэгенэратыўныя захворваньні й само па сабе старэньне не ўзьнікаюць за адзін дзень, а разьвіваюцца працяглы пэрыяд. Таму важна думаць пра сваё харчаваньне ў доўгатэрміновай пэрспэктыве, бо нашае жыцьцё – гэта ня спрынт «схуднець да лета», а маратон «быць здаровым ды энэргічным доўгія гады і прадухіліць раньняе старэньне». Усе дыеты падобныя тым, што працуюць у кароткатэрміновай пэрспэктыве і правальваюцца ў доўгатэрміновай. Чаму? Таму што правільнае харчаваньне – гэта лад жыцьця. Мы будзем гаварыць і пра тое, як уплываюць на наш мэтабалізм стрэс, сьвятло, тэмпэратура, фізычная актыўнасьць, нават настрой. Так, вы пачулі слушна – радасьць сапраўды спальвае тлушч! Калі вы думаеце аб укараненьні новай харчовай звычкі, падумайце, ці зможаце вы прытрымлівацца яе ўвесь час? Калі не, то вам лепей узяць плянку крыху ніжэй – няма сэнсу ў тым, каб не ўжываць цукар на працягу тыдня, а потым сарвацца. А вось калі ўвесьці звычку ўстрымлівацца ад салодкага хаця б раз на два дні, то ўжо праз год гэта дабратворна адаб'ецца на вашым здароўі.

Кожны дзейсны інструмэнт уплыву на здароўе мае й пабочныя эфэкты. Яда не выключэньне. Сёньня існуюць дзьве супрацьлеглыя тэндэнцыі – харчовая залежнасьць, пераяданьне і артарэксія (1). 

Таму перадусім я хачу даць вам некалькі правілаў тэхнікі бясьпекі.

1. Вашае здароўе – найвышэйшы прыярытэт, таму сумнявайцеся ў любой інфармацыі, у тым ліку і ў той, якую даю я. Памятайце, што агульныя правілы ў вашым канкрэтным выпадку могуць мець індывідуальныя нюансы. Стаўцеся да інфармацыі як дасьледчык – экспэрымэнтуйце й прымайце ўзважанае рашэньне.
2.  Ня ўсё будзе зразумела й магчыма зрабіць зь першага разу. Гэта нармальна, вучыцеся на сваіх памылках, рабіце высновы, знаходзьце свае слабыя месцы. Гэта дазволіць вам больш эфэктыўна дзейнічаць у будучыні. Кепская не памылка, а тое, што вы нічому не навучыліся.
3.  Пачуцьцё гумару. Стаўцеся да ўсяго прасьцей і не забывайце пасьмяяцца, у тым ліку і зь сябе. Калі вам ня сьмешна, дык вы ўсталі на шлях, што вядзе да парушэньняў харчовых паводзінаў. Вы заўсёды ўсё можаце зьесьці, ежа – гэта выдатная крыніца энэргіі, а ня зло. Проста трэба акрэсьліць зь ежай межы для вашага агульнага дабра.
4.  Ня ўлазьце ў спрэчкі й наогул будзьце партызанам у пляне зьменаў харчаваньня, але дапамагайце тым, хто шукае дапамогі. Многія людзі і аўтары будуць пераконваць вас, што толькі іх схэма харчаваньня – адзіная слушная ў сьвеце. Памятайце, што існуе мноства варыяцый здаровага харчаваньня, і ваша мэта – знайсьці такую для сябе асабіста. Устрымлівайцеся ад маральнай ацэнкі людзей паводле таго, што яны ядуць. Не лічыце сябе лепшым за іншых, калі ясьцё правільна, – гэта артарэксія. Вы асабіста ды іншыя людзі ня лепшыя й ня горшыя праз розны зьмест талерак.
5.  Здаровае харчаваньне – гэта ня проста ежа, а стыль жыцьця. Гэта стравы зь дзяцінства, гэта хатнія рытуалы, гэта новыя месцы, гэта прыемная кампанія сяброў, гэта тое, што прыносіць вам глыбокае задавальненьне й радасьць ад жыцьця. Нясмачная ежа аніяк ня можа быць здаровай.
6.  Знайдзіце балянс паміж гнуткасьцю ды стабільнасьцю. Выпрацоўвайце правілы й прынцыпы, а не прывязвайцеся да пэўных прадуктаў ці рэцэптаў. Але адначасова з гэтым напрацуйце смачныя рэцэпты з прадуктамі, якія заўсёды ёсьць у вас пад рукой. Дынамічна зьмяняйце рацыён у залежнасьці ад сэзону, фізычнай актыўнасьці і запатрабаваньняў арганізму.
7.  Здаровае харчаваньне пачынаецца не з панядзелка, а са сьняданку. Паспрабуйце не мяняць рэзка свой рацыён, дзейнічайце плаўна. Нам цяжка адначасова ўкараняць больш за тры звычкі, хоць канчатковае іх высьпяваньне ў мозгу займае да трох месяцаў! А перагрузка адно ўзмацняе стрэс і рызыку зрыву. Памятайце пра этапы зьменаў, пра абмежаванасьць сілы волі, сілу звычкі й важнасьць малых, але штодзённых дзеяў. Дзейнічайце мякка, але настойліва. Кожны прыём ежы – гэта магчымасьць трэнаваньня для выпрацоўкі новых, карысных звычак.


\part{Асноўныя правілы рэжыму харчаваньня}

\chapter{Чыстыя прамежкі}

Чысты прамежак~--- гэта поўная адсутнасьць калёрыяў любога кшталту паміж прыёмамі ежы, уключаючы падсілкаваньні, «вадкія» калёрыі (кава, гарбата, малако, сок) і да т.~п. Здаровыя харчовыя паводзіны мяркуюць, што мы занятыя іншымі пытаньнямі ў~прамежках паміж прыёмамі ежы: мы не перакусваем, не гаворым пра ежу, не чытаем рэцэпты і не глядзім кулінарныя шоў. Цяпер многія праблемы са здароўем і харчаваньнем узьнікаюць менавіта праз звычку пастаянна перакусваць, прычым многія людзі недаацэньваюць яе важкасьць.

Вядзеньне «дзёньніка перакусаў» дазволіць убачыць сотні й~тысячы лішніх калёрыяў, неўзаметку зьедзеных за дзень. Многія перакусы ваш мозг і зусім не заўважае~--- вы ясьцё на аўтамаце. Апроч калёрыяў, для нас важны і ўплыў перакусаў на рэжым харчаваньня ды мэтабалізм. Уявіце сабе, што прыём ежы~--- гэта працэс мыцьця. Вы зьбіраеце розную бялізну, загружаеце яе ў~пральную машыну, засыпаеце парашок і ўключаеце пэўную праграму. Калі вы забыліся памыць ручнік, дык не выключаеце пралку, выдзіраючы дзьверцы? Напраўду гэтак жа й~зь ежай: пасьля яды ня варта ўмешвацца ў~працу страўнікава-кішачнага тракту.

\section{Як зьявілася праблема?}

Традыцыйна прыём ежы быў строга рэглямэнтаваны, людзі елі за агульным сталом у~вызначаны час. Нашы бабулі забаранялі «цягаць ежу», «псаваць апэтыт» і лічылі вельмі важным «апэтыт нагуляць» і «есьці за сталом». У~шматлікіх краінах, напрыклад у~Францыі, перакусы пад забаронай. На жаль, паступова гэтая культура харчаваньня разбураецца: так, да 72\,\% жанчын у~заходніх краінах перакусваюць. Структураванае харчаваньне (у арганізаваныя прыёмы ежы) замяняецца ежай на хаду, і гэта абумоўлівае зьяўленьне шматлікіх праблем са здароўем.

\subsection{Раней людзі елі радзей.}
З прычыны адсутнасьці лядоўняў і гатовых прадуктаў правіла чыстых прамежкаў выконвалася ў~ранейшыя часы няўхільна, але сёньня нас увесь час атачае мноства ўжо гатовай да спажываньня ежы. Традыцыйна сталаваньне было рэглямэнтаванае ў~розных культурах, прыступаць да яды можна было толькі за абеднім сталом, выконваючы пэўныя рытуалы. Нават у~поле работнікі бралі абрус і елі ў~пэўны час. Перакусы, ежа на хаду, ежа ў~адзіноце былі вельмі рэдкія. Дасьледнікі выявілі, што з~1950-х гадоў колькасьць прыёмаў ежы няўхільна ўзрастае.

\tipbox{Традыцыйна сталаваньне было рэглямэнтаванае ў~розных культурах, прыступаць да яды можна было толькі за абеднім сталом, выконваючы пэўныя рытуалы. Нават у~поле работнікі бралі абрус і елі ў~пэўны час.}

\subsection{Цяпер людзі ядуць вельмі часта.}
Сёньня мы назіраем эпідэмію перакусаў~--- людзі ядуць часта, на хаду, па-за спэцыяльнымі зонамі й~часавымі правіламі. Прыём ежы стаў выпадковы і хаатычны. Гэта зьвязана са зьяўленьнем вялікай разнастайнасьці й~паўсюднай даступнасьці гатовай ежы, з~разбурэньнем традыцыйнай культуры харчаваньня, з~агрэсіўнай рэклямай. Ежа паўсюль, дзе толькі магчыма, пачала, на жаль, успрымацца як норма. Чалавек стаў кіравацца ня голадам, а~вонкавымі стымуламі. Немагчыма выбудаваць здаровыя харчовыя звычкі пры такім хаатычным харчаваньні! Мне хочацца, каб кожны зразумеў, што рэжым харчаваньня~--- гэта хрыбет вашай дыеты, безь яго немагчыма прытрымлівацца здаровых звычак у~доўгатэрміновай пэрспэктыве!

\section{Як гэта ўплывае на здароўе?}

\subsection{Парушэньне мэтабалізму й~працы гармонаў.}
Перакусы зьмяншаюць узровень гармону грэліну, які адказвае за адчуваньне голаду, а~гэта, у~сваю чаргу, можа павялічваць трывогу й~зьмяншаць узровень задаволенасьці. Калі ў~страўнік трапляе нават адносна невялікая колькасьць ежы, адразу пачынаецца выдзяленьне стрававальных фэрмэнтаў, выпрацоўка інсуліну, зьніжэньне ўзроўню грэліну. Гэтыя ваганьні, якія паўтараюцца шматкроць на працягу дня, могуць мець неспрыяльны эфэкт у~выглядзе зьніжэньня адчувальнасьці да гармонаў (рэзыстэнтнасьць), што шкодна для здароўя. Асабліва шкодныя парушэньні выпрацоўкі гармону лептыну, які разам з~грэлінам зьяўляецца галоўным рэгулятарам апэтыту. Перакусы правакуюць ваганьні глюкозы ў~крыві («цукровыя арэлі»). Напэўна, вы заўважалі, што нават маленькі кавалачак ежы на тле сытасьці можа абудзіць пачуцьцё голаду.

\subsection{Пераяданьне.}
Адзін з~пабочных эфэктаў перакусаў~--- гэта незаўважнае ўжываньне вялікай колькасьці калёрыяў па-за прыёмамі ежы. Прадукты для перакусаў часьцей за ўсё маюць высокую колькасьць калёрыяў і валодаюць здольнасьцю ўзмацняць апэтыт. Выпадкова пад'едзены кавалачак на працы, у~стрэсе, з~гарбатай~--- усё гэта ператвараецца ў~незаўважнае для нас, але выяўнае пераяданьне. Яно шкодзіць здароўю, нават калі пакуль не адбіваецца на выглядзе цела, бо самы небясьпечны тлушч не падскурны, а~нутравы (вісцэральны). Ежа на хаду, за стырном, падчас тэлефоннай размовы таксама зьніжае ўзровень насычэньня й~вядзе да залішняга спажываньня ежы. Перакусы шкодныя й~людзям без атлусьценьня. Так, навукоўцы высьветлі, што й~людзі з~нармальнай вагой палепшаць сваё здароўе, адмовіўшыся ад 200--300\,ккал на дзень, а~дзеці часам зьядаюць падчас перакусаў больш, чым у~асноўныя прыёмы ежы, і бацькі дзівяцца адсутнасьці ў~іх апэтыту. Адмоўцеся ад перакусаў~--- і вы вернеце сабе апэтыт!

\subsection{Парушэньне харчовага рэжыму і харчовых паводзінаў.}
Чым болей перакусаў, тым меней карыснай ежы вы зьясьцё ў~асноўныя прыёмы ежы. Фармуецца заганнае кола: перакусваючы, вы менш зьядаеце ў~асноўны прыём ежы і зноў хочаце есьці. Гэта парушае нармальныя харчовыя паводзіны й~закладае нездаровыя звычкі. Я~бачу, як людзі зьвяртаюцца да перакусаў для ўзьняцьця настрою пры стрэсе: такая звычка можа прывесьці да парушэньня нармальнай працы мозгу й~павялічыць рызыку дэпрэсіі.

\tipbox{Дзеці часам зьядаюць падчас перакусаў больш, чым у~асноўныя прыёмы ежы, і бацькі дзівяцца адсутнасьці ў~іх апэтыту. Адмоўцеся ад перакусаў~--- і вы вернеце сабе апэтыт!}

\subsection{Іншыя прычыны.}
Пэрыядычныя, альбо хаатычныя, перакусы павялічваюць рызыку для здароўя, а~вось іх перавагі для здароўя маюць слабую навуковую базу. Перакусы павялічваюць рызыку захворваньняў печані, такіх як тлушчавая дыстрафія печані. Зьніжаецца інтэнсіўнасьць клеткавага аднаўленьня і самаачышчэньня~--- аўтафагіі. Акрамя таго, павелічэньне частаты харчаваньня ўзмацняе нагрузку на печань і ўзровень кіслотнай нагрузкі на зубы. На жаль, мы бачым актыўную прапаганду перакусаў у~дыеталёгіі, што ня йдзе на карысьць здароўю. Актыўную прапаганду перакусаў вядуць і вытворцы розных батончыкаў ды гатовых прадуктаў.

\section{Асноўныя прынцыпы}

\subsection{Датрымлівайцеся чыстых прамежкаў паміж прыёмамі ежы.}

Важным правілам здаровага рэжыму харчаваньня зьяўляецца пільнаваньне бескалярыйных інтэрвалаў паміж прыёмамі ежы. Памятайце, што магія адбываецца не тады, калі вы ясьцё, а~тады, калі не ясьцё! Скасуйце паступленьне калёрыяў у~любым выглядзе, у~тым ліку з~напоямі.

Важна разумець, што нават невялікая колькасьць калёрыяў у~выглядзе перакусаў~--- гэта ня проста мэханічны дадатак да сумы зьедзенага, а, па сутнасьці, дадатковы прыём ежы, запуск усяго працэсу страваваньня (выдзяленьне фэрмэнтаў, зьмяненьне маторыкі кішачніка, выдзяленьне кішачных і мэтабалічных гармонаў). Датрыманьне правіла чыстых прамежкаў~--- гэта падмурак здаровых харчовых паводзінаў, умацаваньне самадысцыпліны й~паляпшэньня мэтабалізму.

\subsection{Сэнсарная стымуляцыя} (выгляд і пах ежы, размовы пра ежу і г.~д.). Навукова даведзена, што пах і выгляд ежы могуць уплываць на выдзяленьне інсуліну. Таму найболей камфортнай для працы будзе асяродак без харчовых стымулаў. Не працуйце за абедзенным сталом і не абедайце за працоўным.

\subsection{Толькі вада.}

Падчас пустых інтэрвалаў дапушчальна піць толькі негазаваную ваду. У~рэдкіх выключэньнях~--- некалярыйныя напоі без кафеіну, зёлкі, гарбату каркадэ, газаваную ваду, але лепш абысьціся бязь іх. Гарбату й~каву варта піць безь вяршкоў, цукру, цукразаменьнікаў і, вядома ж, дэсэрту. Пажадана абысьціся й~бяз жуйкі.

\section{Як трымацца правіла? Ідэі ды парады}

\subsection{Рэжым харчаваньня.}

Памятайце, што рэжым~--- перадусім. Рэжым харчаваньня~--- гэта лад, структура харчаваньня, самае галоўнае ў~паляпшэньні вашага здароўя. Генэтычна мы схільныя кантраляваць доступ ежы не абмежаваньнем калёрыяў, а~абмежаваньнем часу доступу да яе. Паступова зь цягам часу такі рэжым харчаваньня стане звычкай, якая будзе апірышчам, хрыбтом вашага харчаваньня.

\subsection{Устрыманасьць~--- гэта прыкмета годнасьці.}
Бо калі вы захацелі ў~прыбіральню, вы ж не здавальняеце свой позыў проста на вуліцы. Чаму павінна быць інакш зь ежай? Стаўцеся да чыстых прамежкаў як да трэніроўкі сваёй умеранасьці й~праявы дысцыпліны. Гартуйце свой дух.

\subsection{Зьмяніце харчовы асяродак.} 
Вядома, ставіць замкі на лядоўню неабавязкова, але ня варта правакаваць сябе лішні раз. Калі вы зьбіраецеся есьці прысмакі, дык рабіце гэта ў~асноўны прыём ежы, а~ня ў~чысты прамежак. Наша сіла волі абмежаваная, ня трэба дазваляць ежы зьнясільваць яе.

\subsection{Не таргуйцеся з~сабой.} 
Ня варта абмяркоўваць забароны ці таргавацца з~сабой. Пераключыце ўвагу на свае штодзённыя клопаты, а~пра ежу думайце падчас ежы. Мы заўсёды схільныя паддавацца харчовым спакусам, калі адсутнічае правіла «ня есьці».

\tipbox{Чым мацней вы гоніце думкі пра ежу, тым часьцей яны прыходзяць. Паспрабуйце пажартаваць зь сябе й~пасьмяяцца са спакусы. Гэта сапраўды сьмешна, бо вы~--- дарослы чалавек, які пакутуе праз кавалак смажанага цеста з~тлушчам!}

\subsection{Неадчэпныя думкі пра ежу.}
Чым мацней вы гоніце думкі пра ежу, тым часьцей яны прыходзяць. Пераключайце ўвагу: прайдзіцеся, папрысядайце. Паспрабуйце давесьці думкі да абсурду (думкі пра пончык: супэргерой чалавек-пончык, машына-пончык, колькі пончыкаў я зьем, пакуль мяне не парве, і г.~д.). Пажартуйце зь сябе і пасьмейцеся са спакусы. Ня сьмешна? Але ж вы дарослы чалавек, які пакутуе ад кавалка смажанага цеста з~тлушчам! Гэта сапраўды сьмешна.

\subsection{Не глядзіце фуд-порна.}
Пад тэрмінам «фуд-порна» разумеюць узбуджальныя выявы ежы ў~сацсетках, часопісах і на сайтах. Устрымлівайцеся ад разгляданьня карцінак з~такой ежай, абмеркаваньня рэцэптаў, прагляду кулінарных перадач. Ежа~--- гэта толькі частка жыцьця, ня трэба запаўняць ёю больш часу, чым неабходна. Навукова даведзена, што разгляд выяваў ежы можа выклікаць голад і жаданьне есьці ў~абсалютна сытага чалавека.

\tipbox{Ежце толькі за абедзенным сталом, а~не на хаду ці за ноўтбукам. Апошняе небясьпечнае выпрацоўкай умоўнага рэфлексу, калі адно выгляд ноўтбука будзе наганяць апэтыт.}

\subsection{Не забараняйце, а~ўводзьце правілы.}
Не факусуйцеся на тым, што вам забаронена або нельга. Бо забароны могуць узмацняць цягу да ежы. Стварэньне свайго рэжыму харчаваньня~--- гэта не забароны, а~правілы харчаваньня. Вы можаце зьесьці тое, што хочаце, але зрабіце гэта ў~наступны прыём ежы. Забаронаў няма, ды ёсьць дакладныя правілы. Структуруйце і парадкуйце сваё сілкаваньне, а~не забараняйце сабе ўсё запар!

\subsection{Стварайце пазытыўныя харчовыя звычкі.}
Ежце толькі за абедзенным сталом, а~не на хаду ці за ноўтбукам. Апошняе небясьпечнае выпрацоўкай умоўнага рэфлексу, калі адно выгляд ноўтбука будзе наганяць апэтыт. Выразна вызначыце абедзеннае месца, упрыгожце стол абрусам~--- гэта створыць настрой.

\subsection{Апэтыт~--- гэта добра.}
Нармальны апэтыт~--- прыкмета здароўя. За некалькі гадзінаў устрыманьня ад ежы вы не сапсуяце сабе кіслатой страўнік, не запаволіце мэтабалізм і не зьясьцё ўдвая болей пасьля. Гэта міты, шмат якія зь іх мы разьбяром па ходзе кнігі.

\subsection{Перабіце цягу.}
Калі вам складана кантраляваць апэтыт, дапамогуць інтэнсіўныя смакі~--- кіслы або востры. Запараныя вострыя прыправы (карыца, бадзян, кардамон) або каркадэ карысныя для здароўя, практычна ня ўтрымліваюць калёрыяў і дапамагаюць кантраляваць апэтыт.

\subsection{Найменей шкодны перакус.}
Калі ўжо сытуацыя такая невыносная, што конча трэба перакусіць, то ўстрымайцеся ад мучных, салодкіх прадуктаў з~высокай глікемічнай нагрузкай, альбо індэксам, аддайце перавагу сырой расьліннай ежы ці тлушчам. Гэта могуць быць сырыя арэхі, сырая гародніна, якую можна пагрызьці (морква, салера, перац), або зялёная салата. Таксама можна дадаць у~гарбату ці ўжыць асобна лыжку масла ці алею~--- сьметанковага, какосавага, аліўкавага,~--- тлушчы эфэктыўна сьцішваюць цягу да салодкага.

\subsection{Звычкі будуць вас падводзіць, але не хвалюйцеся.}
Спачатку можа быць складана, магчымыя зрывы нават аўтаматычныя, калі вы задумаліся аб нечым і апрытомнелі ўжо зь ежай у~руках. Гэта вынік вашых доўгатэрміновых звычак і сумневаў. Калі вы перакусвалі ў~машыне, то мозг будзе прапанаваць зрабіць вам гэта зноў. Не падмацоўвайце звычкі~--- і яны счэзнуць.

\chapter{Харчовае вакно}

Харчовае вакно~--- гэта час паміж першым і апошнім прыёмам ежы на працягу сутак. Таксама вылучаюць харчовую паўзу~--- начны час ад апошняга да першага прыёму ежы на наступны дзень. У~цэлым чым вузейшае харчовае вакно і шырэйшая харчовая паўза (у разумных межах), тым болей карысьці для здароўя нават безь сьвядомага абмежаваньня каляражу. 

Без паступленьня калёрыяў у~арганізьме паскараецца працэс самаачышчэньня клетак (аўтафагіі), спаленьня тлушчу, павялічваецца адчувальнасьць арганізму да розных гармонаў, зьмяншаецца ўзровень запаленьня, а~таксама зь цягам часу павялічваецца частка цягліцавай масы. Дасьледаваньні паказалі, што такая дыета прадухіляе зьяўленьне парушэньняў абмену рэчываў, у~тым ліку дыябэт другога тыпу, пячоначны стэатоз (залішні тлушч у~печані) і высокі ўзровень халестэрыну ў~крыві.

\tipbox{Харчовае вакно~--- гэта час паміж першым і апошнім прыёмам ежы на працягу сутак. Чым карацейшае харчовае вакно і даўжэйшая харчовая паўза, тым болей карысьці для здароўя.}

Уявіце сабе, што час, калі мы ямо, і час, калі не ямо, па-рознаму ўплывае на наша здароўе. Што пераважыць на шалях? Дзіўна, але ўсе цуды (спаленьне тлушчу, аўтафагія, павышэньне адчувальнасьці да гармонаў) адбываюцца ў~той прамежак, калі мы не ямо. Таму, каб паскорыць станоўчыя працэсы, трэба даўжэйшы прамежак часу ў~соднях устрымлівацца ад ежы! Калі ў~нас 12 на 12, то гэта ўсяго толькі роўныя ўмовы, але нават невялікі зрух на 2 гадзіны (10~на~14) ужо дае нашаму арганізму пэўныя перавагі, і шалі схіляюцца да спаленьня тлушчу. Абмежаваньне харчовага вакна ў~8 гадзін дае ўдвая большую перавагу і яшчэ мацней схіляе шалі ў~бок жаданых эфэктаў!

\subsection{Як зьвілася праблема?}

\paragraph{Раней ежа была даступная кароткі час цягам содняў.}

Сучасны чалавек атрымаў у~спадчыну ад сваіх продкаў асаблівыя «сквапныя» або «ашчадныя» гены. Рэч у~тым, што раней ежы было мала, але ў~некаторыя сэзоны была яе празьмернасьць (сэзон паляваньня, ураджаю). Прылад для захоўваньня ежы не было, таму нашы «сквапныя» гены дапамагалі нам запасіць яе ў~выглядзе тлушчу, каб перажыць голад ці недахоп ежы. Цяпер нашы гены штурхаюць нас пад'есьці ў~любы час.

\paragraph{Цяпер ежа даступная цэлыя содні і ў~гатовым выглядзе.}

Вельмі часта наша харчовае вакно расьцягваецца на 14 і больш гадзін, людзі ядуць з~самай раніцы і да позьняга вечара. А~калі ежа, асабліва высокакалярыйная, увесь час ёсьць у~доступе, мы схільныя есьці ў~запас. Пры гэтым працяглыя прамежкі часу, калі мы маглі б спаліць назапашаны тлушч і павысіць адчувальнасьць да гармонаў, адсутнічаюць. Чым шырэйшае харчовае вакно, тым вышэйшая імавернасьць набору вагі. Больш за тое, прыём ежы блякуе працэсы самаачышчэньня клетак, таму нашаму арганізму застаецца менш часу на самааднаўленьне.

\subsection{Як гэта ўплывае на здароўе?}

\paragraph{Прыводзіць да пераяданьня.}
Заканамернасьць простая: чым шырэйшае харчовае вакно, тым вышэйшая рызыка пераяданьня. Скарачэньне часавага прамежку паміж першым і апошнім прыёмамі ежы аўтаматычна вядзе да таго, што вы ясьцё менш, але пры гэтым не абмяжоўваеце сябе сьвядома ў~колькасьці калёрыяў.

\paragraph{Спаленьне тлушчу.}
Калі мы ямо, узровень гармону інсуліну павялічваецца. Больш высокія ўзроўні яго ўтрыманьня блякуюць спаленьне тлушчу, таму пры частых прыёмах ежы нашы шанцы на тлушчаспаленьне зьніжаюцца. Зь цягам часу гэта можа прывесьці да зьніжэньня адчувальнасьці нашага арганізму да інсуліну, што шкодна для здароўя.

\paragraph{Узмацняе рызыку іншых хваробаў.}
Кожны прыём ежы актывуе адмысловыя малекулярныя комплексы mTORС\footnote{Мішэнь рапаміцыну млекакормячых, mammalian target of rapamycin
complex (mTORС)~--- фэрмэнт, які грае галоўную ролю ў~клеткавым 
росьце.~--- \emph{Аўтарская нататка.}} у~нашых клетках. Яго хранічная гіпэрактыўнасьць павялічвае рызыку алергічных і аўтаімунных захворваньняў, падвышанага артэрыяльнага ціску.

\paragraph{Запавольвае самааднаўленьне.}\!
Пастаянная актыўнасьць mTORС у~нашых клетках, выкліканая шырокім харчовым вакном, прыводзіць да зьніжэньня актыўнасьці аўтафагіі.

\paragraph{Цыркадныя рытмы.}
Чым шырэйшае ваша харчовае вакно, тым мацней зьбіваецца работа сутачных рытмаў. Так, прыём ежы позна ўначы здольны парушыць нармальную працу цыркадных рытмаў. А~парушэньне працы нашых унутраных гадзінаў (дэсынхранізацыя) ляжыць у~аснове шматлікіх захворваньняў.

\paragraph{Парушае працу гармонаў.}
Пастаянны прыём ежы павышае ўзровень інсуліну і зьніжае ўзровень гармону росту і грэліну. Першае ўзмацняе сымптомы дэпрэсіі, зьніжаючы нэўрапластычнасьць. Другое запавольвае аўтафагію, спаленьне тлушчу і рост цягліцавай масы, памяншае адчувальнасьць да інсуліну.

\paragraph{Рызыка карыесу.}
Чым большая нагрузка на зубы, тым горш спраўляецца сыстэма самаачышчэньня ў~ротавай поласьці. А~здароўе зубоў~--- гэта ня толькі карыес, але яшчэ і рызыка сардэчна-сасудзістых захворваньняў!

\paragraph{Эфэктыўнае і пры захворваньнях.}
Вузкае харчовае вакно павышае адчувальнасьць да інсуліну ў~дыябэтыкаў, зьніжае аксыдатыўны стрэс, узровень запаленьня, запавольвае старэньне, зьніжае рызыку гіпэртэнзіі, паляпшае ліпідны профіль. Дасьледаваньні на жывёлах паказваюць, што абмежаваньне часу харчаваньня стымулюе рост новых нэрвовых клетак, палягчае сымптомы дэпрэсіі й~нават паляпшае стан пры нэўрадэгенэратыўных захворваньнях, у~тым ліку хваробах Альцгеймэра і Паркінсона.

\subsection{Асноўныя прынцыпы}

\paragraph{Звужайце харчовае вакно.}
Умоўна падзяліце содні на два прамежкі. У~адзін зь іх можна есьці (харчовае вакно), у~іншы прамежак неабходна абыходзіцца без калёрыяў (харчовая паўза). Безадносна колькасьці спажываных калёрыяў, звужэньне харчовага вакна і павелічэньне часу бязь ежы станоўча ўплываюць на здароўе.

\tipbox{Для пачатку паспрабуйце пакінуць 3--4 вольныя ад прыёму ежы гадзіны перад сном. Напрыклад, сьняданак а~7-й, вячэра а~19-й, сон а~23-й.}

Існуе некалькі варыянтаў абмежаваньня харчаваньня, але іх аб'ядноўвае агульны прынцып: пэрыяд голаду і пэрыяд прыёму ежы, або харчовае вакно. Часта звужэньне харчовага вакна называюць абмежаваньнем часу харчаваньня~--- TRF (time restricted feeding), часам гэтае правіла гучыць як «ня есьці пасьля 17-й або 18-й» і інш. Шмат хто дзеля скарачэньня харчовага вакна прапускае сьняданак, але гэта не зусім правільна, чаму~--- абмяркуем у~наступным разьдзеле.

\paragraph{12/12}
Калі вы ня маеце абмежаваньняў для яды, для пачатку паспрабуйце пакінуць 3--4 вольныя ад прыёму ежы гадзіны перад сном. Напрыклад, сьняданак а~7-й, вячэра а~19-й, сон а~23-й гадзіне. Вядома, гэта яшчэ не звужэньне харчовага вакна, але першы крок да гэтага. Трымайце цалкам чысты прамежак ад вячэры да сну.

\paragraph{10/14}
Гэта аблегчаная вэрсія звужэньня харчовага вакна, часта вядомая як правіла «ня есьці пасьля 18:00». Вы сьнедаеце а~7-й, вячэраеце а~17-й і пасьля 18-й нічога не ясьцё.

\paragraph{8/16}
Усе прыёмы ежы ўкладаюцца ў~8 гадзін, а~інтэрвалы паміж імі~--- у~16. Самая распаўсюджаная практыка: па сутнасьці, харчовая паўза акурат у~два разы даўжэйшая за харчовае вакно. Даем двухразовую перавагу спаленьня тлушчу і аўтафагіі. Выконваць правіла проста: укладзіце ўсе свае прыёмы ежы ў~прамежак 8 гадзін. Не абмяжоўвайце сябе ў~прадуктах і памеры порцыі, ежце як звычайна і ўволю. Па харчовым вакне трымайце чысты прамежак у~16 гадзін. Напрыклад, вы пасьнедалі а~8-й, паабедалі а~12-й і павячэралі а~16-й. Выдатна працуе, у~тым ліку і для атлетаў. Зручны варыянт і для тых, хто працуе, бо можа зьмяшчаць як два, так і тры прыёмы ежы.

\tipbox{Без паступленьня калёрыяў паскараецца працэс самаачышчэньня клетак, спаленьня тлушчу, павялічваецца адчувальнасьць арганізму да розных гармонаў.}

\paragraph{6/18}
Вы сьнедаеце і абедаеце грунтоўна ды прапускаеце вячэру. Важна сытна паабедаць і пасьнедаць, тады да ночы апэтыт добра кантралюецца. Зручная схэма для людзей, якія працуюць у~офісе. У~8:00 вы сытна сьнедаеце, у~14:00 сытна абедаеце. Цікава, што такая схэма выяўляецца эфэктыўнай ня толькі для здаровых людзей, але й~для тых, хто хварэе на цукровы дыябэт II тыпу. Дыябэтыкі на двухразовым харчаваньні хутчэй худнеюць і аднаўляюць адчувальнасьць да інсуліну.

\paragraph{4/20}
Часам вядомая як «дыета ваяра». У~арыгінальным варыянце харчовае вакно ўвечары, але яно можа быць і днём. Есьці ў~4 гадзіны болей пасуе мужчынам і ў~больш рэдкіх сытуацыях. Не рэкамэндуецца для рэгулярнага выкарыстаньня.

\paragraph{Лічыце гадзіны, а~не калёрыі.}
Перавага харчовага вакна ў~тым, што вы скарачаеце колькасьць калёрыяў у~ежы безь іх вылічэньня. Пры абмежаваньні харчаваньня, скажам, на 10-гадзіннае вакно людзі аўтаматычна ядуць на 20\,\% менш без падліку калёрыяў і самапрымусу. Гэта не ўплывае на здольнасьць набіраць цягліцы, атлеты могуць нарошчваць масу, сілкуючыся і ў~чатырохгадзіннае вакно. Чым даўжэйшае вакно, тым вышэй імавернасьць пераяданьня.

\subsection{Як трымацца правіла? Ідэі ды парады}

\paragraph{Дзейнічайце спакваля.}
Спачатку можна звузіць харчовае вакно да 12 гадзінаў (12 гадзінаў для прыёму ежы і 12 гадзінаў устрыманьня), станоўчыя эфэкты бачныя пры харчовым вакне ў~10 гадзінаў, але найлепшыя вынікі пры добрым трываньні дасягаюцца пры харчовым вакне ў~8 гадзінаў. У~апошнім выпадку час ад першага да апошняга прыёму ежы~--- 8 гадзінаў, а~вось харчовае ўстрыманьне роўна ўдвая даўжэйшае~--- 16 гадзінаў. Сістэма 8 на 16 простая і эфэктыўная. Таксама існуюць методыкі 4 на 20 ці 6 на 18, але да іх лепш пераходзіць на больш прасунутым этапе, і яны пасуюць ня ўсім.

\paragraph{Частата харчовых вокнаў 7/7, 6/7, 5/7.}
Гэты тып дыеты мае на ўвазе штодзённае датрыманьне такога раскладу. Аднак дасьледаваньні паказалі, што абмежаваньне часу харчаваньня ў~5/2 таксама працуе. Таму для пачатку вы можаце датрымлівацца правіла вузкага харчовага вакна ў~будні, а~ў выходныя вячэраць паводле іншага графіку~--- для падтрымкі сацыяльных сувязяў. Другі варыянт~--- паставіць адну познюю вячэру сярод тыдня, а~другую~--- на выходных. Падобныя схэмы 2\kern0.5pt+1 таксама паказваюць эфэктыўнасьць (два дні абмежаваньня, адзін дзень звычайнага харчаваньня) і могуць пасаваць пачаткоўцам.

\paragraph{Стрэс і харчовае вакно.}
Чым вышэйшы стрэс, тым шырэйшым павінна быць харчовае акно. Стрэс павышае ўзровень гармону картызолу і на галодны страўнік здольны нашкодзіць здароўю. Чым ніжэйшы стрэс, тым вузейшым можа быць харчовае вакно.

\paragraph{Ежце ўволю і не лічыце калёрыі (у разумных межах, вядома).}
Вельмі важна не зьмяншаць занадта рэзка колькасьць спажываных калёрыяў. Для гэтага трэба зьядаць больш, чым вы звыклі за адзін прыём ежы, бо часу на ежу цяпер меней. Перавага звужэньня харчовага вакна ў~параўнаньні зь іншымі дыетамі ў~тым, што падчас харчовага вакна адсутнічаюць абмежаваньні на памеры порцыі, вы можаце есьці ўволю. Вам ня трэба лічыць калёрыі і ўвесь час баяцца пераесьці. Ежце ўволю, з~задавальненьнем, у~адведзены час.

\paragraph{Трэніруйцеся сьмела.}
Вы можаце сумяшчаць правіла вузкага харчовага вакна з~трэніроўкамі, у~т.~л. і~зь сілавымі, і гэта ня будзе замінаць набору цягліцавай масы.

\paragraph{Сьвяты.}
Вы можаце мінімізаваць пабочныя эфэкты тлустай і салодкай ежы, калі ў~гэты дзень будзеце выконваць правіла харчовага вакна. Абмежаваньне харчаваньня дазваляе зьмякчыць нэгатыўнае ўзьдзеяньне высокага ўзроўню спажываньня тлушчу і цукру.

\tipbox{Якасны бялок, здаровыя тлушчы, складаныя вугляводы зь нізкім глікемічным індэксам дапамогуць захоўваць устойлівае насычэньне доўга, а~на паўфабрыкатах і фастфудзе вас з~большай імавернасьцю будзе мучыць голад.}

\paragraph{Ня бойцеся.}
Нягледзячы на ўсе асьцярогі, вы не расьцягнеце страўнік і не пераясьцё. «Расьцягнуты страўнік»~--- гэта міт. Мае пацыенты спачатку баяліся, што да вечара будуць адчуваць моцны голад, але на справе вынікла, што, калі ў~першай палове дня яны ядуць уволю, голаду не ўзьнікае. Трэнэры часта палохаюць стратай цяглічнай масы, калі вы не паясьцё пасьля трэніроўкі, але гэта ня мае належных падставаў. Вы можаце й~ня есьці пасьля трэніроўкі~--- гэта не нашкодзіць вашым цягліцам.

\paragraph{На гатовай ежы будзе складана.}
Спажывайце больш бялку і тлушчу для насычэньня, дадавайце гародніну і бабовыя. Праблемы з~голадам у~вас будуць, калі ёсьць вялікая колькасьць гатовай ежы, якая стымулюе пераяданьне. Якасны бялок, тлушч, складаныя вугляводы зь нізкім глікемічным індэксам дапамогуць захоўваць устойлівае насычэньне доўга, а~на паўфабрыкатах і фастфудзе вас з~большай імавернасьцю будзе мучыць голад.

\paragraph{Есьці ўволю з~задавальненьнем і без забаронаў.}
Чым часьцей і даўжэй цягам дня вы ясьцё, тым менш атрымліваеце задавальненьня ад ежы. Скарачэньне часу харчаваньня павялічвае задавальненьне ад ежы, пры гэтым у~вас няма жорсткіх забаронаў. Гэта паляпшае харчовыя паводзіны.

\paragraph{Парушэньні харчовых паводзінаў.}
Калі вы маеце схільнасьць да парушэньняў харчовых паводзінаў, вам варта пазьбягаць жорсткіх абмежаваньняў.

\chapter{Харчовы хранатып}

Харчовы хранатып~--- часавыя перавагі людзей, зьвязаныя з~тым, калі яны зьядаюць большую частку свайго штодзённага каляражу. Навукоўцы высьветлілі, што работа стрававальнай сыстэмы наўпрост зьвязаная з~работай сыстэмы нашых цыркадных (штодзённых) рытмаў, якія кантралююцца ня толькі з~мозгу (наш цэнтральны «гадзіньнік»~--- гэта адмысловая частка мозгу пад назвай «супрахіязматычнае ядро»), але й~з перыферычнага «гадзіньніка»~--- печані, цягліцаў, страўнікава-кішачнага тракту, белай тлушчавай тканкі й~іншых. Выпрацоўка гармонаў, адчувальнасьць да інсуліну, мэтабалічная актыўнасьць печані мяняюцца ў~розны час дня. У~працэсе эвалюцыі арганізм адаптаваўся да ваганьняў дня і ночы і адаптаваў да іх сваю працу, актыўнасьць шматлікіх генаў мяняецца ў~цыркадных рытмах.

У ідэале вакно харчаваньня мусіць супадаць са сьветлавым вакном дня і з~часам максымальнай актыўнасьці чалавека. Чым мацнейшае разыходжаньне часу харчовага вакна са сьветлавым днём, тым большая праблема. Зьядайце ранкам і ўдзень ня меней за 80\,\% агульнага дзённага каляражу. Разьмеркаваньне калёрыяў паміж сьняданкам, абедам і вячэрай у~суадносінах 50, 30 і 20\,\% паляпшае здароўе. Навуковыя дасьледаваньні паказалі, што адна й~тая ж ежа, зьедзеная ў~розны час дня, дзейнічае па-рознаму, і позьнія багатыя вячэры могуць павысіць узровень сыстэмнага запаленьня, дыябэту, пухлінаў і парушыць якасьць сну.

\subsection{Як зьявілася праблема?}

\paragraph{Адкладзены лад жыцьця.}
Са зьяўленьнем штучных крыніц сьвятла, працы раніцай і адпачынку ўвечары наш графік дзейнасьці паступова ссоўваецца на пазьнейшыя гадзіны, т.~зв. «адкладзены лад жыцьця». Ранкам мы сьпяшаемся на працу, прапусьціўшы сьняданак, увечары мы ямо шмат і сытна. Вялікая колькасьць штучных крыніц сьвятла, асабліва сьвятлодыёдаў, блякуе вытворчасьць мэлятаніну і зьбівае працу ўнутранага гадзіньніка і парушае апэтыт. Чым пазьнейшая вячэра, тым менш хочацца есьці ў~час сьняданку, так і ўтвараецца заганнае кола.

\paragraph{Начныя здавальненьні.}
Спрабуючы атрымаць болей здавальненьня, мы пачынаем есьці пазьней увечары. Ежа ў~цемры прыносіць нам больш асалоды, гэтая асаблівасьць зьвязаная з~рэцэптарамі мэлятаніну і дафаміну ў~мозгу. Уначы адчувальнасьць рэцэптараў дафаміну вышэйшая, і гэта вядзе да таго, што эклер і серыял чым пазьнейшыя, тым смачнейшыя. Такім чынам, людзі несьвядома падсаджваюцца на «позьняе здавальненьне»: чым менш радасьці й~прыемнасьцяў яны атрымліваюць на працягу дня, тым мацней хочуць расьцягнуць іх увечары, як многія кажуць~--- «жыць прынамсі ўвечары». Часта зрух харчовага вакна прыводзіць да «сындрому начной яды», калі людзі сытна сталуюцца позна ўвечары і ўночы, бадай што ня маючы змогі супрацьстаяць імпульсам падсілкавацца.

\subsection{Як гэта ўплывае на здароўе?}

\paragraph{Парушэньні вугляводнага абмену.}
Важны гармон інсулін і начны гармон мэлятанін шчыльна ўзаемадзейнічаюць. У~норме ўзроўні гармону сну мэлятаніну і інсуліну, які рэгулюе вугляводны абмен, знаходзяцца ва ўзаемазваротных адносінах. Такім чынам, лішак сьвятла і вугляводаў увечары парушае якасьць сну, а~вячэра ў~позьні час пагаршае засваеньне вугляводаў, што можа выклікаць мэтабалічныя парушэньні. У~рэшце рэшт, высокі інсулін можа зьнізіць узровень мэлятаніну і перашкодзіць яго выдзяленьню. А~зьніжэньне мэлятаніну й~працягласьці сну зь цягам часу зьнізіць адчувальнасьць да інсуліну.

\paragraph{Адчувальнасьць да інсуліну.}
Добрая адчувальнасьць да інсуліну зьвязаная з~мацнейшым здароўем. У~чалавека адчувальнасьць да інсуліну рэгулюецца цыркадным рытмам, зьмяншаючыся ўвечары і ноччу. Гэты працэс зьвязаны ня толькі з~нашым цэнтральным гадзіньнікам, супрахіязматычным ядром гіпаталямусу, але й~з~гадзіньнікамі печані, цягліцаў, падстраўніцы (падстраўнікавай залозы) і тлушчавай тканкі. Раніцай і ў~другой палове дня інсулінарэзыстэнтнасьць мінімальная, але ўвечары зьніжаецца адчувальнасьць і тлушчавых, і цягліцавых тканак і печані да інсуліну. Тыя, хто зьядаў траціну ці больш содневага рацыёну пасьля 18-й, мелі больш высокія ўзроўні глюкозы і глікіраванага гемаглабіну (маркер глікемічнага кантролю) у~крыві, больш высокі інсулін і вышэйшую рызыку дыябэту і павышанага артэрыяльнага ціску.

Спалучэньне сытнага сьняданку і лёгкай вячэры дапамагае лепей кантраляваць узровень глюкозы ў~крыві ў~параўнаньні са спалучэньнем «лёгкі сьняданак~--- сытная вячэра», розьніца ўзроўняў інсуліну і глюкозы ў~эксьперымэнтальных групах складала 20\,\%.

\paragraph{Сыстэмнае запаленьне і пухліны.}
Чым вышэйшы ўзровень інсуліну і сыстэмнага запаленьня, тым большая рызыка раку. Навукоўцы высьветлілі, што позьняя ежа ўвечары павялічвае рызыку раку малочнай залозы. Кожныя тры гадзіны, што мы не ямо да ночы, на 20\,\% зьніжаюць узровень глікіраванага гемаглабіну. І~наадварот, кожныя 10\,\% калёрыяў, зьедзеных пасьля 17-й, павялічваюць узровень запаленьня (ультраадчувальны C-рэактыўны бялок) на 3\,\%.

\paragraph{Іншыя праблемы і хваробы.}
Звычка есьці позна небясьпечная рызыкай атлусьценьня, саркапэніі (страта колькасьці цягліц), парушэньняў харчовых паводзінаў, дэпрэсіі, трывожных разладаў. Але раньняе харчовае вакно зьніжае ўзровень акісьляльнага стрэсу і прыводзіць да зьніжэньня вечаровага апэтыту! Яда на ноч пагаршае працу самататропнага гармону (СTГ), зьвязанага з~начной аўтафагіяй (самаачышчэньне клетак). Чым больш ежы ноччу, тым горшыя сон і аднаўленьне.

\subsection{Асноўныя прынцыпы}

\paragraph{Перасуньце харчовае вакно на дзённы час.}
Харчовае вакно, ссунутае на больш раньнія гадзіны, карыснае для здароўя, бо сынхранізуе наш графік харчаваньня і содневыя біярытмы. Аптымальна есьці ў~той час, калі ў~нашай крыві высокі ўзровень гармону актыўнасьці й~стрэсу картызолу і самы нізкі начнога гармону мэлятаніну. Зрушваючы харчовае вакно, мы сынхранізуем нашыя цыркадныя рытмы, лепей кантралюем апэтыт і голад. На сьветлы час дня прыпадае найлепшая адчувальнасьць нашага арганізму да інсуліну, таму ежа аптымальней засвойваецца, а~рызыка атлусьценьня ды іншых праблемаў са здароўем памяншаецца.

Гэты мэтад называецца па-рознаму: напрыклад, early Time-Restricted Feeding (еTRF)~--- раньняе кармленьне з~часавым абмежаваньнем. Падобны зрух харчовага вакна таксама выкарыстоўваецца і ў~некаторых традыцыйных практыках для падтрыманьня яснасьці розуму, таму будысцкім манахам дазваляецца есьці толькі да 12-й дня.

\paragraph{Адкінуць яду на ноч.}
Некаторыя людзі маюць звычку піць кефір альбо малако для моцнага сну і г.~д. Гэтага трэба пазьбягаць: ніякай яды ноччу.

\paragraph{Перагрупоўка каляражу.}
Захоўваючы ранейшы рэжым харчаваньня, пачніце разгружаць вячэру і павялічваць каляраж і аб'ём сьняданку і абеду. Для многіх людзей менавіта вячэра зьяўляецца самай сытнай ядой, таму захавайце яе аб'ём дзеля падтрымкі сытасьці ды зьменшыце каляраж. Для гэтага дадайце больш прадуктаў зь нізкай адноснай калярыйнай шчыльнасьцю: зеляніна, сырая і вараная гародніна. Дадавайце тлушчы, каб стымуляваць выдзяленьне гармону халецыстакініну, што падтрымлівае адчуваньне сытасьці. Для некаторых людзей наедным можа быць і бялок, але калі вы кепска сьпіце, то яго трэба пазьбягаць на вячэру. Дасьледаваньні паказалі, што зрушэньне каляражу зь вячэры на сьняданак з~захаваньнем часу яды~--- цалкам эфэктыўная стратэгія.

\paragraph{Непасрэдны зрух харчовага вакна.}
Сынхронна на паўгадзіны ці гадзіну зрушце на больш раньні час сьняданак і вячэру. Пры гэтым не імкніцеся да крайнасьці, зрух нават на гадзіну можа быць карысны і палепшыць якасьць сну. Для нас агульнай мэтай стане плыўны зрух і эвалюцыя вашага рэжыму ў~наступнай пасьлядоўнасьці:

✓ ня есьці на ноч;

✓ ня есьці за 2 гадзіны да сну;

✓ ня есьці за 3 гадзіны да сну;

✓ ня есьці за 4 гадзіны да сну (можа быць і 5, але 4 гадзіны дастаткова для звычайнай працы ўсіх гармонаў).

\paragraph{Прыклад.}
Вы сьнедаеце аб 11-й і вячэраеце а~21-й, кладзяцеся спаць а~23-й. Вы можаце паступова ссоўваць сваё харчовае вакно гэтак: 10:00–20:00, 9:00–19:00, 8:00–18:00. У~той жа час, вядома, вам давядзецца перакласьці свой графік засынаньня і абуджэньня.
Такім чынам, наша мэта можа быць сфармаваная так: скончыць есьці да 16-й, але таксама цалкам рэальныя і маюць абгрунтаваньне мэты скончыць да 17-й і нават клясыка «ня есьці пасьля 18-й». Для краінаў, разьмешчаных далей ад экватару, графік можа мяняцца ў~залежнасьці ад сэзону. У~такім разе скрайні вечаровы час на яду складае 19:00, а~для нашых шыротаў~--- 20:00 у~вяснова-летні час з~больш працяглым сьветлавым днём. Пры гэтым час засынаньня мусіць быць па магчымасьці не пазьней за 22:00, а~ўлетку з~доўгім сьветлавым днём магчыма й~да 23:00.

\paragraph{Вечаровы харчовы хранатып.}
У шэрагу выпадкаў могуць быць выключэньні з~правіла. Напрыклад, культурныя асаблівасьці ў~гарачых краінах, дзе празь сьпёку цягам дня ня хочацца есьці. У~многіх культурах вячэра грае сацыяльную ролю, калі яна зьбірае сваякоў і сяброў за адным сталом для зносінаў і адпачынку пасьля працы. Людзі, якія працуюць па вечарох, цалкам могуць аддаваць перавагу вечароваму харчоваму хранатыпу.

\subsection{Як трымацца правіла? Ідэі і парады.}

\paragraph{Дзейнічайце спакваля.}
Хутка зьмяніць свае звычкі цяжка і небясьпечна на зрыў. Таму разварочвайце свае біярытмы пакрысе і паступова. Крыху меней зьясьцё на вячэру~--- на сьняданак будзе больш апэтыту. Чым болей зьясьцё на сьняданак, тым менш захочацца на вячэру. Памятайце, што сыты вечар закладаецца раніцай. Тыя, хто прапускае сьняданак, як правіла, ядуць шмат на ноч. Такім чынам, бясплённа «ня есьці вечарам», калі вы кепска пасьнедалі і паабедалі; змагацца з~пачуцьцём голаду ўвечары, калі вы ўжо стаміліся і рэсурс сілы волі абмежаваны,~--- гэта сьвядома выракаць сябе на няўдачу. Пачынайце зьмены ад сьняданку.

\paragraph{Няма апэтыту на сьняданак?}
Калі няма апэтыту на сьняданак, то рабіце вячэру больш лёгкай альбо прапускайце яе. Тады раніцай у~вас будзе выдатны апэтыт і жаданьне есьці, вы зможаце зьесьці дастатковы каляраж. Па першым часе апэтыт будзе нестабільны, таму вам спатрэбіцца крыху прымусу, каб добра пасьнедаць. Калі вы сытна павячэралі, то ўсё роўна не прапускайце сьняданак.

\paragraph{Прачынайцеся крыху раней.}
Калі ваш графік дазваляе, прачынайцеся крыху раней, да восьмай раніцы вы ўжо мусіце пасьнедаць. Зразумела, гэта не заўсёды і не для ўсіх магчымае правіла. Калі ваш пік рабочай актыўнасьці прыпадае на вечар, то занадта раньняя вячэра можа не пасаваць вам.

\paragraph{Меней гатовай ежы ўдома.}
Калі ў~вас праблемы зь вячэрнім самакантролем, то трымайце менш гатовай ежы дома, а~таксама падумайце: магчыма, вы можаце павячэраць у~месцы зь якаснай ежай па дарозе дадому?

\paragraph{Майце плян на сьняданак.}
Падумайце, што і як вы будзеце гатаваць на сьняданак. Калі пляну няма, то імавернасьць зрабіць нясмачны і няправільны сьняданак вельмі высокая. У~ідэале прыгатаваньне ежы павінна заняць раніцай ня больш за 20 хвілін.

\paragraph{Болей задавальненьня ўдзень.}
Часта на пераяданьне ўвечары нас падштурхоўвае жаданьне атрымаць больш задавальненьня пасьля бязрадаснага дня. Таму абавязкова плянуйце сабе як мінімум адно асабістае задавальненьне на кожны вечар~--- і гэта адцягне вас ад ежы.

\paragraph{Калі такі ясьцё вячэру позна.}
У ідэале ежце нешта сырое зь вялікай колькасьцю клятчаткі, напрыклад пагрызіце моркву. Працэс дбайнага жаваньня здольны аслабіць стрэс і паменшыць голад. Сырыя расьлінныя прадукты (за выключэньнем садавіны) зь невялікай колькасьцю тлушчу будуць мець мінімальны нэгатыўны эфэкт, нават зьедзеныя позна ўвечары.

\chapter{Сьняданак}

Сняданак~--- гэта найважнейшы прыём ежы. У~ранішні час арганізм заводзіць наш унутраны гадзіньнік і настройвае біярытмы. Тое, як мы пасьнедаем, будзе шмат у~чым вызначаць нашае самаадчуваньне і энэргічнасьць днём, голад і сытасьць, а~таксама наш настрой і нават сон. Для многіх людзей раніца~--- гэта час без апэтыту і радасьці, але гэта няправільна. Наладзіўшы сваю раніцу, вы будзеце прачынацца са смакам да жыцьця і добрым апэтытам~--- і гэта значыць, што вы на слушным шляху да здароўя! Калі мае кліенты сустракаюць раніцу з~задавальненьнем і вяртаюць апэтыт, яны хутчэй здаравеюць.

Добры сьняданак дапамагае добра пачаць дзень і захаваць энэргічнасьць да самага вечара! Успомніце, калі вы адпраўляецеся ў~падарожжа і бэнзіну мала, вы адразу запраўляе поўны бак, каб затым ехаць спакойна і не баяцца таго, што паліва скончыцца. Пачынаючы дзень, важна папоўніць свае энэргетычныя запасы, каб цягам дня ваш арганізм спраўна функцыянаваў.

\tipbox{Для многіх людзей раніца~--- гэта час без апэтыту і радасьці, але гэта няправільна. Наладзіўшы сваю раніцу, вы будзеце прачынацца са смакам да жыцьця і добрым апэтытам~--- і гэта значыць, што вы на слушным шляху да здароўя!}

\subsection{Як зьявілася праблема?}
Павелічэньне тэмпу жыцьця і адсутнасьць рэжыму дня зрушваюць наш графік, мы вячэраем і кладзёмся спаць пазьней. Позьняя вячэра і ўздым прыводзяць да адсутнасьці часу для гатаваньня сьняданку, і ў~нас да таго ж няма апэтыту раніцай. Усё гэта фармуе заганнае кола, і мы ссоўваем гадзіны прыёмаў ежы на ўсё пазьнейшы час. Цяпер увогуле ў~сярэднім 30\,\% людзей ня сьнедаюць, многія пераходзяць на позьні сьняданак, а~колькасьць ужытых ранкам калёрыяў імкліва зьніжаецца.

\subsection{Як гэта ўплывае на здароўе?}

\paragraph{Сардэчна-сасудзістыя захворваньні.}
Навуковыя дасьледаваньні пацьвердзілі, што рэгулярны сьняданак станоўча ўплывае на здароўе сардэчна-сасудзістай сыстэмы і зьніжае рызыку многіх захворваньняў. Адсутнасьць сьняданку прыводзіць да парушэньня ліпіднага профілю крыві, павелічэньня атэрасклератычнай паразы сасудаў. Агульная колькасьць атэрасклератычных бляшак у~паўтара разы вышэйшая ў~тых, хто прапускае сьняданак. Акрамя таго, павялічваецца рызыка сардэчнага прыступу ды ішэміі сэрца са сьмяротным вынікам.

\paragraph{Атлусьценьне і цукровы дыябэт.}
Сьняданак зьніжае рызыку мэтабалічнага сындрому (атлусьценьне, дыябэт, гіпэртэнзія і да т.~п.). Большасьць людзей, якія пакутуюць на атлусьценьне, ня маюць звычкі сьнедаць або зьядаюць за сьняданкам вельмі мала. Тыя, хто рэгулярна сьнедае, маюць на 30\,\% меншую імавернасьць атлусьценьня і ў~два разы зьніжаюць рызыку цукровага дыябэту.

\paragraph{Фэртыльнасьць.}
Сьняданак уплывае нават на разьвіцьцё захворваньняў, якія на першы погляд складана зьвязаць з~харчаваньнем. Напрыклад, на полікістоз яечнікаў, што ўзьнікае ў~сувязі з~інсулінарэзыстэнтнасьцю. Сытны сьняданак і нізкакалярыйная вячэра дапамагаюць павысіць адчувальнасьць да інсуліну і палепшыць гарманальны профіль, падвышаючы шанцы на зачацьце.

\paragraph{Сыстэма «голад~--- сытасьць».}
Адсутнасьць сьняданку прыводзіць да парушэньня насычэньня і паніжанага кантролю голаду на працягу ўсяго дня. Навукоўцы высьветлілі, што маладыя людзі, якія прапускаюць сьняданак, зьядаюць на 40\,\% больш прысмакаў цягам дня.

\tipbox{Спалучэньне сьняданку зь яркім сонечным сьвятлом, фізычнай актыўнасьцю і прыемным абуджэньнем~--- гарантыя добрага пачатку дня.}

\paragraph{Прадуктыўнасьць, стрэс.}
Рэгулярны сьняданак спрыяе паляпшэньню канцэнтрацыі ўвагі, стабілізуе ўзровень цукру ў~крыві. IQ дзяцей, якія рэгулярна сьнедаюць, вышэйшае. Пры адсутнасьці сьняданку зьмяншаецца наша стрэсаўстойлівасьць, бо раніцай самы высокі ўзровень картызолу, што на тле адсутнасьці ежы павялічвае рызыку разбурэньня цягліц і назапашваньня тлушчу.

\subsection{Асноўныя прынцыпы}

Здаровы сьняданак~--- гэта рэгулярны, сытны, высокабялковы раньні сьняданак пры дастатковым узроўні асьвятленьня.
Такі сьняданак спрыяе спаленьню тлушчу, пры гэтым мацней выдаткоўваецца менавіта вісцэральны тлушч, а~не падскурны (інтэнсіўней зьніжаецца ахоп таліі). Сьняданак дапамагае кантраляваць узровень грэліну, таму вы пачуваецеся сытым на працягу ўсяго дня. Сьняданак стабілізуе ваганьні цукру ў~крыві, паляпшае настрой і прадуктыўнасьць. Спалучэньне сьняданку зь яркім сьвятлом, фізычнай актыўнасьцю і прыемным абуджэньнем зьяўляецца гарантыяй добрага пачатку дня. Бо менавіта раніца крытычна важная для нашага здароўя: яна дазваляе перазагрузіць, наладзіць і сынхранізаваць працу цыркадных рытмаў.

\paragraph{Высокабялковы сьняданак.}
Высокабялковыя прадукты на сьняданак засвойваюцца і насычаюць лепей, чым у~любы іншы прыём ежы. Так, сньяданак з~двух яйкаў паказаў сябе лепш, чым аналягічная па калёрыях колькасьць вугляводаў. Пасьля сьняданку з~падвышанай колькасьцю бялку моладзь ела на вячэру на 200\,ккал менш, а~на працягу дня не адчувала жаданьня чым-небудзь перакусіць. Сьняданак павінен утрымоўваць 50--80 грамаў якаснага бялку і дастатковую колькасьць тлушчаў. Вугляводы на сьняданак павінны складаць меншую долю калёрыяў, кашу варта замяніць гароднінай.

\paragraph{Сытны па аб'ёме (breakfast diet).}
Сьняданак павінен складаць больш за 30\,\% аб'ёму сутачнага спажываньня калёрыяў. Ухапіць кавалачак чаго-небудзь з~кавай ня лічыцца паўнавартасным сьняданкам. Важнасьць сытнага сьняданку~--- гэта факт, які знайшоў адлюстраваньне ня толькі ў~навуковых дасьледаваньнях, але і ў~традыцыйнай мудрасьці: «Сьняданак~--- золата, абед~--- срэбра, вячэра~--- медзь», «Сьняданак зьеж сам, абедам падзяліся зь сябрам, а~вячэру аддай ворагу», «Сьнедай як кароль, вячэрай як жабрак» і шматлікія іншыя.

\paragraph{Раньні сьняданак.}
Сьнедаць трэба ў~першую гадзіну пасьля абуджэньня. Каб пасьпяваць, вы мусіце прыгатаваць усё зь вечара. Прачнуліся~--- уключайце пліту і займайцеся ранішнімі справамі.

\tipbox{Паўгадзіны сонечнага сьвятла раніцай~--- гэта навукова абгрунтаваная, эфэктыўная працэдура для здаровай працы цыркадных рытмаў. Зімой можна сьнедаць з~уключанымі яркімі лямпамі для фотатэрапіі.}

\paragraph{Яркі сьняданак.}
З самае раніцы трэба ўключаць яркае сьвятло, якое зьнізіць узровень мэлятаніну і картызолу. Ранішняе сьвятло павялічвае адчувальнасьць да інсуліну і, як вынік, спрыяе пахудзеньню. Сонечнае сьвятло палепшыць настрой і павялічыць выпрацоўку дафаміну, бо дафамінавыя нэўроны ёсьць і ў~сятчатцы вока, дзе ўтвараюць асаблівую рэтына-дафамінавую сыстэму. Паўгадзіны сонечнага сьвятла раніцай~--- гэта навукова абгрунтаваная, эфэктыўная працэдура для здаровай працы цыркадных рытмаў. Зімой можна сьнедаць з~уключанымі яркімі лямпамі для фотатэрапіі.

\subsection{Як трымацца правіла? Ідэі і парады}

\paragraph{Хуткі сьняданак.}
У ідэале плянуйце такі сьняданак, каб прыгатаваць яго за 10--15 хвілінаў. Складаныя стравы, прадукты з~працяглым часам гатаваньня не пасуюць. Актыўна выкарыстоўвайце спэцыі і травы, каб надаць яркасьці звыклым прадуктам.

\paragraph{Прачынайцеся лёгка.}
Лёгкае і прыемнае ранішняе абуджэньне~--- гэта важная прыкмета моцнага здароўя. Усталюйце на будзільнік прыемную мэлёдыю з~паступовым павелічэньнем гучнасьці. Увечары візуалізуйце час абуджэньня, няхай лічбы застануцца ў~вас у~памяці. Прайграйце некалькі разоў у~галаве працэс абуджэньня, як вы лёгка прачынаецеся за пяць хвілінаў да сыгналу будзільніка і адчуваеце сябе выдатна. Сфакусуйцеся на прадчуваньні ад прыемнага абуджэньня.

\paragraph{Няма апэтыту.}
Пры адсутнасьці апэтыту раніцай рабіце лёгкую вячэру ці зусім прапускайце яе. Часам пры парушэньні сыстэмы «голад~--- сытасьць» раніцай апэтыту няма, але па меры нармалізацыі вашага стану апэтыт будзе прыходзіць.

\paragraph{Прачынайцеся заўсёды ў~адзін час.}
І ў~працоўны, і ў~выходны дзень прачынайцеся ў~аднолькавы час. Ваша цела абвыкне абуджацца ў~правільнай фазе сну, і вам будзе лягчэй уставаць. Можна паэкспэрымэнтаваць, ссоўваючы час абуджэньня на 10--20 хвілінаў у~абодва бакі. А~што рабіць, калі трэба выспацца? Усё проста: кладзіцеся раней спаць на наступны дзень. Гэтае проста правіла: прачынаецеся ў~адзін час, засынаеце, калі захацелася.

\paragraph{Выкарыстоўвайце сьветлавы будзільнік.}
Сьветлавы будзільнік плыўна павышае яркасьць сьвятла, і вы прачынаецеся лёгка і хутка. Гэта добры і карысны спосаб, не такі траўматычны, як гучны будзільнік.

\paragraph{Выйдзіце на сонца.}
Кароткі шпацыр або прабежка на вуліцы пад сонцам выдатна наладзяць вас і вернуць апэтыт. Цудоўна, калі ў~вас ёсьць сабака! Памятайце, што сонечнае сьвятло~--- гэта дзівосная крыніца здароўя і даўгалецьця, выкарыстоўвайце любую магчымасьць пабываць пад сонцам, асабліва раніцай!

\tipbox{Сьветлавы будзільнік плыўна павышае яркасьць сьвятла, і вы прачынаецеся лёгка і хутка. Гэта добры і карысны спосаб, не такі траўматычны, як гучны будзільнік.}

\paragraph{Выкарыстоўвайце тэмпэратурны будзільнік.}
Тэмпэратура таксама зьяўляецца сыгналам для нашых унутраных гадзінаў. Уначы тэмпэратура падае, і сасуды скуры звужаюцца, у~мозг трапляе больш крыві, і мы лягчэй прачынаемся. Зрабіце меншым ацяпленьне на ноч~--- і ранкам прачняцеся ў~прыемнай і бадзёрай прахалодзе. Падвышэньне тэмпэратуры з~дапамогай водных працэдур і зарадкі дапаможа вам зарадзіцца энэргіяй і лягчэй прачнуцца. Разнасьцежце вокны, выйдзіце на балькон, абліцеся вадой.

\paragraph{Нагуляць апэтыт.}
Зрабіце інтэнсіўную зарадку, але не даўжэй за 20 хвілінаў. Халодны душ са старанным расьціраньнем цела, заняткі спортам на сьвежым паветры добра падвышаюць апэтыт.

\paragraph{Шклянка вады.}
Выпіце шклянку вады пасьля абуджэньня. За ноч вы страцілі крыху вадкасьці, зараз самы час аднавіць водны балянс.

\paragraph{Пазьбягайце канцэнтраваных вугляводаў.}
Дапускаюцца нізкавугляводныя гародніна і зеляніна: капуста, салера, шпінат, цыбуля і г.~д. У~звычайным варыянце можна разгледзець сярэднекрухмалістую гародніну. Непажадана рабіць сьняданак з~адной толькі кашы, няхай і карыснай, ня кажучы ўжо пра макарону.

\paragraph{Прадоўжыце сытасьць да абеду.}
Для прадаўжэньня сытасьці выкарыстоўвайце тлушчы. У~ідэале ваш сьняданак павінен быць такім, каб думкі аб ежы не прыходзілі вам у~галаву да абеду. Для прадаўжэньня сытасьці дадавайце тлушчы: сьметанковае масла, какосавы, аліўкавы алей, ялавічны тлушч.

\paragraph{Дэсэрт на сьняданак.}
На дэсэрт разгледзьце авакада, гарэхі, кавалачкі какосу, ягады, садавіну. Гарэхі выдатна насычаюць і добра пасуюць да гарбаты і кавы.

\paragraph{Аўсянка, сэр.}
Насуперак мітам, ангельскі сьняданак~--- гэта не аўсянка, а~яйкі, бекон і фасолю, т.~е. высокабялковыя прадукты. Многія вытворцы шматкоў актыўна іх прасоўваюць, але кашы на сьняданак~--- гэта не самае лепшае рашэньне, так як іх не хапае для кантролю апэтыту і энэргіі да самага абеду.
\chapter{Вячэра}

Вечар~--- гэта важная частка нашых цыркадных рытмаў. Добрая вячэра і якасны сон забясьпечваюць паўнавартаснае аднаўленьне арганізму ўначы. Вячэрнія пераяданьні, начныя перакусы, празьмерны стрэс, лішак сьвятла і стымуляцыі парушаюць вячэрні спакой, структуру сну і яго працягласьць. Чалавек часам сьпіць столькі ж, але аднаўляецца горай. Чым горшы сон, тым горшая работа мэтабалізму, сыстэмы «голад~--- сытасьць», слабейшы самакантроль. Вячэрайце лёгка, рана і правільна~--- гэта падмурак паўнавартаснага сну і здароўя!

На прыкладзе сваіх пацыентаў я бачу, што вечар~--- гэта ўразьлівае месца для многіх людзей. Чаму? Яны ня ўлічваюць стрэс і парушэньні сьветлавога рэжыму. Прывяду прыклад. Ёсьць такая глыбакаводная і вельмі страшная на выгляд рыба~--- марскі д'ябал. У~яе ёсьць нарост на галаве~--- «вуда»,~--- які сьвеціцца ў~цёмнай глыбокай вадзе. Прывабленыя сьвятлом рыбы падплываюць і трапляюць проста ў~пашчу марскога д'ябла. Так і ў~нас залішняе сьвятло і стымуляцыя ўвечары зьбіваюць цыркадныя рытмы, прымушаюць думаць, што цяпер дзень, прыліпаць да экранаў ды шмат і сытна есьці ўвечары. Не трапляйце ў~сьветлавую пастку!

\subsection{Як зьявілася праблема?}

Розныя краіны маюць розныя традыцыі. У~адных краінах вячэра раньняя і лёгкая, у~іншых~--- позьняя і досыць сытная, што зьвязана з~працай удзень і гарачым дзённым кліматам. Але ў~цэлым усюды мы бачым стаўленьне да вячэры як да лягчэйшага прыёму ежы («вячэру аддай ворагу», «вячэрай як жабрак» і да т.~п.). А~вось сучасныя людзі пры адсутнасьці культуры харчаваньня сутыкаюцца зь вечаровай спакусай ежай і часта паддаюцца ёй.

\tipbox{Велізарная колькасьць сьвятлодыёдных выпраменьваньняў з~высокаю доляй сьвятла сіняга спэктру (тэлефоны, кампутары, пляншэты, тэлевізары, сьвятлодыёдныя лямпы і інш.) зьбіваюць нашы ўнутраныя гадзіньнікі. Сьвятло такога спэктру дае сыгнал аб тым, што цяпер дзень, і абвастрае пачуцьцё голаду.}

\paragraph{Цыркадная гіпэрфагія (пераяданьне).}
Пытаньне вячэрняга харчаваньня і вячэры шчыльна зьвязана з~работай нашых цыркадных біярытмаў. Велізарная колькасьць сьвятлодыёдных выпраменьваньняў з~высокаю доляй сьвятла сіняга спэктру (тэлефоны, кампутары, пляншэты, тэлевізары, сьвятлодыёдныя лямпы і інш.) зьбіваюць нашы ўнутраныя гадзіньнікі. Сьвятло такога спэктру дае сыгнал аб тым, што цяпер дзень, і абвастрае пачуцьцё голаду.

\paragraph{Гіперфагія і эпідэмія стрэсу.}
Хранічны стрэс становіцца распаўсюджанай зьявай. Калі востры стрэс часьцей прытупляе апэтыт, то хранічны~--- наадварот, узмацняе з~прычыны падвышанай выпрацоўкі грэліну. Стомленасьць і голад на глебе стрэсу павялічваюцца ўвечары, што прыводзіць да пераяданьня. Больш за тое, грэлін узбуджае нэрвовую сыстэму, таму людзі вымушаныя есьці для таго, каб заснуць, выкарыстоўваючы ежу як снатворнае. Злоўжываньне кафеінам і алькаголем таксама пагаршае разлады прыёму ежы.

\paragraph{Геданічная гіпэрфагія} (імкненьне есьці дзеля атрыманьня задавальненьня пры адсутнасьці дэфіцыту энэргіі).
Акрамя стрэсу, існуе і сындром дэфіцыту задаволенасьці, калі назіраецца зьніжэньне здольнасьці атрымліваць задавальненьне ад жыцьця. Такія людзі ядуць, бо ім сумна, тужліва, самотна, мала радасьцяў, і выбіраюць пры гэтым максымальна калярыйныя прадукты, нездаровыя спалучэньні мучнога, тлустага, салодкага, смажанага, салёнага і інш.

\tipbox{У людзей, якія пакутуюць на начныя перакусы, парушаная работа біялягічных рытмаў, фізыялягічная работа шматлікіх гармонаў, нармальны абмен рэчываў. Чым больш і чым часьцей чалавек есьць позна, тым больш выяўленымі могуць быць яго сымптомы і парушэньні сну.}

\paragraph{Працоўная нагрузка.}
Інтэнсіўная працоўная нагрузка, адсутнасьць паўнавартаснага сьняданку і часу на абед прыводзяць да зьяўленьня фізыялягічнага інтэнсіўнага голаду ўвечары і пераяданьня. Працяглыя прамежкі бязь ежы раніцай і днём прыводзяць да падвышанай выпрацоўкі картызолу, што дадаткова ўзмацняе стрэс.

У людзей, якія пакутуюць на начныя перакусы, парушаная работа біялягічных рытмаў, фізыялягічная работа шматлікіх гармонаў, нармальны абмен рэчываў. Чым больш і чым часьцей чалавек есьць позна, тым больш выяўленымі могуць быць яго сымптомы і парушэньні сну. Позьнія вячэры і начныя перакусы прыводзяць да атлусьценьня, падвышанай рызыкі цукровага дыябэту 2-га тыпу, а~таксама псыхалягічных праблемаў. Дасьледаваньні пацьвердзілі, што падвышаны апэтыт у~начны час зьвязаны з~прыгнечанасьцю, схаванай дэпрэсіяй, эмацыйнай напругай, трывожнасьцю.

\subsection{Як гэта ўплывае на здароўе?}

Шэраг нэгатыўных узьдзеяньняў позьніх прыёмаў ежы разабраны ў~папярэднім разьдзеле. Мы адзначым, што існыя дасьледаваньні даюць далёка не заўсёды адназначны адказ. Па сутнасьці, калі лічыць колькасьць спажываных калёрыяў (што далёка не заўсёды рэальна ў~доўгатэрміновай пэрспэктыве), то ўплыў вячэры на структуру цела не такі вялікі. Так, ва ўмовах лёгкага дэфіцыту калёрыяў, харчаваньне раніцай і ўвечары на працягу 12 тыдняў не паўплывала на ўзровень тлушчу. Але большая колькасьць дасьледаваньняў знаходзіць сувязь паміж позьняй сытнай вячэрай і атлусьценьнем.

Рэжым харчаваньня зьвязаны з~запаленьнем у~тлушчавай тканцы. Прыём ежы ўначы актывуе імунныя клеткі макрафагаў і выклікае большы запаленчы адказ у~тлушчавай тканцы ў~параўнаньні зь ядой удзень. Яда на ноч больш чым у~7 разоў павялічвае імавернасьць пякоткі ў~параўнаньні зь ядой за 3 гадзіны да сну. Існуюць дасьледаваньні, якія паказваюць, што позьнія прыёмы ежы зьніжаюць навучальнасьць і прыводзяць да парушэньняў работы гіпакампу.

Ёсьць і іншыя праблемы. Так, рызыка разьвіцьця ішэмічнай хваробы сэрца на 55\,\% вышэйшая ў~тых, хто есьць позна ноччу. Цікава, што рызыка разьвіцьця гарманальных парушэньняў, у~тым ліку раку грудзей і раку падкарэньнічнай залозы, таксама зьвязаная з~позьнімі прыёмамі ежы. Рызыка разьвіцьця раку малочнай залозы зьніжаецца на 16\,\%, а~раку падкарэньніцы~--- на 26\,\% у~тых, хто вячэрае мінімум за дзьве гадзіны да сну.

\paragraph{Сындром начной яды.}
Скрайнім парушэньнем харчовых біярытмаў зьяўляецца сындром начной яды. Яго асноўныя сымптомы~--- гэта частыя (больш за 2 разы на тыдзень) эпізоды прыёму ежы позна ўвечары ці ноччу, за якія паглынаецца больш за 25\,\% сутачнай дозы калёрыяў, пэрыядычная (больш за 4 разы на тыдзень) адсутнасьць апэтыту раніцай, пачуцьцё віны і сораму праз харчовыя паводзіны, зьніжэньне якасьці жыцьця. Падобны разлад патрабуе кансультацыі псыхатэрапэўта і работы над нармалізацыяй біярытмаў, перамагчы яго адной сілай волі цяжка. Сындром начнога апэтыту небясьпечны і для людзей са звычайнай вагой, бо пагаршае якасьць сну. Чым большую долю дзённых калёрыяў вы зьядаеце ўвечары і ўначы, тым вышэйшая рызыка атлусьценьня для вас у~будучыні.

\subsection{Асноўныя прынцыпы}

\paragraph{Вячэрайце рана, умерана і карысна.}
Памятайце, што ня толькі сьняданак уплывае на сон, але й~вячэра закладае падмурак для сьняданку. Лёгкая раньняя вячэра~--- гэта гарантыя лёгкага абуджэньня і наяўнасьці апэтыту раніцай. Напрыклад, навукоўцы высьветлілі, што прадукты зь нізкім глікемічным індэксам на вячэру стабілізуюць цукар у~крыві на ўвесь наступны дзень!

Неабходна вячэраць як мінімум за 3--4 гадзіны да сну. Калярыйнасьць вячэры павінна складаць ад 20 да 30\,\% ад содневай калярыйнасьці. Аптымальныя прадукты~--- сярэдне-і нізкакрухмалістыя гародніна і зеляніна, карысныя тлушчы (сьметанковае масла, аліўкавы, какосавы алеі). Порцыя можа быць вялікай, і не забывайце ўжываць і сырую гародніну. Абмяжуйце выкарыстаньне крупаў, мучных вырабаў, паўфабрыкатаў і вялікай колькасьці бялковых прадуктаў.

\tipbox{Аддайце перавагу таму, што ўтрымлівае больш клятчаткі: гародніна і зеляніна. Рафінаваныя вугляводы павялічваюць рызыку бессані, а~вось клятчатка на вячэру можа падоўжыць фазу павольнага сну!}

\paragraph{Раньняя вячэра.}
Раньняя вячэра дапамагае павялічыць харчовую паўзу, а~акрамя таго, неўзаметку скарачае колькасьць калёрыяў. Навукоўцы давялі, што поўнае абмежаваньне яды з~7-й вечара да 6-й раніцы ў~здаровых людзей без абмежаваьньня каляражу прыводзіць да таго, што яны ў~сярэднім пачынаюць есьці менш на 240\,ккал у~содні! А~гэта ў~доўгатэрміновай пэрспэктыве дае дастаткова прыкметнае паніжэньне вагі! Раньняя вячэра спрыяе выпрацоўцы гармону росту, больш за 70\,\% колькасьці якога выдзяляецца ноччу. Гармон росту стымулюе рост цяглічнай тканкі і спальваньне тлушчу. Яда на ноч, у~сваю чаргу, спрыяе выдзяленьню інсуліну, які зьмяншае ўзровень гармону росту.

\paragraph{Вугляводы.}
Нягледзячы на тое, што вугляводы могуць даць прыемную дрымоту, ня варта на вячэру есьці крухмалістыя вугляводы, варта абмежаваць мучное і салодкае. Аддайце перавагу таму, што зьмяшчае больш клятчаткі,~--- гародніне і зеляніне. Рафінаваныя вугляводы павялічваюць рызыку бессані, а~вось клятчатка на вячэру можа падоўжыць фазу павольнага сну! Таму складаная салата~--- гэта выдатнае рашэньне! Навуковыя дасьледаваньні паказваюць, што цукар на вячэру выклікае павелічэньне эпізодаў мікраабуджэньня цягам ночы, што заўважна зьніжае якасьць сну.

\paragraph{Тлушчы.}
Тлушчы таксама зьяўляюцца важным кампанэнтам вячэры, але толькі ў~спалучэньні з~клятчаткай. Тлушчы не выклікаюць уздыму інсуліну, стымулююць выдзяленьне гармону халецістакініну, які дапамагае адчуваць сытасьць. Какосавы, аліўкавы алей, сьметанковае масла і да т.~п. будуць найлепшыя. Нават кавалачак сала пойдзе на карысьць. Таму сьмела запраўляйце гародніну тлушчамі, але захоўвайце пачуцьцё меры.

\tipbox{Багацьце бялковай ежы не пасавацьме трывожным людзям з~праблемным засынаньнем. Але для тых, хто трэніруецца і моцна сьпіць, працуе дапазна, бялок на вячэру таксама можа быць прыдатным варыянтам.}

\paragraph{Бялкі.}
Пытаньне бялковых прадуктаў на вячэру не зусім адназначнае і павінна вырашацца індывідуальна. З~аднаго боку, бялкі ўзмацняюць актыўнасьць і стымулююць нэрвовую сыстэму, што ня вельмі добра для сну. Багацьце бялковай ежы не пасавацьме трывожным людзям з~праблемным засынаньнем. Зь іншага боку, бялкі выдатна спаталяюць голад і доўга падтрымліваюць апэтыт. Таму для тых, хто трэніруецца і моцна сьпіць, працуе дапазна, бялок на вячэру таксама можа быць прыдатным варыянтам.

\subsection{Як трымацца правіла? Ідэі і парады}

\paragraph{Вячэрняе асьвятленьне.}
Для правільнай работы цыркадных біярытмаў сьвятло ўдоме павінна па спэктры супадаць са сьвятлом на захадзе сонца, быць жоўтым, расьсеяным, няяркім. Для гэтага ўвечары трэба ўключаць не сьвятлодыёдныя лямпы, а~лямпы напальваньня, якія дадуць патрэбны спэктар. Але большую частку сьвятлодыёднага сьвятла даюць смартфоны, ноўтбукі, тэлевізары~--- для гэтага таксама ёсьць варыянты. Можна ўсталяваць праграмы накшталт f.lux, twilight\footnote{f.lux, twilight~--- гэта праграмы, што рэгулююць колеравую тэмпэратуру кампутарнага манітора ў~адпаведнасьці з~геаграфічным месцівам і часам сутак карыстальніка.}, скарыстацца функцыяй night shift. Але найлепшым варыянтам будзе купіць спэцыяльныя акуляры, якія блякуюць сінюю частку спэктру.

\paragraph{Тэмпэратура ўвечары.}
Тэмпэратура таксама зьяўляецца важным сыгналам для нашага цела. Тэмпэратура нашага асяродку павінна зьніжацца прыкладна зь 19-й. Усё, што замінае нам «астыць», будзе замінаць расслабіцца і заснуць. Таму такія парады, як пагуляць перад сном, праветрыць спальню, паменшыць інтэнсіўнасьць апалу,~--- выдатна працуюць.

\paragraph{Стрэс увечары.}
Вячэрні стрэс~--- частая прычына страты кантролю над харчовымі паводзінамі і пераяданьня. Пры стрэсе нас цягне на тлустае, салодкае і салёнае. Гэта частка ахоўнай рэакцыі, бо наш арганізм успрымае стрэс як фізычную актыўнасьць («нападаю~--- уцякаю»), таму стымулюе нас папоўніць каляраж. Антыстрэсавыя тэхнікі выдатна памяншаюць апэтыт і спрыяюць глыбейшаму сну: цягліцавае паслабленьне, вядзеньне дзёньніка, глыбокае раўнамернае дыханьне, масажны кілімок з~іголкамі, фрырайтынг\footnote{Фрырайтынг (ці вольнае пісьмо)~--- тэхніка і методыка пісьма, якая дапамагае знайсьці неардынарныя рашэньні і ідэі, падобная з~мэтадам мазгавога штурму.}, ёга дапамагаюць палепшыць самаадчуваньне і зьнізіць узровень стрэсу, а~адпаведна і апэтыту. Усе працэдуры, накіраваныя на цела, павялічваюць выпрацоўку антыстрэсавага гармону аксытацыну: водныя працэдуры, масаж, расьціраньні, саўна, самамасаж і да т.~п.

\paragraph{Кафеін.}
Кафеін утрымліваецца ня толькі ў~каве, але і ў~гарбаце, шматлікіх напоях, батончыках і леках. Яго сярэдні пэрыяд паўраспаду складае 4--5 гадзінаў, але дзеяньне можа адчувацца і ўсе восем гадзінаў. Таму калі вы хочаце пазьбегнуць узьдзеяньня кафеіну на сон, то ня варта ўжываць каву, гарбату ды іншыя кафеінавыя напоі пасьля 12:00–14:00 гадзінаў. Навукоўцы высьветлілі, што кафеін ссоўвае біярытмы на больш позьні час, скарачае час павольнага сну. Залішняе ўжываньне кафеіну прыводзіць да парушэньня сну і зьяўленьня ранішняй стомленасьці. Ёсьць людзі, генэтычна вельмі адчувальныя да кафеіну,~--- у~іх ён выклікае трывогу і ўзмацняе нэўроз. Такім людзям абмежаваньне кафеіну вельмі станоўча адаб'ецца на здароўе.

\tipbox{Сярэдні пэрыяд кафеіну паўраспаду складае 4--5 гадзінаў, але дзеяньне можа адчувацца і ўсе восем гадзінаў. Таму калі вы хочаце пазьбегнуць узьдзеяньня кафеіну на сон, то ня варта ўжываць каву, гарбату ды іншыя кафеінавыя напоі пасьля 12:00–14:00 гадзінаў.}

\paragraph{Алькаголь.}
Алькаголь~--- вельмі калярыйны прадукт. Ён хоць і паскарае засынаньне, але зьмяншае якасьць сну, асабліва ў~другой палове ночы, узмацняе храп, выклікае абязводжваньне, пры працяглым ужываньні можа выклікаць бессань. Кладзіцеся спаць цьвярозымі.

\paragraph{Ніякіх паўфабрыкатаў і гатовай ежы.}
Яны ўтрымліваюць вялікую колькасьць цукру, солі і ўзмацняльнікаў смаку. Салёныя прадукты на вячэру прывядуць да затрымкі вадкасьці і да таго, што вы будзеце выглядаць раніцай горш.

\paragraph{Менш тыраміну.}
На вячэру ня варта есьці прадукты, якія ўтрымліваюць шмат біягенных амінаў, напрыклад тыраміну. Яны ўзмацняюць выдзяленьне адрэналіну і пагаршаюць сон. Тырамін ёсьць у~чакалядзе, віне, соўсах, сырах і шэрагу бялковых і малочных паўфабрыкатаў.

\paragraph{Задавальненьне ўвечары.}
Плянуйце прыемнае на вечар загадзя і рабіце гэта, нават калі ня хочацца. Такі прынцып ляжыць у~аснове мэтаду паводзіннай актывацыі, вельмі эфэктыўнай пры дэпрэсіі. Тут пасуе хобі, камунікацыя з~роднымі і сябрамі, любое некалярыйнае задавальненьне. Калі мы стамляемся, то можам адмаўляцца ад зносінаў, значных для нас справаў. Але ж калі мозг пазбаўляецца эмацыйнага падсілкоўваньня, ён шукае спосабы папоўніць яго дэфіцыт~--- і знаходзіць у~ежы! Са свайго досьведу я бачу, што вячэрні голад вельмі часта мае эмацыйныя карані.

\paragraph{Вымушаная вячэра.}
Часта людзі вымушаныя вячэраць позна, калі няма магчымасьці есьці паўнавартасна і спакойна на працы. Гэта не ідэальны рэжым, але ня варта палохаць сябе жахалкамі, што ежа будзе «гніць уначы ў~кішачніку» і да т.~п. Паступова адбываецца адаптацыя арганізму~--- і ежа засвойваецца нармальна. Жахалкі, што ўначы страўнік і кішачнік не працуюць, няслушныя: іх функцыя крыху зьніжаная, але не крытычна. А~ў выпадку адаптацыі яны могуць прыстасавацца да вашага рытму жыцьця.
\chapter{Колькасьць прыёмаў ежы}

Згодна з~дасьледаваньнямі, тры асноўныя прыёмы ежы ў~дзень зьяўляюцца аптымальнымі для доўгатэрміновага падтрыманьня здаровай вагі і самаадчуваньня. Такое правіла існуе ў~шматлікіх культурах і традыцыях, суседнічаючы з~двума прыёмамі ежы.

Колькі разоў на дзень есьці? Модная цяпер парада~--- павялічыць колькасьць прыёмаў ежы, але ў~доўгатэрміновай пэрспэктыве гэта працуе супраць нас. Чалавек у~сучасным сьвеце жыве пад узьдзеяньнем вялікай колькасьці стрэсаў і задачаў, кожная зь якіх канкуруе за яго ўвагу і час. Парада лічыць калёрыі і вылучаць 5--6 паўнавартасных рэгулярных прыёмаў ежы, як таго патрабуе падыход дробавага харчаваньня, патрабуе вялікіх затрат часу, ня мае перавагаў перад 2--3-разовым харчаваньнем, але пры гэтым падтрымліваецца дзякуючы вялізнай колькасьці мітаў, якія мы з~вамі абмяркуем.

\tipbox{Модная цяпер парада~--- павялічыць колькасьць прыёмаў ежы, але ў~доўгатэрміновай пэрспэктыве гэта працуе супраць нас. Неабходнасьць лічыць калёрыі і рабіць 5--6 паўнавартасных прыёмаў ежы, як таго патрабуе падыход дробавага харчаваньня,~--- гэта як мінімум вялікія часавыя выдаткі.}

Зьежце дастатковую колькасьць ежы, каб быць сытымі і ня думаць пра ежу, а~затым спакойна займайцеся сваімі справамі. У~рэшце рэшт, вы ж і бак машыны запраўляе адразу, а~не па 5 літраў за заезд на запраўку? Гэта эканоміць і час, і здароўе!

\subsection{Як зьявілася праблема?}

Першапачаткова дробавае харчаваньне як лячэбная дыета распрацоўвалася для людзей з~захворваньнямі страўнікава-кішачнага тракту (язвавая хвароба, гастраэзафагіяльны рэфлюкс, халецыстэктамія і інш.), а~таксама для аслабленых пацыентаў пасьля апэрацый. І~гэта цалкам апраўдана.

У наш час дробавае харчаваньне ўзьнікла першапачаткова на памылковай здагадцы, што павелічэньне частаты харчаваньня здольнае «разагнаць» мэтабалізм. Навуковыя дасьледаваньні паказваюць, што ніякай разгонкі не адбываецца, а~тэрмічны эфэкт ежы прапарцыйны содневай колькасьці калёрыяў, а~ня колькасьці прыёмаў ежы. Вядома, важкім укладам у~фармаваньне міта пра дробавае харчаваньне зьяўляецца міт «утаймаваньня голаду», які, па сутнасьці, зводзіцца да таго, каб «перабіць» апэтыт. У~ісьце гэта нездаровы падыход, які толькі патурае нашаму жаданьню есьці.

Калі мы зірнём на вынікі навуковых дасьледаваньняў, то ў~кароткатэрміновым пэрыядзе ў~некалькі месяцаў розьніцы паміж трыма і шасьцю прыёмамі ежы пры аднолькавай колькасьці калёрыяў няма, то-бок частае харчаваньне не палепшыць стан здароўя (вага, сытасьць, аналізы). Але калі мы возьмем больш працяглы пэрыяд, у~некалькі гадоў, дык існуе выразная заканамернасьць паміж колькасьцю прыёмаў ежы і здароўем. Тыя, хто еў 2 разы на дзень, мелі тэндэнцыю зьніжэньня вагі, тыя, хто еў 3 разы, захоўвалі вагу, а~тыя, хто еў часьцей за 3 разы на дзень, мелі няўхільную тэндэнцыю павелічэньня вагі з~узростам.

Чаму так адбываецца? Як звычайна, уся справа ў~калёрыях і звычках. Нам цяжка сьвядома кантраляваць дакладнае спажываньне калёрыяў гадамі. Стрэсы, перагрузка, стомленасьць прыводзяць да зьніжэньня сьвядомага кантролю, і наша харчаваньне пераходзіць пад кантроль звычак. Сілкуючыся 5--6 разоў на дзень, мы павялічваем імавернасьць пераяданьня ў~параўнаньні з~харчаваньнем 2--3 разы на дзень. Вядома, важна разумець, што на кароткатэрміновым этапе любая, нават абсурдная, дыета можа прывесьці да пэўных вынікаў выключна за кошт большай увагі да харчаваньня. Памятайце пра тое, што для чалавека нашмат фізыялягічней абмяжоўваць доступ да ежы, а~не скарачаць колькасьць калёрыяў.

Аднак заўсёды ўзьнікае пытаньне: чаму столькі дыетолягаў гавораць аб дробавым харчаваньні і столькі людзей кажуць, што схуднелі на ім? Адказ просты: парада есьці часьцей заўсёды гучыць і ўспрымаецца больш пазытыўна, чым парада есьці радзей. Акрамя гэтага, частыя прыёмы ежы зьніжаюць узровень грэліну\index{грэлін}, чым прыносяць часовае палягчэньне. Аднак гэта ня вельмі карысна ў~доўгатэрміновай пэрспэктыве (гл. правіла «Ежце, калі галодныя»).

\subsection{Як гэта ўплывае на здароўе?}

Частыя прыёмы ежы не працуюць у~доўгатэрміновай пэрспэктыве. Штодзённы кантроль за гатоўляй пяці-шасьці прыёмаў ежы з~падлікам калёрыяў вельмі энэргаёмісты. Чалавек і так прымае 300--400 харчовых рашэньняў за содні, таму дастаткова будзе любога стрэсу або стомленасьці, каб гэта перастала працаваць. Схуднець хутка лёгка на любой дыеце, а~вось утрымаць вагу на 5--10 гадоў ужо складаней. Праз час кантроль за падлікам калёрыяў слабне, а~звычка есьці часта~--- застаецца, што і прыводзіць да набору вагі.

Дробавае харчаваньне парушае нармальную працу сыстэмы «голад~--- сытасьць». Нягледзячы на тое, што яно зьніжае голад, разам з~тым пакутуе і пачуцьцё сытасьці. Трохразовае харчаваньне дазваляе падтрымліваць больш стабільнае пачуцьцё насычэньня.

\paragraph{Апантанасьць ежай.}
Частыя думкі аб ежы, фіксацыя на ёй, падлік калёрыяў могуць лёгка прывесьці да парушэньня харчовых паводзінаў. Здаровае стаўленьне да ежы~--- калі вы думаеце пра яе, але яна не зьяўляецца галоўным чыньнікам, які структуруе вашае жыцьцё і кіруе ім. Ніколі не забывайце, што ежа слугуе вам, а~ня вы слугуеце ежы.

\paragraph{Вы не худнееце.}
Навуковыя дасьледаваньні паказваюць, што павелічэньне частаты харчаваньня нават да 9--10 разоў у~дзень пры агульнай аднолькавай колькасьці калёрыяў ніяк не дапамагае худнець. Калі вы схільныя пераядаць, то пры 5--6-разовым харчаваньні будзеце зьядаць прыкметна больш ежы, чым пры 3-разовым.

\tipbox{Частыя думкі аб ежы, фіксацыя на ёй, падлік калёрыяў могуць лёгка прывесьці да парушэньня харчовых паводзінаў. Здаровае стаўленьне да ежы~--- калі вы думаеце пра яе, але яна не зьяўляецца галоўным чыньнікам, які кіруе вашым жыцьцём.}

\paragraph{Адсутнасьць гнуткасьці.}
Прывучаючы сябе да строгага прыёму ежы празь невялікія прамежкі, мы пакутуем, калі пачынаем прапускаць гэтыя прыёмы праз зрухі працоўнага графіку.

\paragraph{Парушэньне працы шэрагу гармонаў.}
Працяглае дробавае харчаваньне можа прывесьці да ўзмацненьня інсулінарэзыстэнтнасьці, зьніжэньня ўзроўню грэліну\index{грэлін}, самататропнага гармону. Карысныя ўласьцівасьці гармону грэліну\index{грэлін} ўключаюць абарону сэрца і нырак, павышэньне нэўрагенэзу\index{нэўрагенэз}, антыдэпрэсіўнае дзеяньне і іншыя.

\subsection{Асноўныя прынцыпы}

Для дзяцей будуць аптымальнымі 4 прыёмы ежы, для жанчынаў~--- 3--4, для мужчынаў~--- 2--3. Павелічэньне частаты прыёмаў ежы больш за тры для здаровых людзей не нясе перавагаў для здароўя і не спрыяе пахуданьню. Тым, хто есьць часта ці хаатычна, парадкаваньне прыёмаў ежы дазваляе стварыць просты і зразумелы рэжым харчаваньяня, які будзе іх падтрымліваць і структураваць харчовы рэжым. Такім чынам, давайце разгледзім наступныя рэжымы харчаваньня.

\paragraph{Больш за 4 прыёмы ежы.}
Пасуе дзецям, а~таксама ў~якасьці лекавай дыеты пры некаторых хваробах страўнікава-кішачнага тракту. Падыдзе прафэсійным атлетам, якія маюць патрэбу ў~вялікай колькасьці калёрыяў праз інтэнсіўныя трэніроўкі. Зьвярніце ўвагу, што тыя, хто па-аматарску займаецца спортам па гадзіне 3--4 разы на тыдзень, ня маюць патрэбы ў~павелічэньні частаты харчаваньня.

\tipbox{Павелічэньне частаты харчаваньня нават да 9--10 разоў на дзень пры агульнай аднолькавай колькасьці калёрыяў аніяк не дапамагае худнець. Калі вы схільныя пераядаць, то пры 5--6-разовым харчаваньні будзеце зьядаць прыкметна больш ежы, чым пры 3-разовым.}

\paragraph{4 прыёмы ежы.}
Пасуе дзецям пасьля спыненьня груднога кармленьня і больш старэйшага ўзросту, атлетам, тым, хто аднаўляюцца пасьля хваробы.

\paragraph{3 прыёмы ежы.}
Стандартная частата прыёму ежы (сьняданак, абед і вячэра). У~цэлым падыдзе большасьці людзей.

\paragraph{2 прыёмы ежы.}
Сьняданак і абед, для людзей зь вячэрнім тыпам харчаваньня~--- абед і вячэра. Два прыёмы ежы на дзень уласьцівыя шматлікім традыцыйным культурам ад Усходу да Захаду. Так, старажытныя грэкі елі 2 разы~--- ланч (ariston) і вячэру (deipnon). Стратэгія есьці 2 разы на дзень можа падысьці для пахудзеньня і падтрыманьня вагі. Большасьць мужчын пераносяць яе лёгка, а~вось для некаторых жанчын харчавацца тройчы на дзень можа быць больш прыярытэтнай стратэгіяй, улічваючы іх гарманальныя асаблівасьці.

\paragraph{1 прыём ежы на дзень.}
Па сутнасьці, такі падыход уяўляе сабой радыкальнае памяншэньне харчовага вакна да 1 гадзіны на содні (23/1). Дасьледаваньні, праведзеныя на жывёлах, паказалі, што такі мэтад запавольвае працэс старэньня ў~мышэй. Традыцыйна такая дыета была распаўсюджаная сярод, напрыклад, рымскіх легіянэраў («дыета ваяра») або будыйскіх манахаў («Тры разы на дзень ядуць жывёлы, два разы чалавек і адзін раз~--- тыя, хто ідзе шляхам Ісьціны», «Сутра 42 кіраўнікоў, сказаная Будам»). Мы можам разглядаць такое харчаваньне як варыянт разгрузачнага дня на 24 гадзіны і карыстацца ім 1--2 разы на тыдзень. Сучаснымі прыхільнікамі такога харчаваньня зьяўляюцца некаторыя біяхакеры, аднак для шырокай аўдыторыі гэты падыход нельга рэкамэндаваць.

\subsection{Як трымацца правіла? Ідэі і парады}

\paragraph{Плыўнасьць і адаптацыя.}
Часам частата прыёмаў ежы зьяўляецца адно звычкай, але часьцей за ўсё яна абумоўленая асаблівасьцямі працы нашага страўніка. У~нашым страўніку ёсьць мэханарэцэптары, якія рэагуюць на ступень яго расьцяжэньня ежай. Іх адчувальнасьць можа мяняцца. Гэта значыць, што чым большыя аб'ёмы ежы вы зьядаеце, тым лягчэй вам гэта зрабіць без пачуцьця цяжару. Калі пачаць сілкавацца меншымі аб'ёмамі ежы, то адчувальнасьць рэцэптараў павялічваецца, і большы аб'ём ежы выклікае пачуцьцё перапаўненьня страўніка. Дасьледаваньні паказалі, што адаптацыя да зьменаў у~аб'ёме порцыі займае ў~сярэднім каля 4 тыдняў.

Рэальнага перарасьцяжэньня страўніка няма. Мэханізм такі ж, як пры адмове ад солі,~--- спачатку ўсё становіцца прэсным, але праз пару тыдняў адчувальнасьць смакавых рэцэптараў аднаўляецца, і вы зноў паўнавартасна адчуваеце ўсе смакі. Ня трэба баяцца «расьцягнуць» страўнік, вы ж не баіцеся расьцягнуць свой мачавы пухір? Страўнік~--- гэта цягліцавы орган, ён можа павялічваць свой аб'ём у~4--5 разоў і вяртацца ў~норму, гэтая ўласьцівасьць нам і патрэбная, каб мы маглі зьесьці шмат за адзін прыём ежы.

\paragraph{Варыятыўнасьць.}
Зусім неабавязкова прытрымлівацца стро\-га фіксаванай колькасьці прыёмаў ежы. Калі ў~вас быў інтэнсіўны дзень трэніровак, можна зьесьці больш. А~калі гэта дзень зь нізкай фізычнай актыўнасьцю, то можна прапусьціць прыём ежы.

\tipbox{Дрымотнасьць пасьля ежы зьвязаная або з~паабедзенным зьніжэньнем картызолу, або зь вялікай колькасьцю вугляводаў. У~першым выпадку можна абедаць пазьней, а~ў другім~--- дадаць больш бялку ў~іншы прыём ежы.}

\paragraph{Яда не для настрою.}
Часам людзі пераходзяць на больш частае харчаваньне дзеля таго, каб падняць сабе настрой ежай. Але важна разумець, што такі падыход небясьпечны і спробы «ўзбадзёрыцца ежай» могуць прывесьці да парушэньня харчовых паводзінаў і іншым праблемаў са здароўем. Стымуляцыя настрою ежай можа толькі разгайдаць нашыя «цукровыя арэлі», разбалянсаваўшы настрой.

\paragraph{Цяжар пасьля сытнай ежы.}
Цяжар пасьля ежы мае некалькі прычынаў. Першая~--- гэта адчувальнасьць мэханарэцэптараў страўніка, пра якую мы ўжо казалі. Другая~--- гэта высокая хуткасьць яды. Калі сталаваньне вялікае, то зьядаць яго трэба не хутчэй чым за 20 хвілін не сьпяшаючыся. Трэцяя прычына~--- недастатковае перажоўваньне. Ежу трэба старанна перажоўваць, не глытаць вялікія кавалкі.

\paragraph{Дрымотнасьць пасьля ежы.}
Часьцей за ўсё дрымотнасьць пасьля ежы зьвязаная або з~паабедзенным зьніжэньнем картызолу (час сыесты), або зь вялікай колькасьцю вугляводаў. У~першым выпадку можна абедаць пазьней, а~ў другім~--- дадаць больш бялку ў~прыём ежы, што не дае такой рэакцыі.

\paragraph{Беражыце зубы.}
Кожны прыём ежы~--- гэта кіслотная атака на зубы. Старанна палашчыце рот пасьля кожнага прыёму ежы. Зубная эмаль\index{зубная эмаль} разьмякчаецца, калі кіслотнасьць у~роце падае ніжэй за 5{,}5\,pH, патрабуецца да гадзіны часу, каб аднавіць кіслотны балянс. Калі ўзьдзеяньне ежы занадта частае і перадусім утрымлівае камбінацыі кіслата + цукар (сокі ці ахаладжальныя напоі), то натуральная абарона сьліны не працуе. Памятайце, што адразу пасьля прыёму ежы (асабліва садавіны) нельга чысьціць зубы!

\paragraph{Вы ня страціце цяглічную масу.}
Паводле дасьледаваньняў, атлеты выдатна набіраюць масу нават у~вузкім харчовым вакне і з~двума прыёмамі ежы за дзень.

\paragraph{Меней думайце пра ежу і эканомце сілы.}
Меншая колькасьць прыёмаў ежы дае вам куды больш гнуткасьці (бо ня трэба плянаваць 5--6 прыёмаў ежы), менш мыцьця талерак, кантэйнэраў, думак, калі б і дзе паесьці. Чым меней вы думаеце аб ежы, тым лепей для вас.

\paragraph{Ежце больш сур'ёзна.}
Скарачаючы колькасьць прыёмаў ежы, не забывайце аб разнастайнасьці. Сьмела ўключайце і салату, і садавіну, і гарэхі да гарбаты, рабіце вашу ежу разнастайнай і складанай. Памятайце пра тое, што чым менш разоў вы ясьцё, тым больш старанна трэба есьці, каб ня думаць пра ежу да наступнага сталаваньня.

\paragraph{Дзеці і дробавае харчаваньне.}
Дзецям, старэйшым за паўтара году, харчаваньне часьцейшае за 4 разы на дзень не патрэбнае. Часта бацькі спрабуюць зрабіць ім перакусы, падсоўваючы то сушкі, «каб зубы лепш рэзаліся», то сок, «каб не абязводзіўся», то проста цукеркі~--- «парадаваць». Пры гэтым суцяшаюць сябе думкамі, што «дзеці растуць», таму ўсё сыдзе ў~рост. Няма надзейных навуковых доказаў, што перакусы ў~дзяцей паляпшаюць паказчыкі іх здароўя (пры 4-разовым паўнацэнным харчаваньні). Дзеці, якія часьцей  перакусваюць, спажываюць больш калёрыяў, чым за базавыя прыёмы ежы.

\part{Удасканаленьне рэжыму}

\chapter{Ежце, калі галодныя}

Калі я кажу пра карысьць голаду, шмат хто палохаецца. І дарма, бо здаровы голад нам ня вораг, а супольнік. Здаровы фізыялягічны апэтыт – гэта больш, чым проста жаданьне есьці. Здаровае жаданьне есьці – гэта прыкмета здароўя, прыкмета «смаку да жыцьця». Тое, што мы называем пачуцьцём невялікага, лёгкага голаду, – гэта абсалютна нармальны, натуральны стан чалавека. Для правільнага абмену рэчываў, добрага самаадчуваньня і здаровых харчовых паводзінаў важна выконваць простае правіла: есьці, калі вы галодныя, і есьці дасхочу. Нягледзячы на ўяўную простасьць, гэтае правіла мае мноства нюансаў.

Пераборлівасьць у ежы і паўнавартасныя сталаваньні адлюстроўваюць добрую работу сыстэмы «голад – сытасьць», у аснове якой ляжыць балянс двух гармонаў, грэліну і лептыну. Пры гэтым важна разумець, што слова «голад» зьяўляецца вельмі недакладным, бо ёсьць суплётам розных фізыялягічных зьяваў. Мы можам вылучыць як мінімум тры розныя зьявы: апэтыт (3-5 гадзінаў, вычарпаньне глікагену), устрыманьне (пост, фастынг [5], да 70 гадзінаў) і сапраўдны голад (выяўныя гарманальныя зьмены).

Лёгкі голад – гэта натуральны стан здаровага, актыўнага чалавека. Для правільнага абмену рэчываў, добрага самаадчуваньня і здаровых харчовых паводзінаў важна выконваць простае правіла: есьці, калі вы напраўду галодныя, і есьці дасхочу.

Навучыцеся ставіцца да голаду як да сыгналу. Уявіце сабе, што вы едзеце машынай, і раптам у вас загараецца чырвоны індыкатар бэнзабаку. Вы ж ня цісьніце газ у падлогу і не зрываецеся ў бліжэйшую запраўку, кінуўшы ўсе справы? У вас ёсьць запас яшчэ 50-100 кіламетраў, можна дарабіць справы і тады ўжо заехаць на запраўку. Так і голад – гэта сыгнал, што ўзровень глікагену ў печані зьніжаецца: зазвычай хочацца есьці, калі застаецца яшчэ 20% запасу энэргіі. Таму цалкам можна гадзіну-паўтары пачакаць і потым добра пад'есьці. Але калі вы будзеце ігнараваць голад, то ён можа папрасіць і гвалтам, павялічыўшы ўзровень стрэсавага гармону картызолу. Таму прыслухоўвайцеся да сыгналаў!

\section{Як зьявілася праблема?}

\subsection{Сквапныя гены.}
Сучасную эпідэмію атлусьценьня і пераяданьня навукоўцы апісваюць «тэорыяй сквапных генаў». Даўным-даўно, калі ежы было вобмаль, важнай для выжываньня нашых продкаў стратэгіяй было зьесьці ў запас калярыйную ежу, бо захоўваць і насіць яе з сабой было складана. Цяпер актыўнасьць гэтых «сквапных генаў» стымулюе пераяданьне, бо мы выпрабоўваемся сталым сэнсарным бамбаваньнем высокакалярыйнай ежай, што смачна пахне і апэтытна выглядае (фуд-порна), а ва ўмовах стрэсу яна становіцца асабліва прывабнай.

\subsection{Стрэс і ежа.}
Галоўнай праблемай выжываньня для нашых продкаў быў голад, таму ў выпадку хранічнага стрэсу наш апэтыт павялічваецца. Так ужо атрымалася, што грэлін павялічваецца пры стрэсе. Таму псыхалягічны стрэс без належнага кантролю правакуе пераяданьне. Высокакалярыйная ежа выклікае ўздым дафаміну, замацоўваючы падобныя паводзіны і «палягчаючы» стрэс. Атрымліваецца, што мы ежай падмацоўваем сваю стрэсавую яду і так трапляем у заганнае кола яшчэ зь дзяцінства. Пры стрэсе ўзровень грэліну падвышаецца, што штурхае нас на пошук рэсурсаў і рашэньняў нашых праблем. Так, мы становімся больш раздражняльнымі і рэзкімі, але гэта дапамагае нам лепей вырашаць праблемы і нават узмацняе «волю да жыцьця». Але, на жаль, стрэсавае павышэньне грэліну часта прыводзіць да таго, што мы проста ямо, а не вырашаем нашы праблемы. А атлусьценьне – гэта не проста лішнія кіляграмы, але часта і страта «смаку да жыцьця».

\subsection{Разбалянсаваны голад.}
Калі мы выкарыстоўваем ежу не па прызначэньні (здавальненьне фізычнага голаду), а зь іншымі мэтамі (заткнуць дзіця, супакоіцца, падняць настрой, заесьці стрэс), то мы руйнуем натуральную зьвязку «голад – ежа», пачуцьцё голаду пачынае асацыявацца зусім зь іншымі трыгерамі, напрыклад «стрэс (нуда) – ежа». Такі ж эфэкт мае ежа строга па раскладзе, без увагі на апэтыт. Прымушаць есьці сябе ці іншага – гэта няправільна!

Голад – гэта сыгнал, што ўзровень глікагену ў печані зьніжаецца, звычайна хочацца есьці, калі застаецца яшчэ 20% запасу энэргіі. Таму цалкам можна пачакаць гадзіну-паўтары і затым добра падсілкавацца.

\section{Як гэта ўплывае на здароўе?}

\subsection{Голад і смак.}
Як сьцьвярджае народная мудрасьць, «голад – найлепшая закраса». Грэлін павялічвае дафамінавы водгук на звычайныя прадукты, прыкметна ўзмацняючы задавальненьне ад ежы. Таму, калі мы галодныя, звычайная ежа здаецца нам вельмі смачнай і ямо мы яе з задавальненьнем. І тут узьнікае пытаньне і частая асьцярога: калі голад узмацняе смак ежы, ці ёсьць рызыка пераесьці? Людзі баяцца, што, «нагуляўшы апэтыт», яны зьядуць лішку. Так, грэлін павялічвае колькасьць зьедзенага пажытку, але робіць гэта ён павялічваючы частату прыёмаў ежы, а ня разавы аб'ём. Бо грэлін выпрацоўваецца сьценкамі страўніка, таму, калі вы зьядаеце дастатковы аб'ём ежы (уключаючы гародніну і зеляніну) і датрымліваецеся выразнага рэжыму харчаваньня, то падставаў баяцца грэліну ў вас няма.

\subsection{Грэлін і галаўны мозг.}
Сэнс дзеяньня гармону голаду грэліну – не пазбавіць вас сілаў, а, наадварот, прастымуляваць, каб вы змаглі рухацца хутчэй і лепей цяміць. Менавіта таму грэлін мае вялікую колькасьць карысных дзеяньняў. Так, ён стымулюе выдзяленьне самататропнага гармону (гармон голаду), які, у сваю чаргу, стымулюе аўтафагію.

Вельмі дабратворна ўзьдзейнічаюць голад і грэлін на працу галаўнога мозгу. Грэлін стымулюе выпрацоўку нэўратрафічнага чыньніка BDNF, нэўрагенэз, дапамагае абараніць мозг ад старэньня. Голад паляпшае памяць, здольнасьць вучыцца («поўнае бруха да навукі глуха»), падтрымлівае нармальную працу гіпакампу, зьніжае рызыку разьвіцьця нэўрадэгенэратыўных захворваньняў, у тым ліку хваробаў Паркінсана і Альцгеймэра. Важным зьяўляецца і антыдэпрэсіўнае дзеяньне грэліну. Так, голад прыкметна зьніжае рызыку дэпрэсіі і стымулюе «пошук навізны», выдзяленьне дафаміну і захаваньне «смаку да жыцьця». Калі ўвесьці грэлін экспэрымэнтальным жывёлам, то яны актыўней пачынаюць дасьледаваць ды імкнуцца вывучаць новае; у чалавека – усё тое ж самае.

\subsection{Голад і матывацыя.}
Грэлін узмацняе ня толькі імкненьне да ежы, але і ў цэлым «матывацыю». На жывёлах дакладна даведзена: чым больш грэліну, тым больш гатовая жывёла працаваць для атрыманьня ўзнагароды. Больш грэліну – больш матывацыі!

Павелічэньне ўзроўню гармону голаду грэліну паляпшае памяць, падтрымлівае нармальную працу гіпакампу, зьніжае рызыку разьвіцьця нэўрадэгенэратыўных захворваньняў, узьдзейнічае як антыдэпрэсант. Менавіта голад прыкметна зьніжае рызыку дэпрэсіі і стымулюе «пошук навізны», выдзяленьне дафаміну і захаваньне «смаку да жыцьця».

\subsection{Голад і імунітэт.}
Шмат хто ведае, што пры хваробах (траўмы, інфэкцыі, пухліны ды інш.) зьнікае апэтыт, і гэта заканамерна, а вось вяртаньне апэтыту зьяўляецца добрай прагнастычнай прыкметай. Аказалася, што грэлін здольны ўплываць на імунную функцыю, зьніжаючы ўзровень запаленьня і стымулюючы рост лімфоіднай тканкі. Грэлін і самататропны гармон аказваюць супрацьзапаленчае ўзьдзеяньне, а вось лептын і інсулін – пераважна празапаленчае. Грэлін дапамагае пры шматлікіх аўтаімунных захворваньнях, падтрымлівае нармальную працу тымусу (вілавіцы – вілаватай залозы) і нармальную выпрацоўку імунных клетак.

Вядома, голад мае і шмат чорных бакоў. Калі мы прапускаем прыёмы ежы падчас інтэнсіўнай стрэсавай працы, то наш арганізм бярэ патрэбную для сябе энэргію «гвалтам», павялічваючы ўзровень картызолу, разбураючы цягліцы і выклікаючы ператамленьне. Грэлін, зь яго стымуляцыяй дзеяў, таксама мае нэгатыўныя бакі. Так, падвышаны грэлін можа штурхаць на «пошук навізны», павялічваючы рызыку наркаманіі і рызыкоўных паводзінаў. Таму тым, хто выходзіць з залежнасьці, важна трымацца правіла: пазьбягаць «HALT» (hungry, angry, lonely, tired – галодны, злы, самотны, стомлены), не галадаючы залішне.

\section{Асноўныя прынцыпы}

Правіла гучыць так: еж, калі напраўду галодны, да пачуцьця сапраўднага насычэньня.

Таму цалкам заканамерна і лягічна ставіць абед і вячэру такім чынам, каб да іх часу ўжо было пачуцьцё голаду.

\subsection{Гнуткасьць у абедзе і вячэры.}
Голад – гэта сыгнал гатоўнасьці прымаць ежу. Гэтае правіла, вядома, можа быць трохі нязвыклым, калі мы ўвесь час чуем: «Сядай бяз голаду, уставай бяз сытасьці». Такім чынам, для выкананьня гэтага правіла варта «нагуляць апэтыт» і «не перабіваць апэтыту».

\subsection{«Нагуляць апэтыт».}
Гэта працяглыя прамежкі паміж прыёмамі ежы. Сапраўдны голад не ўзьнікае імгненна, ён зьяўляецца паступова, па меры выкарыстаньня запасаў глікагену ў цягліцах і печані. Імгненнае ўзьнікненьне жаданьня нешта зьесьці – гэта прыкмета голаду на глебе стрэсу.

\subsection{«Не перабіваць апэтыту».}
Не перакусваць. Гэта важна, бо нават невялікі перакус («каліўца ў рот укінуць») можа прывесьці да падзеньня ўзроўню грэліну і прытупленьня пачуцьця голаду. Гэты прынцып выкарыстоўваецца ў дробавым харчаваньні, якое я не рэкамэндую. Для звычайнай працы сыстэмы «голад – сытасьць» нам важны добры апэтыт да моманту прыёму ежы. Часта людзі перабіваюцца перакусамі, спажываючы мала калёрыяў, замест таго каб паўнавартасна паесьці. Такая стратэгія можа прывесьці як да недаяданьня, так і да пераяданьня. Пастаянныя перакусы ствараюць падвышаную нагрузку на страўнікава-кішачны тракт – ад зубоў да стрававальных залозаў.

Сапраўдны голад не ўзьнікае імгненна, ён зьяўляецца паступова, па меры выкарыстаньня запасаў глікагену ў цягліцах і печані. Імгненнае ўзьнікненьне жаданьня нешта зьесьці – гэта прыкмета голаду на глебе стрэсу.

\subsection{«Апэтыт прыходзіць у час яды».}
Узровень грэліну прыкметна павялічваецца, калі мы ўжо садзімся за стол. Выгляд ежы, яе пах яшчэ мацней стымулююць апэтыт. Паўза перад ядой разабраная асобна (гл. разьдзел «Паўза перад ядой»). У здаровых людзей узровень гармону голаду грэліну максымальны нашча, пасьля пачатку яды (праз 20 хвілінаў), затым зьніжаецца на 35-55% і захоўваецца на такім узроўні дзьве-чатыры гадзіны. Пад рэзыстэнтнасьцю да грэліну (зьніжэньне адчувальнасьці) разумеюць недастатковае зьніжэньне грэліну плязмы пасьля прыёму ежы, звычайна гэта зьвязана і з парушэньнямі апэтыту.

\subsection{Адрозьнівайце эмацыйны і фізычны голад.}
Фізычны голад узьнікае паступова пасьля яды, вы хочаце зьесьці любую ежу, і яна ўся здаецца прывабнай. Фізычны голад хоць і настойлівы, але яго лёгка адкласьці на гадзіну-другую, пацярпець. У адказ на фізычны голад вам хочацца есьці шмат, а паеўшы, вы адчуваеце палёгку, задаволенасьць, голад зьнікае.

Эмацыйны голад узьнікае раптоўна, часта не зьвязаны з прыёмам ежы, адразу прыходзіць дастаткова інтэнсіўны.

Ён захоплівае ўвагу, патрабуе хуткага задавальненьня. Пры гэтым вам хочацца есьці менавіта спэцыфічную страву (тлустую, вострую, хрумсткую: піцу, чыпсы, цукеркі і да т.п.), але ня шмат, а «кавалачак». Пры гэтым звычайныя стравы вам есьці ня хочацца. Часта, наеўшыся ў адказ на эмацыйны голад, мы атрымліваем посмак у выглядзе напругі, шкадаваньня і агіды. Важна разумець, што каўбасой мы ня вырашым пытаньне эмацыйнага голаду. А вось тэхнікі ўсьвядомленасьці, узяцьця чагосьці пад кантроль, тэхнікі рэляксацыі дапамогуць зьнізіць напал эмацыйнага голаду.

\subsection{Кантралюйце апэтыт.}
Кантроль голаду – гэта найважнейшая ўмова для доўгатэрміновага выкарыстаньня любой дыеты. Калі голад ня ўзяты пад кантроль, то наўрад ці вы зможаце прытрымлівацца яе ў доўгатэрміновай пэрспэктыве. У ідэале схэма харчаваньня павінна выбудоўвацца наступным чынам: сытны сьняданак, такі, каб яго хапіла да абеду і зьявіўся апэтыт на абед. Добры абед, каб вы былі сытыя да вячэры, але на вячэру быў апэтыт. Вячэра такая, каб вам яе хапіла да сну. Для гэтага важна кантраляваць свой голад, калі ён узьнікае, і старанна яго задавальняць, каб сытасьці вам хапала надоўга.

\subsection{Ежце ўволю.}
Як мы з вамі ведаем, насычэньне лепей за ўсё ўспрымаецца на фоне папярэдняга голаду. Грэлін хутка зьніжаецца пасьля 20 хвілінаў прыёму ежы і на 2-4 гадзіны. Таму насычэньне – гэта зьнікненьне голаду, і гэта самы надзейны індыкатар насычэньня. «Больш ня лезе» – гэта ўжо сымптом пераяданьня, а не фізыялягічнай сытасьці. Насычэньне – гэта шматузроўневы працэс, які ўключае этап цэфалічнага (мазгавога) насычэньня пры выглядзе і паху ежы, мэханічны этап (стымуляцыя мэханарэцэптараў страўніка), кішачны (усмоктваньне глюкозы і амінакіслотаў), дзеяньне гармонаў (гастраінтэстынальны пэптыд, халецыстакінін і інш.) і яшчэ шэраг мэханізмаў. Чым больш мы задзейнічалі розных мэханізмаў насычэньня, тым хутчэй яно надыходзіць. Асобна варта адрозьніваць сытасьць – падтрыманьне насычэньня на працягу часу. Сытасьць падтрымліваецца дзякуючы вугляводам зь ніжэйшым глікемічным індэксам (напрыклад, бабовыя), распушчальнай вязкай клятчатцы (багавіньне, гародніна, садавіна), бялкам і да т. п. Выбар правільных прадуктаў – гэта ключ да кантролю насычэньня і сытасьці.

Эмацыйны голад захоплівае ўвагу, патрабуе хуткага задавальненьня. Пры гэтым вам хочацца есьці менавіта спэцыфічную страву (тлустую, вострую, хрумсткую: піцу, чыпсы, цукеркі і да т.п.), але ня шмат, а «кавалачак». Звычайныя стравы вам есьці не хочацца. Часта, наеўшыся ў адказ на эмацыйны голад, мы атрымліваем посмак у выглядзе напругі, шкадаваньня і агіды.

\section{Як трымацца правіла? Ідэі і парады}

\subsection{Фальшывы голад.}
Важна адрозьніваць праўдзівы фізыялягічны і фальшывы голад. Фальшывы голад можа быць зьвязаны са стрэсам, стомленасьцю, недасыпам, нэгатыўнымі эмоцыямі і многімі іншымі нехарчовымі аспэктамі. Стрэс павялічвае ўзровень грэліну. Ежа ў гэтым выпадку шкодзіць фігуры і ніяк не спрыяе рашэньню эмацыйнай праблемы, а часьцяком адно пагаршае яе.

\subsection{Недасыпаньне.}
Дэфіцыт сну, няякасны сон прыводзяць да зьніжэньня ўзроўню гармону сытасьці лептыну і павышэньня ўзроўню грэліну, гармону голаду. Адна гадзіна недасыпаньня можа прывесьці да пераяданьня на наступны дзень да 250 ккал: чым меней вы сьпіце, тым, імаверна, болей зьясьцё на наступны дзень. Таму сьпіце добра, якасны сон – важны крок на шляху кантролю голаду!

\subsection{Гіпадынамія.}
Пастаяннае сядзеньне (нават у зручным або эрганамічным крэсьле) нэгатыўна ўплывае на ўзровень стрэсу і шкодзіць абмену рэчываў. Працяглае сядзеньне і дэфіцыт руху ўзмацняюць голад, а вось нават невялікая актыўнасьць (праца стоячы, хада, уздым па лесьвіцы) спрыяюць лепшаму кантролю голаду. Часьцей рабіце невялікія перапынкі і рухайцеся!

Дэфіцыт сну, няякасны сон прыводзяць да зьніжэньня ўзроўню гармону сытасьці лептыну і павышэньню ўзроўню грэліну. Адна гадзіна недасыпаньня можа прывесьці да пераяданьня на наступны дзень да 250 ккал! Чым менш вы сьпіце, тым больш зьясьцё на наступны дзень. Таму сьпіце добра, якасны сон – важны крок на шляху кантролю голаду!

\subsection{Абязводжваньне.}
Смагу мы часта прымаем за голад. Дастаткова бывае выпіць усяго глыток вады, каб зразумець розьніцу (гл. разьдзелы «Водны балянс», «Балянс натрый – калій»).

\subsection{Гарманальныя зьмены.}
Часта прычынай зьменаў апэтыту зьяўляюцца пэўныя гарманальныя зьмены. Так, зьніжэньне адчувальнасьці да інсуліну (інсулінарэзыстэнтнасьць), зьніжэньне адчувальнасьці да лептыну (лептынарэзыстэнтнасьць), парушэньні працы палавых гармонаў могуць прыкметна ўзмацняць апэтыт. Мэтабалічная жорсткасьць – частая прычына падвышанага голаду (гл. разьдзел «Мэтабалічная гнуткасьць і цыркадная сынхранізацыя»).

\subsection{Рэчывы, якія перашкаджаюць насычэньню.}
Існуе шмат рэчываў, якія могуць стымуляваць пераяданьне і тармазіць насычэньне. Да іх адносяцца соль (натрый), узмацняльнікі смаку, араматызатары, цукразамяняльнікі, цукар, рэчывы, якія ўтвараюцца пры вэнджаньні і смажаньні. Лішак натрыю стымулюе апэтыт і пераяданьне, паменшыце колькасьць солі ў ежы – і вы будзеце есьці менш, нават вады вам спатрэбіцца менш. Нават звычайныя араматызатары, у якіх няма калёрыяў, стымулююць голад і могуць прымусіць вас зьесьці на 10% больш, чым вы б зьелі бязь іх. Цукразамяняльнікі пры працяглым выкарыстаньні павялічваюць голад і цягу да салодкага, зьніжаюць сытасьць. Адмоўцеся ад іх выкарыстаньня на рэгулярнай аснове. Спалучэньне «тлушч і цукар» найбольш эфэктыўна разбурае насычэньне і распальвае апэтыт.

\subsection{Спэцыі.}
Горкі, востры, кіслы смакі, вострыя закрасы могуць тармазіць апэтыт і спрыяць лепшаму насычэньню. Акрамя гэтага, спэцыі ўтрымліваюць велізарную колькасьць біялягічна актыўных рэчываў, нікчэмную колькасьць калёрыяў. Таму даданьне спэцыяў і закрасаў – гэта важны элемэнт здаровага харчаваньня. Спэцыі паляпшаюць насычэньне рознымі спосабамі: больш стымуляцыі, больш горкага, мацней жоўцеаддзяленьне, робіцца лепшай маторыка, шчыльнейшым – сэнсарны «ўсьвядомлены кантакт» зь ежай. У сярэднім чалавек зьядае на 200 ккал меней ежы са спэцыямі.

\subsection{Клятчатка і зеляніна.}
Клятчатка (у першую чаргу распушчальная) – гэта найважнейшы кампанэнт, які забясьпечвае доўгатэрміновае насычэньне і яшчэ шэраг вельмі карысных уласьцівасьцяў. Болей за ўсё клятчаткі ў зеляніне, гародніне, багавіньні. Ужывайце клятчатку не ў парашках, а ў складзе цэльных прадуктаў, дамагайцеся яе максымальнай разнастайнасьці – гэта карысна для мікрафлёры. Зеляніна месьціць шэраг біялягічна актыўных рэчываў, якія падтрымліваюць сытасьць.

Спэцыі ўтрымліваюць велізарную колькасьць біялягічна актыўных рэчываў і нікчэмную колькасьць калёрыяў, гэта важны элемэнт здаровага харчаваньня. У сярэднім чалавек зьядае на 200 ккал менш ежы са спэцыямі.

\subsection{Паважайце ежу.}
Чым болей вы ведаеце пра ежу і ўпэўненыя, што яна вас насыціць, тым даўжэйшым будзе пачуцьцё сытасьці.

\subsection{Бялок.}
Бялок – гэта выдатны нутрыент, які забясьпечвае доўгае пачуцьцё насычэньня. Таму ежце бялок першым у час сталаваньня, дадавайце яго для кантролю сытасьці.

\subsection{Тлушчы.}
Тлушчы спрыяюць сытасьці, павялічваючы выпрацоўку халецыстакініну. Але праблема для таго, хто худнее, у іх высокай калярыйнасьці. Аптымальнае спажываньне тлушчаў альбо зь бялкамі (у цэльным складзе рыбы ці мяса), альбо ў выглядзе салаты зь зелянінай і гароднінай.

\subsection{Нармалізацыя голаду.}
Пры пахудзеньні ўзровень грэліну і адчувальнасьць да яго расьце, што часта зьяўляецца прычынай адскоку і зваротнага набору вагі. Таму плыўнасьць працэсу зьмяншае гэтыя рызыкі. На жаль, гэты працэс у людзей вельмі працяглы. Так, пахуданьне на 8,5% ад масы цела і падтрыманьне гэтай вагі  цягам паўгоду выявіла наступнае: паўгоду грэлін расьце, пакуль чалавек худнее, і потым застаецца яшчэ паўгоду павышаным, а ўжо затым прыходзіць у норму.

\subsection{Не магу адрозьніць від голаду.}
Ёсьць простае працоўнае рашэньне: калі вы ня можаце зразумець, адчуваеце вы эмацыйны ці фізыялягічны голад, то тады ня ежце. Ня ўпэўнены – ня еж!

\subsection{Стварыце багацьце.}
Сам факт багацьця, раскашаваньня ва ўсім, пачынаючы ад падзякі навакольным людзям, ежы, сабе, дастатковая разнастайнасьць прадуктаў, задаволенасьць жыцьцём – усё гэта спрыяе большай сытасьці і зьніжае голад. Шукайце задавальненьне і задаволенасьць у ежы, насычаючы ня толькі цела, але й розум.

\subsection{Не хадзіце ў краму галоднымі.}
Як паказваюць дасьледаваньні, галодныя людзі схільныя купляць больш калярыйныя і менш карысныя прадукты. Купляючы ці замаўляючы дастаўку ў інтэрнэце, уносьце карысныя прадукты ў абраныя сьпісы.

\chapter{Паўза перад ядой}

Паўза перад ядой~--- гэта прамежак часу паміж момантам, калі вы селі за стол зь ежай, і да моманту пачатку яды. Паўза перад ядой мае вялікае значэньне для наладкі на ежу, зьніжэньня ўзроўню стрэсу, павелічэньня апэтыту, стымуляцыі мазгавой фазы сакрэцыі інсуліну, паляпшэньня смаку і насычэньня, стымуляцыі выдзяленьня стрававальных фэрмэнтаў. Усе гэтыя чыньнікі ў~сукупнасьці вельмі станоўча ўплываюць на харчовыя паводзіны і страваваньне. Я часта бачу, як людзі ядуць нецярпліва і похапкам накідваюцца на ежу, сам быў такі. Але гэта не здаровыя харчовыя паводзіны. Для таго каб складаныя працэсы працякалі правільна і з~задавальненьнем, да іх трэба падрыхтавацца. Бо мы ведаем, што сакрэт добрага сэксу~--- у~прэлюдыі. Гэтак жа сама і зь ежай: накідвацца прагна на ежу, толькі сеўшы за стол,~--- гэта шмат чаго сябе пазбаўляць. Зрабіце прэлюдыю перад ядой~--- вы атрымаеце нашмат больш задавальненьня і карысьці!

\subsection{Як зьявілася праблема?}

Праблема «накідваньня на ежу» зьяўляецца, калі гэтаму папярэднічае доўгі пэрыяд голаду, высокі ўзровень стрэсу. У~спробе суняць пачуцьцё голаду і стрэсу людзі выкарыстоўваюць ежу, каб як мага хутчэй пазбавіцца ад непрыемных адчуваньняў, «набіць» страўнік, каб адчуць расслабленьне і спакой. Такія дзеяньні могуць прывесьці да пераяданьня і да горшага засваеньня ежы.
Час прыёму ежы зьмяншаецца, ежу ўсё часьцей мы купляем. Гэта прыводзіць да таго, што мы пачынаем есьці практычна бяз паўзы перад ядой. Але такая дробная на першы погляд дэталь істотна ўзьдзейнінчае на наш працэс страваваньня.

\subsection{Як гэта ўплывае на здароўе?}

\paragraph{Мазгавая фаза страваваньня.}
Выгляд смачнай ежы, яе водар стымулююць выдзяленьне інсуліну ў~невялікіх колькасьцях, што рыхтуе нашу сыстэму страваваньня да прыёму ежы. Такая падрыхтоўка дапамагае лепей кантраляваць апэтыт і меней пераядаць.

\paragraph{Стрэс.}
Рэзкае пераключэньне з~працы на яду непажаданае. Паўза перад ядой дапамагае настроіцца на прыняцьце ежы. Бо стрэсавая сымпацыйная сыстэма прыгнятае выдзяленьне стрававальных фэрмэнтаў і зьніжае маторыку кішачніка. А~вось расслабленьне зьвязанае з~актывацыяй парасымпацыйнай антыстрэсавай сыстэмы, якая стымулюе выдзяленьне сьліны ды іншых сакрэтаў, стымулюе маторыку кішачніка, паляпшае крывацёк у~страўнікава-кішачным тракце.

\paragraph{Аўтаматычнае пераяданьне і ўсьвядомленасьць.}
Паўза перад ядой зьніжае сьпех і аўтаматычнае рэагаваньне, дапамагае пакінуць нэгатыўныя эмоцыі не за сталом, атрымаць больш задавальненьня і зрабіць больш здаровы выбар стравы.

\tipbox{Выгляд смачнай ежы, яе пах стымулююць вылучэньне інсуліну ў~невялікіх колькасьцях, што рыхтуе нашу сыстэму страваваньня да прыёму ежы. Такая падрыхтоўка дапамагае лепей кантраляваць апэтыт і меней пераядаць.}

\paragraph{Апэтыт.}
Як гаворыць народная мудрасьць, «апэтыт прыходзіць падчас яды». Узровень грэліну ўплывае на задавальненьне ад ежы, на колькасьць зьедзенага. Чым вышэйшы апэтыт перад прыёмам ежы, тым лягчэй вам своечасова заўважыць сытасьць. Зьніжэньне пачуцьця голаду~--- гэта і ёсьць зьяўленьне сытасьці. Калі вы пачынаеце есьці без пачуцьця голаду, то можаце аўтаматычна пераесьці.

\paragraph{Паляпшэньне смаку ежы і задавальненьня ад ежы.}
У адным дасьледаваньні навукоўцы паказалі, што паўза перад ядой, запоўненая фатаграфаваньнем ежы і публікацыяй фота, палепшыла смак і павысіла задавальненьне ў~90\,\% удзельнікаў экспэрымэнту. Аўтары дасьледаваньня лічаць, што паўза перад ядой павялічвае асалоду. Таму карыстайцеся невялікім чаканьнем, каб узмацніць смак.

\paragraph{Сытасьць.}
Калі вы ясьцё зьлёту, ігнаруеце тое, што вы ясьцё, нічога ня ведаеце і ня думаеце аб карысьці ежы і харчовай вартасьці, то такая ежа горай спаталяе голад. Так, у~адным дасьледаваньні падыспытныя атрымлівалі малочны кактэйль аднолькавай калярыйнасьці, але з~рознымі этыкеткамі (на адной указвалася істотна меншая калярыйнасьць, на другой~--- вялікая). Аказалася, што падзеньне гармону голаду пасьля кактэйлю залежала ад этыкеткі, а~не ад рэальнай калярыйнасьці. Такім чынам, псыхалягічная ўстаноўка на сытасьць і дастатковасьць ежы рэальна ўплывае на ўзровень гармону голаду ў~крыві чалавека.

\subsection{Асноўныя прынцыпы}

Зрабіце паўзу перад ядой ад 2 да 5 хвілінаў, разгледзьце ежу, адчуйце яе пах, падумайце аб тым, для чаго вам энэргія ежы, адкуль зьявілася гэтая ежа і што ў~ёй карыснага і дзейснага.

\paragraph{Час.}
Дасьледаваньні паказваюць, што ў~першыя 30 сэкундаў пах ежы распальвае апэтыт і жаданьне зьесьці менавіта больш высокакалярыйных страваў. Але вось дзьве і больш хвіліны нюху ежы зьніжалі цягу да высокакалярыйнай ежы і дазвалялі рабіць больш здаровы харчовы выбар.

\paragraph{Усьвядомленасьць.}
З пункту гледжаньня ўсьвядомленасьці самая эфэктыўная яе форма, якая зьніжае стрэс,~--- гэта падзяка. Мы можам падзякаваць тым людзям, хто лавіў, вырошчваў, захоўваў, перавозіў, гатаваў нашую сёньняшнюю ежу. Мы можам падзякаваць нашым абставінам, што можам спакойна есьці нашую ежу сёньня і з~намі ўсё ў~парадку.

\paragraph{Вывучэньне ежы.}
Разгледзьце ежу ўважліва. Сфакусуйцеся на той энэргіі, якую яна вам дорыць. Як вы ёй карыстаецеся? Назавіце ўсе візуальныя інгрэдыенты і водары. Што карыснага ёсьць у~кожным зь іх? Як гэтая ежа можа дапамагчы вам у~дасягненьні вашых мэтаў? Такі падыход дапаможа вам зрабіць найболей здаровы выбар у~ежы.

\paragraph{Камунікацыя.}
Пагаварыце пра ежу. Падзякуйце таму, хто яе прыгатаваў. Вы можаце сказаць і тост (і без алькаголю): «Няхай гэтая ежа дасьць сілы нам скончыць сёньняшні праект!» Чаканьне і размовы, як і ў~рэстаране, робяць ежу смачнейшай.

\subsection{Як трымацца правіла? Ідэі і парады}

\paragraph{Ці гатовыя вы да ежы?}
Ці зьнізіўся стрэс? Гэта лёгка праверыць, бо пры стрэсе прыгнятаецца выпрацоўка сьліны і ротавая поласьць сухаватая. Калі пры выглядзе ежы ў~вас сухі рот, то трэба яшчэ адпачыць, калі сьліна зьяўляецца~--- можна есьці!

\paragraph{Гарачая ежа.}
Занадта гарачая ежа і напоі пры працяглым выкарыстаньні шкодзяць сьлізьніцы і павялічваюць рызыку апёку і раку рота, гартані і стрававода. А~рэзкая зьмена гарачай і халоднай ежы можа сапсаваць зубную эмаль. Таму паўза перад ядой добрая, каб даць астыць ежы, бясьпечная тэмпэратура~--- 40\,°C і ніжэй.

\paragraph{Халодная вада.}
Вымыйце рукі, можна і твар, перад дой. Халодная вада зьніжае стрэс, стымулюе тонус вагусу (блукальнае нэрвы)\index{вагус (блукальны нэрв)}. Высокі тонус вагусу\index{вагус (блукальны нэрв)}, у~сваю чаргу, спрыяе лепшаму страваваньню.

\paragraph{Дыхайце.}
Дыхальныя практыкаваньні дапамагаюць супакоіцца і паслабіцца. Некалькі глыбокіх выдыхаў зь невялікай затрымкай на выдыху дапамогуць вам. Можна выкарыстоўваць адмысловы дадатак на смартфоне.

\paragraph{Паважайце ежу.}
Важна ставіцца да ежы з~павагай! Калі мы садзімся за стол з~думкай «не памерці б з~голаду ад гэтай травы», то ані задавальненьня, ані сытасьці мы не атрымаем. Калі вы ўспрымаеце ежу як крыніцу энэргіі, з~удзячнасьцю і падзякай, задавальненьнем, гэта будзе карысна для здаровых харчовых паводзінаў. А~калі ведаеце, колькі там вітамінаў, мінэралаў і карысных рэчываў, колькі энэргіі~--- то гэтыя веды і павага трансфармуюцца ў~сытасьць і задавальненьне. Думайце пра ежу як пра крыніцу энэргіі. Бо гэта сапраўды так: лічаныя месяцы таму энэргія, зьмешчаная ў~хімічных сувязях вашай брокалі, была фатонамі, якія нарадзіліся ў~тэрмаядзерных сонечных рэакцыях! Ведайце і паважайце ежу~--- і яна адкажа вам узаемнасьцю!

\tipbox{Занадта гарачая ежа і напоі пры працяглым выкарыстаньні шкодзяць сьлізьніцы і павялічваюць рызыку апёку і раку рота, гартані і стрававода. Рэзкая зьмена гарачай і халоднай ежы можа пашкодзіць эмаль зубоў. Бясьпечная тэмпэратура ежы~--- 40\,°C і ніжэй.}

\paragraph{Задайце сабе правільныя пытаньні.}
Ці ўзбагачае гэтая ежа мой рацыён? Настолькі я галодны? Ці можна зьесьці больш здаровую альтэрнатыву? Якую порцыю я зьбіраюся зьесьці? Дасьледаваньні паказалі, што візуалізацыя меншай порцыі падчас ежы прыводзіць да зьніжэньня колькасьці зьедзеных калёрыяў.

\paragraph{Ці можна піць перад ежай?}
У цэлым можна, калі адчуваеце смагу. Пасьля шклянкі вады няхай пройдзе яшчэ пара хвілінаў да прыёму ежы. Водна-солевы балянс разабраны ў~разьдзелах «Водны балянс», «Балянс натрый~--- калій».

\tipbox{Думайце пра ежу як пра крыніцу энэргіі. Бо гэта сапраўды так: лічаныя месяцы таму энэргія, зьмешчаная ў~хімічных сувязях вашай брокалі, была фатонамі, якія нарадзіліся ў~тэрмаядзерных сонечных рэакцыях! Ведайце і паважайце ежу~--- і яна адкажа вам узаемнасьцю!}

\paragraph{Малітва.}
Вы можаце прыдумаць (ці ўзяць існую) малітву, якую будзеце чытаць перад ядой. Стаўшы часткай вашага рытуалу, яна выдатна супакоіць і цудоўна настроіць на сталаваньне.

\chapter{Павольная яда}

Павольная яда – працягласьць прыёму ежы, што складае ня менш за 20 хвілінаў. Такі час дазваляе паўнавартасна ўключыцца сыстэме насычэньня, расслабіцца, уключыць работу залозы страўнікава-кішачнага тракту. Час паглынаньня ежы непасрэдна ўплывае на стан здароўя. Пры іншых роўных абставінах больш высокая хуткасьць яды зьвязаная з горшым здароўем. Паводле меркаваньня некаторых аўтараў, мы можам лічыць хуткай ядой паглынаньне больш за 100 грамаў ежы за хвіліну, а павольнай – менш за 60 грамаў. Што да зручнейшага часавага паказчыка, мы можам лічыць адрэзак часу ў 20 хвілінаў (мінімум), як найболей аптымальным для прыёму ежы, але можна (а ў добрай кампаніі і трэба) і больш.

Я часта бачу сярод сваіх кліентаў, што павелічэньне часу сталаваньня прыводзіць да прыкметнага зьніжэньня колькасьці зьедзеных калёрыяў без усялякага самапрымусу. Мы звыклі лічыць, што фастфуд – гэта ежа, якая прадаецца толькі ў забягайлаўках. На жаль, фастфудам можа стаць любая ежа, зьедзеная хутка! Уявіце сабе, як натоўп людзей арганізавана і павольна праходзіць праз вузкія дзьверы. Не сьпяшаючыся ўсе пасьпеюць без праблемаў. А калі пачынаецца сьпех, можа ўзьнікнуць цісканіна з сур'ёзнымі праблемамі, а выйсьці становіцца і зусім немагчыма. Пазьбягайце цісканіны прадуктаў, арганізуйце спакойнае і раўнамернае іх паступленьне!

\subsection{Як узьнікла праблема?}

Цягам тысячаў гадоў гісторыі чалавецтва людзі елі ў кампаніі сваіх блізкіх. Дзіця з маці, дарослы ў коле супляменьнікаў, стары сярод сямейнікаў. Адмова сталавацца разам была адной з самых страшных пакараньняў. Яда была строгім рытуалам з мноствам традыцый, якія рабілі яе ня проста актам загрузкі нутрыентаў у СКТ, а амаль сьвятым актам, важным для выжываньня.

Цяпер культура сталаваньня руйнуецца пад узьдзеяньнем безьлічы чыньнікаў. Гэта і экспансія фастфуду, яда па-за домам, яда на хаду, што падштурхоўвае да хуткага паглынаньня ежы. Паслабленьне сямейных сувязяў атамізуе грамадства, і нават чальцы адной сям'і ядуць у розных пакоях ці перад тэлевізарам. Эвалюцыя гатовай ежы прывяла да таго, што сёньня можна не карыстацца сталовымі прыборамі, ежу практычна ня трэба жаваць, яна сама распадаецца ў роце. Усе гэтыя чыньнікі павялічваюць хуткасьць прыёму ежы, што, з аднаго боку, выклікае рызыку шэрагу хваробаў, а з другога – прыкметна павялічвае колькасьць зьедзеных калёрыяў і пагаршае ўсе праблемы, якія адсюль вынікаюць.

\section{Як гэта ўплывае на здароўе?}

\subsection{Атлусьценьне.}
Ёсьць выразная ўзаемасувязь паміж тым, як хутка вы ясьцё, і тым, як хутка набіраеце вагу. Дасьледаваньні паказваюць, што вага тых, хто есьць хутка, перавышае сярэдні паказьнік на 4 кіляграмы, а вага тых, хто есьць павольна, складае на 3 кіляграмы менш за сярэдні паказьнік. Рызыка атлусьценьня падвойваецца! Рэч у тым, што павольнае харчаваньне дапамагае сыстэме «голад - насычэньне» спрацаваць лепш і насыціцца раней. Мозгу для вызначэньня сытасьці патрэбны час. Так, запаволеньне працэсу адразу прывядзе да таго, што вы зьясцё на 88 ккал менш у кожны прыём ежы. А цягам года гэта прывядзе да страты да 10 кіляграмаў нават бязь зьмены рацыёну!

\subsection{Перажоўваньне.}
Недастатковае перажоўваньне – важная праблема фастфуду. Сёньня мы ўжываем усё больш здробненай, гатовай ежы – катлеты замест мяса, смузі замест садавіны, пюрэ замест цэльнай гародніны і т. п. Дбайнае перажоўваньне ежы важнае для стымуляваньня мясцовага сьлізістага імунітэту, здароўя зубоў, фармаваньня тварнага шкілета ў дзяцей, лепшага насычэньня і кантролю голаду, яно станоўча ўплывае на мозг і на стрэсаўстойлівасьць. Больш за тое, зьмяншаецца актыўнасьць стрэсавай восі, паляпшаецца мазгавы крывацёк, адбываецца стымуляцыя актыўнага мысьленьня, прэфрантальнай кары, паляпшаецца работа памяці і нэўрагенэз, зьмяншаецца рызыка нэўрадэгенэратыўных разладаў. Надмер мяккай здрабнёнай ежы зьніжае вашу стрэсаўстойлівасьць і здароўе зубоў. Чым больш інтэнсіўнае перажоўваньне, тым больш зьніжаецца ўзровень картызолу і адрэналіну. Так, стараннае перажоўваньне зьніжае ўзровень картызолу на 26% за 20 хвілінаў.

\subsection{Іншыя праблемы.}
Хуткае харчаваньне павялічвае рызыку разьвіцьця цукровага дыябэту і зьяўленьня пякоткі больш чым у два разы, таксама павялічваецца рызыка разьвіцьця артэрыяльнай гіпэртэнзіі. Часта хуткая яда і прагнае глытаньне прыводзяць да заглынаньня паветра, якое потым выклікае дыскамфорт у страўніку. У цэлым павольная яда зьмяншае дыскамфорт нават пасьля багатага прыёму ежы. Цікава, што тыя, хто есьць павольна, менш ужываюць солі. Імаверна, гэта зьвязана з тым, што яны атрымліваюць нашмат больш задавальненьня ад ежы, таму не імкнуцца кампэнсаваць задавальненьне сольлю ці павелічэньнем порцый.

\subsection{Культура харчаваньня.}
У дзяцей яда на самоце павялічвае рызыку парушэньняў харчовых паводзінаў у будучыні. Так, дзеці, якія мінімум двойчы на дзень сілкуюцца асобна ад бацькоў, маюць рызыку атлусьценьня на 40% больш. Дзеці, якія ядуць зь сям'ёй больш за 5 разоў на тыдзень, маюць меншую рызыку парушэньняў харчовых паводзінаў, ядуць больш здаровую ежу і лепш вучацца. Чым большая колькасьць сямейных абедаў, тым большую колькасьць гародніны ядуць людзі.

\section{Асноўныя прынцыпы}

Датрымлівацца правіла павольнай яды адначасова і складана, і проста – ежце павольна, мінімум 20 хвілінаў. Падчас ежы старанна жуйце, рабіце паўзы, не адцягвайцеся (не чытайце газэту і не глядзіце ў тэлефон), размаўляйце. Памятайце, што фастфуд – гэта ня толькі тое, што прадаецца ў пунктах хуткага харчаваньня, гэта ўсё, што вы зьелі хутка. Таму і добрая ежа можа стаць фастфудам, калі вы зьясьцё яе хутка.

Мозгу для вызначэньня сытасьці патрэбны час. Так, запаволеньне працэсу адразу прывядзе да таго, што вы зьясьцё на 88 ккал менш у кожны прыём ежы. А цягам году гэта прывядзе да страты ля 10 кіляграмаў тлушчу нават бязь зьмены рацыёну!

\subsection{Час.}
Працягласьць сталаваньня аптымальна павінна складаць ня менш за 20 хвілінаў. Вы можаце выкарыстоўваць розныя шляхі працягнуць сталаваньне: старанна перажоўваць, факусаваць увагу на смаку, паху, тэкстуры ежы. Актыўна выкарыстоўвайце сталовыя прыборы і рытуалы харчаваньня, камунікуйце. Калі вы зьядаеце сваю порцыю ежы за 5 хвілінаў, то 20 хвілінаў будуць для вас пакутлівыя. Дадавайце 5 хвілінаў кожныя тры дні – і паступова ўцягнецеся. Выкарыстоўвайце сэкундамер або будзільнік на гадзіньніку або тэлефоне, каб выпрацаваць пачуцьцё часу, неабходнае для сталаваньня.

Перажоўваньне пачынаецца з адкусваньня кавалка такога памеру, каб вы маглі яго камфортна пражаваць. У працэсе зьвярніце ўвагу, як ежа рухаецца да задняй часткі ротавай поласьці, здрабняючыся на зубах і затым падаючы на дно ротавай поласьці. Самая частая памылка адбываецца, калі вы хочаце «прысьпешыць» перажоўваньне і языком перакідваеце ежу на самыя далёкія зубы. Не сьпяшайцеся, няхай рух адбываецца сам сабой. Старанна разжаваны харчовы камяк, зьмяшаны са сьлінай, дае выдатны старт наступным працэсам страваваньня ў страўніку і кішачніку. Розныя аўтары прапануюць розную колькасьць жавальных рухаў, але важна разжаваць да аднароднага стану. Пасьля гэтага глытайце, пасьля глытаньня зрабіце невялікую сэкундную паўзу, калі ваша ротавая поласьць вольная.

Фастфуд – гэта ня толькі тое, што прадаецца ў пунктах хуткага харчаваньня, гэта ўсё, што вы зьелі хутка. Таму і добрая ежа можа стаць фастфудам, калі вы зьясьцё яе хутка. Паспрабуйце есьці павольна, мінімум 20 хвілінаў.

\subsection{Паўзы.}
Чыстыя прамежкі, паўзы пры перажоўваньні паміж асобнымі адкушанымі кавалачкамі дапамагаюць нам супрацьстаяць аўтаматычнаму жаваньню. Вы можаце спытаць сябе цягам гэтых паўзаў, ці хочаце вы зьесьці яшчэ кавалачак. Магчыма, вы хочаце зьесьці нешта іншае? Вы можаце пагутарыць у гэтыя паўзы, агледзецца па баках, адпачыць. Звыкайце да сэрыі невялікіх паўзаў, абавязкова робячы іх паміж рознымі стравамі.

\subsection{Датрымлівайцеся парадку страваў.}
Розныя нутрыенты па-рознаму ўплываюць на выпрацоўку асаблівага гармону GLP-1. Гэтае злучэньне запавольвае апаражненьне страўніка, зьмяншае ўзровень саляной кіслаты. Запаволеньне апаражненьня страўніка зьвязанае зь лепшым насычэньнем, бо ежа пазьней паступае ў тонкі кішачнік, які ўсмоктвае неабходныя арганізму рэчывы. Чым павольней працякаюць гэтыя працэсы, тым меншая імавернасьць празьмернага падвышэньня глюкозы ў крыві пасьля яды. Занадта высокі ўзровень глюкозы пасьля прыёму ежы зьяўляецца раньняй прыкметай дыябэту, ён таксама можа сустракацца і ў здаровых людзей, узмацняючы выяўленасьць глікацыі, аксыдантнага стрэсу і падвышаючы рызыку сардэчна-сасудзістых хвароб. Дасьледаваньні давялі, што гародніна, бялок і тлушч, якія зьядаюцца ў пачатку сталаваньня, могуць зьменшыць уздым глюкозы ад вугляводнай ежы. Навукоўцы вывучылі пасьлядоўнасьць прыёму прадуктаў: рыс – рыба, рыба – рыс, мяса – рыс, рыс – мяса (прапорцыі і каляраж аднолькавыя).

Аказалася, што прыём рыбы ці мяса ў першую чаргу прыводзіць да большага насычэньня, запавольвае эвакуацыю са страўніка, спрыяе меншаму павышэньню ўзроўню глюкозы, глікеміі і больш высокай выпрацоўцы гармону GLP-1 як у дыябэтыкаў, так і ў здаровых удзельнікаў дасьледаваньня! Пачынаючы сталаваньне зь бялковых і тлустых прадуктаў, зеляніны і гародніны, багатых клятчаткай, мы ўзмацняем выпрацоўку GLP-1. Яны хутчэй насычаюць і памяншаюць дзеяньне вугляводаў. Ідэальны парадак страў: салата зь зелянінай і гароднінай (можна з аліўкавым алеем), затым бялковая ежа, і толькі пасьля – больш шчыльныя вугляводы або садавіна. Такая пасьлядоўнасьць страў дае лепшы глікемічны кантроль.

\subsection{Кампанія.}
Для чалавека натуральна і карысна есьці ў кампаніі. Пачынаючы зь евангельскага «праламленьня хлеба» і заканчваючы дадзенымі навуковых дасьледаваньняў, вядома, што адзінота павялічвае выпрацоўку грэліну, а вось кампанія, наадварот, стымулюе выпрацоўку антыстрэсавага гармону аксытацыну, які стымулюе блукальную нэрву, паляпшаючы насычэньне. Самая прыемная ежа тая, якую дзеліш з прыемнымі табе людзьмі.

\section{Як трымацца правіла? Ідэі і парады}

\subsection{Складаная ежа.}
Чым больш увагі мы зьвяртаем на ежу, тым павольней ямо і тым яна смачнейшая. Адзін са спосабаў запаволеньня – складаная ежа, якая патрабуе намаганьняў і ўвагі пры спажываньні. Напрыклад, рыба з косткамі (а ня рыбныя катлеты), гарэхі са шкарлупінай (грэцкія), гранат, крабы, какос, перапёлкі, вараныя яйкі і да т.п.

\subsection{Пасьлядоўнасьць страваў.}
Пасьлядоўнасьць прыёму ежы арганізуйце паводле ўдзельнай калярыйнасьці. Пачынайце з прадуктаў зь нізкай удзельнай калярыйнасьцю (салаты, супы і т. п.). Затым можна есьці рыбу, мяса, гарнір. І толькі пасьля гэтага можна завяршыць прыём ежы карысным дэсэртам. У ідэале падача страваў паступовая.

\subsection{Ежа для перажоўваньня.}
Выкарыстоўвайце такую ежу, якую можна пагрызьці і старанна пажаваць: мяса, салера каранёвая, морква і г. д. Грызіце і жуйце ежу, а не сябе і навакольных!

Сталовыя прыборы дапамагаюць есьці павольней. Выкарыстоўвайце нож, адразаючы па кавалачку ад порцыі. Дзеля экспэрымэнту паспрабуйце зьесьці суп не сталовай, а чайнай лыжкай і адчуйце розьніцу ў насычэньні.

\subsection{Адхіляйцеся ад стала падчас паўзаў.}
Падчас паўзаў вы можаце адсунуцца ад стала, каб вам прасьцей было перадыхнуць і адцягнуцца.

\subsection{Сталовыя прыборы.}
Не трымайце сталовыя прыборы ў руках увесь час, адкладайце іх падчас паўзаў. Адкладваючы прыборы, вы ясьцё павольней неўпрыкмет для сябе. Выкарыстоўвайце нож, адразаючы па кавалачку ад порцыі. Дзеля экспэрымэнту паспрабуйце зьесьці суп не сталовай, а чайнай лыжкай і адчуйце розьніцу ў насычэньні.

\subsection{Камфортнае месца.}
Выбірайце зручнае, камфортнае месца для яды, дзе вы можаце адчуваць сябе расслабленымі і дзе вам хочацца затрымацца. Можа, гэта будзе месца ля акна з займальным відам ці куток, дзе спакойна. Сьпех і стрэс псуюць ваш апэтыт і стымулююць пераяданьне. Гучная музыка можа памяншаць задавальненьне ад ежы. Выбірайце ціхае спакойнае месца.

\subsection{Дзелавыя і сяброўскія кантакты.}
Выкарыстоўвайце ежу для нэтворкінгу. Ежа расслабляе, дазваляе лепей спазнаць чалавека, спрыяе фармаваньню блізкіх даверных кантактаў, а жаваньне дае магчымасьць абдумаць свой адказ. Яда на самоце хутчэй за ўсё пазбаўляе нас адчуваньня шчасьця.

\subsection{Выпрастайцеся.}
Сядзьце проста, паслабце тугую вопратку, няхай ваша пастава будзе выказваць расслабленьне і ўласную годнасьць. Не сядзіце згорбіўшыся над талеркай, гэта стымулюе больш высокую хуткасьць яды. Ежце высакародна. Калі вы ня ўпэўненыя ў тым, як седзіцё, зьніміце свой прыём ежы на камэру і затым праглядзіце: погляд збоку адкрые вам шмат карыснага!

Стварыце прастору для прыёму ежы. Няхай на стале будзе толькі тое, што мае дачыненьне да прыёму ежы. Прыбярыце ключы, тэлефоны, нататнікі са стала. Калі ясьцё, толькі ежце, і ўсё!

Выкарыстоўвайце ежу для нэтворкінгу. Ежа расслабляе, дазваляе лепей спазнаць чалавека, спрыяе фармаваньню блізкіх даверных кантактаў, а жаваньне дае магчымасьць абдумаць свой адказ. Яда на самоце хутчэй за ўсё пазбаўляе нас адчуваньня шчасьця.

\subsection{Усьвядомленае харчаваньне.}
Зважайце на тэкстуру, колер, кансыстэнцыю, пах, смак ежы. Падбярыце словы, каб назваць іх.

\subsection{Завяршыце прыём ежы.}
Пасьля апошняй калёрыі завяршыце прыём ежы ачысткай ротавай поласьці. Для гэтага прапалашчыце рот або скарыстайцеся жавальнай гумкай. Гэта дапаможа ачысьціць рот ад рэшткаў ежы, палегчыць гігіену ротавай поласьці ды псыхалягічна залацьвіць датрыманьне наступнага чыстага харчовага прамежку.

Правіла 10. Харчовае ўстрыманьне (фастынг)

Харчовае ўстрыманьне (пост, пэрыядычнае галаданьне, далей – фастынг) – гэта добраахвотнае сьвядомае абмежаваньне ежы для дасягненьня пэўных вынікаў на пэўны тэрмін. Харчовае ўстрыманьне мае працяглую гісторыю, яго выкарыстоўвалі для духоўнага ўдасканаленьня, паляпшэньня работы мозгу, валявой загартоўкі. Цяпер часьцей за ўсё фастынг выкарыстоўваецца як інструмэнт для паляпшэньня мэтабалічнага здароўя, у першую чаргу для пахуданьня. Ён зьвязаны найперш з рэжымам харчаваньня, але непазьбежна прыводзіць да зьніжэньня колькасьці калёрыяў. Я не рэкамэндую фастынг працягласьцю больш за 24-36 гадзінаў раз на тыдзень для большасьці людзей, даўжэйшае харчовае ўстрыманьне павінна праходзіць пасьля кансультацыі адмыслоўца або пад мэдычным назіраньнем.
Наш арганізм схільны запашаць калёрыі на чорны дзень, як мядзьведзь на зіму. Мядзьведзь набірае шмат тлушчу, але затым засынае і спальвае яго. А людзі часта ядуць бясконца, а вось зіма і сьпячка так і не надыходзяць! І, як мядзьведзь-шатун, мы блукаем тоўстыя і санлівыя! І гэты тлушч, і ўсё назапашанае пачынаюць дзейнічаць супраць нас. Фастынг – гэта як маленькая зіма, якая дазваляе нам скінуць баляст і пачаць пачувацца лепш.

Як зьявілася праблема?

Здавён-даўна, калі нашы продкі вялі лад жыцьця паляўнічых-зьбіральнікаў, яны ўвесь час сутыкаліся з праблемай нерэгулярнага доступу да ежы. Пэрыяды ўдалага паляваньня, калі ежы было шмат, спалучаліся з пэрыядамі голаду (сухмень, бясплённае паляваньне або зьбіральніцтва і інш.). Пры гэтым пошук ежы суправаджаўся прыкметным узмацненьнем фізычнай актыўнасьці. Такое спалучэньне прывяло да рэгулярных цыкляў «голад – баляваньне». Гэтыя старажытныя адаптацыі прывялі да фармаваньня «сквапных генаў», якія рэгулююць як наш мэтабалізм, так і фізычную актыўнасьць.

Такая цыклічная даступнасьць ежы заснаваная на рабоце гармонаў лептыну і інсуліну. Лептын – гэта галоўны гармон, які кантралюе энэргетычны балянс. Ён дазваляе захоўваць энэргію ў той час, калі ежа недаступная, уключаючы «рэжым дэфіцыту». Частай праблемай людзей сёньня зьяўляецца парушэньне адчувальнасьці да лептыну, што прыводзіць да шэрагу захворваньняў – ад хранічнай стомленасьці і дэпрэсіі да праблем са шчытападобнай залозай (шчытавіцай) і фэртыльнасьцю. Добрая адчувальнасьць да лептыну дазваляе падтрымліваць структуру цела, пазьбягаючы адкладаў вісцэральнага (унутранага) тлушчу. Лептынавы цыкль мае пэўную асымэтрыю: яго ўзровень вельмі хутка падымаецца ў час яды, затым павольна падае па меры яе адсутнасьці. Пэрыядычны фастынг дапамагае аднавіць адчувальнасьць да лептыну. Аднак пачынаць лепш вельмі паступова, бо людзям зь нізкай адчувальнасьцю да лептыну працяглы фастынг проціпаказаны.

Харчовае ўстрыманьне выкарыстоўвалі для духоўнага ўдасканаленьня, паляпшэньня работы мозгу, валявой загартоўкі. Сёньня часьцей за ўсё да фастынгу зьвяртаюцца з мэтай паляпшэньня мэтабалічнага здароўя і пахуданьня.

Цяпер мы ў сытуацыі, калі сутыкаемся з пастаянным доступам да высокакалярыйнай ежы. Гэта прыводзіць да таго, што нашы «сквапныя гены» нястомна назапашваюць пажытак «на зіму, на голад», але такія часы не наступаюць. Праз гэта губляецца адчувальнасьць да лептыну і інсуліну. Інсулінарэзыстэнтнасьць цягліцаў, печані, тлушчавай тканкі небясьпечная праблемамі са здароўем і можа выяўляцца ня толькі ў атлусьценьні, але і ў шэрагу іншых праблемаў: ад артэрыяльнай гіпэртэнзіі да сындрому полікістозных яечнікаў.
Цыклічнасьць актуальная ня толькі на ўзроўні гармонаў, але і на ўзроўні клетак. Напрыклад, клеткавы сыгнальны шлях mTORС аптымальна павінен працаваць у перарывістым рэжыме. Калі мы ямо, ён актывуецца, гэта дапамагае клеткам сынтэзаваць новыя рэчывы, расьці. Але ўвесьчасная яго актывацыя прыводзіць да назапашваньня «клеткавага сьмецьця», узмацненьня запаленьня, заўчаснага старэньня, павелічэньня рызыкі анкалёгіі і аўтаімунных хваробаў. А вось рэгулярны фастынг, асабліва ў спалучэньні зь фізычнай актыўнасьцю, зьмяншае актыўнасьць mTORС. Яго нізкая актыўнасьць уключае мэханізм аўтафагіі, «самаачышчэньня» клетак, зьніжэньня запаленьня. Як бачыце, такая цыклічнасьць закладзеная ў нашым целе, і датрыманьне яе – важны прынцып захаваньня здароўя.

Як гэта ўплывае на здароўе?

Аўтафагія і mTORС.
Фастынг зьмяншае актыўнасьць mTORС і паскарае працэс аўтафагіі, што важна для запаволеньня старэньня, паляпшэньня ўзнаўленьня клетак, падаўжэньня жыцьця ды іншых карысных эфэктаў.

Гармоны.
Фастынг павялічвае ўзроўні грэліну і гармону росту, якія станоўча ўплываюць на шмат якія органы і сыстэмы. Зьніжаецца ўзровень інсуліну, лептыну і павышаецца адчувальнасьць да іх. Таксама падае ўзровень гармону IGF-1, што многія дасьледчыкі зьвязваюць са зьніжэньнем рызыкі анкалягічных захворваньняў і падаўжэньнем жыцьця. Але, на жаль, працяглы фастынг можа ўзьдзейнічаць і нэгатыўна, зьмяншаючы ўзровень палавых гармонаў і гармонаў шчытавіцы. Праўда, гэтыя рызыкі вялікія толькі пры працяглым (больш за 72 гадзін) галаданьні, асабліва з падвышанай фізычнай актыўнасьцю. Пры высокім узроўні стрэсу можа падвышацца ўзровень картызолу, які падчас фастынгу нэгатыўна ўплывае на цягліцы.

Імунітэт і фастынг.
Кароткачасовы фастынг мадулюе актыўнасьць імуннай сыстэмы ў некалькіх кірунках. Ён палягчае сымптомы большасьці аўтаіммунных захворваньняў (астма, рэўматоідны артрыт, расьсеяны склероз), памяншае выяўленасьць запаленьня. Фастынг і нізкакалярыйнае харчаваньне зьніжаюць рызыку пухлінаў, павялічваюць эфэктыўнасьць хіміятэрапіі, стымулююць абнаўленьне клетак імуннай сыстэмы нават для пацыентаў на хіміятэрапіі.

Даўгалецьце і фастынг.
Нізкакалярыйнае харчаваньне (а мы памятаем, што пры рэгулярным галаданьні таксама адбываецца зьніжэньне калярыйнасьці харчаваньня) прыводзіць да падаўжэньня жыцьця большасьці жывёлаў.

Мозг і фастынг.
Харчовае ўстрыманьне паляпшае работу мозгу, слых, зрок. Яно таксама павялічвае выпрацоўку нэўратрафічнага чыньніка мозгу BDNF, стымулюе нэўрапластычнасьць. Паляпшаюцца практычна ўсе кагнітыўныя функцыі, запавольваецца іх узроставае зьніжэньне. Фастынг прыкметна зьніжае ўзровень аксыдантнага стрэсу ў нэўронах, павялічвае ўзровень нэўрамэдыятара дафаміну, памяншае рызыку разьвіцьця нэўрадэгенэратыўных захворваньняў, уключаючы самыя распаўсюджаныя – хвароба Альцгеймэра і хвароба Паркінсана. Павелічэньне выпрацоўкі нэўратрафічнага чыньніка мозгу BDNF – гэта істотная перавага харчовага ўстрыманьня над нізкакалярыйнай дыетай, пры якой ён зьніжаецца.

Працяглы фастынг можа аказваць і нэгатыўнае ўздзеяньне, зьніжаючы ўзровень палавых гармонаў і гармонаў шчытападобнай залозы. Гэтая рызыка вялікая адно пры працяглым (больш за 72 гадзіны) галаданьні, асабліва з падвышанай фізычнай актыўнасьцю.

Фастынг, сэрца і структура цела.
У параўнаньні са звычайнай нізкакалярыйнай дыетай фастынг дапамагае зьменшыць колькасьць падскурнага тлушчу практычна бяз страты цягліцавай масы. Акрамя гэтага, рэгулярнае ўстрыманьне ад ежы дапамагае эфэктыўна зьнізіць колькасьць вісцэральнага тлушчу, зьніжае рызыку сардэчна-сасудзістых захворваньняў, артэрыяльны ціск, памяншае прагрэсаваньне атэрасклератычных паражэньняў сасудаў.

Асноўныя прынцыпы

Выпадковы пропуск прыёму ежы (таксама вядомы як RMS Random Meal Skipping).
Усё вельмі проста: калі ў вас выдаўся дзень без нагрузкі, калі няма апэтыту на абед ці вячэру, вы цалкам можаце прапусьціць іх. Гэта можа быць у палёце, пры адсутнасьці апэтыту або немагчымасьці паўнавартасна паесьці. Лепей прапусьціць вячэру, каб на сьняданак у вас быў апэтыт. Дапушчальна да 2-3 выпадковых пропускаў прыёму ежы на тыдзень, але ня больш, каб захоўваць рэжым харчаваньня.

24- або 36-гадзінны фастынг раз на тыдзень (Eat Stop Eat).
Такі фастынг не патрабуе падрыхтоўкі, ды лепей яго плянаваць у найменш стрэсавы дзень, калі вы можаце быць далей ад харчовых стымулаў. 24 гадзіны выглядаюць наступным чынам: вы сьнедаеце і далей нічога не ясьцё да сьняданку наступнага дня. Пост на 36 гадзін: вы вячэраеце, нічога не ясьцё ўвесь наступны дзень і сьнедаеце празь дзень. Пост на 24 гадзіны зручнейшы для тых, хто працуе, бо сьняданак дасьць энэргіі на паўнавартасны працоўны дзень безь зьніжэньня мазгавой актыўнасьці. Я раю пачынаць з 24 гадзінаў і не рабіць два 36-гадзінныя фастынгі на тыдзень, асабліва пры разумовых і спартовых нагрузках.

5:2 сыстэма (The Fast Diet).
Гэта кампрамісны варыянт абмежаваньня калярыйнасьці працягласьцю два дні на тыдзень. Тут вы абмежаваныя на два дні фастынгу ежай калярыйнасьцю не больш за 500-600 ккал, якую можна падзяліць на адзін ці два разы, аптымальна гэта сьняданак і абед. Два дні фастынгу можна ставіць у любыя зь дзён тыдня, падбіраючы іх пад свой расклад. Па сутнасьці, гэты метад нічым асаблівым не адрозьніваецца ад 24-гадзіннага посту раз на тыдзень.

Фастынг празь дзень (alternate day fasting, 36/12, кожны іншы дзень Every-Other-Day Diet (EODD), UpDayDownDay).
Гэтая цыклічная сыстэма мае наўвеце чаргаваньне 12-гадзінных прамежкаў прыёмаў ежы без абмежаваньняў і 36 гадзінаў фастынгу. Напрыклад, вы павячэралі ў сераду, увесь чацвер вы ўстрымліваецеся і пачынаеце есьці ў пятніцу раніцай, атрымліваецца 36 гадзінаў устрыманьня і 12 гадзінаў ежы. Палегчаныя версіі гэтай сыстэмы мяркуюць спалучэньне звычайнага дня і зьніжэньня калярыйнасьці на кожны наступны да 20-25% ад каляражу, альбо альтэрнатыўна – толькі адзін прыём ежы кожны другі дзень.

Тыднёвыя сыстэмы фастынгу.
Таксама вы можаце стварыць свае ўласныя тыднёвыя схэмы фастынгу, ідэальна пасоўныя да вашага ладу жыцьця і сыстэмы трэніровак. Часам эфэктыўнымі аказваюцца схэмы накшталт 2+1, калі вы два дні ясьцё як звычайна, а трэці дзень толькі сьнедаеце, ці схэмы з паніжэньнем калярыйнасьці 1+1+1, якія спалучаюць адзін звычайны дзень, затым дзень безь вячэры і трэці дзень з адным сьняданкам. Ёсьць розныя традыцыйныя схэмы, напрыклад у хрысьціянстве прынята пасьціцца ў сераду і пятніцу кожны тыдзень.

Абмежаваньні на нізкакалярыйнай дыеце вымотваюць, бо патрабуюць увесьчаснага падліку калёрыяў, што складана рабіць абсалютнай большасьці людзей у доўгатэрміновай пэрспэктыве. Лічыце гадзіны, а не калёрыі!

Дыета, якая імітуе галаданьне (FMD Fasting Mimicking Diet).
Гэта абмежаваньне калярыйнасьці і выключэньне некаторых прадуктаў на 5 дзён раз на месяц. Гэтая дыета не зьяўляецца шырока рэкамэндаванай, бо яна цалкам яшчэ ня вывучаная. Пры такім фастынгу назіраецца прыкметнае зьніжэньне маркераў старэньня, рызыкі дыябэту і сардэчна-сасудзістых захворваньняў, паляпшэньне імуннай функцыі. Сутнасьць такога посту ў скарачэньні калярыйнасьці напалову ў першы дзень, а затым спажываецца толькі траціна звычайнага звыклага рацыёну, без жывёльных бялкоў (10% расьлінных), 56% калёрыяў павінны складаць тлушчы, 34% – зеляніна і гародніна, ягады, грыбы. Аўтары рэкамэндуюць яе для здаровых людзей адзін цыкл на 3-6 месяцаў, аднак маё меркаваньне такое, што яе рэкамэндаваць павінен толькі спэцыяліст, бо 5 дзён абмежаваньня калярыйнасьці могуць нэгатыўна паўплываць на здароўе, асабліва пры ўтоеных рызыках або парушэньні тэхнікі бясьпекі.

У параўнаньні з працяглай нізкакалярыйнай дыетай, фастынг мае шмат плюсаў. Гэта і выйгрыш па часе (чым меней прыёмаў ежы, тым болей часу), гэта і меншая рызыка пераесьці, болей задавальненьня, бо хоць рэдка, але вы можаце есьці ўволю і з задавальненьнем. Абмежаваньні на нізкакалярыйнай дыеце вымотваюць, бо патрабуюць сталага падліку калёрыяў, што складана рабіць абсалютнай большасьці людзей у доўгатэрміновай пэрспэктыве. Лічыце гадзіны, а не калёрыі!

Як трымацца правіла? Ідэі і парады

Паступовасьць і адаптацыя.
Пэрыядычны фастынг патрабуе адаптацыі, жорсткія рэжымы накшталт яды празь дзень далёка ня ўсім пасуюць! Для многіх людзей самая простая тэхніка – 24 гадзіны раз на тыдзень – будзе вельмі эфэктыўнай і цалкам дастатковай для падтрыманьня вагі і здароўя.

Спорт і цягліцы.
Да 40 гадзінаў фастынгу, паводле дасьледаваньняў, не прыводзяць да страты цяглічнай масы. Больш за тое, ён дапамагае адначасова спальваць тлушчы і нарошчваць цягліцы. Займацца ўмеранай актыўнасьцю можна і падчас галаданьня, але з разумнымі абмежаваньнямі для красфіту і сілавых відаў спорту. Нізкаінтэнсіўныя віды спорту без абмежаваньняў.

Пераяданьне.
Ня бойцеся пераесьці пасьля фастынгу. Вы зьясьцё ня больш за 25% лішку на наступны дзень, што ніяк не пераважыць 100% прапушчанага каляражу.

Кагнітыўныя здольнасьці.
Пры дэфіцыце калёрыяў можа быць зьніжэньне крэатыўнасьці і стратэгічнага мысьленьня, але гэта адно праявы дзеяньня стрэсу ад галаданьня. Пры гэтым здольнасьць рашаць дэталёвыя задачы захоўваецца.

Мэтабалізм.
Кароткатэрміновы фастынг хутчэй павялічвае базавы ўзровень мэтабалізму, яго зьніжэньне не адбываецца аж да 60-72 гадзінаў поўнай адсутнасьці калёрыяў. Так што 24 і 36 гадзінаў галаданьня абсалютна бясьпечныя.

Абязводжваньне.
Важна ўжываць дастатковую колькасьць вадкасьці і пазьбягаць абязводжваньня. Газаваная вада не рэкамэндуецца. Людзі з моцным страўнікам могуць пакінуць гарбату ці каву, аднак для большага эфэкту іх лепей прыбраць.

Кароткатэрміновы фастынг хутчэй павялічвае базавы ўзровень мэтабалізму, яго зьніжэньне не адбываецца аж да 60-72 гадзінаў поўнай адсутнасьці калёрыяў. Так што 24 і 36 гадзінаў галаданьня абсалютна бясьпечныя.

Чакайце хвалі голаду.
Часам жаданьне пад'есьці ўзьнікае як прыступ, дасягаючы максімуму, а потым зьніжаючыся. Памятайце пра гэта і чакайце хвалі голаду.

Рухайцеся.
Любая актыўнасьць палягчае працэс тлушчаспальваньня і пераноснасьць галаданьня. Гуляйце, працуйце ў садзе, катайцеся на ровары, займіцеся ўборкай – любая дзейнасьць дапамагае.

Галаўныя болі.
Галаўныя болі – часты пабочны эфэкт устрыманьня ад ежы. Дзейнічайце плыўна, піце больш вадкасьці, часта дадатковая порцыя солі можа зьмякчыць гэты эфэкт.

Заставайцеся занятымі.
Прадумайце, як структураваць ваш час так, каб быць занятымі і не дапускаць прамежкаў лайдацтва. Незанятасьць спрыяе большай адчувальнасьці да голаду і непатрэбным разважаньням.

Падтрымка.
Заручыцеся падтрымкай сяброў. Не зьвяртайцеся да тых, хто ставіцца да гэтага скептычна. Не спрачайцеся і не пераконвайце іншых людзей, у першую чаргу гэта нашкодзіць больш вам.

Нізкакалярыйная дыета (chronic caloric restriction) таксама вельмі эфэктыўная, але перавагаў рэгулярны фастынг мае больш: меншая рызыка наступстваў абмежаваньня калёрыяў (уплыў на самаадчуваньне, на палавыя гармоны), вышэйшая псыхічная задаволенасьць, лягчэйшы кантроль (людзям лацьвей кантраляваць час, чым дакладную колькасьць калёрыяў). Тым ня менш заўсёды магчымы выключэньні.

Пахуданьне для здаровых.
Дасьледаваньні паказваюць, што невялікае зьніжэньне вагі вельмі карыснае і для здаровых людзей са звычайнай вагой. Трохі схуднеўшы, яны паляпшаюць якасьць сну, павялічваюць лібіда, памяншаюць вісцэральны тлушч і павялічваюць цяглічную масу.

Пабочныя эфэкты фастынгу.
Усе яны часовыя і пакрысе сыходзяць з практыкай галаданьня. Часам патрабуецца 1-2 месяцы для зьніжэньня моцнага голаду. Такім чынам, недахопы – гэта навязьлівыя думкі аб ежы (затое потым простая ежа смачнейшая), адчуваньне падзеньня энэргіі (але цукар у крыві ня падае ў здаровых людзей!), ваганьні настрою (з паляпшэньнем на наступны дзень).

Групы рызыкі.
Неабходна атрымаць кансультацыю спэцыяліста людзям зь недастатковай вагой, з разладамі харчовых паводзінаў (булімія, анарэксія), цяжарным жанчынам і жанчынам, якія кормяць грудзьмі. Акрамя таго, у групу рызыкі ўваходзяць тыя, хто перанёс сур'ёзнае захворваньне, нядаўнюю апэрацыю і/або знаходзіцца на мэдыкамэнтозным лячэньні.
Дасьледаваньні паказваюць, што невялікае зьніжэньне вагі вельмі карыснае і для здаровых людзей са звычайнай вагой. Трохі схуднеўшы, яны паляпшаюць якасьць сну, лібіда, у іх зьмяншаецца вісцэральны тлушч і павялічваецца цяглічная маса.

Працяглыя галаданьні.
Акрамя апісаных тут, магчымыя і больш працяглыя галаданьні, але яны небясьпечныя, іх мінусы могуць пераважваць плюсы, праводзіцца яны могуць толькі ў спэцыялізаваных установах пад мэдыцынскім кантролем. Посты працягласьцю больш за 72 гадзін – гэта зона павышанай рызыкі.

\chapter{Рэфіды і «салодкія дні»}

Рэфіды і «салодкія дні»~--- гэта пэрыядычнае павелічэньне каляражу з~мэтай псыхалягічнай разгрузкі («салодкія дні») і з~мэтай павышэньня ўзроўню лептыну ды пераадоленьня мэтабалічнай адаптацыі да дыеты, загрузкі цягліцаў глікагенам (рэфід). Пастаяннае працяглае абмежаваньне калёрыяў запускае ў~нашым целе шэраг мэханізмаў, закліканых захоўваць колькасьць тлушчу, таму хуткасьць страты вагі зьніжаецца. Таксама харчовыя абмежаваньні і забароны павялічваюць узровень стрэсу, абмяжоўваюць сацыяльную актыўнасьць, павялічваюць рызыку парушэньняў харчовых паводзін, зрываў, таму іх доўгатэрміновае выкарыстаньне не зусім здаровае рашэньне. Плянавае ўвядзеньне пэрыядычных павелічэньняў каляражу дапаможа захаваць псыхічнае і мэтабалічнае здароўе і, як ні парадаксальна, паскорыць пахуданьне.

Багата хто схільны да крайнасьцяў і забаронаў, але яны не зьяўляюцца карыснымі. Памятаеце гісторыю «ня думай пра белага мядзьведзя»? Гэтак жа і з~харчовымі забаронамі: забараняючы, мы толькі ўзмацняем значнасьць гэтых прадуктаў, яны здаюцца нам яшчэ больш прывабнымі і жаданымі. Забарона як бумеранг~--- чым мацней забароніш, тым мацней ударыць. Факусуйцеся на парадкаваньні дыеты, адводзячы ў~ёй месца любой ежы, якая здаецца вам асабліва апэтытнай.

\subsection{Як зьявілася праблема?}

Пры кантролі свайго харчаваньня чалавек сутыкаецца з~шэрагам пабочных эфэктаў. Яны спалучаныя зь неабходнасьцю стварэньня дэфіцыту калёрыяў цягам доўгага пэрыяду часу, адмовай ад пэўных прадуктаў і зьменай ладу жыцьця. Так як новае харчаваньне ня стала яшчэ звычкай, то для яго падтрыманьня выдаткуецца вялікая колькасьць псыхічных рэсурсаў, і гэта патрабуе валявога намаганьня. Праблема яшчэ ўскладняецца тым, што зьмяненьне складу прадуктаў прыводзіць да таго, што тыя стравы, якія давалі задавальненьне і дафамінавы выкід (салодкія і тлустыя), выключаныя з~рацыёну, і чалавек сутыкаецца з~дэфіцытам задавальненьня.

\paragraph{Мэтабалічная адаптацыя да дыеты.}
Пры працяглым абмежаваньні калярыйнасьці ўзровень лептыну паступова зьніжаецца. Зьніжэньне лептыну зьяўляецца сыгналам да пераходу ў~рэжым эканоміі калёрыяў, што прыводзіць да нэгатыўнага ўплыву на ўзровень палавых гармонаў і гармонаў шчытавіцы, падзеньня энэргічнасьці, настрою і запаволеньня тлушчаспаленьня.

\paragraph{Вычарпаньне глікагенавых дэпо.}
Зьніжэньне колькасьці цяглічнага глікагену (запас глюкозы ў~цягліцах) пры трэніроўцы зьяўляецца эфэктыўным спосабам стымуляваць спаленьне тлушчу. Аднак моцнае яго зьніжэньне ў~спалучэньні з~абмежаваньнем вугляводаў можа прывесьці да пагаршэньня самаадчуваньня і павышэньня ўзроўню картызолу. У~такім выпадку рэфід дапаможа аднавіць энэргетычныя запасы ў~цягліцах (гэта больш актуальна для тых, хто займаецца сілавымі відамі спорту). Папярэднічаць рэфіду павінна інтэнсіўная трэніроўка на ўсё цела.

\paragraph{Хуткасьць мэтабалізму ў~спакоі.}
Посты і абмежаваньне калярыйнасьці на працягу доўгага часу (больш за 20 дзён) прыкметна зьніжаюць хуткасьць мэтабалізму. Пэрыядычныя рэфіды дапамагаюць часткова прадухіліць яго зьніжэньне і павялічыць эфэктыўнасьць дыеты.

\paragraph{Небясьпека забаронаў.}
Працяглая забарона ўжываньня пэўнай ежы стварае рызыку парушэньняў харчовых паводзінаў. «Забаронены плод салодкі», і забарона ўзмацняе фіксацыю на ежы, выклікае навязьлівыя думкі, павышае ўзровень стрэсу. Таму правільней будзе дзейнічаць, зыходзячы з~пазыцыі кароткатэрміновых абмежаваньняў, а~не забаронаў, успрымаючы абмежаваньне як структураваньне, парадкаваньне свайго харчаваньня, а~не татальную забарону. Жорсткія забароны павялічваюць рызыку стрэсу і зрываў, якія могуць зьвесьці на нішто ўсе нашы намаганьні нармалізацыі рэжыму.

\tipbox{Забарона ўзмацняе фіксацыю на ежы, выклікае дакучлівыя думкі, павялічвае ўзровень стрэсу. Таму правільней будзе дзейнічаць, зыходзячы з~пазыцыі кароткатэрміновых абмежаваньняў, а~не забаронаў, успрымаючы абмежаваньне як структураваньне, упарадкаваньне свайго харчаваньня, а~не татальную забарону.}

\paragraph{Сацыяльная ізаляцыя.}
Пастаяннае жорсткае датрыманьне пэўнай дыеты часта можа ствараць пэўныя сацыяльныя нязручнасьці, калі вы ў~нейкі час устрымліваецеся ад ежы на сяброўскіх сустрэчах, афіцыйных мерапрыемствах і да т.~п. Гэта можа прывесьці да непаразуменьня і адчужэньня. Мая парада~--- сацыяльная актыўнасьць, у~тым ліку зьвязаная зь ежай, павінна захоўвацца, няхай і ў~кампрамісным выглядзе.

\subsection{Як гэта ўплывае на здароўе?}

\paragraph{Павышэньне эфэктыўнасьці пахудзеньня.}
Аптымальны ўзровень гармонаў шчытавіцы і ўзровень мэтабалізму ў~спакоі ды шэраг іншых паказчыкаў важныя для падтрыманьня прагрэсу. Рэфіды дапамагаюць падтрымліваць посьпех пахудзеньня. Лептын напрамую стымулюе актыўнасьць сымпатаадреналавай сыстэмы, узмацняючы спаленьне тлушчу. З~усіх нутрыентаў мацней за ўсё ўзровень лептыну паднімаюць вугляводы, таму менавіта яны часьцей выкарыстоўваюцца ў~рэфідах. Паводле дасьледаваньняў, рэфідная група пахудзеньня пасьля завяршэньня праграмы практычна ня мела эфэкту адскоку (рыкашэту) вагі~--- і гэта вельмі важна для захаваньня дасягнутых посьпехаў.

Асіметрыя лептынавага цыклу заключаецца ў~наступным. Калі вы ясьцё мала некалькі дзён, то ўзровень лептыну падае і спаленьне тлушчу запавольваецца. Але калі вы зьядаеце больш вугляводаў у~1--2 прыёмы ежы, узровень лептыну ўзрастае і тлушчаспаленьне працягваецца. Такім чынам, узровень лептыну ўзрастае хутка, а~зьніжаецца павольна. Таму кароткі абмежаваны рэфід на тле дыеты дапамагае падтрымаць дастатковы ўзровень лептыну.

\paragraph{Памяншэньне псыхалягічнага стрэсу.}
Рэфіды і «салодкія дні» дапамагаюць зьменшыць стрэс і зьнізіць узровень картызолу, захаваць сацыяльную актыўнасьць.

\paragraph{Ежа як узнагарода.}
Ежа, асабліва смачная, выклікае выкід нэўрамэдыятара дафаміну, што спрыяе фармаваньню ўмоўнага рэфлексу з~папярэднімі паводзінамі. Менавіта з~гэтай прычыны маленькімі кавалачкамі ежы дрэсіроўнік можа прымусіць жывёлу рабіць розныя фокусы бяз гвалту. Наш мозг таксама лёгка стварае прычынна-выніковыя сувязі, таму задумайцеся, што адбываецца, калі вы ясьце салодкаё, будучы ў~стрэсе. Сувязь наступная: «Я няўдачнік, таму вось мне ўзнагарода~--- ежа. Каб наступным разам атрымаць смачную ежу, трэба быць у~стрэсе». Гэта вельмі небясьпечная і нездаровая стратэгія, таму я ня раю есьці «сьвяточную» ежу ў~стрэсе. Аднак з~пункту гледжаньня паводзінаў адзначыць свае посьпехі ў~коле блізкіх сяброў смачнай ежай, узнагародзіць сябе за дасягненьні~--- гэта цалкам працоўная стратэгія, закліканая ў~будучыні ўзмацніць вашую матывацыю на далейшыя перамогі.

\subsection{Асноўныя прынцыпы}

Ёсьць шмат назоваў для розных схэмаў рэфідаў. Гэта і дыета для пахудзеньня з~павышэньнем калёрыяў (CSD), калі вы 11 дзён на дыеце, а~затым ідуць 3 дні рэфіду. Таксама маюцца й~іншыя схэмы, накшталт 3+1, калі на тры дні дыеты прыпадае адзін зь вялікай колькасьцю калёрыяў.

\paragraph{Рэфід} (ад анг. re-feed, feed~--- харчаваньне, ежа)~--- гэта мэтавае павышэньне калярыйнасьці і вугляводаў (вугляводная загрузка). Рэфід карысны для тых, хто знаходзіцца на выразнай дыеце, і для тых, хто займаецца спортам. Рэфід аптымальна рабіць вугляводнымі прадуктамі, зь невялікай колькасьцю бялку, але бяз тлушчу і фруктозы (батат, бульба, рыс і да т.~п.). Колькасьць дадатковых калёрыяў таксама мусіць быць умераная, каб не перакрэсьліць усе вашыя вынікі.

Для спартоўцаў рэфіды шчыльна зьвязаныя з~графікам трэніровак (EOD refeeds), чаргаваньне дзён з~розным харчаваньнем, калі чаргуюцца фазы абмежаваньня калярыйнасьці (мала калёрыяў і вугляводаў + больш тлушчаў для спальваньня тлушчу) і фаза нарошчваньня цягліц (шмат калёрыяў і вугляводаў + мала тлушчаў для нарошчваньня цягліцаў).

\tipbox{Ежа, асабліва смачная, выклікае выкід дафаміна, што спрыяе фармаваньню ўмоўнага рэфлексу з~папярэднімі паводзінамі. Менавіта з~гэтай прычыны маленькімі кавалачкамі ежы дрэсіроўнік можа прымусіць жывёлу рабіць розныя фокусы бяз гвалту.}

\paragraph{Зрабіце салодкі дзень, або чытміл.}
Cheat meal перакладаецца як яда-падман, яда-парушэньне. Гэта адзін прыём ежы раз на тыдзень, калі вы ясьцё ўсё, што захочаце. Зьвярніце ўвагу, што «салодкі дзень» пераносіць нельга, ён павінен быць рэгулярным. «Салодкі дзень» зьніжае ціск забаронаў, дае псыхалягічную разгрузку, дапамагае захаваць сацыяльную актыўнасьць, зьніжае рызыку зрываў. Пры гэтым неабходна датрымлівацца агульнага рэжыму харчаваньня ў~свой «салодкі дзень». Калі ў~вас моцная залежнасьць ад ежы, то вы можаце спачатку рабіць «дыету празь дзень», чаргуючы прадукты, а~затым паступова кожны тыдзень-два зьніжаць колькасьць «салодкіх дзён».

\paragraph{Частасьць і працягласьць.}
Адзін дзень, яшчэ лепей~--- адзін прыём ежы. Для рэфіду можна зрабіць 1--2 прыёмы ежы, а~вось у~«салодкі дзень» (чытміл) лепш есьці ўсё, што вы любіце, але ў~межках аднаго прыёму ежы. Людзі з~высокім адсоткам тлушчу могуць рабіць рэфід не часьцей за адзін раз на два тыдні, з~сярэднім~--- раз на тыдзень, а~зь нізкім узроўнем (ды інтэнсіўнымі трэніроўкамі)~--- два рэфіды на тыдзень.

\subsection{Як трымацца правіла? Ідэі і парады}

\paragraph{Дух свабоды.}
Памятайце, што ежа~--- цудоўная, гэта крыніца энэргіі, і вы заўсёды можаце зьесьці ўсё, што захочаце. «Усё можна, але ня вам і ня сёньня». Рабіце забароны кароткатэрміновымі~--- на тыдзень, на месяц, з~магчымасьцю падаўжэньня. Думкі аб тым, што да канца жыцьця вы ня будзеце есьці цукар, сапраўды не дададуць вам здароўя і ўпэўненасьці.

\paragraph{Ня дома.}
Рабіце «салодкі дзень» ня дома. Наведайце ўлюбёную кавярню, каб прыём такой ежы ўдома не ўваходзіў у~звычку. Старанна і густоўна выбірайце прадукты для «салодкага дня».

\paragraph{Структуруйце ежу.}
Памятайце, што «сьвяточная ежа»~--- толькі на сьвяты (але не падманвайце сябе). Падумайце, што асаблівага вы можаце прыгатаваць да традыцыйных ці сямейных сьвятаў, каб падкрэсьліць іх атмасфэру. Многія стравы традыцыйна выкарыстоўваюць толькі на сьвяты. Ня ежце сьвяточную ежу ў~будні.

\paragraph{Абарона часам.}
Калі цяга моцная, дазвольце сабе гэта зьесьці заўтра раніцай. Калі вы высьпіцеся і сілы дададуцца, магчыма, вы зьменіце сваё рашэньне.

\tipbox{Памятайце, што ежа~--- цудоўная, гэта крыніца энэргіі, і вы заўсёды можаце зьесьці ўсё, што захочаце. Рабіце забароны кароткатэрміновымі~--- на тыдзень, на месяц, з~магчымасьцю працягу. Думкі аб тым, што да канца жыцьця вы ня будзеце есьці цукар, сапраўды не дададуць вам здароўя і ўпэўненасьці.}

\paragraph{Абарона адлегласьцю.}
Чым далей вы ад ежы, тым лягчэй трымацца. Таму ня майце гатовай ежы дома: так меней рызыкі яе зьесьці.

\paragraph{Ня ежце «сьвяточную» ежу падчас стрэсу.}
Ніколі ня ладзьце «салодкія дні» ў~стане стрэсу. Спачатку антыстрэсавыя працэдуры і рэлаксацыя, заспакаеньне~--- і толькі тады яда.

\needspace{3\baselineskip}
\paragraph{Калі можна прызначыць «салодкі дзень».}
\begin{enumerate}[itemindent=3em,labelwidth=1.5em,leftmargin=0pt,nosep]
  \item У~вас дастаткова часу паесьці і атрымаць асалоду ад ежы.
  \item Няма фізыялягічных і псыхалягічных прыкметаў стрэсу.
  \item Вы здаровыя.
  \item Усьвядомленасьць і фокус на працэсе: вы знаходзіцеся «тут і цяпер», факусуецеся на працэсе «як я буду есьці».
  \item Рытуалы і правілы «салодкага дня»: у~вас ёсьць фіксаваны рытуал спажываньня: «салодкі дзень», пэўнае месца, іншыя ўмовы.
  \item Узнагарода: вы дамагліся посьпеху ў~складанай справе, якую доўга адкладалі. Абгрунтаваная ўзнагарода замацоўвае атрыманы посьпех на ўзроўні рэфлексу.
  \item Балянс «жаданьне» і «асалода»: вы атрымліваеце шмат задавальненьня ў~працэсе, ня менш чым ад самога чаканьня і прадчуваньня.
  \item Добры, спакойны посмак, расслабленьне.
\end{enumerate}

\paragraph{Калі нельга прызначаць «салодкі дзень».}
\begin{enumerate}[itemindent=3em,labelwidth=1.5em,leftmargin=0pt,nosep]
  \item Вы ясьцё ў~сьпеху, на хаду, час абмежаваны.
  \item Адчуваеце стрэс, злосьць і нэгатыўныя эмоцыі, цягліцавую напругу, падвышаны ціск, пачашчанае дыханьне.
  \item Вы хварэеце.
  \item Усьвядомленасьць і фокус на працэсе: вы занураныя ў~праблемы мінулага ці будучыя справы, сфакусаваныя на мэце «трэба зьесьці, і мне стане лепш».
  \item Рытуалы і правілы «салодкага дня»: вы ня ведаеце ці не выконваеце.
  \item Узнагарода: вы нічога не дамагліся або вашыя дасягненьні звычайныя. Пустая ўзнагарода толькі зьніжае ўзровень матывацыі.
  \item Балянс «жаданьне» і «асалода»: задавальненьне ад чаканьня мацнейшае, чым ад працэсу насалоджваньня. Так-так, у~гэтым выпадку нельга!
  \item Нядобры посмак, шкадаваньне, трывога.
\end{enumerate}
\chapter[Мэтабалічная гнуткасьць і цыркадная сынхранізацыя][Мэтабалічная гнуткасьць]{Мэтабалічная гнуткасьць і цыркадная сынхранізацыя}

Мэтабалічная гнуткасьць~--- гэта здольнасьць пераключацца з~адной крыніцы «паліва» на іншую (ясьцё тлушч~--- спальваеце тлушч, ясьцё вугляводы~--- спальваеце вугляводы), выкарыстоўваць у~сапраўдны момант найболей аптымальнае паліва для адпаведнага віду фізычнай актыўнасьці (нізкая актыўнасьць~--- тлушчы, высокая актыўнасьць~--- вугляводы), а~акрамя гэтага~--- выдатна спальваць тлушчы ў~прамежках паміж ядой і эфэктыўна выкарыстоўваць ды разьмяркоўваць глюкозу падчас яды.

Мэтабалічная «жорсткасьць»~--- адваротны панятак, гэта немагчымасьць (або абмежаваньне магчымасьці) паўнавартаснага пераключэньня з~адной крыніцы энэргіі на іншую, звычайна гэта парушэньне спальваньня тлушчаў. Пры жорсткасьці наш арганізм горай рэагуе на зьмены харчаваньня і трэніроўкі, мы адчуваем хранічную стомленасьць.

Уявіце сабе, што вы плывяце на лодцы пасярод акіяну. Вы ня ўмееце плаваць, ня ўмееце ныраць і вымушаныя чапляцца за лядашчы чаўначок. А~вось навык плаваць і ныраць адразу пашырыць вашыя магчымасьці і прыстасоўвальнасьць, зробіць больш гнуткім. Лодачка~--- гэта вугляводы, а~ўменьне плаваць~--- тлушчы. Калі мы кепска спальваем тлушчы, то становімся меней адаптыўнымі. А~мэтабалічная гнуткасьць~--- гэта магчымасьць добра пачувацца і ў~лодцы, і ў~вадзе, і з~вугляводамі, і з~тлушчамі.

Цыркадная сынхранізацыя\index{цыркадныя рытмы}~--- гэта сумяшчэньне свайго рэжыму харчаваньня з~рэжымам дня. Содневыя (цыркадныя) рытмы\index{цыркадныя рытмы} кіруюцца шэрагам унутраных гадзіньнікаў арганізму (стветлавыя, тэмпэратурныя, харчовыя і да т.~п.). Калі ўсе гэтыя гадзіньнікі паказваюць розны час, то гэта зьбівае наш арганізм~--- і зьяўляюцца праблемы. Вельмі важна сябе сынхранізаваць, для гэтага правільна заводзіце ўсе нашы ўнутраныя гадзіньнікі кожны дзень!

\tipbox{Мэтабалічная «жорсткасьць»~--- гэта немагчымасьць (ці абмежаваньне магчымасьці) паўнавартаснага пераключэньня з~адной крыніцы энэргіі на іншую, звычайна гэта парушэньне спальваньня тлушчаў. Пры жорсткасьці наш арганізм горш рэагуе на зьмены харчаваньня і трэніроўкі, мы адчуваем хранічную стомленасьць.}

\subsection{Як зьявілася праблема?}

\paragraph{Паляўнічыя-зьбіральнікі.}
Традыцыйна садавіна, клубні, мёд былі даступныя абмежаваны час цягам году, паляваньне таксама было пасьпяховым не заўсёды, а~эпізадычна. Таму ў~нашых продкаў чаргаваліся пэрыяды «вугляводы~--- не вугляводы» (тлушч і бялок). Цяпер мы ямо шмат тлушчу, бялку і вугляводаў, у~тым ліку цукар, адначасова ў~адным прыёме ежы. Такія спалучэньні, як тлушч з~цукрам, асабліва небясьпечныя для нашага апэтыту і мэтабалізму. Частыя прыёмы ежы і пераяданьне пагаршаюць праблему.

\paragraph{Гіпадынамія\index{гіпадынамія}.}
Навукоўцы называюць сядзеньне паводле ўплыву на здароўе «новым курэньнем». Чым даўжэй мы сядзім, тым больш растуць рызыкі захворваньняў і раньняй сьмерці, тым ніжэй становіцца адчувальнасьць да інсуліну\index{інсулін}. Нізкаінтэнсіўная актыўнасьць~--- гэта ўсё, што вы робіце, калі не сядзіце і не ляжыце, яе нам у~жыцьці вельмі не хапае. Сакрэт жыцьця доўгажыхароў ня ў~спорце, а~ў пастаяннай рухальнай актыўнасьці на працягу ўсяго дня.

\paragraph{Стрэс.}
Акрамя эпідэміі сядзячага ладу жыцьця мы сутыкаемся з~эпідэміяй стрэсу. Непрадказальнасьць нашага жыцьця, надмер інфармацыі, нявызначанасьць прымушаюць больш перажываць. Наша цела кепска трывае падвышаныя ўзроўні картызолу\index{картызол} і рэагуе на хранічны стрэс старажытным спосабам~--- узмацненьнем апэтыту і запасаньнем тлушчу «на чорны дзень». Антыстрэсавая праграма~--- гэта абавязковы кампанэнт у~нармалізацыі харчаваньня. Стрэс~--- гэта ня толькі праблемы на працы, але і такія незаўважныя фактары, як начны шум. Навукоўцы лічаць, што «шумавое забруджваньне» вельмі шкодна ўплывае на наш арганізм.

\paragraph{Сьвятло.}
Лішак яркага сьвятла ўвечары зьбівае нашыя біярытмы, павялічвае апэтыт увечары. Пры парушаным цыркадным рытме\index{цыркадныя рытмы} тлушчаспальваньне прыгнятаецца і клеткі часьцей выкарыстоўваюць глюкозу. У~экспэрымэнтах чым ярчэй у~спальні, тым больш лішніх кіляграмаў! Навукоўцы кажуць пра сьветлавое забруджваньне, якое можа ўзмацняць шэраг захворваньняў, уключаючы пухлінныя.

\tipbox{Антыстрэсавая праграма~--- гэта абавязковы кампанэнт у~нармалізацыі харчаваньня. Стрэс~--- гэта ня толькі праблемы на працы, але і такія фактары, як начны шум. Шумавое забруджваньне вельмі згубна ўплывае на наш арганізм.}

\paragraph{Тэмпэратура.}
Раней ваганьні навакольнага асяродзьдзя днём і ноччу, у~розныя паравіны году ўплывалі на нас вельмі заўважна. Тэмпэратура кіруе нашым сном, актыўнасьцю, спальваньнем тлушчу. Цяпер мы сутыкаемся з~аднолькавай тэмпэратурай днём і ноччу, што адмоўна ўплывае на нас, зь лішкам цяпла, лішкам адзеньня. Такое «цеплавое забруджваньне» робіць нас нездаровымі.

\subsection{Як гэта ўплывае на здароўе?}

Сымптомы мэтабалічнай жорсткасьці разнастайныя: вы санлівыя пасьля прыёму вугляводаў, вам цяжка трываць прамежкі больш за 4--5 гадзінаў бязь ежы, вам хочацца перакусваць для падтрымкі самаадчуваньня, падчас посту вы страчваеце цягліцы, а~ня тлушч, вам цяжка функцыянаваць без кафэіну (нізкі ўзровень энэргіі). Мэтабалічная жорсткасьць прыводзіць да таго, што, калі ў~вас зьніжаецца ўзровень глюкозы ў~крыві, вы адразу адчуваеце стомленасьць і слабасьць, расшчапленьне тлушчаў не ўключаецца. Людзі з~мэтабалічнай жорсткасьцю ў~стане спакою атрымліваюць нашмат менш энэргіі з~тлушчаў.

Пры мэтабалічнай жорсткасьці ўзьнікаюць праблемы як з~тлушчамі, так і з~глюкозай (два бакі аднаго медаля~--- мэтабалізму). Праблемы з~глюкозай зьвязаныя з~тым, што клеткі ня могуць эфэктыўна паглынаць і спальваць глюкозу (інсулінарэзыстэнтнасьць\index{інсулінарэзыстэнтнасьць}), праблема з~тлушчам выяўляецца як парушэньні акісьленьня тлушчаў (мітахондрыі\index{мітахондрыі} цягліцаў) і павышэньне ўзроўню свабоднай тлустай кіслаты (ліпатаксічнасьць).

\paragraph{Мітахондрыі.}
Пры мэтабалічнай жорсткасьці вельмі цярпяць мітахондрыі\index{мітахондрыі} (гэта своеасаблівыя электрастанцыі клетак, якія выпрацоўваюць энэргію з~паліва). Калі зашмат і тлушчаў, і вугляводаў, то мітахондрыі\index{мітахондрыі} «перагружаюцца», шляхі іх акісьленьня могуць нэгатыўна ўплываць адзін на аднаго. Людзі зь нізкай мэтабалічнай гнуткасьцю маюць менш мітахондрыяў у~цягліцах і спальваюць мала тлушчу, пры гэтым нават тыя нешматлікія мітахондрыі працуюць кепска, схільныя да большага акісьляльнаму стрэсу і падчас працы даюць вялікую ўцечку актыўных формаў кіслароду. А~аксыдантны стрэс пашкоджвае клеткавыя структуры. Добрыя мітахондрыі~--- даўжэйшае жыцьцё. Калі мітахондрыі\index{мітахондрыі} эфэктыўна спальваюць тлушч, то пры гэтым утвараецца менш актыўных формаў кіслароду і менш пашкоджваюцца клеткі. Гэта можа спрыяць запаволеньню старэньня і нават павялічваць працягласьць жыцьця.

\tipbox{Насуперак мітам, нашаму мозгу не патрабуецца сталае падсілкоўваньне глюкозай. Наадварот, яна можа прывесьці да ваганьня ўзроўню цукру, што нэгатыўна паўплывае на нашыя разумовыя здольнасьці, прывядзе да раздражняльнасьці і агрэсіўнасьці.}

\paragraph{Інсулін\index{інсулін}.}
Пры перагрузцы калёрыямі ўзьнікае рэзыстэнтнасьць да інсуліну\index{інсулін}. Пры падвышаным узроўні інсуліну блякуецца расшчапленьне глікагену\index{глікаген} і тлушчу, а~вось назапашваньне тлушчу толькі павялічваецца, роўна як і затрымка вадкасьці. Рэзыстэнтнасьць да інсуліну\index{інсулін}~--- гэта прычына разьвіцьця многіх захворваньняў.

\paragraph{Фігура і харчаваньне.}
Чым вышэйшая гнуткасьць, тым лягчэй мы можам засвойваць як тлушчы, так і вугляводы. Пастаяннае спажываньне вугляводаў зьніжае адчувальнасьць да інсуліну\index{інсулін}, яго ўзровень павышаецца, і гэта запавольвае тлушчаспальваньне. А~высокая мэтабалічная гнуткасьць дазваляе лягчэй вытрымліваць пост без пачуцьця голаду, што палягчае пахуданьне. Калі вы лёгка спальваеце і тлушчы, і вугляводы, тыя вы зможаце падтрымліваць добрую постаць доўгі час, не знаходзячыся на цьвёрдай дыеце.

\paragraph{Паляпшаецца сон.}
Сон і спаленьне тлушчу шчыльна зьвязаныя між сабой. Увечары працэс тлушчаспаленьня ўзмацняецца, што дазваляе ўначы нам не прачынацца, каб паесьці. Узровень глюкозы падае, гэта прыводзіць да павелічэньня выкіду гармону росту і гармонаў шчытавіцы, што таксама спрыяе тлушчаспаленьню. І~ўсе гэтыя працэсы спрыяюць глыбокаму аднаўленчаму сну. Калі спаленьне тлушчу парушана, то сон становіцца менш глыбокім, гэта выклікае частыя абуджэньні і фрагмэнтацыю сну.

\paragraph{Стабільны настрой.}
Насуперак мітам, нашаму мозгу не патрабуецца сталае падсілкоўваньне глюкозай. Наадварот, яна можа прывесьці да ваганьня ўзроўню цукру, што нэгатыўна ўплывае на нашыя разумовыя здольнасьці, вядзе да раздражняльнасьці і агрэсіўнасьці. Нават калі прыкметна абмежаваць колькасьць вугляводаў, усё роўна мозгу хопіць глюкозы, ён атрымае яе з~глікагену\index{глікаген} печані. Лішак вугляводаў і нізкая адчувальнасьць да інсуліну\index{інсулін} павялічваюць рызыку хваробы Альцгаймэра\index{хвароба!Альцгаймэра}, якую навукоўцы называюць дыябэтам мозгу.

\subsection{Асноўныя прынцыпы}

Як мы з~вамі ведаем, на шпагат імгненна сесьці немагчыма, патрэбная паступовая расьцяжка для павелічэньня гнуткасьці. Мэтабалічная гнуткасьць таксама разьвіваецца ад чаргаваньня пэўных актыўнасьцяў з~паступовым павелічэньнем амплітуды. Ёсьць 4 асноўныя вэктары мэтабалічнай гнуткасьці і 2 дапаможныя вэктары (сьвятло, тэмпэратура), зьвязаныя з~цыркаднай сынхранізацыяй. Усе вэктары мэтабалічнай гнуткасьці зьвязаныя паміж сабой і ўзмацняюць адзін аднаго. Напрыклад, ніжэйшыя тэмпэратуры ўзмацняюць тлушчаспаленьне і паляпшаюць сон, а~вось фізычная актыўнасьць зьмяншае стрэс і трывогу.

\paragraph{Рэжым харчаваньня} (чаргаваньне «ем шмат»~--- «ня ем, ем мала»).
Гэты вэктар падрабязна апісаны ў~папярэдніх разьдзелах, ад інтэрвалаў і скарачэньня харчовага вакна да розных формаў посту.

\paragraph{Вар'іраваньне макранутрыентаў\index{макранутрыенты}} (чаргаваньне «вугляводы»~--- «бялкі, тлушчы»).
Як я ўжо пісаў, для нашых продкаў было характэрна паасобнае паступленьне нутрыентаў. Зьбіральніцтва давала расьлінныя вугляводныя прадукты, а~паляваньне прыносіла бялкі і тлушчы. Умеранае вар'іраваньне макранутрыентаў\index{макранутрыенты} на працягу тыдня дапаможа вам аптымальней падтрымліваць мэтабалічную гнуткасьць. Гэта значыць, што лепей зьесьці больш мяса зь зялёнай салатай без гарніру кшталту рысу, а~калі вы ясьцё порцыю грэчкі, дык лепей дадаць гародніну, зеляніну і закрасы, а~не катлету.
Гіпадынамія\index{гіпадынамія} ўзмацняе мэтабалічную жорсткасьць, таму больш працы пераносьце на ногі~--- працуйце стоячы, гаварыце па тэлефоне стоячы, хадзіце ў~перапынках. Шматгадзіннае сядзеньне не кампэнсуецца гадзінай трэніроўкі!

\paragraph{Віды фізычнай актыўнасьці} (чаргаваньне «высокаінтэнсіўная»~--- «нізкаінтэнсіўная» актыўнасьць).
Фізычная актыўнасьць вельмі важная для падтрымкі мэтабалічнай гнуткасьці. Кажучы проста, чым больш цягліцаў, тым вышэйшая мэтабалічная гнуткасьць. Наша цяглічная тканка запасіць глікаген\index{глікаген}, спальвае тлушчы і вугляводы, розныя тыпы валокнаў цяглічнай тканкі маюць дачыненьне да розных субстратаў. А~добрая адчувальнасьць цяглічнай тканкі да інсуліну\index{інсулін} важная для падтрымкі гнуткасьці.

Важна памятаць, што на трэніроўкі мы трацім малую частку энэргіі, а~вось паўсядзённая нізкаінтэнсіўная дзейнасьць адбірае ў~3--4 разы больш калёрыяў! Таму базісам зьяўляецца пастаянная нізкаінтэнсіўная рухальная актыўнасьць, зьніжэньне часу сядзеньня! Гіпадынамія\index{гіпадынамія} ўзмацняе мэтабалічную жорсткасьць, таму больш працы пераносьце на ногі: працуйце стоячы, размаўляйце па тэлефоне стоячы, хадзіце ў~перапынках. Шматгадзіннае сядзеньне не кампэнсуецца гадзінай трэніроўкі! Нізкаінтэнсіўная фізычная актыўнасьць~--- гэта аснова рухальнай піраміды, яе сярэдні ўзровень~--- гэта сярэднеінтэнсіўная актыўнасьць (гульнявыя віды спорту, танцы, кардыё, бег, ровар і да т.~п. 45 хвілінаў аэробных практыкаваньняў умеранай інтэнсіўнасьці 3--5 разоў на тыдзень або 30 хвілінаў умеранай інтэнсіўнасьці), а~на вяршыні піраміды~--- высокаінтэнсіўная кароткатэрміновая актыўнасьць (спрынт, кросфіт, сілавыя і да т.~п. 1--3 гадзіны на тыдзень).

\subsection{Спалучэньне фізычнай актыўнасьці ды іншых вэктараў}

\paragraph{Сілкаваньне і прынцып «больш~--- менш».}
Больш калёрыяў у~дні інтэнсіўных трэніровак. Калі менш фізычнай актыўнасьці~--- то і менш калёрыяў (прыёмаў ежы). Больш нізкаінтэнсіўнай актыўнасьці пры малой колькасьці вугляводаў (і высокім тлушчаў). Больш вугляводаў пры большай высокаінтэнсіўнай фізычнай актыўнасьці. Падвышаны ўзровень трэніровак абавязкова трэба кампэнсаваць павелічэньнем калярыйнасьці, прынцып «менш ясі~--- больш трэніруесься» небясьпечны для здароўя!

\paragraph{Час дня.}
Для прасунутай групы дапушчальныя кароткія трэніроўкі на пусты страўнік для спаленьня тлушчу (мала глікагену\index{глікаген} ў~печані, высокі картызол\index{картызол}). А~вось для набору цягліцаў лепей выбіраць час увечары, трэніруючыся пасьля апошняга прыёму ежы (нізкі картызол\index{картызол}, высокі гармон росту без дэфіцыту калёрыяў). Інтэнсіўныя сілавыя трэніроўкі на пусты страўнік не рэкамэндуюцца!

\paragraph{Стрэс.}
Максымум нагрузкі даваць у~дні рэфідаў і пры малым узроўні фізычнага і псыхалягічнага стрэсу, у~дні мінімальнай стомленасьці. Мінімум нагрузкі~--- у~дні стрэсу, стомленасьці або харчовага ўстрыманьня (посту).

\paragraph{Рэжым актыўнасьці і адпачынак} (чаргаваньне «стрэс»~--- «сон, адпачынак»).
Для аптымальнага здароўя нам трэба больш карысных вострых стрэсаў, зьніжэньне ўзроўню хранічнага стрэсу і дастатковае аднаўленьне (рэляксацыя, адпачынак, сон). Востры карысны стрэс можа паляпшаць тлушчаспаленьне за кошт адрэналіну і сымпацыйнай вэгетатыўнай сыстэмы, а~вось хранічны стрэс стымулюе адклад тлушчу і зьніжэньне адчувальнасьці да інсуліну\index{інсулін} празь дзеяньне картызолу\index{картызол}. Пры хранічным стрэсе і няўпэўненасьці ў~будучыні чалавек схільны зьядаць больш «у запас». Тэорыя «харчовай няўпэўненасьці» сьцьвярджае, што менавіта стрэс і сацыяльны статус уплываюць на ўзровень апэтыту. Пры высокім стрэсе ня раю выкарыстоўваць сур'ёзныя забароны і выяўныя абмежаваньні, высокаінтэнсіўныя трэніроўкі, бо яны могуць яшчэ больш павысіць агульны ўзровень стрэсу і прывесьці да зрыву. Для фастынгу лепей выбіраць найменш стрэсавыя дні.

\paragraph{Сьвятло і мэтабалічная гнуткасьць.}
Сьвятло аказвае наўпроставае ўзьдзеяньне на галоўны цэнтар кіраваньня цыркаднымі рытмамі\index{цыркадныя рытмы} ў~мозгу~--- супрахіязматычнае ядро\index{супрахіязматычнае ядро} гіпаталямусу\index{гіпаталямус}. Ёсьць 4 асноўныя правілы нармалізацыі сьветлавога рэжыму, зьвязаныя з~часам дня. Такім чынам, ранішняе сьвятло~--- вельмі важнае, нават паўгадзіны яркага сьвятла дапамогуць узбадзёрыцца, павысіць адчувальнасьць да інсуліну\index{інсулін}, палепшыць сон. Найлепшае сьвятло~--- сонечнае з~ультрафіялетам. Калі раніцай у~вас сьвятла мала~--- тае бяды: можна выкарыстоўваць прылады для фотатэрапіі, пажадана з~добрым спэктрам і яркасьцю ня менш за 10 000 люкс.

Днём таксама важна бываць на вуліцы з~розных прычын: чым болей дзённага сьвятла, тым вышэйшы ўзровень сэратаніну\index{сэратанін}, а~значыць, і самаадчуваньня, яркае сьвятло прытупляе апэтыт, сонечнае сьвятло спрыяе назапашваньню вітаміну D. Аднак улетку і ў~гарачых краінах неабходна датрымлівацца і фотаабароны для прадухіленьня пашкоджаньня скуры. Увечары варта спыніцца на няяркім расьсеяным жаўтлявым сьвятле, ідэальна ад лямпаў напальваньня. Сьвятлодыёдныя крыніцы (смартфоны, тэлевізары, лямпы) варта абмежаваць. Як варыянт~--- можна выкарыстоўваць e-ink-рыдары для чытаньня электронных кніг або акуляры, якія блякуюць сінюю частку спэктру. Уначы~--- поўная цемра без кампрамісаў. Яна можа быць дасягнутая маскай для сну, шчыльнымі шторамі, заклейваньнем усіх магчымых індыкатараў і сьвятлодыёдаў.

\tipbox{Тэмпэратура ўплывае на якасьць сну: у~прахалоднай спальні сон глыбейшы. Дасьледаваньні высьветлілі, што 2 гадзіны ў~дзень пры тэмпэратуры 17\,°C прыводзяць да страты дадатковых 110 ккал цягам тыдня, а~да канца другога тыдня ўдзельнікі губляюць ужо 290 ккал! 10-хвіліннае знаходжаньне на холадзе штодня прыводзіць да страты 3--4 кіляграмаў на месяц.}

\paragraph{Тэмпэратура і мэтабалічная гнуткасьць.}
Халадовае ўздзеяньне (цыкль «цяпло~--- холад»)~--- гэта эфэктыўны і вельмі недаацэнены спосаб узмацніць сваё здароўе. Наша цела захоўвае цыклічныя ваганьні тэмпэратуры: раніцай тэмпэратура цела павышаецца, днём падтрымліваецца, увечары зьніжаецца, ноччу мінімальная. Усе халадовыя працэдуры стымулююць работу бурага і белага тлушчаў, у~якіх адбываецца расшчапленьне зьмесьціва клетак для выпрацоўкі цяпла. Актывацыя бурага і белага тлушчаў прыводзіць ня толькі да пахудзеньня, але яшчэ мае шэраг карысных уласьцівасьцяў, такіх як зьніжэньне запаленьня, паляпшэньне адчувальнасьці да інсуліну\index{інсулін} і г.~д. Пачынаць можна з~элемэнтарнай паветранай ванны, затым працягнуць кантрасным (халодным) душам, трэніроўкамі ў~лёгкай спартовай вопратцы, паніжэньнем тэмпэратуры ўдома, сном з~прыадчыненым вакном і да т.~п. Але да ледзяной вады прыстасавацца нельга, таму экстрэмальныя маржаваньні ці наўрад прынясуць шмат карысьці.

Тут пасуе адзеньне, якое можна апісаць формулай «цёпла рухацца, холадна стаяць». Тэмпэратура ўплывае на якасьць сну, у~прахалоднай спальні сон глыбейшы. Дасьледаваньні давялі, што 2 гадзіны на дзень пры тэмпэратуры 17\,°C прыводзяць да страты дадатковых 110\,ккал на працягу тыдня, а~да канца другога тыдня ўдзельнікі губляюць ужо 290\,ккал! 10-хвіліннае знаходжаньне на холадзе штодня прыводзіць да страты 3--4 кіляграмаў на месяц.

\subsection{Як трымацца правіла? Ідэі і парады}

\paragraph{Рух.}
Нават невялікая актыўнасьць, 5 хвілінаў кожную гадзіну, прыкметна аслабляе апэтыт, зьніжае ўзровень стомленасьці і павялічвае настрой да канца працоўнага дня. Рухайцеся ўсюды і шукайце для гэтага магчымасьці!

\paragraph{Праца стоячы.}
Праца стоячы~--- гэта выдатны спосаб дадаць рух, ня рухаючыся. Калі мы стаім, то цягліцы, якія ўтрымліваюць нас у~раўнавазе, актыўныя і спальваюць калёрыі. А~акрамя таго, праца стоячы прыкметна паляпшае нашу прадуктыўнасьць.

\paragraph{Карысныя стрэсы.}
Даданьне карысных стрэсаў (прыемная паездка, саўна, незвычайны спорт) выклікае выкід адрэналіну і паляпшае ваша самаадчуваньне.

\paragraph{Прадукты для тлушчаспаленьня.}
Ёсьць прадукты, якія спрыяюць большаму тлушчаспаленьню на холадзе і актывуюць буры тлушч. Да іх можна аднесьці аліўкавы алей, каратыноіды (морква, таматы), амэга-3 тлустыя кіслоты, чырвоны перац, чорны перац, гарчыцу, хрэн, часнык, гваздзік, імберац. Дадавайце больш гэтых прадуктаў і спэцыяў для бясьпечнага гартаваньня.

\paragraph{Тэмпэратура цела.}
Днём карысная высокая тэмпэратура (у межах нормы), яна выдатна стымулюе і павялічвае працаздольнасьць. Таму прысяданьні ўзбадзёраць вас ня горш за кафэін. Калі вы мерзьнеце ці кепска трываеце сьпякоту, варта праверыць працу шчытавіцы, а~таксама дэфіцыт амэга-3 тлустых кіслотаў.

\paragraph{Радасьць і тлушчаспаленьне.}
Пазытыўныя эмоцыі, задавальненьне павялічваюць узровень дафаміну\index{дафамін} і паляпшаюць адчувальнасьць да інсуліну\index{інсулін}. Проста так адмовіцца ад задавальненьняў высокакалярыйнай ежы цяжка, таму знайдзіце сабе як мага болей іншых крыніц задавальненьня і радасьці! Новае хобі, натхняльныя пляны, прыемныя знаёмствы дапамогуць зьнізіць апэтыт.

\paragraph{Стрэс мяняе цела.}
Цікава, што пры стрэсе зьмяняецца сам тып адкладу тлушчу, ён больш інтэнсіўна адкладаецца ўнутры жывата (вісцэральны тлушч\index{вісцэральны тлушч}), на баках («выратавальны пояс»), на грудзях, на твары і ў~«тлушчавых пастках». Без кантролю стрэсу складана дасягнуць доўгатэрміновых мэтаў у~харчаваньні.

\paragraph{Гульнявыя віды спорту.}
Пры хранічным стрэсе эфэктыўнае даданьне гульнявых відаў спорту (тэніс, футбол, баскетбол, сквош і інш.), танцаў і танцавальнага фітнэсу. Яны даюць выдатную папуску ад стомленасьці і зараджаюць энэргіяй, а~таксама міжволі залучаюць у~актыўнасьць.

\tipbox{Пры стрэсе зьмяняецца сам тып адкладу тлушчу, ён больш інтэнсіўна адкладаецца ўнутры жывата (вісцэральная тлушч\index{вісцэральны тлушч}), на баках («выратавальны пояс»), на грудзях, на твары і ў~«тлушчавых пастках». Без кантролю стрэсу складана дасягнуць доўгатэрміновых мэтаў у~харчаваньні.}

\paragraph{Паравіны году і мэтабалічная гнуткасьць.}
Цалкам натуральна ўлетку, калі болей сонца, ужываць больш вугляводаў і абмяжоўваць іх узімку. Што тычыцца трэніровак, зімой можна есьці і трэніравацца менш, з~позьняй вясны і да сярэдзіны восені~--- трэніравацца і есьці больш. А~вось каб згладзіць пераходы, увесну~--- больш актыўнасьці і крыху менш ежы, а~ўвосень~--- наадварот.


\part{Асноўныя правілы выбару прадуктаў}

\chapter{Цэльныя прадукты}

Правіла цэльных прадуктаў заключаецца ў~тым, каб пераважна купляць і есьці тое, што вырасла само, а~не зьяўляецца камбінацыяй з~парашкападобных сумесяў і дабавак (харчовыя рэчывы). Гэтае правіла вельмі простае, але зьяўляецца базавым у~выбары прадуктаў. Значная колькасьць нашых праблемаў са здароўем і самаадчуваньнем зьвязаная з~тым, што мы ямо мала «сапраўднай ежы», аддаём перавагу імітацыям прадуктаў і харчовым рэчывам. Вельмі часта людзі трапляюць у~маркетынгавую пастку «палепшанай ежы»: тварагу бяз тлушчу, сьвежавыціснутага соку бязь мякаці, «палепшаных» расьлінных алеяў. Я часам бачу, што людзі з~розных прычынаў зусім пазьбягаюць есьці цэльную ежу: яны ядуць толькі перапрацаваную, прычым вельмі глыбока. Гэта і шматкі, гатовыя замарожаныя паўфабрыкаты, мучныя вырабы. Вядома, няма нічога небясьпечнага ў~тым, каб часам зьесьці й~гатовую ежу, але, калі яна пачынае складаць большую частку рацыёну, могуць узьнікнуць праблемы са здароўем.

\tipbox{Значная колькасьць нашых праблемаў са здароўем і самаадчуваньнем зьвязаная з~тым, што мы ямо мала «сапраўднай», цэльнай ежы, аддаём перавагу імітацыям прадуктаў і харчовым рэчывам. Вельмі часта людзі трапляюць у~маркетынгавую пастку «палепшанай ежы»: тварагу бяз тлушчу, сьвежавыціснутага соку бязь мякаці, «палепшаных» расьлінных алеяў.}

Сапраўдная ежа~--- гэта расьліны і жывёлы, якія існуюць у~сьвеце вакол нас, прыгатаваныя з~умеранай і зберагалай апрацоўкай (цэльныя прадукты): рыба, грэцкі арэх, буракі, чарніцы.

Харчовыя рэчывы~--- гэта натуральныя або штучныя (розьніцы няма) субстанцыі, што зазвычай маюць кшталт парашкоў, вадкасьцяў, алеяў, высокай ступені прамысловай перапрацоўкі, звычайна высокаачышчаныя і высокаканцэнтраваныя. Прыклады: цукар, соль, мука, яечны бялок, арахісавы парашок, соевы пратэін, сухое малако, сурымі, сухі бульбяны крухмал, солад і інш. Паасобку вы наўрад ці зможаце зьесьці шмат кожнага з~гэтых кампанэнтаў, але, зьмяшаныя разам у~пэўных прапорцыях, яны могуць узламаць вашу сыстэму насычэньня й~мэтабалізм і стымуляваць пераяданьне.

\subsection{Як зьявілася праблема?}

Склад цэльных прадуктаў утварыўся паступова, пры працяглым эвалюцыйным узаемадзеяньні расьлінаў і жывёлаў. Нашыя целы аптымальна прыстасаваныя ўзаемадзейнічаць з~цэльнай ежай, таму прадукты~--- гэта больш чым мэханічная сума асобных нутрыентаў, гэта яшчэ іх складаная арганізацыя і велізарная колькасьць біялягічна актыўных рэчываў, якія немагчыма сабраць цалкам на заводзе. У~працэсе эвалюцыі разьвіцьцё жывёльнага і расьліннага сьветаў ішло ўзаемазьвязана. І~мы зьвязаныя з~расьлінамі й~жывёламі тысячамі сувязяў і маем нашмат больш агульнага, чым нам здаецца.

Наш абмен рэчываў і стрававальная сыстэма адаптаваныя да пэўных прадуктаў, разам зь якімі адбывалася наша эвалюцыя. Менавіта таму цела чалавека генэтычна адаптаванае для спажываньня сапраўднай ежы, а~не высокаачышчаных рэчываў. Сёньня мы ямо занадта шмат прадуктаў, вырабленых з~харчовых рэчываў: мучныя вырабы, піца, дэсерты, хлеб, сасіскі, газіроўка, супы хуткага прыгатаваньня, кансэрвы і інш.

\paragraph{Імітацыя сапраўдных прадуктаў.}
Імітацыя сапраўднай ежы~--- гэта не сапраўдная ежа, хоць і вельмі да яе падобная. Яна зробленая шляхам камбінаваньня вычышчаных харчовых рэчываў. Напрыклад, хлеб зь вялікай колькасьцю цукру і солі, штучна зроблены сыр ці ёгурт з~сухога малака з~высокай канцэнтрацыяй цукру і сыр натуральнага высьпяваньня з~бактэрыямі, сапраўднае сьметанковае масла ці спрэд на аснове транстлушчаў\index{транстлушчы}.

Цяпер зьявілася магчымасьць мадыфікаваць прадукты харчаваньня. Вядома, у~некаторых выпадках гэта сапраўды карысна для здароўя (даданьне ёду ў~соль, ГМА-рыс зь вітамінам а~для прафіляктыкі сьлепаты, чорныя таматы з~антацыянамі і да т.~п.). Але, на жаль, часьцей мадыфікацыі маюць камэрцыйную мэту прадаць вам больш і прымусіць зьесьці больш менавіта гэтага прадукту, схаваць недахопы, патаньніць вытворчасьць, павялічыць устойлівасьць да пэстыцыдаў, працягнуць тэрмін захоўваньня. А~такія мэты маюць наўвеце даданьне вялікай колькасьці інгрэдыентаў кшталту араматызатараў, фарбавальнікаў, цукру, тлушчу і да т.~п. Гэта вядзе да пераяданьня і адпаведных рызыкаў для здароўя.

\subsection{Як гэта ўплывае на здароўе?}

\paragraph{Харчовыя паводзіны.}
Нашыя органы пачуцьцяў могуць правільна ацэньваць уласьцівасьці толькі тых прадуктаў, да якіх мы прыстасаваныя. Так, мы без праблемаў адрозьніваем сасьпелы і нясьпелы яблык, якаснае ці сапсаванае мяса і таму можам верыць сваім адчуваньням, калі ямо сапраўдную ежу. Харчовыя рэчывы зьмяшчаюць такія камбінацыі рэчываў, якія рэдка сустракаюцца ў~прыродзе (тлушч, цукар і соль у~адным прадукце), ня кажучы ўжо пра араматызацыю і ўзмацненьне смаку. Такая ежа паведамляе фальшывую інфармацыю нашым органам пачуцьцяў і парушае харчовыя паводзіны, узмацняе апэтыт і блякуе насычэньне. Падобныя камбінацыі высокаканцэнтраваных цукраў, тлушчаў, солі і ўзмацняльнікаў смаку не існуюць у~прыродзе і дзейнічаюць разбуральна на нашыя харчовыя паводзіны. Ад такой ежы лёгка фармуецца залежнасьць, і ва ўмовах стрэсу мы пачынаем ужываць яе зь лішкам.

\paragraph{Пераяданьне.}
Адчуць насычэньне харчовымі рэчывамі практычна немагчыма, гэтак жа як і зьесьці іх у~меру. Хачу зьвярнуць вашую ўвагу, што рэч нават ня ў~колькасьці калёрыяў~--- харчовыя рэчывы парушаюць харчовыя паводзіны нават пры спажываньні звычайнай колькасьці калёрыяў. Паасобку вы не зьясьцё шмат цукру, тлушчу ці солі, але калі іх зьмяшаць, то вашым органам пачуцьцяў будзе цяжка выстаяць. І, вядома, важная асаблівасьць харчовых рэчываў~--- гэта неймаверная калярыйнасьць у~невялікіх памерах (высокая ўдзельная калярыйнасьць). У~шматлікіх гатовых прадуктах вялікая колькасьць цукру хаваецца за кіслотамі, таму вы неўзаметку можаце зьесьці неймаверную колькасьць цукру ў~кіслым ці вострым соўсе ці выпіць у~выглядзе напою, зьесьці чыстым такую ж колькасьць цукру было б вельмі складана.

\tipbox{Нашы органы пачуцьцяў могуць правільна ацэньваць уласьцівасьці толькі тых прадуктаў, да якіх мы прыстасаваныя. Харчовыя рэчывы зьмяшчаюць такія камбінацыі рэчываў, якія рэдка сустракаюцца ў~прыродзе (тлушч, цукар і соль у~адным прадукце), не гаворачы ўжо пра араматызацыю і ўзмацненьне смаку. Такая ежа паведамляе фальшывую інфармацыю нашым органам пачуцьцяў і парушае нашыя харчовыя паводзіны, узмацняе апэтыт і блякуе насычэньне.}

\paragraph{Складаная ежа.}
Пры пэўным вонкавым падабенстве сапраўдная ежа і харчовыя рэчывы маюць сур'ёзныя адрозьненьні нават у~засваеньні. Так, усё жывое складаецца з~клетак, у~якіх аптымальна спакаваныя харчовыя рэчывы і якія руйнуюцца пры страваваньні паступова. Сапраўдная ежа патрабуе больш увагі і намаганьняў у~працэсе сталаваньня (дарослая ежа): уменьня даставаць косткі, разьбіраць яе і карыстацца інструмэнтамі (рыба, гарэхі, малюскі), а~таксама дае больш нагрузкі на жавальную сыстэму (параўнайце мяса і фарш, цэльныя гарэхі і арэхавую пасту). Складанасьць ежы гарантуе больш часу для сталаваньня, больш увагі і ўсьвядомленасьць у~час яды. Харчовыя рэчывы лёгка адкусваюцца, адломліваюцца і распадаюцца да вадкасьці без асаблівых намаганьняў жаваньня, што правакуе пераяданьне і аўтаматычную яду. Нават калі назва прадукту застаецца ранейшай і мы ў~яго нічога не дадаём, усё роўна перапрацоўка істотна зьмяняе яго ўласьцівасьці. Так, чым глыбейшая перапрацоўка аўсяных шматкоў, тым мацней павялічваецца іх глікеміческій індэкс (ступень уплыву на ваганьні цукру ў~крыві). Шматкі, якія патрабуюць варкі, маюць індэкс 40, а~тыя, што трэба адно заліць кіпнем,~--- 80. Так, крухмал застаецца крухмалам, але мяняецца даўжыня яго ланцужкоў.

\paragraph{Бясьпека і карысьць.}
Пры глыбокай перапрацоўцы харчовых рэчываў адбываецца страта шматлікіх карысных некалярыйных кампанэнтаў (мінэралы, вітаміны), акісьленьне нутрыентаў (амінакіслоты, поліненасычаныя тлустыя кіслоты) пры павышанай тэмпэратуры і інтэнсіўным кантакце з~кіслародам паветра, для паляпшэньня вонкавага выгляду прымяняюцца таксычныя хімічныя рэагенты (адбельваньне мукі і г.~д.), хаваюцца дэфэкты вытворчасьці (чым глыбейшая перапрацоўка, тым вышэйшая рызыка забруджваньня таксынамі). У~цэльным прадукце ёсьць лупіна, плёнкі, перагародкі, клеткавыя мэмбраны, свае антыаксыданты\index{антыаксыданты}, якія абараняюць нутрыенты ад акісьленьня. Менавіта таму экстракты прадуктаў патэнцыйна маюць горшую якасьць, чым цэльныя прадукты, і ня могуць іх цалкам замяніць. Навуковыя дасьледаваньні паказваюць, што харчовыя рэчывы значна больш небясьпечныя для нашага здароўя. Так, перапрацаванае мяса больш павялічвае рызыку раку, чым цэльнае. А~сьвежавыціснутыя сокі~--- рызыку дыябэту, нашмат больш, чым цэльная садавіна.

\paragraph{Харчовыя дабаўкі.}
Нягледзячы на тое, што большасьць дабавак у~цэлым адносна бясьпечныя для здароўя, сярод іх ёсьць шэраг рэчываў, якія нэгатыўна ўплываюць на яго. Акрамя гэтага, дастаткова складана прадказаць іх сумеснае дзеяньне, патэнцыяваньне эфэктаў можа прыводзіць да непрадказальных і непажаданых наступстваў.

\tipbox{У цэльным прадукце ёсьць лупіна, плёнкі, перагародкі, клеткавыя мэмбраны, свае антыаксыданты\index{антыаксыданты}, якія абараняюць нутрыенты ад акісьленьня. Менавіта таму экстракты прадуктаў патэнцыйна маюць горшую якасьць, чым цэльныя, і ня могуць іх замяніць.}

\subsection{Асноўныя прынцыпы}

\paragraph{Правіла 80\,\%.}
80\,\% свайго рацыёну складаць з~цэльных прадуктаў, пакідаючы 20\,\% на свой густ. Аддаваць перавагу сапраўднай ежы~--- гэта вельмі простая і выйгрышная стратэгія харчаваньня. Для пачаткоўцаў~--- хай гэта будзе 50\,\% цэльных прадуктаў, у~ідэале~--- 80\,\%. Вядома, можна і больш, але гэта можа парушаць вашую гнуткасьць і сацыяльную адаптацыю.

\paragraph{Прыклады.}
Давайце разьбяром пары «сапраўдны прадукт»~--- «харчовае рэчыва»: цэльнае мяса~--- сасіскі (перапрацаванае мяса), яблыкі~--- яблычны сок, трускаўка~--- клубнічнае жэле, часнык~--- часнычны марынад, памідор~--- кетчуп. Суцэльны авёс~--- цукровыя аўсяныя шматкі хуткага прыгатаваньня. У~такой справе важна абыходзіцца бяз крайнасьцяў. У~рэальнасьці мы сутыкаемся зь вялікай колькасьцю прамежкавых кампанэнтаў, і ня ўсе яны адназначна кепскія. Вось пара: сьвежая рыба~--- рыбная мука (сурымі). Насамрэч тут больш прамежкавых зьвёнаў: сьвежая рыба~--- рыба астуджаная~--- цэльная рыба глыбокай замарозкі~--- растрыбушаная тушка глыбокай замарозкі~--- кавалкі рыбы з~косткамі і скурай~--- філе бяз скуры і костак~--- кансэрвы ва ўласным соку~--- рыбны фарш~--- паўфабрыкаты з~рыбнага фаршу~--- сурымі. Варта аддаваць перавагу верхняму полюсу, выбіраючы даступныя вам варыянты на дадзены момант часу. Розьніца паміж сырой і сьвежазамарожанай рыбай не такая ўжо вялікая, а~вось паміж сурымі і цэльнай рыбай~--- вельмі істотная. Цалкам вартымі варыянтамі будуць сьвежазамарожаная зеляніна і таматная паста (з таматаў, без даданьня цукру і крухмалу).

\paragraph{Сьвежыя прадукты.}
Сьвежасьць зьяўляецца важнай умовай карысьці і бясьпекі прадуктаў, але многія цэльныя прадукты маюць нядоўгі тэрмін захоўваньня. Пры працяглым захоўваньні ў~бялковых прадуктах адбываецца распад амінакіслотаў да біягенных амінаў\index{аміны!біягенныя}, што пагаршае якасьць ежы і можа выклікаць непажаданую рэакцыю. Тлушчы пры працяглым захоўваньні пачынаюць акісьляцца, есьці такія акісьленыя тлушчы вельмі шкодна. Вугляводы захоўваюцца лепш, але садавіна і гародніна працяглага захоўваньня заўважна губляюць свае карысныя ўласьцівасьці. Сьвежую гародніну, якую прывозяцца здалёк, часта зрываюць яшчэ нясьпелай і зь нізкім утрыманьнем карысных кампанэнтаў. Менавіта таму глыбокая замарозка~--- гэта ідэальны варыянт захоўваньня шматлікіх прадуктаў у~цэльным выглядзе, большасьць сваіх карысных уласьцівасьцяў прадукты пры ёй захоўваюць. Заўсёды выбірайце найболей сьвежыя прадукты, навучыцеся адрозьніваць іх вонкава.

\paragraph{Правіла «маладой» ежы.}
Сута гэтага правіла вельмі простая: імкнуцца выбіраць у~якасьці ежы маладзейшыя расьліны (маладое лісьце і парасткі, пупышкі, верхавіны), маладзейшую рыбу і мяса. Вядома, «маладыя» прадукты харчаваньня здаўна шанаваліся як больш лёгказасваяльныя і пажыўныя, але навуковае абгрунтаваньне гэтаму ўдалося атрымаць нядаўна. Навукоўцы давялі, што «маладая» дыета падаўжае жыцьцё, а~«старая» скарачае жыцьцё ў~параўнаньні з~«маладой». У~маладых уцёках чайнага куста відавочна больш карысных рэчываў. А~вось максімальная канцэнтрацыя сульфарафану\index{сульфарафан} ўтрымліваецца менавіта ў~парастках брокалі. Што да шкодных спалучэньняў, то ў~маладых прадуктах іх меней. Так, у~буйной і дарослай рыбе ртуці больш, чым у~дробнай і маладой.

\tipbox{Пры працяглым захоўваньні ў~бялковых прадуктах адбываецца распад амінакіслотаў да біягенных амінаў\index{аміны!біягенныя}, што пагаршае якасьць ежы. Тлушчы пры працяглым захоўваньні пачынаюць акісьляцца, есьці іх вельмі шкодна. Вугляводы захоўваюцца лепей, але садавіна і гародніна працяглага захоўваньня відавочна губляюць свае карысныя ўласьцівасьці.}

\subsection{Як трымацца правіла? Ідэі і парады}

\paragraph{Смакавыя рэцэптары.}
Калі вы звыклі есьці шмат салодкага, салёнага і да т.~п., то простыя цэльныя прадукты пададуцца вам спачатку нясмачнымі і прэснымі. Ня бойцеся, так будзе не заўсёды. За некалькі тыдняў вашыя смакавыя рэцэптары вернуць сабе адчувальнасьць і ежа здабудзе яшчэ больш адценьняў і смакаў.

\paragraph{Пачніце з~простага.}
Выберыце самыя смачныя цэльныя прадукты (гарэхі, гародніна і да т.~п.), якія вы можаце лёгка гатаваць і пазнаваць, пачніце калекцыянаваць рэцэпты. Больш складаную ежу, такую як морапрадукты ці грыбы, дадавайце пазьней. Няхай першыя рэцэпты будуць простымі і хуткімі.

\paragraph{Стварыце запас дома.}
Памятайце, што можна ствараць запас і сапраўдных прадуктаў: яны добра захоўваюцца дома як у~сьвежазамарожаным выглядзе, так і ў~цёмных шафах у~прахалодным месцы. Сапраўдныя прадукты~--- гэта не заўсёды ўтомная і доўгая гатоўля, іх можна прыгатаваць хутка. Прыклад хуткага сьняданку з~маразільні: спаржавая фасоля і цэльны ласось. Перакладзіце іх з~маразільні ў~лядоўню ўвечары, а~раніцай прамыйце вадой і зварыце разам за 10 хвілінаў~--- будзе хутка і смачна.

\paragraph{Паступовы пераход.}
Дайце кішачніку адаптавацца да клятчаткі\index{клятчатка}. Рэзкі пераход можа выклікаць пэўныя праблемы з~кішачнікам. Ня бойцеся іх, яны неўзабаве зьнікнуць, але клятчатку\index{клятчатка} варта павялічваць у~рацыёне паступова.

\paragraph{Мясцовыя прадукты.}
Аддавайце па магчымасьці перавагу мясцовым (ці блізкім да іх) сэзонным прадуктам. Яны мінімальна перавозяцца і захоўваюцца. Вы можаце дасьледаваць лякальныя рынкі, уведаць добрых пастаўнікоў.

\paragraph{«Экстрагаваныя прадукты».}
Па магчымасьці старайцеся папоўніць умераны дэфіцыт не з~дапамогай БАДаў, а~ежай. Так, рыба зьніжае рызыку хваробы Альцгаймэра\index{хвароба!Альцгаймэра}, а~паасобку вітамін D або амэга-3~--- не, клятчатка\index{клятчатка} ў~складзе садавіны зьніжае рызыку раку кутніцы (кутняй кішкі), а~вось яна ж праз дабаўкі~--- не зьніжае. Зразумела, сур'ёзны дэфіцыт ежай скампэнсаваць цяжэй, тады ўжо можна зьвярнуцца і да таблетак. Але стратэгія «БАДы жменяй замест нармальнай ежы» ня можа быць названая здаровай.

\paragraph{Арэол здаровага харчаваньня.}
Гэта маркетынгавая пастка, калі цэльная здаровая ежа спалучаецца зь ня вельмі карыснай. Напрыклад, лісток салаты на піцы, даданьне гарэхаў у~марозіва, даданьне кунжутных семак у~батон. Гэтыя мікраскапічныя дабаўкі ня зьменяць дзеяньне асноўнай стравы, і, вядома ж, пэктын\index{пэктын} у~зэфіры ані не заменіць яблыкаў і не адменіць дзеяньне цукру.

\paragraph{«Натуральныя інгрэдыенты».}
Натуральнасьць зусім не зьяўляецца закладам здароўя. Цукар у~натуральным соку дзейнічае гэтак жа, як і ў~газіроўцы, і маецца ў~фруктовых соках у~супастаўнай колькасьці. У~некаторых краінах сэртыфікацыя «арганік» можа нешта значыць, у~шматлікіх іншых гэтае слова~--- толькі маркетынгавы крок, роўна як і словы «бія», «эка» і да т.~п.

\tipbox{Лісток салаты на піцы, даданьне гарэхаў у~марозіва, даданьне кунжутных семак у~батон~--- усё гэта маркетынгавыя пасткі. Гэтыя мікраскапічныя дабаўкі ня зьменяць дзеяньне асноўнай стравы.}

\paragraph{Здаровы глузд.}
Вядома ж, даданьне харчовых дабавак не псуе ежу, а~часта і абараняе яе, многія зь іх бясшкодныя і нават карысныя. Таму шануйце здаровы глузд і ня кідайцеся ў~скрайнасьці.

\paragraph{Палеадыета.}
Сэнс дыеты ў~пазьбяганьні «гатовых сучасных» прадуктаў, уключна са збажыною, бабовымі, расьліннымі тлушчамі і малочнымі прадуктамі. Плюс гэтай дыеты ў~тым, што вы пачынаеце есьці больш цэльных прадуктаў, мінус у~тым, што толкам ніхто ня ведае, як менавіта елі нашыя продкі. Дасьледаваньні паказваюць, што нашыя продкі былі ўсяеднымі, таму елі і збажыну, ды і ўсё астатняе, што трапляла пад руку. Мне падабаецца ў~палеападыходзе ўлік нашых генэтычных і эвалюцыйных асаблівасьцяў, што важна для разуменьня таго, чаму нам нешта пасуе, а~нешта~--- не. Бо кантэкст эвалюцыі вызначае наш аптымальны выбар.


Правіла 14. Калярыйная і нутрыентная шчыльнасьць ежы

Правіла шчыльнасьці заключаецца ў тым, каб выбіраць сабе ежу зь нізкай удзельнай калярыйнай шчыльнасьцю і з высокай шчыльнасьцю нутрыентаў (клятчатка, вітаміны, антыаксыданты, макра- і мікраэлемэнты). Калярыйная шчыльнасьць (энэргетычная шчыльнасьць) – гэта колькасьць калёрыяў на грам прадукту, чым вышэйшвя шчыльнасьць, тым меншы аб'ём прадукту і вышэйшая колькасьць калёрыяў у ім. Так, напрыклад, гародніна ўтрымоўвае 30 ккал на 100 грамаў прадукту, а вось чакалядка – да 550 ккал на 100 грамаў прадукту.

Можна ўявіць сабе сытуацыю і па-іншаму: так, 400 ккал – гэта 100 грамаў чакаляднага батончыка, амаль 400 грамаў мяса і амаль паўтара кіляграмы брокалі. Менавіта таму сьцьвярджэньні аб тым, што калёрыі цалкам ідэнтычныя, няслушныя: бо 400 ккал батончыка і 400 ккал брокалі будуць па-рознаму ўплываць на пачуцьцё голаду і сытасьці, а гэта шмат у чым вызначае вашы харчовыя паводзіны. Чым вышэйшая калярыйная шчыльнасьць, тым вышэйшая ступень ачысткі прадукту і тым меншая яго нутрыентная шчыльнасьць. Ежце як кароль – толькі багатыя прадукты. «Бедныя» прадукты ўтрымліваюць шмат калёрыяў і мала карысных рэчываў. «Багатыя» прадукты ўтрымліваюць мала калёрыяў, але вельмі шмат біялягічна актыўных рэчываў.

Як зьявілася праблема?

З даўніх часоў у прыродзе чалавек рэдка сустракаўся з багацьцем ежы высокай шчыльнасьці, яна была эпізадычнай ці сэзоннай. Зь цягам часу калярыйная шчыльнасьць ежы пакрысе павялічваецца. Так, узровень тлушчу ў дзікіх жывёл – 5-7% і дасягае 30% у сельскагаспадарчых, тое ж тычыцца многіх расьлінных культураў. Зьяўленьне рафінаваных прадуктаў прывяло да таго, што ежа стала неймаверна калярыйнай, ачыстка прадуктаў прывяла да зьніжэньня іх біялягічнай вартасьці.
Павелічэньне калярыйнай шчыльнасьці часьцей за ўсё суправаджаецца зьніжэньнем нутрыентнай шчыльнасьці ежы (гэта значыць зьмесьціва ў ёй розных некалярыйных, але важных для здароўя прадуктаў, уключаючы клятчатку, вітаміны, антыаксыданты, рэчывы, якія адказваюць за смак і пах, і многае іншае). Таму ежа з высокай калярыйнай шчыльнасьцю, як правіла, бедная на біялягічна актыўныя спалучэньні. Акрамя гэтага, працяглае вырошчваньне ежы на адных і тых жа землях зьнясільвае глебу, зьніжае ўтрыманьне ў ёй шэрагу карысных кампанэнтаў; працяглае захоўваньне і транспарціроўка таксама ўзмацняюць гэты працэс.

У старажытнасьці некаторымі хваробамі ганарыліся, лічылі іх прыкметай вытанчанасьці. Так, падагру называлі хваробай каралёў, бо толькі багатым людзям была даступная такая колькасьць бялковай ежы і алькаголю на працягу доўгага часу ў спалучэньні з маларухомым ладам жыцьця.

Многія «хваробы цывілізацыі», распаўсюджаныя сёньня (алергіі, атлусьценьне, дыябэт, дэпрэсія), у старажытнасьці былі толькі ў некаторых людзей з высокім сацыяльным статусам і даходам, а дакладней – зь няправільным харчаваньнем, гіпадынаміяй і нястрымнасьцю. Больш за тое, некаторымі хваробамі нават ганарыліся, лічачы іх прыкметай вытанчанасьці. Так, падагру называлі хваробай каралёў, паколькі толькі багатым людзям была даступная такая колькасьць бялковай ежы і алькаголю на працягу доўгага часу ў спалучэньні з маларухомым ладам жыцьця. Гэта ж датычыцца і цукровага дыябэту другога тыпу, які сустракаўся, як правіла, толькі ў багатых. Алергія першапачаткова таксама была прыкметай заможнасьці. Сянны катар быў «моднай» хваробай эліты. Лекары адразу заўважылі, што хвароба сустракаецца ў багатых, а ў сялян яе не назіралася. Чаму? Багатыя больш пераядалі, елі больш тлустага, мучнога, салодкага, менш рухаліся, таму і часьцей хварэлі. Цяпер сытуацыя поўнасьцю перакулілася. Калярыйная бедная ежа (мучное, тлустае, цукар) стала таннай, а вось "багатыя" прадукты (морапрадукты, зеляніна, якаснае мяса, ягады і гародніна) сталі даражэйшымі.

Як гэта ўплывае на здароўе?

Пераяданьне.
Дасьледаваньні паказваюць, што чым вышэйшая калярыйная шчыльнасьць ежы, тым больш ежы чалавек зьядае. Зьніжэньне шчыльнасьці ежы прыводзіць да таго, што чалавек аўтаматычна пачынае есьці менш на 300-400 ккал. Навукоўцы давялі, што тыя, хто есьць больш ежы зь нізкай шчыльнасьцю, маюць меншую вагу і абхоп таліі, павелічэньне калярыйнай шчыльнасьці можа да 56% павялічыць спажываньне калёрыяў, у экспэрымэнтах людзі зьядалі ў сярэднім на 425 ккал больш пры пераходзе на шчыльную сытную ежу.

Кантроль апэтыту.
Чым вышэйшая ўдзельная калярыйная шчыльнасьць, тым менш месца займае гэтая ежа ў страўніку, такім чынам, менш стымулюе мэханарэцэптары страўніка, якія ўспрымаюць ціск на яго сьценкі. Акрамя гэтага, сама шчыльнасьць ежы стымулюе апэтыт. З эвалюцыйнага пункту гледжаньня прадукты з высокай калярыйнай шчыльнасьцю былі рэдкія, таму іх важна было зьесьці і пераўтварыць у тлушчавыя запасы, якія могуць дапамагчы ў будучыні. Расьлінныя прадукты зь нізкай удзельнай калярыйнасьцю былі шырока распаўсюджаныя, таму яны не выклікаюць такога павышэньня апэтыту. Многія сумуюць па выпечцы, але мала хто сумуе па брокалі.

Забясьпечанасьць вітамінамі і мінэраламі.
Многія калярыйныя прадукты бедныя вітамінамі і мікраэлемэнтамі. Сэлекцыя, якая прыводзіць да павелічэньня ўтрыманьня крухмалу і тлушчу, паскарэньня росту, адначасова вядзе і да зьніжэньня ўтрыманьня карысных рэчываў. Чым больш чалавек зьядае шчыльнай ежы, тым менш у яго рацыёне багатых вітамінамі і мінэраламі прадуктаў, што можа прыводзіць да шэрагу дэфіцытных станаў.

Калярыйныя прадукты бедныя вітамінамі і мікраэлемэнтамі. Сэлекцыя, якая прыводзіць да павелічэньня ўтрыманьня крухмалу і тлушчу, паскарэньня росту, адначасова вядзе і да зьніжэньня ўтрыманьня карысных рэчываў.

Удзельная шчыльнасьць вугляводаў і кішачнік.
Рэч у тым, што навуковыя дасьледаваньні паказалі: менавіта ўдзельная шчыльнасьць вугляводаў, а не глікемічны індэкс, уплывае на іх дзеяньне. Чым вышэйшая шчыльнасьць, тым мацнейшы яе ўплыў на мікрафлёру і тым мацнейшая запаленчая рэакцыя, незалежна ад агульнай колькасьці вугляводаў. Пасьля прыёму канцэнтраваных вугляводаў узьнікае «мэтабалічная эндатаксэмія» (павелічэньне ўзроўню таксынаў), якая трымаецца да 5 гадзінаў пасьля яды. Гэта выклікае хранічнае запаленьне, што прыводзіць да лептынарэзыстэнтнасьці і парушэньня працы вагусу, зьніжае насычэньне, гнетучы актыўнасьць такіх мэдыятараў насычэньня, як халецыстакінін, пэптыд YY. Даведзена, што харчаваньне "старажытнымі" вугляводамі прыводзіць да зьмены мікрафлёры, зьніжэньня пранікальнасьці кішачніка і памяншэньня выяўнасьці запаленчых зьяваў. Аналягічная колькасьць "сучасных" вугляводаў не аказвае падобнага эфэкту. Чым больш «сучасных» вугляводаў, тым больш неспрыяльнай мікрафлёры як у ротавай поласьці, так і ў тонкім кішачніку, і тым мацнейшы запаленчы статут арганізму.
Чым вышэйшая удзельная калярыйная шчыльнасьць, тым ніжэйшая разнастайнасьць мікрафлёры. Так, узровень Bacteroidetes вышэйшы на дыеце ў 2 Ккал / грамаў, а ўзровень Firmicutes вышэйшы на дыеце ў 4 Ккал / грамаў. Узровень Parabacteroides вышэйшы на дыеце ў 2 Ккал / грамаў, а ўзровень Barnesiella вышэйшы на дыеце ў 4 Ккал / грамаў.

Асноўныя прынцыпы

Павялічце долю прадуктаў зь нізкай калярыйнай шчыльнасьцю, то-бок «багатых» прадуктаў, дзе мала калёрыяў. Вы можаце есьці без абмежаваньняў зеляніну, нізкакрухмалістую гародніну, ягады, багавіньне і г. д. Форма прадукту таксама мае значэньне: так, калярыйная шчыльнасьць вінаграду ў пяць разоў меншая, чым разынак. Усе вугляводныя прадукты мы можам умоўна падзяліць на «старажытныя клеткавыя», такія як банан, батат, бурак і морква, гарбуз, капуста і інш. І «сучасныя пазаклеткавыя» вугляводы, куды адносяцца крупы і іх перапрацаваныя вытворныя. Важна ня проста памяншаць вугляводы, а радыкальна памяншаць колькасьць «сучасных пазаклеткавых» (мучное любога паходжаньня, перапрацаваныя крупы) вугляводаў. А вось «старажытныя клеткавыя» вугляводы можна спажываць без істотных асьцярогаў. Правіла калярыйнай шчыльнасьці датычыцца ня толькі расьлінных, але і жывёльных прадуктаў. Параўнайце тлустую марскую рыбу і тлустую сьвініну: тлусты селядзец 248 ккал на 100 грамаў, а вось сьвініна – 491 ккал, розьніца вельмі істотная!

Паменшыце долю прадуктаў з высокай калярыйнай шчыльнасьцю, то-бок «бедных» прадуктаў, у якіх шмат калёрыяў і мала нутрыентаў. Нават проста крупы, якія зьядаюцца асобна, ня будуць здаровым выбарам, абавязкова дадавайце ў прыём ежы больш зеляніны і гародніны, каб калярыйныя прадукты займалі меншую частку талеркі. У першую чаргу гэта мучныя вырабы, кандытарскія, якія зьмяшчаюць шмат цукру і тлушчу.

Усе вугляводныя прадукты мы можам умоўна падзяліць на «старажытныя клеткавыя», такія як банан, батат, бурак і морква, гарбуз, капуста і інш., і «сучасныя пазаклеткавыя» вугляводы, куды адносяцца крупы і іх перапрацаваныя вытворныя.

Павялічце долю прадуктаў з высокай нутрыентнай шчыльнасьцю, дзе шмат вітамінаў, антыаксыдантаў, мінэралаў і да т. п. Такія прадукты часта называюць «супэрфудамі». Як мы з вамі ведаем, рабіць дыету на некалькіх, нават вельмі карысных, прадуктах не зусім правільна, але долю такіх прадуктаў варта павялічыць. Для дакладнага вызначэньня дэфіцыту вітамінаў і мінэралаў вы можаце здаць біяхімічныя аналізы крыві, самыя распаўсюджаныя дэфіцыты коратка адлюстраваныя ніжэй. Пад супэрфудамі маюць на ўвазе прадукты, якія ядуцца ў малых колькасьцях і валодаюць высокай удзельнай біялягічнай актыўнасьцю (канцэнтрацыя карыснага на грам прадукту). Пад азначэньне супэрфудаў трапляюць авакада, шпінат, марское багавіньне, гранат, чарніцы, буякі, брокалі, кале і ўсе крыжакветныя, міндаль, какосавы алей, дзікі ласось, аліўкавы алей, зялёная гарбата, какава-бабы, грэцкія гарэхі. Многія з супэрфудаў уяўляюць зь сябе маркетынгавы ход: так, абляпіха нічым не саступае годжы, а шматклеткавае багавіньне карысьнейшае і бясьпечнейшае за сьпіруліну (можа ўтрымоўваць шэраг таксынаў).

Адносна дэфіцыту мікранутрыентаў можна сустрэць розныя вэрсіі, пачынаючы ад той, што «ўся ежа сёньня ўжо ня тая» да «ў ежы ёсьць усё, што трэба». Ісьціна заключаецца ў тым, што нават калі ў ежы ўсё ёсьць, то бываюць індывідуальныя праблемы з засваеньнем неабходных рэчываў. Акрамя гэтага, важна разумець, што ў выпадку ўмеранага дэфіцыту арганізм накіроўвае дэфіцытныя рэчывы на рашэньне галоўных пытаньняў выжываньня, пры гэтым пакутуюць іншыя важныя функцыі. Так, напрыклад, пры ўтоенай жалезадэфіцытнай анэміі эрытрацыты будуць у норме, але вось валасы і энэргічнасьць – не. Аналагічная гісторыя адбываецца і пры дэфіцыце вітаміна D, сэлену, цынку і шэрагу іншых элемэнтаў. Таму важнае харчаваньне багатай нутрыентамі ежай, але, акрамя гэтага, аптымальнае падтрыманьне патрэбных узроўняў розных мікранутрыентаў.

Прапорцыі і талерка.
Калі ў вас мэта схуднець, то зьмяняйце прапорцыі розных харчовых групаў, павялічваючы ў рацыёне долю прадуктаў зь нізкай удзельнай калярыйнай шчыльнасьцю, а калі набраць вагу – то павялічвайце з высокай шчыльнасьцю.

У прынцыпе, мы можам есьці любыя прадукты, але важна памятаць, што чым вышэйшая яго ўдзельная калярыйная шчыльнасьць, тым меншую колькасьць гэтага прадукту нам варта зьесьці. Пра тое, як есьці менш і атрымліваць больш задавальненьня, я расказваю ў разьдзеле «Смачная ежа». Дадаючы ў свой рацыён больш такіх прадуктаў, мы зможам лягчэй спаталяць пачуцьцё голаду. У сярэднім лічыцца, што агульная вага ежы для дасягненьня сытасьці павінна складаць 1,5-2 кіляграмы пры адэкватнай шчыльнасьці. Таму вам важна зьядаць вялікія, аб'ёмныя порцыі ежы зь нізкай шчыльнасьцю.

У катэгорыю супэрфудаў трапляюць авакада, шпінат, марское багавіньне, гранат, чарніцы, буякі, брокалі, кале і ўсе крыжакветныя, мігдал, какосавы алей, дзікі ласось, аліўкавы алей, зялёная гарбата, какава-бабы, грэцкія гарэхі. Многія з супэрфудаў – толькі маркетынгавы ход: так, абляпіха нічым не саступае годжы, а шматклеткавае багавіньне карысьнейшае і бясьпечнейшае за сьпіруліну (можа ўтрымліваць шэраг таксынаў).

Пасьлядоўнасьць прыёму прадуктаў.
Варта пачынаць сталаваньне з прадуктаў нізкай удзельнай калярыйнасьці, зьядаючы іх больш (зялёная ліставая гародніна). Затым пераходзіце да прадуктаў сярэдняй калярыйнай шчыльнасьці (мяса, рыба, крухмалістая гародніна) і толькі ў канцы можна (але не абавязкова) зьесьці дэсэрт (гарэхі, ягады і да т.п.).

Як трымацца правіла? Ідэі і парады

Стоеная зеляніна і гародніна.
Вы можаце прыкметна зьменшыць удзельную калярыйную шчыльнасьць страў, калі будзеце больш дадаваць у іх здробненай зеляніны і гародніны, зьмешваючы з асноўнай стравай і паступова павялічваючы прапорцыю. Гэта зьменшыць колькасьць калёрыяў і павялічыць аб'ём стравы, што добра для сытасьці.

Прадукты з адмоўнай калярыйнасьцю.
Існуе міт пра прадукты з адмоўнай калярыйнасьцю, нібыта на засваеньне якіх арганізм выдаткуе больш энэргіі, чым яны даюць. У цэлым на засваеньне (тэрмічны эфэкт ежы) выдаткуецца ня больш за 5-10%.

Заправіць салату.
Заправіць салату можна ня толькі калярыйнымі алеямі, але і выкарыстоўваць бальзамічны воцат, кефір, цытрынавы сок, здробненыя гарэхі і да т.п.

Спартовае харчаваньне.
Спартхарч, па сутнасьці, можа быць аднесены да «бедных» калёрыяў, бо там шмат чыстых рэчываў, напрыклад бялку, але мала іншых важных злучэньняў. Калі вы не элітны атлет з ужываньнем неймавернай колькасьці энэргіі, вам спартовае харчаваньне ня трэба. Ежце цэльныя прадукты – так вы атрымаеце нашмат больш карысьці.

Багавіньне.
Багавіньне мае высокія канцэнтрацыі мінэралаў, нізкую колькасьць калёрыяў, яно карыснае для мікрафлёры і зручнае тым, што яго можна купляць празапас сушанае і выкарыстоўваць па меры неабходнасьці. Асноўныя віды: лямінарыі (комбу, келп), вакамэ, ірляндзкі мох, фукус, хідзікі і многія іншыя.

Ягады.
Ягады маюць высокую шчыльнасьць і багацьце біялягічна актыўных рэчываў. Магчымасьць замарозкі забясьпечвае іх даступнасьць у любую паравіну года. Выбірайце ягады мясцовыя, сакавітыя, у іх, як правіла, вышэйшая канцэнтрацыя антыаксыдантаў: чарніцы, ажыны, шаўкоўніца, вішня, чарнаплодная рабіна, буякі і многія іншыя. Кававы кубак ягадаў – неабходны мінімум.

Грыбы.
Розныя віды грыбоў прыкметна адрозьніваюцца па складзе, але ў цэлым усе маюць высокую біялягічную вартасьць. Грыбы станоўча ўплываюць на мікрафлёру кішачніка, утрымліваюць шэраг мінэралаў, паляпшаюць ліпідны спэктар крыві, зьмяншаюць запаленьне. Многія грыбы (кардцыцэпс, рэйшы і інш.) валодаюць унікальнымі здольнасьцямі ўплываць на імунны адказ.

Гарэхі.
Гарэхі, нягледзячы на сваю калярыйнасьць, вельмі станоўча ўплываюць на здароўе, зьмяншаюць рызыку сардэчна-сасудзістых захворваньняў, памяншаюць узровень запаленьня. Можна сьмела іх есьці на дэсэрт, але пры гэтым таксама памятайце пра ўмеранасьць. Большасьць гарэхаў маюць даведзенае карыснае дзеяньне: грэцкі гарэх, мігдал, фісташкі, кеш'ю, бразільскі гарэх. Калі гарэхі пашкоджаныя ці здробненыя, іх якасьць падае – купляйце арэхі ня дробненыя, а з цэлай плёнкай ці шкарлупінай.

Спэцыі.
Спэцыі ўтрымоўваюць нікчэмна малую колькасьць калёрыяў, але пры гэтым неймаверна шмат карысных спалучэньняў. Яны зьмяншаюць запаленьне, паляпшаюць насычэньне і адчувальнасьць да інсуліну, стымулююць спаленьне тлушчу і валодаюць яшчэ цэлым шэрагам карысных эфэктаў. Выбірайце розныя спэцыі і дадавайце іх у ежу ў дастатковых колькасьцях: чырвоны перац, размарын, куркума, лаўровы ліст, аніс, кардамон, шафран і многія іншыя. Таксама дадавайце ў ежу і такія прадукты, як імберац, цыбуля, часнык, хрэн, гарчыца: яны ня толькі смачныя, але й вельмі карысныя.

Зеляніна.
Зеляніна даступная ня толькі ў сьвежым выглядзе, што ідэальна, але можа актыўна ўжывацца і ў сушаным і замарожаным выглядзе. Вы можаце на свой густ выкарыстоўваць шырокі набор: кроп, шпінат, базілік, кінза, пятрушка, лук, мангольд, кале і інш.

Мультывітаміны.
Лягічным рашэньнем зьнізіць рызыку дэфіцыту вітамінаў зьяўляецца прыём дабавак. Аднак дасьледаваньні паказваюць, што мультывітаміны не аказваюць прыкметнага станоўчага ўзьдзеяньня, а ў шматлікіх выпадках могуць толькі нашкодзіць. Таму важна выбіральна падыходзіць да прызначэньня розных дабавак, кантралюючы ўзровень дэфіцыту вітамінаў лябараторнымі аналізамі.

Спэцыі ўтрымоўваюць нікчэмна малую колькасьць калёрыяў, пры гэтым неймаверна шмат карысных спалучэньняў. Дадавайце ў ежу чырвоны перац, размарын, куркуму, ляўровы ліст, аніс, кардамон, шафран.

Найбольш частыя дэфіцыты.
Найбольш частымі дэфіцытамі зьяўляюцца дэфіцыт амэга-3 тлустых кіслот (дадаткі або тлустая марская рыба два разы на тыдзень, кантроль па амэга-індэксе), дэфіцыт сэлену (лепш хелатные формы дабавак або морапрадукты, часнык, бразільскі гарэх), дэфіцыт вітаміну D (бываць часьцей апоўдні на сонцы, есьці больш рыбы, яек або дабаўкі ў высокіх дозах пад лябараторным кантролем), дэфіцыт жалеза (мясныя прадукты або хелатные дабаўкі, кантроль па профілі жалеза ў крыві: жалеза, ферытын, трансфэрын), дэфіцыт цынку (больш мяса і рыбы або дабаўкі), дэфіцыт магнію, дэфіцыт вітамінаў В9 і В12 (лепш прымаць у выглядзе мэтыльных формаў, мэтылфалат і мэтылкабаламін, кантроль па ўзроўні В9, В12, гомацыстэіну), дэфіцыт ёду (цяпер сустракаецца ўсё радзей).

\chapter{Разнастайнасьць харчаваньня}

Штодзённы рацыён павінен утрымліваць прадукты з~розных групаў, каб падтрымаць дастатковы ўзровень разнастайнасьці. Не існуе фіксаванага рацыёну, які б цалкам пакрываў патрэбу арганізму ў~неабходных кампанэнтах: чым большая разнастайнасьць, тым большы спэктар атрыманых рэчываў. У~большасьці дапаможнікаў па здаровым харчаваньні адзначаецца, што аднастайнасьць харчаваньня~--- гэта шкодная звычка, здаровая ежа павінна быць разнастайнай. Правіла разнастайнасьці таксама закранае здароўе кішачнай мікрафлёры, якое зьніжаецца пры памяншэньні колькасьці прабіётыкаў і прэбіётыкаў, павелічэньні ўжываньня антыбіётыкаў і антысэптыкаў.

Многія людзі лічаць «здаровым» харчаваньнем грэчку з~курынай грудкай пяць разоў на дзень сем дзён на тыдзень або абмяжоўваюць спэктар прадуктаў, зводзячы свой рацыён да некалькіх «правільных» прадуктаў. Вядома, такая бедната можа прытупіць апэтыт, ды ўсё ж ані ня можа быць карыснай. Няма ідэальнага прадукту, які закрые ўсе нашыя патрэбы, таму важна камбінаваць самыя розныя пазіцыі для эфэктыўнага задавальненьня нашых запытаў!

\subsection{Як зьявілася праблема?}

Для нашых продкаў была характэрнай неймаверная разнастайнасьць прадуктаў харчаваньня, чаму спрыяла сэзоннасьць харчаваньня і актыўны пошук усяго, што пасавала ў~ежу. У~ход ішлі розныя карані, гарэхі, многія віды жывёлаў, яйкі, трава, ягады, плады, жамяра, земнаводныя, птушкі і г.~д. Паляўнічыя-зьбіральнікі елі сотні відаў жывёлаў і расьлінаў! Многія расьлінныя прадукты для павелічэньня смаку ўжываліся фэрмэнтаванымі, што падвышала даступнасьць нутрыентаў і біялягічную вартасьць ежы. Пераход да сельскай гаспадаркі характарызаваўся зьніжэньнем разнастайнасьці харчаваньня, павелічэньнем колькасьці крупаў у~рацыёне, што прывяло да прыкметнага пагаршэньня здароўя. Сёньня наша харчаваньне стала яшчэ больш аднастайным, мы атрымліваем большасьць калёрыяў усяго толькі зь некалькіх відаў расьлінаў і жывёлаў: пшаніца, рыс, бульба, ялавічына, сьвініна, курыца і да т.~п. Многія карысныя групы прадуктаў часта маюць мінімальную вагу ў~нашым рацыёне.

Умовы гатаваньня ежы, як і лад жыцьця, меркавалі кантакт з~мноствам відаў мікраарганізмаў. У~старажытных паляўнічых-зьбіральнікаў была самая высокая разнастайнасьць кішачнай мікрафлёры. У~сучасных гараджан разнастайнасьць кішачнай мікрафлёры зьніжаецца, і гэта вядзе да павелічэньня рызыкі алергічных, аўтаімунных і нават анкалягічных захворваньняў.

\tipbox{Сёньня нашае харчаваньне стала аднастайным, мы атрымліваем большасьць калёрыяў усяго толькі зь некалькіх відаў расьлінаў і жывёлаў: пшаніца, рыс, бульба, ялавічына, сьвініна, курыца і да т.~п. Многія карысныя групы прадуктаў часта маюць мінімальную вагу ў~нашым рацыёне.}

Як высьветлілі навукоўцы, стан нашай кішачнай мікрафлёры ўплывае на работу галаўнога мозгу, мэтабалізм і нават хуткасьць старэньня. У~розных тыпаў бактэрый розныя функцыі і ўплыў на здароўе, але вось іх разнастайнасьць~--- галоўная характарыстыка мікрабіёма нашага кішачніка.

Ды й~тая разнастайнасьць, якая ёсьць, выяўляецца далёка ня самай карыснай. Так, з~гародніны~--- гэта бульба, часта смажаная, а~з садавіны~--- бананы і салодкі вінаград з~высокім утрыманьнем цукру. Вельмі часта з~узростам мы становімся больш кансэрватыўнымі, таму абмяжоўваем свой выбар і баімся каштаваць новае. Часам атручэньне або няўдалая пакупка адварочваюць нас адразу ад цэлай прадуктовай групы. Усё гэта паступова прыводзіць да скарачэньня разнастайнасьці.

\subsection{Як гэта ўплывае на здароўе?}

Дасьледчыкі ўвялі шэраг індэксаў разнастайнасьці дыеты, якія карэлююць з~паказчыкамі здароўя. Чым размаіцейшы рацыён, тым людзі жывуць даўжэй і менш хварэюць. Прынцып «есьці самы карысны прадукт» не працуе, дыета павінна зьмяшчаць якасныя і карысныя прадукты розных тыпаў, каб быць здаровай і збалянсаванай.

\paragraph{Разнастайнасьць і забесьпячэньне.}
Разнастайнасьць харчаваньня мусіць быць дастатковая, каб забясьпечыць арганізм неабходным узроўнем калёрыяў і незаменных нутрыентаў, а~таксама поўным наборам вітамінаў, мікраэлемэнтаў, макраэлемэнтаў і інш. Часам дэфіцыт зьвязаны зь нізкім утрыманьнем асобных элемэнтаў у~глебе (сэлен, ёд), часам~--- з~выкарыстаньнем абмежавальных дыетаў. Так, вэганства вядзе да дэфіцыту жалеза, цынку, вітаміну В12, вітаміну D і яшчэ шэрагу спалучэньняў. Шэраг дэфіцытных у~нашай дыеце спалучэньняў і спосабы іх папаўненьня пазначаныя ў~канцы гэтага разьдзелу.

\paragraph{Разнастайнасьць і задавальненьне.}
Разнастайнасьць харчаваньня стымулюе апэтыт, павялічвае ўзровень задавальненьня і спрыяе таму, каб зьядаць больш. У~гэтым выпадку разнастайнасьць высокакалярыйных прадуктаў можа мець нэгатыўны ўплыў на здароўе і павялічваць пераяданьне. А~вось разнастайнасьць зеляніны, гарэхаў, гародніны таксама павялічвае колькасьць зьедзенага, і ў~такім выпадку гэта мае станоўчае значэньне.

\paragraph{Разнастайнасьць і таксыны.}
Многія прадукты ўтрымліваюць таксычныя рэчывы, якія ўключаюць антыбіётыкі, цяжкія мэталы і інш. Разнастайнае харчаваньне дазваляе паменшыць канцэнтрацыю кожнага з~магчымых таксынаў, што зьнізіць іх нэгатыўнае ўзьдзеяньне на арганізм (разьмеркаваньне шкоды).

\tipbox{Чым разнастайнейшая мікрафлёра, тым лепшыя паказьнікі здароўя і меншая рызыка захворваньняў. Для падтрыманьня такой разнастайнасьці неабходныя прэбіётыкі (пэктыны, альгінаты, інулін і інш.), якія лепей атрымліваць з~цэльных расьлінных прадуктаў ці ўжываць асобна.}

\paragraph{Разнастайнасьць і мэтабалізм.}
Дасьледаваньні пастанавілі, што разнастайнасьць рацыёну зьвязаная з~больш высокім утрыманьнем розных мікраэлемэнтаў. Разнастайнасьць харчаваньня станоўча ўплывае на стан здароўя і зьвязаная са зьніжэньнем узроўню запаленьня, лепшым ліпідным профілем і меншай рызыкай мэтабалічнага сындрому. Аднастайнае і беднае харчаваньне часта прыводзіць да дэфіцыту шматлікіх рэчываў у~арганізьме, якія істотна павялічваюць рызыку захворваньняў, а~таксама ўплываюць ня толькі на фізычны, але й~на разумовы стан. Так, дэфіцыт цынку, ёду і жалеза можа зьменшыць паказьнікі IQ у~дзяцей.

\paragraph{Разнастайнасьць мікрафлёры.}
Чым разнастайнейшая мікрафлёра, тым лепшыя паказьнікі здароўя і меншая рызыка захворваньняў. Бо крыніцай харчаваньня для нашай мікрафлёры зьяўляюцца розныя віды расьліннай ежы з~высокім утрыманьнем харчовых валокнаў. Таму зеляніна, ягады, гародніна і садавіна кормяць ня толькі нас, але й~нашы карысныя бактэрыі. Чым лепей мы кормім сваіх маленькіх сяброў, тым большая іх разнастайнасьць і тым лепей яны клапоцяцца пра нашае здароўе. Для падтрыманьня гэтай разнастайнасьці неабходныя прэбіётыкі (пэктыны, альгінаты, інулін і інш.), якія можна атрымліваць з~прадуктаў ці ўжываць асобна. Пры малой разнастайнасьці расьліннай ежы павялічваецца колькасьць бактэрыяў, якія падвышаюць рызыку атлусьценьня і запаленьняў. Разнастайнасьць нашай мікрафлёры на 30\,\% меншая, чым у~паляўнічых-зьбіральнікаў, пры гэтым норма клятчаткі сёньня 25 грамаў на суткі, сярэдні чалавек есьць 15--20, а~паляўнічы-зьбіральнік~--- да 100 грамаў! У~доўгажыхароў разнастайнасьць мікрафлёры нават вышэйшая, чым у~маладых людзей у~сярэднім!

\subsection{Асноўныя прынцыпы}

Сачыце за тым, каб на працягу дня ў~вас на стале былі розныя прадуктовыя групы. Аптымальна ня менш за пяць прадуктаў на прыём ежы, не паўтарайце стравы-прадукты цягам аднаго дня.

\paragraph{Разнастайнасьць прадуктовых групаў.}
Найважнейшым прынцыпам зьяўляецца разнастайнасьць прадуктовых групаў і слушны балянс паміж імі. Так, часта мы назіраем перавагу высокакяларыйных вугляводных крупаў, празьмер мучнога, лішак дабаўленага цукру, пры гэтым амаль поўная адсутнасьць зялёнай ліставой гародніны (зеляніны), рыбы, морапрадуктаў, ягадаў, гарэхаў, багавіньня, грыбоў. А~гэтыя групы прадуктаў валодаюць высокай біялягічнай вартасьцю і ўтрымліваюць шмат карысных кампанэнтаў.

\paragraph{Разнастайнасьць унутры прадуктовай групы.}
Разнастайнасьць усярэдзіне прадуктовай групы дапамагае знайсьці індывідуальны кампраміс, выгаду, пераноснасьць і абраць максымальную карысьць. Важна імкнуцца есьці ня дзесяць відаў рыбы, а~выбраць зь іх пару відаў, самых смачных для вас. Дасьледаваньні паказваюць, што колькасьць зьедзенай гародніны нашмат мацней уплывае на зьніжэньне рызыкі разьвіцьця раку, чым яе разнастайнасьць.

Важнай асаблівасьцю разнастайнасьці ўнутры прадуктовай групы зьяўляецца і захаваньне смаку, бо часта адно і тое ж надакучвае. Часам у~вас ня вельмі добрая пераноснасьць толькі аднаго прадукту, а~вы выключаеце з~рацыёну ўсю групу! Напрыклад, у~вас кепскавата з~фасоляй. Але ж сярод бабовых ёсьць яшчэ нут, маш, сачавіца, гарох, спаржавая фасоля, бабы! Напэўна, сярод іх вы знойдзеце тое, што прыпадзе вам да густу і што вы будзеце добра засвойваць. Такая ж разнастайнасьць ёсьць сярод зеляніны: рукала, кроп, шпінат, базілік, кінза, пятрушка, лук, мангольд і да т.~п. Варыяцыі ўнутры прадуктовай групы падтрымліваюць цікавасьць і смак да ежы, што важна для атрыманьня задавальненьня і задаволенасьці.

\subsection{Як трымацца правіла? Ідэі і парады}

\paragraph{Прадукты адной групы.}
Пералічыце тыя прадукты, якія вы ясьцё, і павялічце сьпіс за кошт прадуктаў той жа групы. Бабовыя (фасоля, нут, бабы, гарох), ягады (чарніцы, вішня, журавіны, брусніцы і…), зеляніна, спэцыі.

\paragraph{Монадыеты.}
Часта можна ўбачыць рэкляму монадыетаў, калі рэкамэндуюць есьці адзін-два прадукты цягам доўгага часу. Аднастайнасьць зьніжае апэтыт і колькасьць зьедзеных прадуктаў, аднак такі падыход ня ёсьць здаровым.

\paragraph{Купляйце новае напаспытак.}
Закупляючыся, паспрабуйце кожны раз па магчымасьці купіць нешта нязвыклае і цікавае. Нават калі большасьць навінак вам не спадабаецца, вы ўсё роўна знойдзеце для сябе нешта новае і вартае ўвагі.

\tipbox{Каб зрабіць харчаваньне разнастайным, складзіце сьпіс прадуктаў, якія вы ясьцё, і павялічце яго за кошт прадуктаў той жа групы. Да прыкладу, замест фасолі можна дадаць у~рацыён нут, сачавіцу ці гарох.}

\paragraph{Дзеці і новыя прадукты.}
Дзеці звычайна вельмі кансэрватыўныя да новых прадуктаў. Ім трэба рассмакаваць новае, для гэтага давайце ім напаспытак маленькі кавалачак новага прадукту, дадавайце да знаёмых страў, падавайце ў~незвычайным выглядзе.

\paragraph{Каштуйце новыя прадукты ў~падарожжах.}
Гэта ўзбагаціць вашыя ўражаньні і, магчыма, мікрафлёру кішачніка.

\paragraph{Дайце ежы шанец.}
Часам, калі нам нешта не даспадобы, мы выключаем гэта назаўжды. Калі нешта не пайшло цяпер, дайце ежы шанец потым, пакаштуйце гэта яшчэ раз.

\paragraph{Тыдзень без паўтарэньня страваў.}
Прыміце выклік~--- розныя стравы кожны дзень! Гэта запатрабуе ад вас большай напругі, чым здаецца на першы погляд!

\paragraph{Рознакаляровая дыета.}
Выбіраючы гародніну, карыстайцеся прынцыпам рознакаляровай дыеты. Кожны колер зьвязаны з~пэўным відам антыаксыданту. Адзінага ўнівэрсальнага антыаксыданту няма, таму камбінацыя розных відаў найболей эфэктыўная і карысная для здароўя.

\paragraph{Больш актыўна ўжывайце фэрмэнтаваныя прадукты.}
Фэрмэнтацыя ўзбагачае прадукты рознымі вітамінамі, прэбіётыкамі і прабіётыкі. Ёсьць шмат розных відаў фэрмэнтаваных прадуктаў: міса, чайны грыб (камбуча), квас, квашаная капуста, тэмпэ, кефір, ёгурт і інш. Квасіць можна розную гародніну, а~купляючы гатовыя прадукты, пераканайцеся, што там няма цукру і лішніх дабавак.

\paragraph{Дадаткі прабіётыкі.}
Існуюць розныя штамы прабіётыкаў, але найбольшы эфэкт назіраўся ад мультыпрабіётыкаў, якія зьмяшчаюць некалькі розных відаў у~вялікшых канцэнтрацыях. З~асобных штамаў вартыя ўвагі Lactobacillus reuteri, Saccharomyces boulardii, Streptococcus salivarius (для ротавай поласьці) і іншыя.

\paragraph{Разнастайнасьць жывёльных прадуктаў.}
Ежце ня толькі мяса, але й~іншыя часткі жывёлаў: косткі (для булёнаў), сэрца, печань і іншыя субпрадукты. Яны таксама ўтрымліваюць шмат карысных рэчываў, іх асабліва шмат, калі гаворка ідзе пра парнакапытных.

\tipbox{Прыміце выклік~--- розныя стравы кожны дзень! Гэта запатрабуе ад вас большай напругі, чым здаецца на першы погляд!}

\paragraph{Разнастайнасьць морапрадуктаў.}
Дадавайце малюскаў, кальмараў, крэвэтак, васьміногаў і крыля. Морапрадукты месьцяць мноства карысных спалучэньняў.

\paragraph{Разнастайнасьць садавіны.}
Выбірайце зь іх найбольш яркія, з~больш выразным смакам і меншым утрыманьнем цукру, сэзонную садавіну зь меншым тэрмінам захоўваньня. Некаторая садавіна, напрыклад ківі ці авакада, утрымліваюць болей карысных кампанэнтаў.

\paragraph{Працуйце ў~садзе.}
Праца ў~садзе павялічвае кантакт з~глебавымі бактэрыямі, што дабратворна адбіваецца на здароўі. Часьцей бывайце на прыродзе, кантактуйце з~хатнімі жывёламі.

\paragraph{Заплянаваная і лёгкая разнастайнасьць.}
Кожнае харчовае рашэньне патрабуе ад вас увагі і энэргіі, таму стварыце свой тыповы разнастайны рацыён і сьпіс пакупак загадзя. Ідэальна, калі вы 80\,\% свайго рацыёну зрабілі рутынным, гэта выдатна зэканоміць вашыя сілы. Таксама важна памятаць пра тое, што найлепшы плян харчаваньня~--- гэта той, які вам лёгка і зручна выконваць!

\paragraph{Нялюбыя прадукты.}
Добры спосаб зрабіць ня вельмі любімы прадукт жаданым~--- зьесьці яго за кампанію зь сябрамі або ў~незвычайным месцы (рэстарацыя, паход, паездка). Навукоўцы вызначылі, што нашы смакавыя перавагі шмат у~чым зьяўляюцца сацыяльнымі, а~не біялягічнымі.
Паасобнае харчаваньне.

\tipbox{Паасобнае харчаваньне~--- гэта харчовы міт, які ня мае пад сабой падставы. Фэрмэнты для расшчапленьня бялкоў, тлушчаў і вугляводаў вылучаюцца разам і ані не замінаюць рабоце адзін аднаго.}

Правіла 16. Хуткая і павольная ежа

Розныя прадукты па-рознаму ўплываюць на ўмоўную хуткасьць мэтабалізму, без увагі на іх калярыйнасьць. Некаторыя прадукты валодаюць большай здольнасьцю стымуляваць паскарэньне працэсаў у клетцы. У аснове гэтага паскарэньня ляжыць малекулярны комплекс mTORC (mammalian target of rapamycin complex, то-бок «мішэнь рапаміцыну ў млекакормячых»), які рэгулюе хуткасьць росту і ачышчэньня клетак.
mTORC – гэта своеасаблівая пэдаль газу для нашага мэтабалізму, падвышаная актыўнасьць якога паскарае рост і размнажэньне клетак, сынтэз новых бялкоў, спрыяе большаму выжываньню клетак. Актыўнасьць mTORС таксама важная для здароўя: ён стымулюе рост цягліцаў, стварае новыя мітахондрыі ды паляпшае сувязі паміж нэрвовымі клеткамі.

Гэты малекулярны комплекс рэагуе на сыгналы ад розных нутрыентаў і зьмяняе актыўнасьць розных фактараў росту клеткі. Нізкая актыўнасьць mTORC запавольвае рост клетак, спрыяе павелічэньню ўзроўню рэпарацый і аўтафагіі (ачысткі клетак), гібелі пашкоджаных і непатрэбных клетак. Залішняя актыўнасьць mTORC ляжыць у аснове разьвіцьця шматлікіх захворваньняў. Для аптымальнага здароўя важна чаргаваць пэрыяды спажываньня прадуктаў, якія падвышаюць mTORC, з больш працяглымі пэрыядамі яды прадуктаў зь нізкай стымуляцыяй актыўнасьці mTORC або з харчовым устрыманьнем.
Уявіце сабе, што нашае жыцьцё – гэта язда на аўтамабілі. Для доўгай і бясьпечнай язды трэба прытармажваць, спыняцца, прапускаць іншыя машыны. Калі ціснуць на пэдаль аўтамабіля вельмі моцна, ды яшчэ і на нізкіх перадачах, то ён, вядома, будзе эфэктна раўці, але гэта можа прывесьці да заўчаснага зносу рухавіка. Плыўны націск, правільнае выкарыстаньне перадач падоўжаць жыцьцё машыне і дазволяць зэканоміць паліва. Так і з нашым арганізмам – пажадана не злаўжываць хуткай ежай, а знайсьці правільны балянс!

Як зьявілася праблема?

Калі б нашага продка спыталі, вэган ён ці мясаед, то яго адказ здзівіў бы нас. У дні вялікіх і малых постаў людзі мяса ня елі, часта былі посты з выразным абмежаваньнем калёрыяў, і спажываньне мяса было абмежаванае такім рэжымам харчаваньня. Так і нашы ўсяедныя продкі паляўнічыя-зьбіральнікі мелі цыклічны рэжым харчаваньня: падчас паляваньня яны пераходзілі на карнівор-дыету (толькі жывёльная ежа), бо мяса захоўваць складана, а ў часы паходаў або няўдалага паляваньня вялі лад жыцьця вэганаў (клубні, карэньні і да т. п.). Павелічэньне спажываньня мяса і тлушчу традыцыйна было зьвязанае зь зімовай парой, калі недахоп корму і зьніжэньне запасаў ежы прыводзілі да забою хатніх жывёлаў.

Нашы ўсяедныя продкі паляўнічыя-зьбіральнікі мелі цыклічны рэжым харчаваньня, падчас паляваньня яны пераходзілі на карнівор-дыету (толькі мяса), бо мяса захоўваць складана, а ў час паходаў або няўдалага паляваньня вялі лад жыцьця вэганаў (клубні, карэньні і г.д.).

Калі ў нас ёсьць шмат ежы (вугляводаў або бялку) і шмат калёрыяў, то гэта падвышае актыўнасьць mTORC, і гэты сыгнал кажа арганізму, што цяпер часы багацьця, значыць, можна расьці ды размнажацца. Уласна, стары жарт пра тое, што мясаеды «раздражняльныя празь мяса» мае навуковае абгрунтаваньне, бо залішняя стымуляцыя mTORС павялічвае ўзровень адрэналіну і актыўнасьць стрэсавай сымпацыйнай сыстэмы. Таму зьніжэньне стымуляцыі прыводзіць да зьніжэньня энэргічнасьці, і мясаеды маюць рацыю, што «слабыя бязь мяса». Пагаворым пра гэта падрабязьней.

Як гэта ўплывае на здароўе?

У аснове шматлікіх хваробаў і старэньня ляжыць залішняя актыўнасьць клетак. Гэта разрастаньне атэрасклератычных бляшак, паскораны рост ракавых клетак, гіпэртрафія міякарду і да т. п. Такія хваробы часта называюць mTORС-залежнымі, бо яны часьцей сустракаюцца ў краінах з высокім узроўнем спажываньня «хуткіх» прадуктаў.

Сыгнальны шлях інсуліну і IGF-1.
Інсулін і IGF-1 зьяўляюцца гармонамі, якія непасрэдна павялічваюць стымуляцыю mTORС і запавольваюць працэсы аўтафагіі. Павелічэньне іх узроўню павышае рызыку ўзьнікненьня многіх тыпаў раку, у тым ліку раку грудзей, прастаты, тоўстай кішкі. Павелічэньне даволі заўважнае ў выпадку раку грудзей, высокі ўзровень IGF-1 на 40% павялічвае рызыку раку грудзей.

Пухліны.
Залішняя актывацыя работы mTORС прыводзіць да бескантрольнага дзяленьня клетак, што азначае ператварэньне іх у ракавыя. mTORС актыўна ўплывае на разьвіцьцё злаякасных пухлінаў, павялічваючы ангіягенэз (рост новых крывяносных сасудаў у самой пухліне і вакол яе), што дапамагае раку расьці.

Акнэ.
Акнэ зьяўляецца вонкавай праявай падвышанай актыўнасьці mTORC і частым прадвесьнікам аддаленай рызыкі для здароўя. Нізкавугляводная дыета, якая ўключае прадукты зь нізкім глікемічным індэксам і выключае малочныя прадукты, прыкметна паляпшае стан скуры, а заадно і вонкавы выгляд. Цікава, што ў карэнных народаў у розных кутках сьвету зьява акнэ ў падлеткаў і ў дарослых цалкам адсутнічае!

Імунітэт.
Залішняя актыўнасьць mTORC спрыяе выжываньню аўтарэактыўных клонаў клетак, што павялічвае рызыку аўтаімунных захворваньняў. Увесьчасна высокая актыўнасьць mTORС павялічвае запаленьне.

Мэтабалізм.
Паніжэньне актыўнасьці mTORС паляпшае адчувальнасьць цяглічных клетак да інсуліну, што абараняе ад разьвіцьця дыябэту ці запавольвае яго разьвіцьцё.

У аснове шматлікіх хвароб і старэньні ляжыць залішняя актыўнасьць клетак. Гэта разрастаньне атэрасклератычных бляшак, паскораны рост ракавых клетак, гіпэртрафія міякарду і г. д. Такія хваробы часта называюць mTORС-залежнымі, бо яны часьцей сустракаюцца ў краінах з высокім узроўнем спажываньня «хуткіх» прадуктаў.

Сардэчна-сасудзістыя захворваньні.
У норме актывацыя mTORC зьмяншае апэтыт і прыводзіць да зьніжэньня вагі праз актывацыю тлушчаспаленьня. Чым вышэйшая актыўнасьць mTORС, тым больш павялічваецца выкід адрэналіну і актыўнасьць стрэсавай сымпацыйнай вэгетатыўнай сыстэмы. Менавіта таму інсулагенныя прадукты і дробавае харчаваньне могуць паляпшаць самаадчуваньне. Але ўвесь час падвышаны тонус сымпацыйнай сыстэмы вядзе да падвышэньня артэрыяльнага ціску і рызыкі гіпэртэнзіі.

Трывога, стрэс і выгараньне.
У выпадку з гіперактывацыяй mTORС нарастае актыўнасьць сымпацыйнай сыстэмы (стрэс), а парасымпацыйнай (расслабленьне) – прыгнятаецца. Працяглае павышэньне сымпацыйнай актыўнасьці ў людзей са схільнасьцю павялічвае ўзровень турботы, трывогі і раздражняльнасьці. Працяглая гіпэрактывацыя сымпацыйнай сыстэмы (рэжым хранічнага стрэсу) прыводзіць да выгараньня (ці заяданьня праблемы).

Аднаўленьне.
Залішняя актывацыя mTORС – гэта ня толькі празьмерны рост клетак, але й прыгнятаньне рэпарацый і аўтафагіі. Пры парушэньні аднаўленьня ўзмацняецца назапашваньне пашкоджваньняў у клетках, павялічваецца колькасьць клеткавага сьмецьця, што зьяўляецца асновай разьвіцьця мноства розных захворваньняў.

Нэўрадэгенэратыўныя захворваньні.
Больш высокі ўзровень IGF-1 назіраецца ў пацыентаў з хваробай Альцгеймэра, і яго ўзровень карэлюе з выяўленасьцю сымптомаў.

Старэньне.
Важна разумець, што старэньне і дзяленьне клетак – гэта два бакі аднаго медаля. Тое, што стымулюе рост колькасьці клетак, у выпадку завяршэньня росту арганізму можа ўзмацняць працэсы старэньня. Чым мацней мы стымулюем клеткі дарослага арганізму надмерам пажыўных рэчываў, тым хутчэй яны старэюць. Мутацыі ў генах інсуліну і IGF-1 відавочна падаўжалі жыцьцё экспэрымэнтальным жывёлам.

Асноўныя прынцыпы

Важна чаргаваць пэрыяды высокай і нізкай mTORС-актыўнасьці.
Пэрыяды неактыўнага mTORС важныя для аднаўленьня клетак. Пастаянная стымуляцыя прыводзіць да таго, што нашы клеткі становяцца «засьмечанымі» і страчваюць адчувальнасьць да сыгналаў арганізму. Зразумела, актыўнасьць mTORС будзе падаць, калі мы не ямо зусім, але паўплываць на актыўнасьць можна ня толькі голадам, але й выбарам прадуктаў.

Прадукты харчаваньня маюць розны ўплыў на актыўнасьць mTORС. Ёсьць нэўтральныя прадукты, якія стымулююць mTORС прапарцыйна колькасьці калёрыяў, а ёсьць «хуткія» прадукты, якія стымулююць mTORС нашмат мацней. Калі чалавек расьце або фізычна актыўны значную частку дня, то асаблівай шкоды для яго няма. Але калі чалавек мае меншую фізычную актыўнасьць, то гэтыя прадукты будуць прыводзіць да росту mTORС-залежных хваробаў, пра якія я казаў раней. Калі гаварыць у цэлым, то для чалавека важна, каб mTORС утрымліваўся на нізкім узроўні з кароткімі часавымі адрэзкамі яго актывацыі. Таму чалавеку важна трымаць здаровы балянс паміж двума станамі: рост / непажаданае сьмецьце і адпачынак / ачыстка.

Старэньне і дзяленьне клетак – гэта два бакі аднаго медаля. Тое, што стымулюе рост колькасьці клетак, у выпадку завяршэньня росту арганізму можа ўзмацняць працэсы старэньня. Чым мацней мы стымулюем клеткі дарослага арганізму лішкам пажыўных рэчываў, тым хутчэй яны старэюць.

Важна датрымлівацца рэжыму харчаваньня, рабіць дні з харчовым устрыманьнем і «павольныя» дні. Стымуляцыя mTORC ежай таксама важная для абнаўленьня клетак і іх рэгенэрацыі. Таму пастаяннае павольнае mTORC-дэфіцытнае харчаваньне можа прыводзіць да дыстрафічных зьяў. Больш «хуткай» ежы могуць бясьпечна дазволіць сабе тыя, хто «расьце» ці аднаўляецца – дзеці, людзі пасьля хваробы, спартоўцы, але людзям, старэйшым за 30, важна абмежаваць «хуткую» ежу.

«Хуткія» прадукты.
Сярод хуткіх прадуктаў трэба вылучыць наступныя групы.

Бялкі.
Мацней за ўсё стымулююць mTORС амінакіслоты з разгалінаваным ланцугом (лейцын, ізалейцын і валін, у першую чаргу лейцын). Лейцын стымулюе mTORС наўпрост, а таксама павялічвае выдзяленьне інсуліну. Больш за ўсё амінакіслотаў з разгалінаваным ланцугом знаходзіцца ў сыроватачным пратэіне, малочных прадуктах, яйках, мясе. Шмат іх і ў некаторых расьлінных бялках, уключна з пшаніцай і сояй. Падобныя дабаўкі амінакіслотаў з разгалінаваным ланцугом – гэта папулярныя дабаўкі для росту цягліцаў. Падобным дзеяньнем валодае і мэтыянін.

Вугляводы.
Чым вышэйшы ўздым інсуліну, тым мацней стымулюецца mTORС. Высокакрухмалістыя прадукты з высокай глікемічнай нагрузкай даюць больш высокі ўзровень стымуляцыі. Сюды ж адносіцца высокае спажываньне цукру. Зьвярніце ўвагу, што на глікемічны індэкс узьдзейнічаюць такія пераменныя, як тэмпэратура, ступень драбненьня і гатоўля прадукту (гл. разьдзел «Гатоўля»).

Агульная колькасьць калёрыяў.
Чым больш калёрыяў, тым вышэйшы ўзровень актыўнасьці mTORС. Зьвярніце ўвагу, што многія з mTORС-хваробаў могуць узьнікаць і ў чалавека нармальнай масы цела, бо ў некаторых людзей ёсьць вялікая ўстойлівасьць да набору вагі.

Рэжым харчаваньня.
Чым часьцей чалавек есьць, тым большую частку часу стымулюецца mTORС.

Камбінацыі.
Праблема становіцца яшчэ горшай, калі розныя прадукты, якія павялічваюць актыўнасьць mTORС, спалучаюцца разам: вугляводы + мяса, салодкія малочныя стравы і да т. п. Спалучэньне: мяса + мучное выклікае большую актывацыю mTORС, чым сыроватачны бялок. Цяпер сухое малако часта дадаецца ў розныя прадукты для ўзмацненьня іх смаку.

«Павольныя» прадукты.
Да іх адносяцца вугляводы зь нізкім інсулінавым і глікемічным індэксам і нізкай глікемічнай нагрузкай, практычна ўсе тлушчы, некаторыя расьлінныя бялкі.

Балянс «хуткія» і «павольныя» прадукты.
Зьмена рацыёну прыводзіць да зьмены актыўнасьці mTORC. У сучасным харчаваньні ёсьць вялікі лішак «хуткіх» прадуктаў, таму важна абмежаваць іх узровень (нашмат менш малочных прадуктаў, есьці мяса, але ня кожны дзень і да т.п.). Можна павялічыць долю «хуткіх» прадуктаў у тыя дні, калі вам патрабуецца большы ўзровень энэргіі і актыўнасьці. Важна зьменшыць долю «хуткіх» прадуктаў у дні адпачынку і пэрыядычна рабіць разгрузкі, асабліва пры нарастаньні сымптомаў стрэсу. Не камбінуйце розныя «хуткія» прадукты ў адзін прыём ежы (калі ясьцё мяса, то з гароднінай, а ня з пастай, калі ясьцё тварог, то дадавайце зеляніну, а не бутэрброд з сочывам, і да т.п.).

Пры памяншэньні колькасьці mTORC-стымулятараў кшталту амінакіслотаў і цукру ў чалавека зьніжаецца артэрыяльны ціск, раздражняльнасьць, ён пачуваецца больш залагоджаным, сьвядомым і спакойным. Таму людзі на расьліннай дыеце відавочна больш спакойныя і запаволеныя, а вось тыя, хто ўжывае малако, мяса, мучныя вырабы, – залішне актыўныя, раздражняльныя і схільныя да аўтаматызмаў.

Знайдзіце свой уласны балянс, ня кідаючыся ў скрайнасьці.
Пры памяншэньні колькасьці mTORC-стымулятараў кшталту амінакіслотаў і цукру ў чалавека зьніжаецца артэрыяльны ціск, раздражняльнасьць, ён адчувае сябе больш залагоджаным, сьвядомым і спакойным. Людзі на расьліннай дыеце відавочна больш спакойныя і запаволеныя, а вось тыя, хто ўжывае малако, мяса, мучныя вырабы, – залішне актыўныя, з падвышаным ціскам, раздражняльныя і схільныя да аўтаматызмаў. Пэрыядычная адмова ад «хуткай» ежы (ці пост) спрыяе расслабленьню і падвышэньню ўсьвядомленасьці. Пры гэтым на кета-дыеце такога эфэкту не назіраецца, бо пры ёй таксама расьце сымпацыйная актыўнасьць, злучаная зь неабходнасьцю ўзмацніць спаленьне тлушчу.

Як трымацца правіла? Ідэі і парады

Амаладжэньне хваробаў.
Чым большая стымуляцыя mTORС, пачынаючы з разьвіцьця ва ўлоньні маці, тым хутчэй захворваньні праходзяць пэрыяд утоенага разьвіцьця. Гэта абумоўлівае амаладжэньне хваробаў, што мы цяпер назіраем. Тыя захворваньні, якія мы лічылі тыповымі для аднаго ўзросту, сёньня сустракаюцца ў маладзейшым узросьце.

Фізычная актыўнасьць.
Фізычная актыўнасьць вядзе да актывацыі кіназы АМРК, што зьмяншае актыўнасьць бялку. Быць у руху і займацца спортам – гэта добрая ідэя.

Інгібітары mTORС.
Існуюць пэўныя лекі і злучэньні, якія памяншаюць актыўнасьць mTORС. Да іх адносяць зялёную гарбату, рэсвэратрол, фісэтын, квэрцэтын, геністэін, сілімарын, элагавая кіслата і некаторыя іншыя, якія маюцца ў нашых прадуктах харчаваньня.

Адпачынак без стымулятараў.
Цалкам выключыце «хуткую» ежу, кафэін і смартфон у дні адпачынку. Гэта дазволіць адпачыць ня толькі мозгу, але й вашаму абмену рэчываў.

Пазьбягайце прадуктаў mTORС-бомбаў.
Часта ў розных батончыках, нават для здаровага харчаваньня, сустракаецца камбінацыя сухога малака, крухмалу, цукру і да т. п. Па магчымасьці пазьбягайце такіх прадуктаў, замяняючы іх паўнавартаснай ежай.

MTORС-дыеты.
Дыета з абмежаваньнем перапрацаваных вугляводаў, зьніжэньнем долі вугляводаў, павелічэньнем долі тлушчаў, умераным ужываньнем мяса. То-бок, па сутнасьці, мы гаворым пра высокатлушчавую сярэднебялковую нізкавугляводную дыету.

Дзеці растуць.
Існуе міт, што калі дзеці растуць, то яны могуць бескантрольна ўжываць шмат мяса, малака ці кандытарскіх вырабаў. Насамрэч гэта ня так: такая ежа не заўсёды прывядзе да атлусьценьня, але можа выклікаць амаладжэньне хваробаў і павелічэньне іх рызыкі ўжо ў дарослых людзей. Важна думаць пра здароўе зь дзіцячага ўзросту.

\chapter{Захоўваньне і гатоўля}

Правільная гатоўля прадуктаў мае ня меншае значэньне, чым іх выбар. З~аднаго прадукту можна прыгатаваць як адназначна карысную, так і досыць шкодную страву. Мэханічная апрацоўка, тэмпэратура гатоўлі, спалучэньне прадуктаў~--- усё мае значэньне. Цікава, што розныя прадукты часьцяком аказваюць супрацьлеглы ўплыў на здароўе. Напрыклад, вараная рыба зьвязаная са здаровым харчаваньнем, а~вось смажаная~--- са шкодным. Садавіна паляпшае здароўе, а~вось фруктовыя сокі могуць павялічыць рызыку шматлікіх захворваньняў. Смажаныя прадукты ды іх няправільныя спалучэньні зьмяншаюць біялягічную вартасьць ежы, скарачаюць утрыманьне ў~ёй карысных злучэньняў, спрыяюць стварэньню таксычных злучэньняў.

\tipbox{Мэханічная апрацоўка, тэмпэратура гатоўлі, спалучэньне прадуктаў~--- усё мае значэньне. З~аднаго прадукту можна прыгатаваць як адназначна карысную, так і досыць шкодную страву.}

У працэсе гатаваньня нутрыенты ўступаюць у~розныя ўзаемадзеяньні паміж сабой, асаблівае значэньне грае нефэрмэнтатыўная рэакцыя між цукрамі і амінакіслотамі (глікаваньне\index{глікаваньне}), у~выніку гэтай рэакцыі ўтвараюцца так званыя «канчатковыя прадукты глікацыі (КПГ)», якія называюцца AGE («узрост», удалая абрэвіятура Advanced Glycosylation End-products). Калі мы ямо смажаныя прадукты, КПГ трапляюць да нас зь ежай. Чым больш вы зьядаеце КПГ зь ежай, тым вышэйшы іх узровень у~крыві. Акрамя гэтага, яны могуць утварацца непасрэдна і ў~нашым целе, калі парушаецца кантроль узроўню глюкозы і яна падвышаецца. Утварэньне ўласных прадуктаў глікаваньня\index{глікаваньне} залежыць ад адчувальнасьці да інсуліну і глікемічнай нагрузкі дыеты. Але зараз размова пойдзе пра першы мэханізм.

Часам людзі купляюць добрыя прадукты, але гатуюць зь іх нешта ня вельмі здаровае. З~садавіны выціскаюць сок, рыбу засмажваюць да стану вугольля, запякаюць гародніну і садавіну так, што яны становяцца як цукеркі. Давайце разьбяромся, як гатаваць больш ашчадным спосабам для захаваньня карысьці прадуктаў і без утварэньня шкодных злучэньняў падчас гатоўлі.

\subsection{Як зьявілася праблема?}

Складана сказаць, наколькі даўно чалавек навучыўся карыстацца агнём. Гэтае адкрыцьцё зьмяніла нас і дапамагло лягчэй засвойваць ежу. Нават на ўзроўні інстынктаў нас адрозьнівае ад жывёлаў цяга да агню. Наш мозг ведае, што прыгатаваная на агні ежа ўтрымлівае лягчэйшыя да засваеньня калорыі. Нядзіўна, што любая вэнджаная на дыме, засмажаная на грылі страва, ад гародніны да мяса, здаецца нам больш смачнай. Цяпер смажаньне прадуктаў актыўна выкарыстоўваецца як спосаб гатоўлі, але гэта не спрыяе паляпшэньню здароўя. Але смажаная ежа для нас прывабнейшыя, мы ямо патрэбнае значна часьцей, чым трэба. Каля 10\,\% КПГ зь ежы трапляюць у~арганізм.

\tipbox{Нават на ўзроўні інстынктаў нас вылучае цяга да агню. Наш мозг ведае, што прыгатаваная на агні ежа ўтрымлівае лягчэйшыя да засваеньня калёрыі. Нядзіўна, што любая вэнджаная на дыме, засмажаная на грылі страва, ад гародніны да мяса, здаецца нам больш смачнай, чым вараная.}

Праблема захоўваньня сёньня вырашаецца вельмі проста. Калі нашы продкі пераважна ўжывалі лякальныя прадукты, то сёньня тэхналёгіі дазваляюць прывозіць прадукты зь любых куткоў сьвету ці кансэрваваць іх з~дапамогай розных дабавак. На жаль, транспарціроўка не заўсёды ідзе ідэальна, таму сьвежасьць прадуктаў~--- гэта таксама важны крытэр іх якасьці. Забруджваньне прадуктаў рознага кшталту рэчывамі, ад складовых элемэнтаў плястыку да цяжкіх мэталаў, уяўляе сабою сур'ёзную праблему, падступіцца да рашэньня якой не заўсёды магчыма. Тым ня менш ёсьць шэраг спосабаў зьнізіць магчымую шкоду свайму здароўю.

\subsection{Як гэта ўплывае на здароўе?}

Існуе мноства злучэньняў, якія ўтвараюцца пры гатоўлі і валодаюць патэнцыйна нэгатыўным уздзеяньнем на здароўе. Гэта акрыламід\index{акрыламід}, мэтылгліяксальляктоза\index{мэтылгліяксаль}, акралеін\index{акралеін} (пры смажаньні на алеі), гетэрацыклічныя аміны, бэнзапірны\index{бэнзапірын} і шэраг іншых злучэньняў. Глікацыя доўгажывучых бялкоў, міжклеткавага матрыксу, ДНК, мітахандрыяльных бялкоў~--- гэта важны паталягічны працэс, які цяпер павярнуць назад практычна немагчыма. Чым больш вы зьядаеце КПГ зь ежай, тым вышэйшы іх узровень у~крыві. Лішак КПГ павялічвае рызыку атэрасклерозу, сардэчна-сасудзістых захворваньняў, ныркавай недастатковасьці, а~таксама шэрагу аўтаімунных захворваньняў.

\paragraph{Адчувальнасьць да інсуліну.}
Нават умеранае скарачэньне прадуктаў, багатых КПГ, павялічвае адчувальнасьць да інсуліну, зьніжае рызыку цукровага дыябэту\index{дыябэт}.

\paragraph{Запаленьне.}
Малекулярныя мэханізмы дзеяньня КПГ вывучаныя і шмат у~чым вызначаюцца іх узаемадзеяньнем з~клеткавым рэцэптарам да іх з~гаваркой назвай RAGE (нянавісьць) (Receptor of Advanced Glycosylation End-products), актывацыя якіх прыводзіць да павелічэньня запаленчых ачагоў у~нашым арганізьме.

\paragraph{Старэньне скуры, злучальнай тканкі, павышэньне крохкасьці сасудаў.}
Пры глікацыі ўтвараецца шэраг злучэньняў, такіх як пэнтасыдын\index{пэнтасыдын} і карбаксымэтылізін\index{карбаксымэтылізін}. Пры гэтым падае функцыя такіх бялкоў, як эласьцін і каляген. Пры парушэньнях вугляводнага абмену і залішнім спажываньні цукру фармуецца т.~зв. «цукровы твар» (цьмяная, абязводжаная скура зь сеткаю зморшчынаў). Таксама гэта памяншае элястычнасьць крывяносных сасудаў, павялічвае іх крохасьць і пранікальнасьць. Старэньне міжклеткавага матрыксу (рэчывы, якія атачаюць клеткі і ўплываюць на іх функцыі) можа быць адным з~ключавых мэханізмаў разбурэньня арганізму. Выключэньне КПГ з~рацыёну экспэрымынтальных жывёлаў прыкметна падаўжала ім жыцьцё.

\tipbox{Пры парушэньнях вугляводнага абмену і залішнім спажываньні цукру фармуецца т.~зв. «цукровы твар» (цьмяная, абязводжаная скура зь сеткай зморшчынаў). Таксама зьмяншаецца элястычнасьць крывяносных сасудаў, павялічваюцца іх крохкасьць і пранікальнасьць.}

\paragraph{Рызыка пухлінаў.}
Частае спажываньне смажанага мяса павялічвае рызыку раку кішачніка. Нярэдка ўжываньне прыгатаванай такім чынам ежы зьвязанае з~павелічэньнем рызыкі раку грудзей, прадкарэньніцы, лёгкіх, падстраўніцы і страваводу.

\subsection{Асноўныя прынцыпы}

\paragraph{Абмяжуйце прадукты з~высокім утрыманьнем КПГ.}
Больш за ўсё КПГ у~прадуктах, якія набываюць бурае адценьне ў~працэсе гатоўлі, як правіла запяканьня або смажаньня, фастфуд~--- чэмпіён па зьмесьціве КПГ. Гэта і смажанае мяса, бульба, выпечка, птушка, рыба, гародніна і да т.~п. Рэкардсмэнам зьяўляецца смажаны бекон. Вялікая колькасьць КПГ утворыцца пры прамочваньні мяса соўсамі на аснове цукру, пры паніроўцы ці смажаньні з~даданьнем цукру і мукі. Шмат КПГ у~прадуктах, якія зьмяшчаюць карамэлізаваныя злучэньні (награваньне цукру): піва, кола і інш. Асабліва шмат КПГ у~паўфабрыкатах, якія зазнавалі ўздзеяньне высокіх тэмпэратураў.

\paragraph{Тэмпэратура.}
Чым вышэйшая тэмпэратура, тым вышэйшы тэмп утварэньня розных КПГ. Так, утварэньне акрыламіду\index{акрыламід} пачынаецца пры рэакцыі паміж амінакіслатой аспарагінам і цукрамі пры тэмпэратурах вышэй 120°C. Ён маецца ў~выпечцы, шашлыку, смажанай бульбе, чыпсах і т.~п. Пры смажаньні~--- тэмпэратура 225\,°C, запяканьні ў~духоўцы 230\,°C, абсмажцы~--- 177\,°C. У~100 грамах сырой ялавічыне ўтрымліваецца амаль у~9 разоў менш канчатковых прадуктаў глікацыі, чым у~смажанай. У~сырой расьліннай ежы ўтрымліваецца мінімальная колькасьць КПГ з~усіх прадуктаў.

\paragraph{Час.}
Чым даўжэйшы час гатаваньня, тым больш утворыцца канчатковых прадуктаў глікацыі.

\tipbox{Бясьпечней за ўсё гатаваць ежу на пары. А~каб рабіць гэта хутчэй, можна крупы адмочваць папярэдне ў~вадзе, а~мяса~--- марынаваць у~кіслых дадатках: воцаце, цытрынавым соку.}

\paragraph{Гатуйце бясьпечна.}
Варка і гатоўля на пары~--- бясьпечныя спосабы прыгатаваньня ежы. Каб скараціць час варэньня, папярэдне можна заліваць крупы вадой на ноч, а~мяса~--- марынаваць у~кіслых дадатках (воцат, цытрына і да т.~п.). Гародніну трэба варыць альдэнтэ, да крыху цьвёрдага стану, а~не разварваць да жэлепадобнага. Al dente~--- значыць, прадукты, цалкам прыгатаваныя, захоўваюць адчувальную пры ўкусе ўнутраную пругкасьць, або народны тэрмін~--- «каб храбусьцела», то-бок гародніна павінна хрумсьцець на зубах. Пасьля таго як прайшлі 2--3 хвіліны з~часу загрузкі гародніны, пакаштуйце яе. Пасьля варкі можна хутка астудзіць гародніну з~дапамогай халоднай вады (з кубікамі лёду), падаючы гародніну пакаёвай тэмпэратуры.

Ашчадная гатоўля таксама спрыяе захаваньню ніжэйшага глікемічнага індэксу. Іншыя спосабы павольнай гатоўлі: пашыраваньне (павольнае гатаваньне пры тэмпературы 88\,°C), бляншаваньне\index{бляншаваньне}~--- апарваньне або нядоўгая, да хвіліны, варка, гатоўля ў~вакууме, тушэньне, прыпусканьне. Ёсьць адмысловыя павольнаваркі або ціхаваркі для такіх працэдураў.

\paragraph{Больш сырой неапрацаванай ежы.}
Часам людзі кажуць, што гародніна «не працуе»~--- запякаюць яе, здрабняюць у~смузі, смажаць… і ня бачаць карыснага эфэкту. Чаму? Адно з~магчымых тлумачэньняў~--- зьмена яе глікемічнага індэксу (ГІ). Любая апрацоўка~--- награваньне, мэханічнае драбненьне і да т.~п.~--- прыкметна павялічвае ГІ. Напрыклад, крупа аўса мае ГІ 40, каша геркулесавая~--- 60, а~каша хуткага прыгатаваньня~--- усё 80! Уласна, калі доўга апрацоўваць любы прадукт, то ён непазьбежна наблізіцца па сваіх паказчыках да мучных вырабаў. Павелічэньне прапорцыі сырой ежы ў~рацыёне дабратворна адбіваецца на здароўі. Але тут важна таксама дзейнічаць бяз скрайнасьцяў, не сьпяшацца навяртацца ў~сыраедства~--- графік карыснасьці сырых прадуктаў мае U-падобную форму, і занадта вялікая колькасьць сырой ежы пагаршае здароўе. У~сыраедаў часта сустракаецца дэфіцыт шматлікіх карысных злучэньняў і падвышаная рызыка для здароўя.

\subsection{Як трымацца правіла? Ідэі і парады}

\paragraph{Тэмпэратура ежы.}
Ежце ежу пакаёвай тэмпэратуры, ня моцна гарачую. Больш гарачая ежа мае вышэйшы глікеміческій індэкс, а~таксама можа пашкодзіць зубы і ротавую паражніну. Частае ўжываньне ежы з~тэмпэратурай 65\,°C павялічвае рызыку раку страваводу.

\paragraph{Без паніровак і скарыначак.}
Карамэлізацыя, паліваньне соўсамі або марынадамі з~цукрам для скарыначкі, паніроўка сухарамі ды іншае~--- усе гэтыя спосабы шматкроць павялічваюць колькасьць КПГ!

\paragraph{Кантакт з~паветрам.}
Чым большы кантакт з~паветрам, тым больш утвараецца КПГ. Гатуйце з~закрытай накрыўкай. Пазьбягайце адкрытага агню і сухога жару, яны таксама павялічваюць КПГ.

\paragraph{Смажаньне з~тлушчам.}
Смажаньне мяса ў~фрыцюры амаль у~10 разоў павялічвае ўтрыманьне КПГ у~параўнаньні з~варкай.

\paragraph{Дадайце больш вады.}
Вялікая колькасьць вады запавольвае глікацыю.

\paragraph{Цукры.}
Самая актыўная малекула ў~працэсе ўтварэньня КПГ~--- гэта фруктоза, затым ідзе ляктоза\index{ляктоза}, а~вось глюкоза~--- на апошнім месцы. Таму непажадана дадаваць цукар у~прадукты і награваць іх да высокіх тэмпэратураў. Пазьбягайце высокага награваньня пры харчовых спалучэньнях бялок + вугляводы (мука і мяса).

\tipbox{Любая апрацоўка, награваньне, мэханічнае драбненьне і да т.~п., прыкметна павялічвае глікемічны індэкс. Крупа аўса мае ГІ 40, каша геркулесавая~--- 60, а~каша хуткага прыгатаваньня~--- усё 80! Уласна, калі доўга апрацоўваць любы прадукт, ён непазьбежна наблізіцца па сваіх паказьніках да мучных вырабаў.}

\paragraph{Марынады.}
Кіслыя марынады са спэцыямі дапамагаюць скараціць час гатаваньня мяса і амаль у~два разы паменшыць колькасьць КПГ (воцат, лайм, цытрына). У~некаторых краінах рыбу традыцыйна гатуюць без награваньня, марынуючы яе. Але гэта можа быць небясьпечным, улічваючы рызыку паразытаў.

\paragraph{Тлушчы.}
Ня смажце на алеі. Захоўвайце алеі ў~лядоўні, старанна закрывайце іх коркамі. Пазьбягайце выкарыстаньня тлушчаў з~пахам і сьлядамі згарчэласьці, купляйце сьвежыя (сачыце за тэрмінам прыдатнасьці). Чым даўжэй захоўваецца ў~вас алей ці тлушч, тым яны больш небясьпечныя.

\paragraph{Глыбокая замарозка.}
Замарозка~--- выдатны спосаб захоўваньня, які дапамагае захаваць большую частку карысных злучэньняў. Замарожаная гародніна хоць і губляе шэраг карысных уласьцівасьцяў, але большая іх частка застаецца.

\paragraph{Ня ежце падчас гатаваньня.}
Так вы можаце перабіць апэтыт і зьесьці неўзаметку для сябе залішнюю колькасьць калёрыяў. Пакаштавалі і выплюнулі.

\paragraph{Аглядайце прадукты.}
Дбайна аглядайце прадукты на прадмет цьвілі, пры яе выяўленьні выкідайце прадукт цалкам.

\paragraph{Замочвайце крупы, збажыну і бабовыя.}
Замочваньне дапамагае паменшыць утрыманьне фіцінавай кіслаты\index{фіцінавая кіслата}\index{кіслата!фіцінавая}, лішак якой можа зьмяншаць якасьць усмоктваньня мікраэлемэнтаў. Крупы патрабуюць меншага часу варкі і даюць меншую глікемічную нагрузку пры папярэднім замочваньні. Напрыклад, грэчку пасьля замочваньня будзе дастаткова давесьці да кіпеньня. Замочваючы, а~затым зьліваючы ваду, мы зьмяншаем канцэнтрацыю магчымых забруджваньняў. Небясьпеку для здароўя ўяўляюць часта нябачныя для вока цьвіль, плесьня, якія ўтрымліваюць мікатаксыны\index{мікатаксыны}. Многія вядомыя традыцыйныя культуры (індзейцы майя~--- кукуруза, азіяты~--- соя, еўрапэйцы~--- пшаніца) не гатавалі збажыну з~дапамогай варкі, а~паляпшалі яе карысныя ўласьцівасьці замочваньнем, прарошчваньнем або фэрмэнтацыяй.

\paragraph{Мыйце гародніну і садавіну.}
Часта садавіна пакрываюць парафінам для павелічэньня тэрміну прыдатнасьці, апрацоўваюць фунгіцыдамі ад цьвілі. Гэтыя рэчывы могуць быць небясьпечныя для здароўя. Прамываньне прадуктаў праточнай вадой~--- гэта просты спосаб ліквідаваць рэшткі такіх злучэньняў. Салёная вада спраўляецца з~гэтай задачай яшчэ лепей. Некаторую зеляніну, напрыклад лісьце салаты складанай формы, складана старанна прамыць, таму нават пры найменшых сумневах у~яе біялагічнай бясьпецы пазьбягайце пакупкі.

\paragraph{Памяншайце ўзровень КПГ.}
У нашым арганізьме ёсьць адмысловая гліяксалазная сыстэма, якая дапамагае паменшыць узровень КПГ, яе стымулятарам зьяўляецца сульфарафан\index{сульфарафан} з~брокалі. Але працуе толькі сырая брокалі!

\paragraph{Ежце зь мясам шмат зеляніны і гародніны.}
Вялікая колькасьць зеляніны, перцу, таматаў, часныку дапамагае абясшкодзіць многія небясьпечныя злучэньні.

\tipbox{Захоўвайце алеі ў~лядоўні, старанна закрывайце іх коркамі. Пазьбягайце выкарыстаньня тлушчаў з~пахам і сьлядамі ёлкасьці, купляйце сьвежыя. Чым даўжэй захоўваюцца алей ці тлушч, тым яны больш небясьпечныя.}

\paragraph{Дадавайце спэцыяў.}
Шмат якія спэцыі абараняюць прадукты і падчас гатоўлі, бо зьяўляюцца магутнымі антыаксыдантамі: размарын, гвазьдзік, куркума і інш.

\paragraph{Зразайце КПГ.}
Зразайце перасмажаныя ці цёмныя кавалкі мяса перад ядой, ня ежце скуру птушак.

\paragraph{Здаровы глузд і балянс.}
Некаторыя рэчывы, якія даюць цёмныя адценьні (мэляідыны), у~невялікіх канцэнтрацыях могуць быць і карысныя для арганізму, таму ня кідайцеся ў~скрайнасьці. Спажываньне смажаных прадуктаў раз на тыдзень і радзей не нясе аніякае рызыкі паводле дасьледаваньняў.

\paragraph{Тэрмаапрацоўка карысная.}
Часта важна праварыць прадукты, асабліва мясныя і рыбу, каб забіць магчымых паразытаў у~ёй. Сырая рыба і недаваранае мяса павялічваюць рызыку заражэньня.

\paragraph{Складанасьць у~ачыстцы.}
Некаторую гародніну зь зялёным лісьцем складанай формы цяжка ачысьціць дакладна, а~яны могуць утрымліваць небясьпечныя бактэрыі. Пры сумневе ў~якасьці пазьбягайце яе купляць.

\paragraph{Не захоўвайце доўга.}
Калі доўга захоўваць ежу, узьнікае рызыка размнажэньня ў~ёй бактэрыяў. Тэмпэратуры ніжэй за 5 і вышэй за 60\,°C спыняюць або запавольваюць рост мікраарганізмаў.

\paragraph{Сьвежасьць.}
Сьвежасьць прадуктаў~--- гэта ключ да здароўя. Найбольш карысным будзе захаваньне сьвежасьці пры харчовых алергіях, асабліва пры непераноснасьці гістаміну. Біягенныя аміны\index{біягенныя аміны} назапашваюцца ў~прадуктах, што схільныя да хуткага закісаньня, гніеньня, перасьпяваньня (асабліва калі гэтыя прадукты вязуць здалёку). Пазьбягайце прадуктаў зь невыразным тэрмінам захоўваньня або зь перапыненым ланцужком астуджэньня (сьляды шматразовай замарозкі).

\chapter{Смачная ежа}

Смачная ежа~--- гэта важны кампанэнт нашага здароўя, як фізычнага, так і псыхалягічнага. Падабаецца гэта нам ці не, але мы створаныя, каб шукаць смачную ежу, яна цудоўная крыніца энэргіі і гарантыя выжываньня. На жаль, цяпер надмер смачнай ежы працуе для нас як пагроза здароўю. Але дасьледаваньні паказваюць, што абмежавальныя дыеты маюць дастаткова нізкую эфэктыўнасьць у~доўгатэрміновай пэрспэктыве, жорсткія харчовыя забароны могуць толькі ўзмацняць цягу да ежы, а~пазбаўленьне сябе смачных прадуктаў павялічвае імавернасьць зрываў. Таму смачная ежа спрыяе большаму задавальненьню і задаволенасьці, гэта кядзе да памяншэньня спажываных калёрыяў і большага псыхалягічнага камфорту.

\subsection{Як зьявілася праблема?}

Першапачаткова смак дапамагаў нам пазьбягаць небясьпечнай ежы і знаходзіць больш пажыўную. Зь цягам часу гэты працэс ускладніўся, таму давайце вызначымся з~паняцьцямі больш дакладна. Вельмі часта мы блытаем два розныя паняцьці, калі гаворым пра адзнаку ежы: цяга і смак. Багата ў~каго наогул уся ежа спрашчаецца да «гэта смачна», «гэта нясмачна», што вельмі кепска для харчовых паводзінаў, бо цягне толькі на вельмі смачнае. Чым больш ежа выбівае з~вас дафаміну, тым мацней вас да яе цягне. Прытым смак гэтай ежы можа быць агідны.

Наш мозг цягнецца да ежы з~высокай даступнасьцю калёрыяў і іншых неабходных кампанэнтаў. Таму для нас будзе прывабнай ежа з~высокай удзельнай калярыйнасьцю, высокім глікемічным індэксам і глікемічнай нагрузкай, вялікім утрыманьнем амінакіслотаў з~разгалінаваным ланцугом (лейцын і г.~д. таксама выклікае прыкметны ўздым інсуліну), больш салёная, смажаная. Таму, калі мы возьмем шмат мукі, дадамо цукру, тлушчу, солі, малака і падсмажым да скарыначкі, наша дафамінавая сыстэма відавочна будзе захопленая і дэзарыентаваная. Чым вышэйшая ўдзельная шчыльнасьць калёрыяў, тым прывабнейшая ежа. Чым вышэйшы ўзровень інсуліну, тым мацнейшая дафамінавая стымуляцыя. Чым вышэйшы стрэс, тым болей трэба зьесьці ў~запас. Гэта сапраўдная дафамінавая цяга, і калі мы гаворым «смачна ці не», дык маем на ўвазе часьцей за ўсё суб'ектыўную цягу, а~ня смак. Так-так, суб'ектыўную, якая мяняецца ў~залежнасьці ад вашага стану: раніцай нясмачна, уначы~--- вельмі смачна.

\tipbox{Чым большая ўдзельная шчыльнасьць калёрыяў, тым прывабнейшая ежа. Чым вышэйшы ўзровень інсуліну, тым мацнейшая дафамінавая стымуляцыя. Чым вышэйшы стрэс, тым больш трэба зьесьці ў~запас. Гэта дафамінавая цяга, і калі мы гаворым «смачна ці не», то мы маем на ўвазе часьцей за ўсё суб'ектыўную цягу, а~ня смак.}

Так фармуецца харчовая залежнасьць, бо чым больш чалавек есьць для задавальненьня, тым больш ежы трэба. Зь цягам часу, як і пры іншых залежнасьцях, узьнікае жаданьне павялічыць дозу смачнага. Адмовіцца ад усяго смачнага таксама вельмі цяжка, бо ежа, у~адрозьненьне ад курэньня, жыцьцёва важны працэс. Рашэньнем можа быць разьвіцьцё смаку і навыкаў атрыманьня некалярыйнага задавальненьня.

Вядома, чалавек мае й~больш складаныя сыстэмы кіраваньня смакам. Культурніцкі смак~--- гэта ўменьне адрозьніваць і сьвядома ідэнтыфікаваць (называць) аб'ектыўныя прыкметы прадукту на аснове фізыялёгіі нашых ворганаў пачуцьцяў. Смак узьнікае ня сам па сабе, а~толькі ў~выніку сэнсарнай адукацыі. Смак~--- гэта эстэтычная катэгорыя, якая характарызуецца выбіральнасьцю і абазначае наяўнасьць уласных перавагаў і меркаваньняў. Смак~--- гэта калі мы кажам: «М-м-м... гэтая брокалі мае лёгкі лятучы ялкава-рэзкі травяны смак». Смак патрабуе ўсьвядомленасьці і накіраванай увагі на прадукты. Чым лепей будзе ў~вас разьвіты смак, тым лягчэй вы будзеце кіраваць сваёй цягай і тым здаравейшым будзе ваш балянс «падабаецца~--- хачу».

Вельмі часта ў~шматлікіх людзей зь пераяданьнем фармуецца своеасаблівая смакавая фрыгіднасьць, калі ім цяжка адрозьніваць смакавыя адценьні, і яны імкнуцца атрымаць задавальненьне, мэханічна набіваючы сабе жывот, кампэнсуючы дэфіцыт смаку лішкам цукру, глутамату, солі і калёрыяў у~цэлым. Нізкая смакавая адчувальнасьць шчыльна зьвязана з~аўтаматычным паглынаньнем, нізкім узроўнем усьвядомленасьці. Цяпер мы сутыкаемся з~тым, што розныя прадукты маюць аднолькавы смак, сфармаваны надмерам падсалоджвальнікаў, узмацняльнікаў смаку. Залішняя стымуляцыя вядзе да памяншэньня нашай смакавай адчувальнасьці, што робіць простую ежу нясмачнай і стымулюе пераяданьне. Калі салодкі смак любяць і нованароджаныя, то смак да больш складаных~--- кіслых, горкіх і рэзкі~--- адценьняў выхоўваецца, роўна як і ўменьне разьбірацца ў~ежы і нюансах смаку. Пры недастатковым выхаваньні, зь нізкай культурай харчаваньня гэтыя навыкі адсутнічаюць, што можа пагаршаць праблему пераяданьня.

\subsection{Як гэта ўплывае на здароўе?}

\paragraph{Харчовая залежнасьць.}
Ежа, якая ўтрымлівае вялікую колькасьць крухмалу, тлушчу, цукраў, солі, стымулюе высокі выкід нэўрамэдыятара дафаміну, што можа пры пэўных умовах прыводзіць да разьвіцьця харчовай залежнасьці, якая вядомая нам ужо сотні гадоў і раней звалася абжорствам. Умеранасьць, адсутнасьць забароны і ўменьне атрымліваць дастаткова задавальненьня ад ежы~--- гэта важныя ўмовы захаваньня здаровых харчовых паводзінаў. Розныя харчовыя абмежаваньні, нэгатыўны вобраз цела ды іншыя фактары могуць прыводзіць да разнастайных парушэньняў харчовых паводзінаў, ад кампульсіўнай яды (цыклі «ўстрыманьне~--- абжорства») да анарэксіі. У~цэлым пры сындроме дэфіцыту ўзнагароджаньня, калі ў~жыцьці нас мала што радуе, мы часта зьвяртаемся да ежы~--- і гэта ня вельмі здаровае рашэньне.

\tipbox{Ежа, якая ўтрымлівае вялікую колькасьць крухмалу, тлушчу, цукраў, солі, стымулюе высокі выкід нэўрамэдыятара дафаміну, што можа пры пэўных умовах прыводзіць да разьвіцьця харчовай залежнасьці, якая вядомая нам ужо сотні гадоў і раней звалася абжорствам.}

\paragraph{Пераяданьне і выкліканыя ім разлады.}
Чым менш задавальненьня мы атрымліваем ад ежы, тым больш пераядаем і тым вышэйшая рызыка шматлікіх праблемаў, выкліканых залішнім спажываньнем калёрыяў (ад атлусьценьня і дыябэту да нэўрадэгенэратыўных і аўтаімунных захворваньняў). Таму задавальненьне і задаволенасьць ад смачнай ежы зьяўляюцца залогам стрыманасьці ў~ежы. Смачная ежа дазваляе больш эфэктыўна прытрымлівацца свайго харчовага пляну.

\paragraph{Здаровы выбар.}
Дастатковую сэнсарную адукацыю і разьвіцьцё смаку спрыяюць больш здароваму выбару прадуктаў, пры якім вы выбіраеце больш цэльных і сьвежых прадуктаў і зьмяншаеце колькасьць нездаровых, пазьбягаеце сапсаваных. Таму разьвіты смак~--- гэта абавязковая ўмова для правільнага харчаваньня ў~доўгатэрміновай пэрспэктыве.

\subsection{Асноўныя прынцыпы}

Правіла смачнай ежы ў~тым, каб атрымліваць больш задавальненьня ад яды, як разьвіваючы смак і ўсьвядомленасьць, так і павялічваючы колькасьць некалярыйнага задавальненьня (рытуал прыёму ежы, посуд, прыборы і інш.).

\paragraph{Сьвядомае харчаваньне.}
Чым больш увагі вы зьвяртаеце на ежу, тым яна смачнейшая! Якія кавалачкі марозіва самыя смачныя? Першы і апошні, але не праз свой асаблівы склад, а~таму што вы больш за ўсё зьвяртаеце на іх увагі. Атрымлівайце асалоду ад выгляду і водару кожнага кавалачка ежы, які кладзяце ў~рот. Сьвядомасьць у~ядзе~--- гэта пастаяннае, бесьперапыннае адсочваньне сваіх адчуваньняў у~сапраўдным моманце, не адцягваючыся на мінулае і не фантазуючы пра будучыню. Важна адсочваць смак, колер, кансыстэнцыю, тэмпэратуру, глейкасьць і іх зьмену падчас жаваньня. Навучыцеся ў~дэталях апісваць усё, што адчуваеце на смак, пры гэтым вызначайце аб'ектыўныя факты, а~не суб'ектыўна ацэньвайце свае адчуваньні. Назіраньне за сабой дапаможа вам заўважыць сваю прагу і пасьпешлівасьць у~ядзе і запаволіцца. Сьвядома засяроджвайцеся на ежы~--- гэта галоўнае ва ўсьвядомленасьці.

\paragraph{100\,\% задавальненьня.}
Зьмена рацыёну зьвязаная са зьніжэньнем надмеру салодкіх, салёных і гіпэркалярыйных прадуктаў, што непазьбежна прывядзе да зьніжэньня агульнай колькасьці задавальненьня, якое вы атрымлівацьмеце ад ежы. Гэта выкліча жаданьне зьесьці нешта яшчэ. Каб пазьбегнуць адхіленьняў ад свайго пляну харчаваньня, загадзя заплянуйце іншыя крыніцы задавальненьня, як зьвязаныя зь ежай (сэрвіроўка, разьвіцьцё смаку, сьвядомае харчаваньне і інш.), так і не зьвязаныя (прагляд сэрыялу і да т.~п.). Зь цягам часу смакавая фіксацыя на ежы аслабне, і вы будзеце пачувацца вальней. Чым больш вы атрымліваеце некалярыйнага задавальненьня ад ежы, тым меншай будзе цяга да пераяданьня: згадайце, якая смачная звычайная салата ў~шыкоўнай рэстарацыі!

\paragraph{Разьвіцьцё смаку (сэнсарная адукацыя)~---} гэта навык актыўнага распазнаньня і апісаньня аб'ектыўных характарыстык ежы (смак, выгляд, пах, гук і да т.~п.).
Чым багацейшыя вашыя смакавыя адчуваньні, тым больш разьвіты ваш мозг, тым больш вы атрымліваеце ўражаньняў і сэнсарных стымулаў. Вывучэньне смакаў пачынаецца з~пашырэньня слоўнікавага запасу і ўменьня правільна яго выкарыстоўваць. Навучыцеся апісваць уласныя адчуваньні правільна. Так, з~дапамогай зроку мы можам апісаць паверхню прадукту (варсістая, гладкая, пузырыстая і інш.), можам апісаць колер паверхні, яе аднастайнасьць, сьвятло і г.~д. Мы можам назваць пах, даючы характарыстыкі яго сіле, устойлівасьці, адценьням. І, вядома ж, вызначыць базавыя смакі: салодкае, кіслае, салёнае і горкае. Кожны з~гэтых смакаў мае свае адценьні. Так, саладосьць можа быць рэзкая, прыемная, прыкрыя. А~кіслотнасьць апісваецца шасьцю тыпамі: рэзкая воцатная, малочная, бурштынавая, цытрынавая, яблычная і вінна-каменная. Ацаніце і структуру ежы ў~роце, яе хрумсткасьць. Кансыстэнцыя можа быць клейкай, рассыпістай, вязкай, ахінальнай і да т.~п. Смакі ўзаемадзейнічаюць міжсобку, яны могуць узмацняць адзін аднаго, перакрываць і т.~п. Узбагачайце ваш слоўнікавы запас!

\tipbox{Сэнсарная адукацыя і разьвіцьцё смаку, у~прыватнасьці,~--- гэта навык актыўнага распазнаньня і апісаньня аб'ектыўных характарыстык ежы. Чым багацейшыя вашыя смакавыя адчуваньні, тым больш разьвіты ваш мозг, вы атрымліваеце больш уражаньняў і сэнсарных стымулаў.}

\subsection{Як трымацца правіла? Ідэі і парады}

\paragraph{Задавальненьне ад сталаваньня~--- у~галаве, а~ня ў~роце.}
Удзельнікам экспэрымэнту давалі адно і тое ж віно, паведамлялі розны кошт, з~дапамогай тамографа вымяраючы актыўнасьць цэнтраў смаку і задавальненьня. Аказалася, што чым даражэйшае віно, тым больш задавальненьня ад яго атрымлівалі падыспытныя, незалежна ад смаку. Гэты прынцып працуе і ў~іншых выпадках: так, кошт банкі энэргетыку ўплывае на хуткасьць рашэньня задач. Чым больш вы ведаеце і знаходзіце ўнікальных характарыстык прадуктаў у~сваім сталаваньні, тым яны здаюцца вам смачнейшымі.

\paragraph{Музыка.}
Музыка прыкметна ўплывае на смакавыя адчуваньні. Прыемная~--- можа палепшыць смак. Экспэрымэнтуйце з~жанрамі і падбярыце аптымальныя для сябе рашэньні.

\paragraph{Чысты стол.}
Трымайце на стале толькі тое, што мае дачыненьне да прыёму ежы. Прыбярыце тэлефон з~поля зроку, не ўключайце тэлевізар і радыё.

\paragraph{Усё, што па-за талеркай.}
Кухня, асяродзьдзе, від з~акна і г.~д.~--- усё гэта ўплывае на вашы смакавыя адчуваньні. Агледзьцеся па баках, магчыма, нешта замінае вашаму задавальненьню.

\paragraph{Посуд.}
Выбірайце самы дарагі і прыгожы посуд і прыборы, якія можаце сабе дазволіць. Есьці зь іх дапамогай тады стане сапраўднай асалодай.

\paragraph{Камунікацыя.}
Ежце побач зь іншымі людзьмі. Аксытацын стымулюе выкід дафаміну і робіць ежу менш важнай. Прыцягвайце сямейнікаў да гатоўлі страваў: уцягнутасьць робіць ежу смачнейшай.

\paragraph{Рытуал.}
Прыдумайце розныя правілы і рытуалы ежы. Выразны парадак рыхтуе ваш стрававальны тракт да ежы.

\paragraph{Афармленьне страваў.}
Навучыцеся прыгожа афармляць стравы, цяпер для гэтага ёсьць мноства формаў і спосабаў. Зьвярніце ўвагу і на традыцыйныя мастацтвы іншых краін, напрыклад бэнто. Выкарыстоўвайце соўсы, ягады, аліўкі, зеляніну для афармленьня страваў, выкладвайце іх у~розныя формы.

\paragraph{Абрус, сурвэткі.}
Іншыя ўпрыгожваньні стала~--- сьвечкі, сурвэткі і абрус. Накрываючы стол абрусам, мы прывучаем сябе есьці толькі за пакрытым сталом і ня есьці ў~іншы час, гэта добрая традыцыя і харчовая звычка.

\paragraph{Табліцы ацэнкі смаку.}
Запампуйце і раздрукуйце традыцыйныя табліцы па ацэнцы смаку гарбаты, кавы, аліўкавага алею і інш., паспрабуйце адрозьніваць смакі розных гатункаў.

\tipbox{Горкі смак тармозіць апэтыт і спрыяе лепшаму насычэньню, бо падчас эвалюцыі мы страцілі частку рэцэптараў да горкага, дзякуючы чаму змаглі есьці больш відаў расьлінаў.}

\paragraph{Спэцыі.}
Выкарыстоўвайце больш спэцыяў, яны дадаюць разнастайнасьці й~індывідуальнасьці гатаваньню, мяняюць пахі, даюць іншыя колеры, паляпшаюць якасьць страваў і колькасьць зьедзенага.

\paragraph{Экспэрымэнты.}
Экспэрымэнтуйце зь ежай і сваімі ворганамі пачуцьцяў. Паспрабуйце есьці рукамі, палачкамі, есьці з~заплюшчанымі вачыма.

\paragraph{Незвычайнае месца.}
Ежа ў~незвычайным месцы заўсёды смачнейшая. Напрыклад, какава ў~тэрмасе ў~парку або рэстарацыя ў~новым раёне.

\paragraph{Белыя талеркі.}
Белы колер аптымальны для ўспрыманьня ежы. А~вось прадукты, якіх вы хочаеце зьесьці менш, лепей класьці на чырвоную талерку.

\paragraph{Цяжкія талеркі і прыборы.}
Чым цяжэйшыя талеркі і прыборы, тым хутчэй мы насычаемся, бо мозгу складаней адрозьніваць вагу прадукту і талеркі.

\paragraph{Называйце сваю ежу.}
Чым смачнейшую і апэтытнейшую назву сваёй страве вы дасьце, тым болей задавальненьня атрымаеце. Не пакідайце вашую гатоўлю безназоўнай, а~давайце ёй яркія назвы! Напрыклад, ``чароўная сьвежая капуста з~градкі бабулі Сьцепаніды, прытушаная з~духмяным топленым вясковым сьметанковым маслам''.

\paragraph{Пашырайце досьвед.}
Дасьледаваньні паказалі, што, калі вы глядзіце прыгожыя фота ежы, можаце зьмяніць стаўленьне да гэтых прадуктаў. Так, гародніна, выдатна пададзеная, падасца вам нашмат смачнейшай.

\paragraph{Горкі смак.}
Цікава, што падчас эвалюцыі людзі страцілі частку рэцэптараў да горкага смаку, дзякуючы чаму змаглі есьці болей відаў расьлінаў. Зьвярніце ўвагу, што горкі смак тармозіць апэтыт і спрыяе лепшаму насычэньню. Пры гэтым кампанэнты ежы, якія даюць горкі смак, часта зьяўляюцца біялягічна актыўнымі рэчывамі (антыаксыданты, біяфлаваноіды і інш.). Горкі мае мноства адценьняў, якімі мы можам насалоджвацца. Гэта дазваляе дэгустатарам вылучаць цудоўныя смакі і пахі кавы, чырвонага віна, чакаляды. Цікава, што смак да горкага выхоўваецца з~узростам, дзіцяці горкае вельмі не падабаецца. Дадаючы больш горкага, ад кавы да рэдзькі, хрэну, горкае чакаляды, мы можам разьвіваць свой смак і трэніраваць рэцэптары.


\part{Прадукты}

\chapter{Вугляводы}

Вугляводы ўяўляюць сабой разнастайную групу, у якой прысутнічаюць як карысныя, так і шкодныя прадукты. Розныя дыеты могуць мець розную долю вугляводаў, ад 85% да вельмі нізкіх значэньняў. Парушэньні вугляводнага абмену ляжаць у аснове шматлікіх захворваньняў, таму правільны харчовы выбар зьяўляецца добрай прафіляктыкай.

Многія людзі ў наш час дэманізуюць вугляводы, абвінавачваючы іх ва ўсіх бедах і пазьбягаючы любой цаной. Часта можна пачуць забаўныя гісторыі, як кавалачак хлеба сапсаваў камусьці ня толькі фігуру, але і жыцьцё. Але насамрэч вугляводныя прадукты – гэта вельмі разнастайная харчовая група, таму трэба навучыцца абыходзіцца зь імі выбіральна. Асобнай увагі заслугоўвае і пытаньне цукру.

\section{Як зьявілася праблема?}

Праблема ўжываньня цукру бярэ свой пачатак у нашай эвалюцыі, мільёны гадоў таму. Навукоўцы выявілі, што падчас аднаго глябальнага пахаладаньня ў нас у клетках адбылася мутацыя, якая паспрыяла павелічэньню ўзроўню мачавой кіслаты пры спажываньні садавіны, запаволеньню яе вывядзеньня. У норме мачавая кіслата ўтвараецца пры мэтабалізьме фруктозы, а яе больш высокія ўзроўні зьмяншаюць адчувальнасьць да інсуліну і спрыяюць набору вагі.

У прыродзе колькасьць цукру зьвязаная з паравінамі году і павялічваецца восеньню, за якой ідзе зіма. Таму здольнасьць набіраць больш тлушчу ад фруктозы дапамагала нашым продкам выжыць. З гэтым жа зьвязаны й факт, што фруктоза адключае сытасьць і павялічвае апэтыт, бо для выжываньня трэба набраць як мага больш тлушчу. Такім чынам, раней гэта была карысная ўласьцівасьць арганізму, а цяпер фруктоза болей нам шкодзіць, бо лішак тлушчу і падвышаны апэтыт толькі павялічваюць хваробы і зьмяншаюць працягласьць жыцьця. Шкаднейшая за ўсё фруктоза пры пераяданьні.

Нашаму арганізму эвалюцыйна карысьнейшыя «клеткавыя» вугляводы, прадстаўленыя клубнямі, лісьцем, гароднінай і садавіной. «Клеткавыя» вугляводы знаходзяцца ў жывых клетках і пры трапленьні ў арганізм павольна руйнуюцца.

Што да крухмалаў, то яшчэ дзесяць тысячаў гадоў таму людзі пачалі генэтычна прыстасоўвацца да яго спажываньня. Так, ёсьць ген AMY1, які кадуе фэрмэнт альфа-амілазу, што расшчапляе крухмал. Дык вось тыя народы, якія здаўна займаюцца сельскай гаспадаркай, маюць больш копіяў гену (аж да 20), а іншыя – усяго некалькі. Чым менш копіяў гена, тым вышэйшая рызыка атлусьценьня. Таму тыя народы, што даўно вырошчваюць крухмалістыя прадукты, лягчэй іх і пераносяць.

\subsection{Рафінаваньне прадуктаў.}
У разьдзеле пра ўдзельную шчыльнасьць калёрыяў мы ўжо згадвалі, што спажываньне вялікай колькасьці вугляводаў у выглядзе цэльнай гародніны, садавіны і зеляніны карыснае для здароўя, а вось вычышчаныя мучныя вырабы аказваюць нэгатыўны ўплыў. Нашаму арганізму эвалюцыйна карысьнейшыя "клеткавыя" вугляводы, прадстаўленыя клубнямі, лісьцем, гароднінай і садавіной. "Клеткавыя" вугляводы знаходзяцца ў жывых клетках і пры трапленьні ў арганізм павольна руйнуюцца і вызваляюцца.

Сёньня мы ямо занадта шмат «няклеткавых», ці «сучасных» вугляводаў, якія атрымліваюцца з мэханічна і тэрмічна апрацаванай збажыны, бабовых і інш. Такія вугляводы расшчапляюцца значна хутчэй і ствараюць больш высокую канцэнтрацыю як у ротавай поласьці, так і ў кішачніку. Яны пагаршаюць кішачную мікрафлёру, спрыяюць больш выяўленаму запаленчаму адказу арганізму, моцна нагружаюць сыстэму рэгуляцыі вугляводнага абмену, зьмяншаюць адчувальнасьць да інсуліну. Цяпер людзі ядуць усё менш цэльнай расьліннай ежы і ўсё больш перапрацаванай у выглядзе вытворных мукі, соевага парашку, высушанага бульбянога крухмалу. Такія прадукты нашмат шкаднейшыя, чым цэльныя расьлінныя. Што да іншых аспэктаў, дык важна зьвярнуць увагу, што вугляводы лепш засвойваюцца на тле больш высокага ўтрыманьня вітаміну D, як гэта бывае ўлетку.

\subsection{Фітанутрыенты і харчовыя валокны.}
Важны ўплыў аказвае на наша здароўе і недахоп розных фітанутрыентаў. Нізкакрухмалістая расьлінная ежа багатая рознымі злучэньнямі з шырокім профілем біялягічнай актыўнасьці, вітамінамі, антыаксыдантамі. Многія з гэтых рэчываў зьяўляюцца сэналітыкамі (забіваюць старыя клеткі) і герапратэктарамі (запавольваюць старэньне).

Шмат у такой ежы яшчэ і рознай клятчаткі, як нераспушчальнай (лігнаны), так і распушчальнай (пэктыны і г. д.). Разнастайнасьць харчовых валокнаў важняа для падтрыманьня мікрафлёры кішачніка, бо здаровае харчаваньне – гэта неад'емная частка здаровага страваваньня. У рафінаванай вугляводнай ежы гэтых кампанэнтаў мала, а вось у цэльнай гародніне, садавіне і зеляніне іх утрымліваецца дастаткова.

\section{Як гэта ўплывае на здароўе?}

\subsection{Цукар.}
Залішняе спажываньне цукру зьвязанае з прыкметным павелічэньнем рызыкі шматлікіх захворваньняў. Малекула цукрозы складаецца з глюкозы і фруктозы. Лішак фруктозы ў спалучэньні з падвышаным спажываньнем калёрыяў шкодзіць як наўпрост, так і ўскосна, павялічваючы імавернасьць атлусьценьня. Лішак цукру павялічвае рызыку інсулінарэзыстэнтнасьці, мэтабалічнага сындрому, цукровага дыябэту другога тыпу, карыесу, тлушчавай дыстрафіі печані, парушае ліпідны профіль крыві і паскарае атэрасклероз. Таксама ён адмоўна ўплывае на работу мозгу, зьніжае ўтварэньне новых нэўронаў, павялічвае трывогу, рызыку дэпрэсіі і павышае рызыку хваробы Альцгеймэра. Лічыцца, што ня менш за траціну выпадкаў сындрому раздражнёнага кішачніка зьвязана з залішнім спажываньнем цукру.

Ускосная шкода ў тым, што цукар можа фармаваць харчовую залежнасьць, павялічваючы ўзровень дафаміну і спрыяючы пераяданьню. Фруктоза нэгатыўна ўплывае на нашыя харчовыя паводзіны, павышаючы ўзровень грэліну і зьніжаючы адчувальнасьць да лептыну. Такое яе дзеяньне вядзе да набору масы цела. У тых людзей, хто атрымлівае 25% калёрыяў з цукру, рызыка сардэчна-сасудзістых захворваньняў удвая вышэйшая, чым у тых, хто атрымлівае зь яго менш за 10% калёрыяў. А вось ужываньне больш за адну порцыю салодкага напою ў дзень прыводзіць таксама да двухразовай рызыкі цукровага дыябэту. Ва ўсіх выпадках гаворка ідзе пра дададзены цукар і ў соках. Адмова ад цукру станоўча ўплывае на стан здароўя нават бязь зьмены вагі.

Надмер цукру ў рацыёне адмоўна ўплывае на работу мозгу, зьмяншае ўтварэньне новых нэўронаў, павялічвае трывогу, рызыку дэпрэсіі і хваробы Альцгеймэра.

\subsection{Лішак канцэнтраваных вугляводаў.}
Як паказваюць дасьледаваньні, важнае значэньне мае ня проста лішак вугляводаў, а лішак перапрацаваных вугляводаў з высокай шчыльнасьцю, якія моцна ўплываюць на ўзровень глюкозы ў крыві (вымяраецца глікемічным індэксам), інсуліну (інсулінавы індэкс), маюць высокую глікемічную нагрузку. Звычайна ў такіх выпадках гаворка ідзе ня проста пра расьлінныя прадукты, а пра моцна перапрацаваныя.

Частае ўжываньне вугляводаў, асабліва ў вялікіх порцыях, перапрацаваных, суправаджаецца падвышаным выдзяленьнем інсуліну. У цэлым, чым ніжэйшы ўзровень інсуліну і вышэйшая да яго адчувальнасьць, тым павольнейшае старэньне і лепшы стан здароўя. Інсулінарэзыстэнтнасьць нават без атлусьценьня зьвязаная з падвышанай рызыкай раку, гіпэртэнзіі ды іншых захворваньняў. Лішак вугляводаў з высокім глікемічным індэксам зьвязаны з больш высокім узроўнем запаленьня і дэпрэсіі. Дасьледаваньні паказваюць, што найбольш высокае ўжываньне такіх вугляводаў на 30% павышае рызыку сьмяротнасьці ў параўнаньні з самым нізкім.

Фітанутрыенты (фітахімічныя рэчывы) – гэта шматлікія біялягічна актыўныя спалучэньні, якія не зьяўляюцца незаменнымі для чалавека, аднак маюць станоўчы ўплыў на здароўе, зьніжаюць рызыку многіх хранічных захворваньняў (сардэчна-сасудзістых, мэтабалічных, анкалягічных і да т.п.). Гэтыя спалучэньні знаходзяцца пераважна ў расьлінах, якія выпрацоўваюць іх як для абароны ад акісьленьня, шкоднікаў, так і для забесьпячэньня колеру, смаку і водару. У сьвеце вядома больш за трыццаць тысячаў такіх рэчываў. У цэльнай расьліннай ежы іх шмат, а вось у рафінаванай – вельмі мала. У порцыі гародніны можа быць да сотні розных фітанутрыентаў!

\section{Асноўныя прынцыпы}

\subsection{Зьмяншайце колькасьць цукру.}
Нормай дададзенага цукру зьяўляецца лічба ў 5% ад агульнай колькасьці калёрыяў, што складае каля 25 грамаў, або 6 лыжак цукру, якія цалкам могуць утрымоўвацца ў выглядзе ўтоенага цукру ў хлебе, соўсе ці паўфабрыкатах. Абмяжуйце дададзены цукар – і вы можаце прыкметна скараціць рызыкі для здароўя. Зьвярніце ўвагу, што як цукар, так і фруктоза біяэквівалентныя, то-бок цукар зь мёду ці соку дзейнічае на арганізм сапраўды гэтак жа, як і звычайны цукар. Таму важна сачыць і за агульнай колькасьцю цукру на дзень з розных крыніц! Цукар хаваецца за рознымі назвамі: глюкозна-фруктозны сыроп, «сыроп фруктозы», ГФС, HFCS, GFS, фінікавы цукар, цукроза, карычневы цукар, какосавы цукар, цукровы трысьнёг, кукурузны цукар, агава, мёд, дошаб, пэкмэз, фруктовы канцэнтрат, кляновы сыроп, цукар-сырэц, сыроп сушанага трысьнягу, інвэртны цукар, сыроп карычневага рысу, сок белага вінаграду і многімі іншымі. Акрамя гэтага, важна ўлічваць і агульную колькасьць фруктозы ў рацыёне, аптымальна заставацца ў межах 20-30 грамаў на содні, пры спажываньні больш за 60 грамаў павялічваюцца рызыкі для здароўя.

\subsection{Замяніце «перапрацаваныя» вугляводы «клетачнымі».}
Ёсьць розныя сыстэмы ацэнкі вугляводаў, якія грунтуюцца на глікемічным індэксе, інсулінавым індэксе, шчыльнасьці вугляводаў, глікемічнай нагрузцы. Камбінаваны паказьнік, які сумяшчае як шчыльнасьць вугляводаў, так і іх глікемічны індэкс, – гэта глікемічная нагрузка. Для яе вылічэньня трэба памножыць колькасьць грамаў вугляводаў у 100 грамах прадукту на яго глікемічны індэкс і падзяліць на 100. У цэлым важна разумець, што, нават калі глікемічны індэкс можа быць высокім, але вугляводаў у прадукце мала, ён ня шкодны для здароўя (бо глікемічная нагрузка будзе нізкай), як у выпадку з гарбузом або кавуном.

Цукар хаваецца за рознымі назвамі: глюкозна-фруктозны сыроп, «сыроп фруктозы», ГФС, HFCS, GFS, фінікавы цукар, цукроза, карычневы цукар, какосавы цукар, цукровы трысьнёг, кукурузны цукар, агава, мёд, дошаб, пэкмэз, фруктовы канцэнтрат, кляновы сыроп, цукар-сырэц, сыроп сушанага трысьнягу, інвэртны цукар, сыроп карычневага рысу, сок белага вінаграду і многімі іншымі.

Практычна гэтыя парады зводзяцца да адной: скарачайце ўжываньне мучных вырабаў, хлеба, выпечкі і паўфабрыкатаў на аснове мукі любых тыпаў. Па магчымасьці важна скараціць і колькасьць крупаў у рацыёне, робячы іх порцыі меншымі і дадаючы да іх больш нізкакалярыйнай зеляніны. Перапрацоўка крупаў робіць са здаровых прадуктаў ня вельмі здаровыя. Так, з аўсянкі з глікемічным індэксам 40-45 можна атрымаць кашу хуткага прыгатаваньня з індэксам 80. У любым выпадку ўсе крупы маюць досыць высокую шчыльнасьць вугляводаў. А вось долю гародніны неабходна павялічваць.

Ёсьць гародніна зь вялікім утрыманьнем крухмалу (бульба, гарбуз, батат, морква, буракі і да т.п.) і меншым утрыманьнем (капусты, цыбуля, салаты, перац, спаржа, струковая фасоля і да т.п.), вы можаце ў розных відах і спалучэньнях выкарыстоўваць іх у сваім рацыёне.

\subsection{Прапорцыі і варыяцыі.}
Вар'іруйце колькасьць вугляводаў. Калі вугляводы складаюць вельмі вялікую частку каляражу ў вашым харчаваньні, то будзе карысным яе паменшыць, замяніўшы вугляводы тлушчамі. Больш вугляводаў можна дадаць улетку, пры інтэнсіўнай фізычнай актыўнасьці, па-за стрэсам. Менш вугляводаў – узімку, пры сядзячай працы, нізкаінтэнсіўнай дзейнасьці, пры стрэсе. Чым менш вы рухаецеся, тым меншая доля вугляводаў павінна быць у вашай дыеце. Аднак многія дасьледаваньні паказваюць, што больш здаровымі ў доўгатэрміновай пэрспэктыве зьяўляюцца дыеты, дзе вугляводы складаюцца каля 50% калярыйнасьці і забясьпечваюцца гароднінай, садавіной, цэльназерневымі прадуктамі.

Існуе каляровае правіла для вугляводаў: для пахудзеньня ўжывайце больш «зялёных» і «жоўта-чырвоных» вугляводаў нізкай і сярэдняй шчыльнасьці і менш «карычневых» вугляводаў (крупы, буры рыс), максімальна абмяжоўвайце долю «белых» вугляводаў (белы рыс, мучныя вырабы і г. д.).

Дзеля простасьці складаньня прапорцый вы можаце выкарыстоўваць каляровае правіла для вугляводаў: для пахуданьня выкарыстоўвайце вялікую долю «зялёных» вугляводаў, то-бок вугляводаў нізкай шчыльнасьці, і «жоўта-чырвоных» вугляводаў (сярэдняй шчыльнасьці), меншую долю так званых «карычневых» вугляводаў ( крупы, буры рыс), максімальна абмяжоўвайце долю «белых» вугляводаў (белы рыс, мучныя вырабы і да т.п.).

\subsection{Разнастайнасьць расьліннай ежы.}
Чым больш разнастайнай ежы вы ясьцё, тым больш атрымліваеце розных фітанутрыентаў, многія зь якіх (капсаіцын, квэрцэтын, фэтызін, куркумін, бэрбэрын і г. д.) маюць добра вывучанае карыснае ўзьдзеяньне на арганізм. Як мінімум імкніцеся зьядаць 500 грамаў зеляніны, гародніны і садавіны ў дзень, большую частку зь іх сырымі або зь мінімальнай апрацоўкай. Бо чым больш вы варыце ці тушыце расьлінную ежу, тым вышэйшы ў яе глікемічны індэкс! Ідэі, як разнастаіць выбар, ёсьць у разьдзеле «Разнастайнасьць». Зьвярніце ўвагу, што БАДы, якія зьмяшчаюць асобныя фітанутрыенты, могуць быць далёка не такімі карыснымі, як цэльная гародніна, і мусяць выкарыстоўвацца толькі ў абмежаваных выпадках па прызначэньні.

\section{Як трымацца правіла? Ідэі і парады}

\subsection{Утоены цукар.}
Прааналізуйце сьпіс прадуктаў, якія вы ўжываеце, і зьвярніце ўвагу на ўтоены цукар, які можа знаходзіцца ў зусім несалодкіх прадуктах: соўсы, хлеб, мясныя паўфабрыкаты і г.д.

\subsection{Бяз скрайнасьцяў.}
Фруктоза ў садавіне не ўяўляе шкоды, асабліва калі вы ўкладаецеся ў норму калёрыяў. Цукар натуральным чынам зьмяшчаецца ў садавіне, у гародніне. Але гэта мала ўплывае на цукар у крыві, бо клятчатка ды іншыя кампанэнты гародніны і садавіны запавольваюць усмоктваньне цукру, хоць вельмі салодкімі ягадамі (вінаград) ня варта захапляцца. А вось фруктовыя сокі валодаюць ужо ўсімі нэгатыўнымі ўласьцівасьцямі цукру.

\subsection{Замяніце дэсэрты.}
Калі вы звыклі да салодкіх дэсэртаў, знайдзіце для сябе карысны сьпіс здаровых дэсэртаў. Гэта могуць быць сырыя гарэхі, горкая чакаляда, ягады, садавіна (ківі, авакада).

Вылучаюць «рэзыстэнтныя» крухмалы, якія карысныя для мікрафлёры і ўтрымліваюцца ў зеляніне, халоднай бульбе, бабовых. Форма крухмалу амілапэктын (бульба, кукуруза, рыс) зьяўляецца маларазгалінаванай і лёгка распадаецца, спрыяючы больш выяўленаму павышэньню глюкозы ў крыві, а вось амілоза (яблыкі, гарох, гародніна) больш шчыльна запакаваная, таму павольней распадаецца ў кішачніку.

\subsection{Схаваная гародніна.}
Калі вам цяжка звыкнуць да гародніны ў штодзённым рацыёне, пачніце патроху дадаваць яе ў свае звычайныя стравы, каб адаптавацца, ужываць у выглядзе кашаў, кактэйляў і да т.п.

\subsection{Розныя крухмалы.}
Нягледзячы на агульны тэрмін «крухмал», гэтыя рэчывы могуць быць зусім рознымі ў пляне ўплыву на наш арганізм. Вылучаюць так званыя «рэзыстэнтныя» крухмалы, якія карысныя для мікрафлёры, іх шмат у зеляніне, халоднай бульбе, бабовых. Форма крухмалу амілапэктын (бульба, кукуруза, рыс) зьяўляецца маларазгалінаванай і лёгка распадаецца, спрыяючы больш выяўленаму павышэньню глюкозы ў крыві, а вось амілоза (яблыкі, гарох, гародніна) больш шчыльна запакаваная, таму павольней распадаецца ў кішачніку. Цікава, што халодная бульба мае ніжэйшы глікемічны індэкс за кошт зьмены формы крухмалу ў бульбе.

\subsection{Ашчаджальная гатоўля.}
Любая апрацоўка прадукту – яго драбненьне, награваньне, гатаваньне пры высокіх тэмпэратурах – павышае яго глікемічны індэкс, прыводзіць да зьніжэньня памеру часьцінак крухмалу. Таму ежа пакаёвай тэмпэратуры мае меншы глікемічны індэкс. Апарваньне кіпнем, варка al dente, фэрмэнтацыя, запарваньне – больш ашчаджальныя спосабы гатаваньня.

\subsection{Харчовыя валокны – гэта важная частка рацыёну.}
Аптымальна атрымліваць іх праз прадукты, а не праз дабаўкі. У дасьледаваньнях выкарыстаньне дабаўленай клятчаткі аказалася неэфэктыўным. Зьвярніце ўвагу, што распушчальныя харчовыя валокны (пэктыны, альгінаты, сьлізі) больш карысныя і лепей насычаюць, чым не распушчальныя (цэлюлоза, лігнін). У гародніне ня менш клятчаткі, чым у збожжавых прадуктах, і аддаваць перавагу лепей ім. Сярод крупаў бабовыя маюць добры набор харчовых валокнаў, а акрамя гэтага яшчэ й нізкі глікемічны індэкс. Прэбіётыкі ўтрымліваюцца ў натуральным відзе ў харчовых прадуктах, але ёсьць асобна і ў выглядзе дабавак: фруктаалігацукрыды, інулін, лактулоза, бэта-глюканы, псіліум і многае іншае.

\subsection{Антынутрыенты.}
Існуюць расьлінныя прадукты, якія ўтрымліваюць шэраг злучэньняў пад назвай «антынутрыенты», што выпрацоўваюцца для абароны. Да іх адносяць фіцінавую кіслату, фітаэстрагены, лектыны і г. д. Пры ўжываньні ў вялікай колькасьці збажыны або бабовых антынутрыенты могуць нэгатыўна ўплываць на арганізм (фітаты са збажыны зьніжаюць усмоктваньне жалеза і цынку, фітаэстрагены з соі ўплываюць на гарманальны балянс, лішак шчаўевай кіслаты павялічвае рызыку камянёў у мачавым пухіры і г. д.). Для зьніжэньня дзеяньня фітаэстрагенаў выкарыстоўваюць замочваньне (на ноч ці на суткі), фэрмэнтацыю, прарошчваньне, кіслае вымочваньне. Але калі вы ясьцё невялікую колькасьць мучнога і крупаў, то можна не турбавацца пра пабочныя эфэктах антынутрыентаў. Пры гэтым заліваць крупы варта – гэта скарачае тэрмін гатаваньня і зьніжае глікемічны індэкс гатовай стравы. Небясьпека іншых антынутрыентаў, асабліва лектынаў, значна перабольшаная!

\subsection{Індывідуальная рэакцыя.}
Важна адсочваць індывідуальную рэакцыю на прадукты, бо яна можа адрозьнівацца ў залежнасьці ад шматлікіх фактараў. Цікава, што харчовыя валокны ў рацыёне паляпшаюць глікемічны кантроль, але толькі праз 24 гадзіны пасьля таго, як вы іх зьелі. Таксама на ваш асабісты глікемічны індэкс уплывае колькасьць солі ў рацыёне, час прыёму ежы з моманту абуджэньня, узровень халестэрыну і нават колькасьць вады ў рацыёне. Зразумела, адэкватны водны рэжым зьвязаны зь лепшымі паказьнікамі. Ключавая прычына гэтага – мікрафлёра кішачніка. Таму важна выбіраць вугляводы ня толькі для сябе, але і з разлікам на сваю мікрафлёру, улічваючы пры гэтым удзельную калярыйную шчыльнасьць вугляводаў, ступень апрацоўкі і ачысткі, від гатаваньня, даданьне харчовых дабавак. Паляпшаючы мікрафлёру, мы палепшым і пераноснасьць вугляводных прадуктаў і свой абмен рэчываў. Маніторынг уздыму глюкозы праз гадзіну і дзьве гадзіны пасьля ежы дапаможа скласьці свой індывідуальны глікемічны профіль, для гэтага можна выкарыстоўваць як кампактны аналізатар крыві з пальца, так і сыстэмы сталага маніторынгу.

\subsection{Дадаткі.}
Даданьне солі павялічвае глікемічны індэкс (натрый паскарае ўсмоктваньне глюкозы), даданьне тлушчаў зьмяншае глікемічны індэкс, даданьне кіслага таксама зьмяншае глікемічны індэкс (карысна цытрынай, лаймам або воцатам запраўляць салату і гародніну).

\subsection{Цукразаменьнікі.}
Існуе наўпроставая (таксычны) і ўскосная шкода цукразамяняльнікаў. Іх таксычнае ўзьдзеяньне слабое (яны адносна бясьпечныя), але яны моцна дзейнічаюць на клеткавыя рэцэптары да салодкага. Гэтыя рэцэптары знаходзяцца яшчэ ў мностве ворганаў: падстраўніцы, мозгу, крывяносных сасудах, касьцях, страўніку, кішачніку, тлушчавай тканкі. Цукразамяняльнікі прыкметна зьмяняюць выдзяленьне кішачных гармонаў, парушаюць экспрэсію бялкоў – пераносчыкаў глюкозы, нават пры поўнай адсутнасьці цукру ўплываюць на выдзяленьне інсуліну і ўзровень глюкозы ў крыві, узмацняюць апэтыт. Таму нядзіўна, што цукразаменьнікі павялічваюць рызыку цукровага дыябэту. Актывацыя рэцэптараў да салодкага прыводзіць да скарачэньня крывацёку ў сасудах галаўнога мозгу. Менавіта таму падсалоджвальнікі павялічваюць частату інсультаў і нэўрадэгенэратыўных захворваньняў. Спажываньне цукразаменьнікаў нэгатыўна ўплывае на дафамінавую сыстэму мозгу і мігдаліцы, парушаючы іх працу. Залішняе ўжываньне цукразамяняльнікаў зьніжае мазгавы водгук на прыём іншых салодкіх прадуктаў, зьніжаючы і адчувальнасьць да салодкага. Цукразамяняльнікі таксама ўплываюць на работу прэфрантальнай кары мозгу. Надзейных доказаў, што цукразаменьнікі дапамагаюць схуднець, ня знойдзена.

Цукразамяняльнікі прыкметна зьмяняюць выдзяленьне кішачных гармонаў, парушаюць экспрэсію бялкоў – пераносчыкаў глюкозы, нават пры поўнай адсутнасьці цукру ўплываюць на выдзяленьне інсуліну і ўзровень глюкозы ў крыві, узмацняюць апэтыт.

\subsection{Каратыноіды.}
Каратыноіды – гэта шырокая група фітанутрыентаў, якая ўключае бэта-каратын, лікапін, лютэін, зэаксантын, астаксантын, крыптаксантын, фукаксантын. Яны тлушчараспушчальныя, таму могуць назапашвацца ў скуры і тлушчавай тканцы, валодаюць імунастымулюючым, фотаахоўным, антыаксыдантным эфэктам, абараняюць сятчатку вока. Пры спажываньні на працягу 12 тыдняў яны могуць назапашвацца ў скуры, што абараняе яе ад фотастарэньня і дае больш здаровае адценьне. Каратыноідаў багата ў зялёнай і жоўтай гародніне (морква, гарбуз, шпінат, зялёны лук), чырвонай гародніне (таматы, перац), бабовых, ягадах, садавіне. Каратыноіды лепш засвойваюцца пры тэрмічнай апрацоўцы і даданьні тлушчу.

\subsection{Флаваноіды.}
Гэта шырокая група, якая ўключае антаціанідыны (яркі колер чарніцаў, чарнаплоднай рабіны і да т.п.), фітаэстрагены, флавоны, ізафлавоны, лігнаны, катэхін (у зялёнай гарбаце). Некаторыя ізафлавоны соі могуць нэгатыўна ўзьдзейнічаць на мужчынскія палавыя гармоны і на работу шчытавіцы.

\subsection{Глюказіналаты.}
Найбольш вядомыя прадстаўнікі – сульфарафан у брокалі (пажадана есьці яе сырой), у хрэне і каляровай капусьце, гэта індол-3-карбінол, які валодае прафіляктычным дзеяньнем у дачыненьні да раку грудзей і падстраўніцы.

\subsection{Алілсульфіды.}
Гэтыя злучэньні ўтрымліваюць серу ў сваім складзе, сустракаюцца ў часныку і розных відах цыбулі. Валодаюць антымутагенным дзеяньнем, паляпшаюць работу печані.

\subsection{Фэрмэнтаваныя вугляводы.}
Фэрмэнтацыя прадуктаў – гэта выдатны спосаб павялічыць іх пажыўныя ўласьцівасьці. Можна як квасіць гародніну, так і рабіць зь яе квасы.

\subsection{Сырая гародніна.}
Аптымальна значную долю, ня менш за палову ад содневай нормы гародніны, садавіны і зеляніны, трэба зьядаць у сырым выглядзе.

\subsection{Багавіньне.}
У багавіньні ўтрымліваюцца фітастэролы, фукаксантын, лямінарыны, фуканы, фікаэрытрын, фікацыянін і т. д.

\subsection{Газаўтварэньне.}
Пры павелічэньні долі расьлінных прадуктаў можна сутыкнуцца з падвышаным газаўтварэньнем. Для прафілактыкі павялічвайце яе паступова, пачніце з тэрмаапрацаваных прадуктаў, вар'іруйце прадукты (сачавіца замест гароху), выкарыстоўвайце мультыпрабіётыкі.

\subsection{FODMAP дыета.}
FODMAP – гэта англамоўны тэрмін-акронім, які пазначае вугляводы з кароткім ланцугом (алігацукрыды, дыцукрыды, монацукрыды, паліолы), што кепска ўсмоктваюцца і могуць прыводзіць да падвышанага газаўтварэньня ды іншым разладаў. Але ў малой колькасьці яны зьяўляюцца прабіётыкамі. Фруктоза, якой шмат у цукры і садавіне, фруктаны ў гародніне (капусты, часнык), галяктаны (соя, бабы), ляктоза (малочны цукар) і іншыя. Іх абмежаваньне можа дапамагчы пры парушэньнях стрававальнага працэсу.

\subsection{Безглютэнавая дыета.}
Людзям, якія маюць генэтычную схільнасьць да цэліякіі, важна прыбраць глютэназьмяшчальныя прадукты з рацыёну. Існуюць зьвесткі, што ёсьць непераноснасьць глютэну, не зьвязаная з цэліякіяй, магчыма, што людзям з сындромам раздражнёнага кішачніка і некаторымі іншымі станамі (шызафрэнія і да т.п.) будзе карысна прыбраць глютэн. Але для абсалютнай большасьці здаровых людзей плюсаў менавіта ад адсутнасьці глютэну ня будзе, а эфэкт "безглютенавай" дыеты зьвязаны толькі са зьмяншэньнем хлебабулачных вырабаў у рацыёне. Але ў такіх прадуктах нічога незаменнага няма!

\subsection{Вэганская дыета.}
Поўная адмова ад прадуктаў жывёльнага паходжаньня ня мае даведзеных карысных уласьцівасьцяў, але пры гэтым павялічвае шэраг рызыкаў для здароўя, спалучаных з дэфіцытам шматлікіх важных злучэньняў, уключаючы вітамін В12, вітамін D, цынк, жалеза і іншыя. Дасьледаваньні паказваюць, што вэганы маюць павышаную рызыку сьмерці ў параўнаньні зь людзьмі, якія прытрымліваюцца іншых клясычных дыетаў, напрыклад міжземнаморскай. З іншага боку, вэгетарыянскія дыеты зьмяншаюць рызыку цукровага дыябету і маюць шэраг карысных уласьцівасьцяў.

У сярэднім низкоуглеводная дыета - гэта менш за 30% калёрыяў са здаровых вугляводаў (60-120 грамаў засваяльных вугляводаў). Такія дыеты маюць шэраг карысных уласьцівасьцяў, у тым ліку лепшую сытасьць і станоўчы ўплыў на мэтабалізм.

\subsection{Пэскетэрыянства.}
Расьлінная дыета з умераным даданьнем у ежу халодакроўных жывёл (рыба, морапрадукты і т. п.). Зьяўляецца дастаткова эфэктыўнай і збалянсаванай.

\subsection{Plant based diet.}
Дыета, заснаваная на цэльнай расьліннай ежы з даданьнем як тлушчаў, так і жывёльных прадуктаў. Ключ да яе – цэльныя расьліны зь мінімальнай апрацоўкай.

\subsection{Нізкавугляводная дыета.}
Паняцце нізкавугляводнай дыеты, у прынцыпе, расьцяглае. Калі вы пачалі менш есьці вугляводаў, то для вас гэта ўжо і ёсьць нізкавугляводная дыета. У адэкватным разуменьні нізкавугляводная дыета мае на ўвазе зьніжэньне колькасьці цукру, сытных мучных і крупяных вырабаў і павелічэньне колькасьці гародніны і зеляніны. Пры гэтым агульны аб'ём вугляводаў не мяняецца, а іх колькасьць зьніжаецца. Зьвярніце ўвагу, што, акрамя агульнага значэньня вугляводаў, значэньне мае і колькасьць клятчаткі, якая не ўваходзіць у агульны лік калёрыяў.

У сярэднім нізкавугляводная дыета – гэта менш за 30% калёрыяў са здаровых вугляводаў (60-120 грамаў засваяльных вугляводаў). Такія дыеты маюць шэраг карысных уласьцівасьцяў, у тым ліку лепшую сытасьць і станоўчы ўплыў на мэтабалізм. Няправільны падыход да нізкавугляводных дыетаў уключае моцнае павелічэньне долі бялку і выкарыстаньне маленькіх порцый мучных вугляводаў. Зьніжэньне калёрыяў з вугляводаў звычайна кампэнсуецца тлушчамі (LCHF (low carb high fat)). Больш радыкальнае зразаньне вугляводаў – гэта кетадыета, яна разглядаецца ў наступным разьдзеле. Мы можам чаргаваць нізка- і высокавугляводныя пэрыяды ў залежнасьці ад узроўню фізычнае актыўнасьці.

\chapter{Тлушчы}

Тлушчы зьяўляюцца важным нутрыентам, што станоўча ўплываюць на насычэньне і мэтабалізм, і выдатнай крыніцай энэргіі. Акісьленьне тлушчаў у~арганізьме адрозьніваецца ад выкарыстаньня вугляводаў больш ашчадным рэжымам. У~залежнасьці ад хімічнай будовы (колькасьць падвойных сувязяў) вылучаюць насычаныя тлушчы (без падвойных сувязяў), монаненасычаныя (з адной сувяззю~--- амэга-9, амэга-7 і іншыя), поліненасычаныя (дзьве падвойныя сувязі~--- амэга-3, амэга-6 і іншыя). Лічбы абазначаюць месца разьмяшчэньня апошняй (амэга) падвойнай сувязі. Тлушчы~--- гэта ня толькі крыніца энэргіі, але й~незаменныя рэчывы, зь якіх утвараюцца сыгнальныя малекулы, што кіруюць шматлікімі працэсамі ў~арганізьме. Акрамя гэтага, важнай зьяўляецца і структурная функцыя тлушчаў: напрыклад, большасьць амэга-3 тлустых кіслотаў знаходзіцца ў~галаўным мозгу.

Рознае стаўленьне да тлушчаў усталявалася паміж тымі, хто лічыць, што гэта «шкодна», і тымі, хто ўпэўнены, што тлушчавая дыета вырашыць усе праблемы. Насамрэч для правільнага выкарыстаньня тлушчаў у~рацыёне трэба ўлічваць спосаб гатоўлі, спалучэньне зь іншымі прадуктамі і шматлікае іншае. Давайце разьбяромся, як збалянсаваць спажываньне тлушчаў.

\tipbox{У залежнасьці ад хімічнай будовы (колькасьць падвойных сувязяў) вылучаюць насычаныя тлушчы (без падвойных сувязяў), монаненасычаныя (з адной сувяззю~--- амэга-9, амэга-7 і іншыя), поліненасычаныя (дзве падвойныя сувязі~--- амэга-3, амэга-6 і іншыя). Лічбы абазначаюць месца разьмяшчэньня апошняй (амэга) падвойнай сувязі.}

\subsection{Як зьявілася праблема?}

З даўніх часоў людзі выкарыстоўвалі тлушчы як надзейную крыніцу энэргіі, для многіх народнасьцяў (эскімосы і да т.~п.) тлушчы былі галоўным нутрыентам. Традыцыйна тлушч часьцей выкарыстоўваўся жыхарамі халаднейшых краінаў. У~нацыянальнай кухні краінаў з~умераным кліматам тлушч таксама складае значную долю. Аліўкавы алей (першага халоднага адціску)~--- самая заўважная асаблівасьць вядомай і добра вывучанай міжземнаморскай дыеты.

Аднак у~50-х гадах адбылося сур'ёзнае зьмяненьне ролі тлушчаў. Пасьля сардэчнага прыступу прэзідэнта ЗША Дуайта Эйзэнхаўэра\index{Эйзэнхаўэр Дуайт} і росту сардэчна-сасудзістых захворваньняў у~ЗША ўвага грамадскасьці была прыцягнутая да пошуку сродкаў прафіляктыкі гэтых захворваньняў. Шырокую вядомасьць і прызнаньне пры гэтым атрымала дзейнасьць Ансэля Кейса, які на аснове сваіх назіраньняў абвясьціў жывёльны тлушч вінаватым у~захворваньнях сэрца. І~хоць яго работы былі ненадзейнымі і крытыкаваліся, гэты пункт гледжаньня стаў агульнапрызнаным і ўвайшоў у~дыеталягічныя рэкамэндацыі. Аўтараў, якія прытрымліваліся іншых пунктаў гледжаньня, такіх як Джон Юдкін\index{Юдкін Джон} (лічыў цукар прычынай праблемы), не ўспрымалі сур'ёзна. Усё гэта прывяло да дэманізацыі халестэрыну і тлушчаў, што выявілася ў~рэзкім скарачэньні колькасьці жывёльнага тлушчу, павелічэньні колькасьці абястлушчаных прадуктаў, долі транстлушчаў\index{транстлушчы}, расьлінных алеяў, павелічэньні ўжываньня вугляводных прадуктаў (паста, мюсьлі, кашы, сокі), цукру. Але гэтыя зьмены не паўплывалі на здароўе станоўча, а~толькі пагоршылі эпідэмію атлусьценьня. Ёсьць меркаваньне, што многія з~гэтых рэкамэндацый былі падтрыманыя вытворцамі пэўных прадуктаў харчаваньня.

\tipbox{З даўніх часоў людзі выкарыстоўвалі тлушчы як надзейную крыніцу энэргіі, для многіх народнасьцяў (эскімосы і да т.~п.) тлушчы былі галоўным нутрыентам. Традыцыйна тлушч часьцей выкарыстоўваўся жыхарамі халаднейшых краінаў.}

\paragraph{Танныя алеі.}
У мінулым стагоддзі ў~рацыёне адбылася зьмена харчовых тлушчаў. Так, людзі сталі спажываць значна больш расьлінных тлушчаў зь лішкам амэга-6 тлустых кіслот (сланечнікавы, баваўняны, соевы, кукурузны алеі), якія рэкамэндаваліся як таньнейшыя і як больш карысныя для арганізму. Транстлушчы\index{транстлушчы} таксама актыўна ўкараняліся ў~рацыён, пазыцыянуючыся як больш карысныя для здароўя ў~параўнаньні з~жывёльнымі насычанымі тлушчамі. Аднак надмер гэтых тлушчаў прывёў адно да павелічэньня росту сардэчна-сасудзістых захворваньняў.

\subsection{Як гэта ўплывае на здароўе?}

Сёньня ўжо вядома, што нізкатлушчавыя дыеты ня маюць ахоўнага ўзьдзеяньня на сэрца. Так, утрыманьне тлушчаў вышэй за 35\,\%, а~вугляводаў менш за 60\,\% прыводзіць да зьніжэньня сардэчна-сасудзістых захворваньняў. У~дасьледаваньнях больш высокае спажываньне тлушчаў у~параўнаньні з~самым нізкім зьніжала сьмяротнасьць на 30\,\%, а~рызыку інсульту на 18\,\%.

\paragraph{Насычаныя тлушчы.}
Насычаныя тлушчы шырока сустракаюцца ў~прыродзе. Варта адзначыць, што іх няма ў~чыстым выглядзе, тлустыя прадукты ўтрымліваюць сумесь розных тлустых кіслотаў. Напрыклад, у~сале насычаныя тлушчы складаюць 42\,\%, монаненасычаныя~--- 44\,\%. У~сьметанковым масьле насычаныя тлушчы складаюць 56\,\%, монаненасычаныя~--- 29\,\%, поліненасычаныя~--- 3\,\%. Большасьць тлустых кіслотаў маюць доўгі ланцужок малекулы, выключэньне~--- сярэднеланцужковыя насычаныя тлустыя кіслоты, такія, як у~какосавым алеі. Яны хутчэй і лепей засвойваюцца, таму какосавы алей карыстаецца заслужанай папулярнасьцю.

Аднак тлушчы з~доўгімі ланцужкамі выдатна і надоўга насычаюць, стымулюючы выдзяленьне гармону халецыстакініну. Насычаныя тлушчы прадстаўлены рознымі малекуламі тлустых кіслотаў, якія адрозьніваюцца сваімі ўласьцівасьцямі. Стэарынавая тлустая кіслата\index{кіслата!стэарынавая} (багата ў~барановым тлушчы) мае станоўчыя ўласьцівасьці, лішак пальмітынавай мае й~нэгатыўныя аспэкты. Утрыманьне пальмітынавай кіслаты\index{кіслата!пальмітынавая} ў~сучасным фастфудзе перавышае палову ад агульнага складу ўсіх тлустых кіслотаў. Пальмітынавая кіслата\index{кіслата!пальмітынавая} дае своеасаблівы «тлусты смак» і зьяўляецца досыць таннай. Аднак уплыў пальмавага алею ва ўмераных колькасьцях на здароўе нэўтральны.

\paragraph{Монаненасычанымі тлустыя кіслоты.}
Уключаюць у~сябе алеінавую (амэга-9), пальмітаалеінавую (амэга-7) і шэраг іншых тлустых кіслот. Найбольш вывучаная алеінавая тлустая кіслата\index{кіслата!алеінавая}, якой шмат у~аліўкавым алеі (76\,\%), авакада (70\,\%), многіх жывёльных прадуктах (у яйках 50\,\%). Багата яе ў~мігдале і лясным гарэху. Алеінавая тлустая кіслата\index{кіслата!алеінавая} здольная зьніжаць рызыку разьвіцьця некаторых відаў раку, павышаць адчувальнасьць да інсуліну, зьніжаць узровень сыстэмнага запаленьня. Монаненасычаныя тлушчы станоўча ўплываюць на здароўе, паляпшаюць як вугляводны, так і тлушчавы абмен, зьніжаюць рызыку разьвіцьця многіх захворваньняў.

\paragraph{Поліненасычаныя тлушчы.}
Поліненасычаныя тлушчы адносяцца да незаменных злучэньняў. З~аднаго боку, іх павелічэньне ў~рацыёне дабратворна ўплывае на сардэчна-сасудзістую сыстэму. Зь іншага~--- поліненасычаныя тлушчы схільныя да перакіснага акісьленьня праз сваю хімічную структуру, таму іх меншая доля ў~клеткавых мэмбранах зьвязаная з~большай працягласьцю жыцьця празь меншую актыўнасьць працэсаў перакіснага пашкоджаньня ліпідаў і ўтварэньня рознага клеткавага сьмецьця.

\paragraph{Дысбалянс амэга-3 і амэга-6 поліненасычаных тлустых кіслотаў.}
Незаменныя тлустыя кіслоты паступаюць зь ежай, частка іх спальваецца, а~частка ідзе на сынтэз вытворных, адмысловых малекулаў эйказаноідаў\index{эйказаноіды}. З~амэга-6 тлустых кіслотаў (лінолевая) утвараюцца эйказаноіды, якія павялічваюць агульны ўзровень запаленьня, павялічваюць пранікальнасьць сасудаў, спрыяюць звадкаваньню крыві, звужэньню бронхаў, утварэньню сьлізі і да т.~п. А~вось з~амэга-3 тлустых кіслотаў (эйказапэнтаенавая і даказагексаенавая) утвараюцца супрацьзапаленчыя эйказаноіды\index{эйказаноіды}, якія валодаюць супрацьлеглымі эфэктамі. Бо фэрмэнт, які ператварае малекулы поліненасычаных тлустых кіслотаў у~эйказаноіды\index{эйказаноіды}, агульны, то й~суадносіны амэга-3 і амэга-6 у~ежы будуць уплываць на суадносіны іх вытворных у~тканках арганізму. Для дакладнага вызначэньня балянсу і падбору дакладнай дазоўкі амэга-3 тлустых кіслотаў можна здаць аналіз на амэга-індэкс, які паказвае іх рэальныя суадносіны ў~арганізьме.

\tipbox{Дастатковая колькасьць амэга-3 тлустых кіслотаў маецца ў~ежы жывёльнага паходжаньня, але сёньня ў~нашым рацыёне зьявілася мноства аналягаў з~высокім утрыманьнем амэга-6 тлустых кіслотаў, якія нясуць шкоду здароўю.}

Раней мы атрымлівалі дастатковую колькасьць амэга-3 зь ежы жывёльнага паходжаньня~--- яйкі, малочныя прадукты, мяса жывёлаў і птушак (трава мае расьлінную амэга-3 кіслату, якую жывёлы ператвараюць у~жывёльныя формы амэга-3). Сёньня ў~кармах для жывёлаў пераважаюць збожжавыя культуры, у~якіх практычна няма амэга-3 тлустых кіслот. Нароўні са скарачэньнем колькасьці амэга-3 у~нашым рацыёне зьявілася мноства танных расьлінных алеяў з~высокім утрыманьнем амэга-6 тлустых кіслотаў. Працэнтнае ўтрыманьне амэга-6 тлустых кіслотаў у~сланечнікавым алеі складае 60\,\%, багата яго ў~соевым, баваўняным, кукурузным ды іншых прамысловых алеях. Праз сваю таннасьць гэтыя маслы выкарыстоўваюцца ў~шматлікіх паўфабрыкатах і гатовай ежы ў~вялікай колькасьці. Лішак амэга-6 тлушчаў у~рацыёне павялічвае ўзровень запаленьня, рызыку сардэчна-сасудзістых, аўтаімунных захворваньняў. Лішак лінолевай амэга-6 тлустай кіслаты\index{кіслата!лінолевая} можа павышаць рызыку інфаркту, дэпрэсій, нэўрадэгенэратыўных захворваньняў, сыстэмнага запаленьня, шэрагу пухлінаў.

\paragraph{Дэфіцыт амэга-3 тлустых кіслотаў} (эйказапэнтаенавая ЭПК\index{кіслата!эйказапэнтаенавая (ЭПК)} і даказагексаенавая ДГК\index{кіслата!даказагексаенавая (ДГК)}~--- кіслоты)~--- гэта даволі частая праблема.
У нашым арганізьме ЭПК і ДГК граюць важную ролю, уваходзячы ў~склад галаўнога мозгу і сятчаткі вачэй. Яны паляпшаюць кагнітыўныя функцыі, зьмяншаюць рызыку дэпрэсіі, зьмяншаюць рызыку запаленьня, карысныя для прафілактыкі сардэчна-сасудзістых захворваньняў і карэкцыі ліпіднага профілю, асабліва важныя для дзяцей і цяжарных. Маюць яны і агульныя ўласьцівасьці, і сваю спэцыфіку, бо ЭПК валодае большай супрацьзапаленчай актыўнасьцю, а~ДГК патрэбнейшы для падтрыманьня ўстойлівых мэмбранаў нэрвовых клетак.

\paragraph{Транстлушчы\index{транстлушчы}.}
Транстлушчы\index{транстлушчы} павышаюць рызыку сардэчна-сасудзістых захворваньняў, могуць павялічваць рызыку дыябэту, раку, дэпрэсій і хваробы Альцгаймэра\index{хвароба!Альцгаймэра} пры высокім узроўні спажываньня.

\subsection{Асноўныя прынцыпы}

\paragraph{Колькасьць тлушчаў у~рацыёне.}
Дастатковая колькасьць тлушчу ў~дыеце на ўзроўні 35--40\,\%~--- гэта цалкам здаровае рашэньне. Аднак павелічэньне долі тлушчаў абавязкова павінна суправаджацца памяншэньнем долі канцэнтраваных вугляводаў, спалучэньне высокакалярыйны тлушч + высокакалярыйныя вугляводы вельмі разбуральнае для мэтабалізму. Канкрэтная прапорцыя нутрыентаў таксама залежыць ад генэтычных асаблівасьцяў. Цыкліраваньне нутрыентаў~--- гэта выхад з~сытуацыі. Зьвярніце ўвагу, што дадаваць тлушчы бескантрольна ня будзе добрым рашэньнем, бо яны маюць вельмі высокую калярыйную шчыльнасьцьі могуць прывесьці да надмернага каляражу.

\paragraph{Суадносіны розных відаў тлустых кіслотаў.}
Большую частку тлушчаў мусяць складаць насычаныя тлушчы з~рознай даўжынёй ланцуга, ад сярэднеланцуговых да доўгаланцуговых, і монаненасычаныя тлушчы. Канкрэтныя іх суадносіны ў~дыеце вызначаюцца генэтычнымі фактарамі, якія можна выявіць пры ДНК-аналізе. Некаторым будзе карысна ўжываць больш насычаных тлушчаў, іншым~--- больш аліўкавага алею. Аптымальнае ўмеранае ўжываньне сьметанковага, ялавічнага, барановага, сьвінога (сала), пальмавага, какосавага ды іншых насычаных тлушчаў у~спалучэньні з~монаненасычанымі тлушчамі (аліўкавы алей халоднага адціску). Больш высокае спажываньне тлушчу зьвязанае з~паніжаным апэтытам.

\tipbox{Важна датрымлівацца аптымальнага балянсу амэга-3 і амэга-6 тлустых кіслотаў. Для гэтага неабходна паменшыць спажываньне амэга-6 тлустых кіслотаў (паўфабрыкаты, алеі накшталт соевага, сланечнікавага, кукурузнага, кунжутнага, канаплянага і да т.~п.) і павялічыць спажываньне жывёльных формаў амэга-3 тлустых кіслотаў (марская рыба, морапрадукты, мяса і яйкі).}

У цэлым трэба скараціць колькасьць поліненасычаных тлустых кіслотаў, асабліва атрыманых з~алеяў. Гэта значыць, што варта пазьбягаць абсалютнай большасьці расьлінных алеяў (уключна з~ільняным алеем), за рэдкім выключэньнем (аліўкавы, какосавы і некаторыя іншыя).

\paragraph{Суадносіны амэга-3 і амэга-6.}
Для падтрыманьня здароўя неабходна датрымлівацца аптымальнага балянсу амэга-3 і амэга-6 тлустых кіслотаў. Для гэтага важна паменшыць спажываньне амэга-6 тлустых кіслотаў (паўфабрыкаты, алеі накшталт соевага, сланечнікавага, кукурузнага, кунжутнага, канаплянага і да т.~п.) і павялічыць спажываньне жывёльных формаў амэга-3 тлустых кіслотаў (эйказапэнтаенавай ЭПК\index{кіслата!эйказапэнтаенавая (ЭПК)} і даказагексаенавай ДГК\index{кіслата!даказагексаенавая (ДГК)}). Шмат амэга-3 тлустых кіслотаў маецца ў~жывёльных прадуктах травянога выпасу і морапрадуктах. Больш за ўсё іх у~тлустай марской рыбе. Так у~селядца 16,8 грамаў амэга-3 на кіляграм сырой масы, у~сардзіне~--- 25, ласосі~--- 12, стаўрыдзе~--- 8. Дастаткова 2--3 разы на тыдзень ужываць порцыю рыбы, каб дастаткова атрымліваць усе неабходныя злучэньні. Ужываньне дабавак амэга-3 тлустых кіслотаў у~вялікіх колькасьцях працяглы час можа быць таксама празьмерным і нават небясьпечным.

\paragraph{Абмежаваць спажываньне транстлушчаў\index{транстлушчы}} (кандытарскія вырабы, выпечку, маргарын і да т.~п.).
Маргарын для выпечкі можа ўтрымліваць 20--40\,\% транстлушчаў\index{транстлушчы}, шмат іх у~кулінарных тлушчах, спрэдах, каўбасных вырабах, цукерках і фастфудзе. Часта яны хаваюцца за агульнай назвай «алей», «гідрагенізаваныя алеі» і т.~п.

\subsection{Як трымацца правіла? Ідэі і парады}

\paragraph{Захоўваньне тлушчаў.}
Тлушчы схільныя да акісьленьня пад узьдзеяньнем тэмпэратуры, сьвятла, паскараюць акісьленьне некаторыя мэталы. Захоўвайце тлушчы ў~цёмным халодным месцы, закрывайце бутэлькі, сачыце за тэрмінам прыдатнасьці. Па магчымасьці купляйце сьвежы аліўкавы алей, сьвежае сала і т.~п.

\paragraph{Гатоўля.}
Што да награваньня і смажаньня, дык ёсьць шмат фактараў, якія ўплываюць на выбар тлушчу (ад выгляду тлушчу, кропкі дымленьня і г.~д.), але перадусім варта памятаць, што смажаньне нават на добрым тлушчы~--- ня самы карысны спосаб гатоўлі. Лепш за ўсё смажыць на насычаных тлушчах у~невялікай колькасьці, яны самыя ўстойлівыя. Напрыклад, нашы бабулі рабілі яечню, змазваючы патэльню кавалачкам сала~--- і гэта нашмат карысьней, чым калі вы выкарыстоўваеце сланечнікавы алей пры смажаньні. Пазьбягайце ў~рацыёне тлушчаў, якія мелі тэрмічную апрацоўку.

\paragraph{Расьлінныя алеі.}
Адмоўцеся ад ужываньня расьлінных алеяў з~высокім утрыманьнем поліненасычаных тлустых кіслотаў, як амэга-3, так і амэга-6. Некаторыя віды алеяў традыцыйна не ўжывалі ў~ежу. Напрыклад, ільняное. Нашы продкі выкарыстоўвалі яго ў~асноўным для апрацоўкі дрэва супраць гніеньня (аліфа), гермэтызацыі вокнаў (паста з~крэйды і льнянога алею), прамочваньня тканінаў. Ільняны алей вельмі хутка полімэрызуецца з~утварэньнем плёнкі, асабліва пры крыху падвышаных тэмпэратурах, трапленьні сонечных прамянёў, кантакце з~паветрам і некаторымі мэталамі. Абсалютная большасьць вытворцаў не гарантуюць датрыманьня строгіх стандартаў яго вытворчасьці, а~надмер поліненасычаных тлустых кіслотаў у~дыеце можа паскараць старэньне.

\tipbox{Традыцыйна льняны алей выкарыстоўваўся пры апрацоўцы дрэва супраць гніеньня, прамочваньні тканінаў, бо ён хутка акісьляецца і полімэрызуецца пры трапленьні сонечных прамянёў, кантакце з~паветрам ці мэталамі.}

\paragraph{Ільняны алей не замена рыбінаму тлушчу і рыбе.}
Ільняны алей неэфэктыўны як крыніца амэга-3. Рэч у~тым, што ўсе амэга-3 тлустыя кіслоты неаднолькавыя, ёсьць расьлінныя амэга-3 (альфа-ліналенавая АЛК) і жывёльныя амэга-3 (эйказапэнтаенавай ЭПК\index{кіслата!эйказапэнтаенавая (ЭПК)} і даказагексаенавай ДГК\index{кіслата!даказагексаенавая (ДГК)}). Чалавек ня можа засвоіць АЛК, яму патрэбныя ЭПК і ДГК. У~нашым арганізьме ёсьць гены, якія канвэртуюць АЛК у~ЭПК, але гэты працэс неэфэктыўны, толькі 1--6\,\% АЛК можна канвэртаваць. А~чым больш вы ясьцё АЛК і амэга-6, тым мацней зьмяншаецца гэтая канвэрсія.

Для эфэктыўнай канвэрсіі АЛК у~ЭПК патрэбная высокая актыўнасьць гена дэсатуразы FASD2. Але ў~85\,\% эўрапэйцаў, якія ўжываюць традыцыйна больш жывёльных прадуктаў, яго актыўнасьць нізкая, таму льняны алей, чыя, алей грэцкага гарэха не зьяўляюцца эфэктыўнымі для папаўненьня дэфіцыту амэга-3. У~экспэрымэнтах ужываньне льнянога масла не ўплывала на ўзровень амэга-3 у~крыві, але пры гэтым павялічвалася рызыка раку падкарэньніцы.

\paragraph{Поліненасычаныя тлустыя кіслоты~--- у~складзе цэльных прадуктаў.}
Як папоўніць дэфіцыт нутрыентаў, не ўжываючы алею? Ежце цэльныя грэцкія гарэхі, пасыпайце ежу кунжутнымі семкамі, разьмяліце насеньне лёну і пасыпце імі салату. У~складзе цэльных прадуктаў поліненасычаныя тлушчы стабільнейшыя.

\paragraph{Амега-7 тлустыя кіслоты.}
Акрамя згаданых амэга-9 монаненасычаных тлустых кіслотаў, ёсьць шэраг і іншых карысных злучэньняў. Напрыклад, пальміталеінавая кіслата\index{кіслата!пальміталеінавая}, гэта асноўны прадстаўнік амэга-7 кіслот. Яе шмат у~рыбе, макадаміі, абляпіхавым алеі. Яна паляпшае мэтабалізм і зьмяншае ўзровень сыстэмнага запаленьня.

\paragraph{Спалучэньне зь зёлкамі.}
Тлушчы выдатна экстрагуюць тлушчараспушчальныя карысныя злучэньні, у~тым ліку і духмяныя. Можна настаяць аліўкавы алей на часныку, размарыне, базіліку і інш.

\paragraph{Запраўка салаты.}
Ня толькі алеем можна заправіць салату ці гародніну. Вы можаце таксама аддаць перавагу і шэрагу менш калярыйных заправак: кефіру, ёгурту, воцату, соку цытрыны і інш.

\paragraph{Кета.}
Кетадыета~--- гэта дасягненьне і ўтрыманьне харчовага кетозу, калі значная частка энэргіі выкарыстоўваецца з~кетонавых целаў. Па сутнасьці, кетадыета імітуе галаданьне. Дзеля дасягненьня кетозу абмяжоўваюцца вугляводы да 30 (15) грамаў, пры гэтым бялкі не павінны складаць больш за 25\,\% ад агульнай калярыйнасьці. Да лекавых уласьцівасьцяў кетадыеты можна аднесьці зьніжэньне прагрэсаваньня некаторых відаў эпілепсіі, раку, ажно да рэмісіі. Кетадыета дапамагае зьнізіць узровень запаленьня, нармалізаваць імунны адказ. Да станоўчых уласьцівасьцяў таксама адносяцца паляпшэньне настрою (больш высокая сімпатаадрэналавая актыўнасьць), высокія кагнітыўныя функцыі, зьніжэньне голаду (кетонавыя целы прыгнятаюць апэтыт), пахудзеньне з~захаваньнем цяглічнай масы, зьніжэньне рызыкі шматлікіх захворваньняў і г.~д. Пры наяўнасьці любых захворваньняў абавязковая кансультацыя са спэцыялістам. У~цэлым сярэдне- і нізкакалярыйныя дыеты паказваюць лепшыя вынікі ў~дачыненьні да працягласьці жыцьця і зьніжэньня запаленьня, чым кетадыета.

\tipbox{Да лекавых уласьцівасьцяў кетадыеты можна аднесьці зьніжэньне прагрэсаваньня эпілепсіі, некаторых відаў раку, ажно да рэмісіі. Кетадыета дапамагае зьнізіць узровень запаленьня, нармалізаваць імунны адказ. Аднак кетадыета пры названых станах зьяўляецца дапаможным, а~не асноўным спосабам лекаваньня.}

Сярод адмоўных бакоў кетадыеты~--- адносна складанае датрыманьне з~кантролем кетонавых целаў і дыетычнымі абмежаваньнямі (любыя абмежавальныя дыеты павялічваюць рызыку парушэньняў харчовых паводзінаў), парушэньні ліпіднага профілю, зьніжэньне адчувальнасьці да інсуліну, павелічэньне нагрузкі на мочапалавую сыстэму, нэгатыўны ўплыў на мікрафлёру кішачніка, дэфіцыт мінэралаў, вітамінаў, клятчаткі. Першаснае ўваходжаньне ў~кетоз суправаджаецца зьніжэньнем працаздольнасьці і шэрагам нэгатыўных чыньнікаў.

Аптымальнай зьяўляецца кетадыета, якая праводзіцца пэрыядычна для карэкцыі пэўных захворваньняў і станаў, узімку, пры гэтым на тле нізкакалярыйнага харчаваньня і ўмеранай колькасьці бялку. Істотна: кетадыета патрабуе дысцыпліны і ўважлівага плана. Эфэкты доўгатэрміновай кетадыеты недастаткова вывучаныя, гэтак жа як і яе ўплыў на працягласьць жыцьця. Часта прыхільнікі кетадыеты робяць шмат памылак, пачынаючы ад выкарыстаньня няякасных тлушчаў да занадта вялікай колькасьці бялку. У~цэлым кетадыета хутчэй лекавая працэдура для людзей, якія дакладна разумеюць, дзеля чаго да яе зьвяртаюцца.

\paragraph{Больш тлушчу.}
Многія людзі часта разумеюць параду «тлушч не такі й~шкодны» як параду есьці больш тлустага. Але проста павелічэньне колькасьці тлушчу бязь іншых зьменаў у~харчаваньні прынясе толькі шкоду, бо высакатлушчавыя дыеты з~цукрам і канцэнтраванымі вугляводамі адно павялічваюць запаленьне, рызыку атлусьценьня, пагаршаюць мікрафлёру кішачніка і г.~д.

\chapter{Бялкі}

Бялкі~--- гэта найважнейшая частка рацыёну, амінакіслоты, якія ўваходзяць у~іх склад, зьяўляюцца істотным будаўнічым матэрыялам, а~таксама крыніцай энэргіі. Важна падтрымліваць аптымальную колькасьць якаснага бялку, каб кантраляваць насычэньне і пры гэтым не падвышаць рызыкі шэрагу захворваньняў. У~дачыненьні да бялкоў многія дыетолягі займаюць процілеглыя пазыцыі, ад пастуляваньня поўнай шкоды жывёльных бялкоў (вэган) да прызнаньня іх абсалютнай карысьці ў~любых колькасьцях (карнівор). Акрамя колькасьці бялку, вельмі важным зьяўляецца яго якасьць, што зьвяза са зьменамі ў~яго апрацоўцы і захоўваньні. Давайце разьбяромся, як нам навучыцца атрымліваць ад бялковых прадуктаў максымум карысьці і пазьбягаць іх неспрыяльных уздзеяньняў на наша здароўе.

\subsection{Як зьявілася праблема?}

У працэсе эвалюцыі нашыя продкі ў~свой час з~расьліннай ежы перайшлі на ўсяеднасьць і павялічылі долю мясной ежы. Гэта дазволіла павялічыць таксама колькасьць больш даступных калёрыяў і паскорыць эвалюцыю. Акрамя таго, бялковая ежа ўтрымлівала шмат важных для разьвіцьця мозгу рэчываў. Першасныя міграцыі людзей адбываліся ўздоўж берагоў акіянаў і мораў, багатых даступнай бялковай ежай ад малюскаў да рыбы. Ёд, амэга-3 тлустыя кіслоты ды іншыя рэчывы з~марской бялковай ежы аказваюць стымулюючае ўздзеяньне на разьвіцьцё мозгу, некаторыя палеаантраполягі бачаць у~гэтым ключ да эвалюцыі мозгу. Расьсяляючыся па зямным шары, нашы продкі пакідалі к'ёкенмэдынгі~--- агромністыя шматмэтровыя горы з~ракавінак зьедзеных малюскаў.

Паляўнічыя-зьбіральнікі елі ня толькі мяса жывёлаў, але й~косткі, храсткі, іншыя субпрадукты. І~гэта правільнае рашэньне, бо ў~печані, напрыклад, утрымліваецца больш вітамінаў і мінэралаў, чым у~чырвоным мясе. Аднак пераход да земляробства прывёў да скарачэньня колькасьці жывёльнай ежы, зьбядненьня рацыёну, павелічэньня спажываньня крупаў, што адбілася на здароўі і вонкавым выглядзе нашых продкаў і нават прывяло да памяншэньня сярэдняга росту.

У шматлікіх традыцыйных культурах спажываньне бялковых, у~першую чаргу мясных, прадуктаў было збалянсаваным. Ад мяса практычна нідзе не адмаўляліся (нават у~будызьме), але ў~цэлым яно абмяжоўвалася. Так, у~хрысьціянстве ў~дні даволі шматлікіх постаў мяса не было. Такі рэжым ужываньня мяса імітуе рэжым харчаваньня паляўнічых-зьбіральнікаў: шмат мяса падчас пасьпяховага паляваньня і шмат расьліннай ежы падчас зьбіральніцтва (няўдалае паляваньне).

\tipbox{Ёд, амэга-3 тлустыя кіслоты ды іншыя рэчывы, якія ўтрымліваюцца ў~марской ежы, аказваюць стымулюючае ўздзеяньне на разьвіццё мозгу, некаторыя палеаантраполягі бачаць у~гэтым ключ да эвалюцыі мозгу.}

У сучасным сьвеце колькасьць мяса ў~рацыёне павялічваецца, пры гэтым часта яно ўжываецца ў~нездаровым харчовым кантэксьце: не зь зелянінай і гароднінай, а~з салодкім і мучным. Бялковыя прадукты прысутнічаюць практычна ў~кожным прыёме ежы. У~структуры бялковых прадуктаў таксама назіраецца перакос у~бок перапрацаванага чырвонага мяса і малочных прадуктаў з~памяншэньнем долі расьліннага бялку, яек і рыбы. Такое высокае спажываньне бялку павялічвае рызыкі шматлікіх «захворваньняў цывілізацыі» (атлусьценьня, дэпрэсіі, аўтаімунных захворваньняў, раку і да т.~п.).

\subsection{Як гэта ўплывае на здароўе?}

\paragraph{Голад і сытасьць.}
Бялкі добра засвойваюцца, сытасьць захоўваецца надоўга, чым і тлумачыцца наступнае зьніжэньне колькасьці спажываных калёрыяў. Высокабялковыя дыеты спрыяюць пахудзеньню, пры гэтым дапамагаючы лепш кантраляваць апэтыт, чым нізкабялковыя ці нізкатлушчавыя дыеты. Тым ня менш уплыў доўгатэрміновых дыетаў розных відаў на вагу не адрозьніваецца.

\paragraph{Агульны тонус.}
Вялікая колькасьць бялку ў~харчаваньні аказвае стымулюючае ўзьдзеяньне, узмацняе сымпацыйны тонус, павялічвае энэргічнасьць і настрой.

\paragraph{Мэтабалізм.}
Надмер шэрагу амінакіслотаў, у~прыватнасьці амінакіслотаў з~разгалінаваным ланцугом (ВСАА), пагаршае адчувальнасьць да інсуліну. Дыеты з~больш высокім утрыманьнем бялку хоць раўназначныя ў~лічбах скінутай вагі, але пры гэтым паказваюць горшую адчувальнасьць да інсуліну ў~параўнаньні з~дыетамі зь нізкім утрыманьнем бялку. Пры гэтым высокае ўтрыманьне бялку можа зьмяншаць узровень трыгліцэрыдаў і колькасьць тлушчу ў~печані.

\tipbox{Бялкі добра засвойваюцца, сытасьць захоўваецца надоўга, чым і тлумачыцца наступнае зьніжэньне колькасьці спажываных калёрыяў. Высокабялковыя дыеты спрыяюць пахудзеньню, пры гэтым дапамагаючы лепш кантраляваць апэтыт.}

\paragraph{Стымулятары mTORС.}
ВСАА амінакіслоты (лейцын, ізалейцын і валін) і метыянін стымулююць актыўнасьць mTORС. Само па сабе гэта карысна пэрыядычна, але залішняя стымуляцыя прыводзіць да павелічэньня рызыкі шэрагу захворваньняў: ад атлусьценьня і аўтаімунных захворваньняў да заўчаснага старэньня, зьніжэньня хуткасьці аўтафагіі і рызыкі пухлінавых захворваньняў (гл. разьдзел «Хуткая і павольная ежа»). Ёсьць дадзеныя, што высокае спажываньне ВСАА прыводзіць да зьніжэньня працягласьці жыцьця.

Больш за ўсё гэтых амінакіслотаў у~малочных прадуктах, асабліва ў~сыроватачным бялку і малацэ, менш~--- у~расьлінным бялку (за выключэньнем соевага бялку). Часта ВСАА неўзаметку прысутнічае ў~выглядзе соевага парашка, сухога малака і г.~д. Важная ўласьцівасьць ВСАА~--- магутная стымуляцыя інсуліну і актывацыя mTORС, што ўзмацняецца ў~спалучэньні з~высокаглікемічнымі вугляводамі. Залішняя ўвесьчасная актывацыя mTORС прыводзіць да залішняй сімпатаадрэналавай актыўнасьці і ўзмацняе стрэс, трывогу, нэўратычнасьць, павялічвае артэрыяльны ціск і пагаршае рэляксацыю і сон.

\paragraph{Рызыка захворваньняў.}
Высокабялковыя дыеты, асабліва на тле высокакалярыйнага харчаваньня, павялічваюць рызыку сардэчна-сасудзістых захворваньняў, нагрузку на ныркі. У~людзей з~утрыманьнем бялку ў~дыеце больш за 20\,\% ад каляражу рызыка захворваньняў вышэйшая, чым у~людзей з~утрыманьнем бялку ў~дыеце 10\,\% і менш. Высокабялковая дыета павялічвае рызыку захворваньня на цукровы дыябэт другога тыпу і рызыку ракавых захворваньняў. Дыета зь нізкай колькасьцю мэтыяніну паляпшае хаду некаторых анкалягічных захворваньняў, зьніжае ўзровень сыстэмнага запаленьня.

\paragraph{Старэньне.}
Лішак бялку ў~ежы, асабліва жывёльнага, скарачае працягласьць жыцьця і паскарае старэньне. Пры гэтым расьлінныя бялкі ня маюць такога ўплыву на захворваньне і працягласьць жыцьця, як жывёльныя (празь меншае ўтрыманьне мэтыяніну і ВСАА). Дыеты зь нізкім утрыманьнем мэтыяніну памяншаюць рызыку разьвіцьця шэрагу захворваньняў і павялічваюць працягласьць жыцьця. Абмежаваньне прадуктаў, багатых лейцынам (уваходзіць у~ВСАА), узьдзейнічае амаль эквівалентна нізкакалярыйнаму харчаваньню і падаўжае жыцьцё.

\paragraph{Засваеньне мінэралаў.}
Мінэралы, якія атрымліваюцца з~жывёльных прадуктаў, засвойваюцца нашмат лепш, чым з~расьліннай ежы. Антынутрыенты запавольваюць усмоктваньне цынку, жалеза і сэлену, таму мяса лепш есьці з~гароднінай, а~ня з~крупамі (збажына, бабовыя). Аднак лішак жалеза (часьцей у~мужчын), назапашваючыся з~узростам, вельмі нэгатыўна ўплывае на здароўе. Жалеза (як і медзь)~--- гэта мэтал зь пераходнай валентнасьцю, таму яго надмер паскарае перакіснае акісьленьне тлушчаў, што павялічвае рызыку шматлікіх захворваньняў~--- ад атлусьценьня і раку да нэўрадэгенэратыўных захворваньняў. Дэфіцыт жалеза (часьцей у~жанчын)~--- таксама сур'ёзная праблема, якая пагаршае самаадчуваньне, вонкавы выгляд і здароўе.

\tipbox{Ёд, цынк, селен, бор і дзясяткі іншых мінэралаў сустракаюцца ў~добрай канцэнтрацыі ў~морапрадуктах. Рыбны бялок заўважна адрозьніваецца ад бялку чырвонага і белага мяса, ён валодае даведзеным антыгіпэртэнзіўным эфэктам, стымулюе фібрыноліз, спрыяе зьніжэньню вагі і зьніжае ўзровень запаленчага маркера С-рэактыўнага бялку, паляпшае адчувальнасьць да інсуліну.}

\paragraph{Рыба і морапрадукты.}
У сучасным рацыёне пераважае ўжываньне бялкоў з~малочных прадуктаў і чырвонага мяса. У~сваю чаргу, рыба і морапрадукты ўтрымліваюць мноства неабходных для здароўя злучэньняў: таўрын, астаксанцін, сэлен, цынк, вітамін D, ёд, амэга-3 тлустыя кіслоты і г.~д. Таўрын~--- гэта важная амінакіслата, якая ахоўна ўзьдзейнічае на шматлікія паказьнікі здароўя і спрыяе даўгалецьцю. Рыбны бялок выгодна адрозьніваецца ад бялку чырвонага і белага мяса, ён валодае даведзеным антыгіпэртэнізіўным эфэктам, стымулюе фібрыноліз, спрыяе зьніжэньню вагі і ўзроўню С-рэактыўнага бялку, паляпшае адчувальнасьць да інсуліну. Большасьць эфэктыўных дыетаў утрымліваюць морапрадукты і рыбу ў~сваім складзе.

Дасьледаваньні паказваюць, што морапрадукты перадухіляюць разьвіццё хваробы Альцгеймэра, страту слыху, павялічваюць інтэлект, захоўваюць жаночае і мужчынскае здароўе, спрыяюць зачацьцю і да т.~п. Для дасягненьня эфэкту ня трэба есьці, як эскімос, дастаткова ўмеранага даданьня ў~рацыён морапрадуктаў і рыбы. Цікава, што асобныя дадаткі амэга-3 ня могуць прыраўняцца да эфэктыўнасьці цэльнага прадукту для прафіляктыкі хваробы Альцгеймэра. Нават адна порцыя рыбы на тыдзень, з'едзеная ў~пажылым узросьце, можа аддаліць наступленьне дэмэнцыі і хваробы Альцгеймэра, у~тым ліку генэтычна дэтэрмінаваных.

\subsection{Асноўныя прынцыпы}

\paragraph{Колькасьць.}
Колькасьць спажыванага бялку можна вар'іраваць у~залежнасьці ад узроўню фізычнай актыўнасьці і ўзросту. Менш бялку могуць ужываць людзі ва ўзросьце 40--55 гадоў. Для людзей, старэйшых за 65 гадоў, разумна павялічыць долю бялку ў~рацыёне. Мінімальная рэкамэндаваная колькасьць бялку на содні складае 0,8 г / кг масы цела. Дыета з~высокім утрыманьнем бялку ўключае больш за 1,5 г / кг, умераным 0,8--1,3 г, менш за 0,8~--- нізкім. Болей бялку вы можаце ўжываць пры больш высокім узроўні фізычнай актыўнасьці.

\paragraph{Рэжым.}
Высокабялковая дыета ў~доўгатэрміновай пэрспэктыве нясе патэнцыйны нэгатыўны ўплыў на здароўе. Аднак для падвышэньня тонусу і энэргічнасьці мы можам ладзіць сабе высокабялковыя дні, ураўнаважваючы іх вэганскімі днямі сярод тыдня. Большую колькасьць бялку ў~рацыёне варта збалянсаваць нізкакалярыйнай зялёнай гароднінай і ўмеранай колькасьцю спажываных калёрыяў, высокабялковыя дні можна рабіць падчас інтэнсіўных трэніровак.

\paragraph{Выбар аптымальных крыніц бялку.}
Найлепшымі крыніцамі бялку зьяўляюцца морапрадукты, яйкі, мяса хатняй жывёлы пашавага выпасу. Расьлінны бялок~--- таксама крыніца важных амінакіслотаў, пры гэтым ён пазбаўлены многіх нэгатыўных аспэктаў жывёльных бялкоў. Чырвонае мяса рэкамэндуецца есьці не часьцей за 2--3 разы на тыдзень.

\paragraph{Марскі бялок.}
Найлепшай крыніцай бялку і мноства іншых незаменных злучэньняў зьяўляецца «марская дыета», якая мяркуе ўжываньне рыбы дзікай лоўлі, ракападобных, малюскаў і да т.~п. Аптымальна есьці марскую рыбу 2--3 разы на тыдзень (але ня смажаную, вэнджаную, а~вараную).

\tipbox{Навукова даведзена, што самым шкодным сярод мясных прадуктаў зьяўляецца перапрацаванае мяса (бэкон, сасіскі, каўбасы, паўфабрыкаты і да т.~п.): яно ўваходзіць у~сьпіс канцэрагенаў. Усяго 50 грамаў такога мяса на содніі дастаткова, каб заўважна павысіць рызыку сардэчна-сасудзістых захворваньняў (на 42\,\%) і дыябэту (на 19\,\%).}

\paragraph{Мяса жывёлаў пашавага выпасу, мяса і яйкі хатняй птушкі.}
Пры гэтым ня трэба факусавацца толькі на мясе, важна есьці й~субпрадукты. Белае і чырвонае мяса можна есьці некалькі разоў на тыдзень у~цэльным выглядзе. Што тычыцца расьліннага бялку, ён менш збалянсаваны па амінакіслотах, таму трэба камбінаваць розныя яго віды для атрыманьня поўнай нормы незаменных амінакіслотаў.

\paragraph{Ідэальная якасьць.}
Якасьць~--- гэта найважнейшае патрабаваньне да бялковых прадуктаў, якія хутка псуюцца, вельмі адчувальныя да награваньня, сьвятла, забруджваньня (у іх размнажаюцца бактэрыі), працяглых тэрмінаў захоўваньня, тэрмаапрацоўкі. Выбірайце і купляйце цэльныя фрагмэнты ці тушы мяса, на костцы і са скурай у~правераных пастаўнікоў і гатуйце іх здаровымі спосабамі. Навукова даведзена, што самым шкодным сярод мясных прадуктаў зьяўляецца перапрацаванае мяса (бэкон, сасіскі, каўбасы, паўфабрыкаты і да т.~п.), аднесенае да сьпісу канцэрагенаў. Усяго 50 грамаў такога мяса на содні дастаткова, каб прыкметна павысіць рызыку сардэчна-сасудзістых захворваньняў (на 42\,\%) і дыябэту (на 19\,\%). Пры смажаньні мяса адбываецца назапашваньне ў~ім канчатковых прадуктаў глікацыі (гл. разьдзел «Захоўваньне і гатоўля»), а~таксама зьніжэньне колькасьці важных злучэньняў.

\subsection{Як трымацца правіла? Ідэі і парады}

\paragraph{Кантроль апэтыту.}
Бялок выдатна насычае, таму, калі вы губляеце кантроль над апэтытам, зрабіце некалькі высокабялковых дзён, каб яго аднавіць.

\paragraph{Бялковыя дні.}
У дні высокай фізычнай ці разумовай нагрузкі вы можаце кароткачасова перайсьці на «рэжым паляўнічага», прыкметна павялічыўшы частку бялку ў~рацыёне. Гэта дазволіць вам выкарыстоўваць яго стымулюючае дзеяньне.

\paragraph{У першую чаргу~--- бялок.}
Бялок варта есьці першым падчас сталаваньня~--- так ён лепш насычае.

\paragraph{Інсулінарэзыстэнтнасьць.}
Высокабялковая дыета вядзе да зьніжэньня адчувальнасьць да інсуліну, нават пры пахудзеньні. Гэта вядзе да захаваньня падвышанай рызыкі інсулінарэзыстэнтнасьці, таму пазьбягайце высокабялковых дыетаў працяглы час.

\paragraph{ВСАА.}
Надмер амінакіслотаў з~разгалінаваным ланцугом у~рацыёне (малочныя прадукты, яйкі, мяса і да т.~п.) або ў~выглядзе спартовых дабавак можа запаволіць цяглічны рост у~атлетаў-пачаткоўцаў (за кошт хутчэйшага аднаўленьня), зьнізіць адчувальнасьць да інсуліну і зрабіць яшчэ шэраг адмоўных узьдзеяньняў. Але мы можам выкарыстоўваць гэтыя дадаткі ў~выпадках экстранагрузак (маратон і г.~д.). Трыптафан зьяўляецца папярэднікам такіх важных малекулаў, як сэратанін і мэлятанін. Рэч у~тым, што трыптафан слаба пранікае праз гематаэнцэфалічны бар'ер, яго перанос залежыць ад суадносінаў у~крыві трыптафан / ВСАА. Таму пасьля бялковай ежы ўзровень трыптафану вырастае ў~плазьме крыві, але ня ў~мозгу. Калі вы зьядаеце расьлінную ежу, то інсулін зьніжае ўзровень ВСАА (але не трыптафану) у~плазьме крыві, узмацняючы іх паглынаньне цягліцамі. Зьніжэньне ўзроўню ВСАА вядзе да зьмены суадносінаў трыптафан / ВСАА, што ўзмацняе перанос трыптафану ў~мозг і паляпшае самаадчуваньне. Даданьне нават невялікіх колькасьцяў бялку да вугляводнай ежы прыводзіць да павелічэньня ўзроўню ВСАА, што блякуе гэты мэханізм. Дастаткова 4\,\% бялку, каб прадухіліць зьмяненьне суадносінаў трыптафан / ВСАА. Таму ня варта дадаваць да кожнага прыёму ежы багатыя ВСАА прадукты!

\tipbox{Зьнізіць узровень жалеза ў~крыві дапаможа донарства. Здавайце 1--3 разы на год кроў і рэгулярна правярайце фэрытын, асабліва калі ў~вас ёсьць генэтычная схільнасьць да назапашваньня жалеза.}

\paragraph{Бялковыя кактэйлі.}
Бялковыя парашкі~--- гэта «бедныя» калёрыі, бо ў~іх шмат калёрыяў і мала дадатковых нутрыентаў (мінэралы, вітаміны і да т.~п.) у~параўнаньні з~цэльнымі прадуктамі. Магчыма, яны будуць карыснымі для прафэсійных атлетаў, але не для тых, хто не займаецца фізычнымі нагрузкамі.

\paragraph{Вэганства і карнівор (zero carb diet).}
Як поўная адмова ад мяса, так і высокабялковая дыета на адным мясе не зьяўляюцца здаровымі варыянтамі харчаваньня, гэтыя скрайнасьці шкодзяць здароўю.

\paragraph{Лішак і недахоп жалеза.}
Вялікая колькасьць чырвонага мяса (асабліва ў~рацыёне мужчынаў) прыводзіць да назапашваньня залішняга жалеза. Праверыць яго варта, здаўшы аналіз на ўзровень фэрытыну. Калі ён высокі, то лепшы спосаб зьніжэньня~--- стаць донарам. Здавайце 1--3 разы на год кроў, рэгулярна правярайце ўзровень жалеза, асабліва калі ў~вас ёсьць генэтычная схільнасьць да яго назапашваньня.

\paragraph{Мэтыянін і кантэкст дыеты.}
Істотна ўплывае і кантэкст дыеты. Напрыклад, у~чалавека, які есьць шмат мяса і мучнога, мала зеляніны і гародніны, спажываньне мэтыяніну можа прыводзіць да павелічэньня ў~крыві гомацыстэіну, што вельмі шкодна для сасудаў. А~вось калі зь мясам есьці больш гародніны, то мэтыянін будзе лепш засвойвацца з~наяўнасьцю вітамінаў В12, В9, прадуктаў-донараў мэтыльных групаў (халін, бэтаін у~капустах і да т.~п.). Людзям з~мутацыямі фолатнага цыклю (MTHFR, MTR, MTRR) і з~падвышаным гомацыстэінам варта таксама больш уважліва ставіцца да спажываньня чырвонага мяса.

\paragraph{Гатоўля.}
Бялковыя прадукты вельмі адчувальныя да гатоўлі (гл. разьдзел па гатаваньні). Аптымальныя ашчадныя спосабы гатоўлі, марынаваньне мяса кіслым (воцат, цытрына) і спэцыямі, варыць, а~ня смажыць. Гэта скарачае тэрмін гатаваньня і прадухіляе ўтварэньне шкодных рэчываў. Цікава, што вараная рыба павялічвае працягласьць жыцьця і паляпшае настрой (зьніжэньне рызыкі дэпрэсіі), а~вось смажаная~--- не.

\paragraph{Мяса, зеляніна і гародніна.}
Карысныя кампанэнты зеляніны і гародніны нават у~страўніку абясшкоджваюць тыя шкодныя рэчывы, якія ўтворацца пры смажаньні мяса. Таму найлепшы гарнір да мяса~--- гэта зеляніна (базілік, рукала, шпінат), закрасы (часнык, лук), крыжакветныя (кале, брокалі, пэкінская капуста і г.~д.).

\paragraph{Бялок і цяглічны рост.}
Вядома, дастатковы ўзровень бялку важны для цяглічнага росту. Але яго ўплыў часта пераацэньваецца. Так, сілавыя трэніроўкі павышаюць сынтэз бялку на 40\,\%, а~выкарыстаньне бялку дадае толькі 10\,\%, то-бок ключавым момантам зьяўляюцца трэніроўкі, а~не бялок. Залішні бялок не павялічвае інтэнсіўнасьці росту цягліцаў.

\paragraph{Малочныя прадукты.}
Малако~--- гэта не звычайны структурны бялок (як мяса), а~сыгнальны, які актывуе рост маладых арганізмаў. У~ім шмат лейцыну (да 14\,\%), ён мае высокі інсулінавы індэкс і пры ўжываньні ў~вялікай колькасьці можа выклікаць шэраг адмоўных эфэктаў. У~малацэ таксама ёсьць цукар галактоза, які паскарае старэньне, і ляктоза, непераноснасьць якой (як прыроджаная, так і набытая) нэгатыўна ўплывае на работу кішачніка. Залішняе спажываньне малака можа павялічваць рызыку разьвіцьця некаторых відаў раку. Асабліва неспрыяльна спалучэньне малочнага, салодкага і мучнога разам. Таму ўжываньне цэльных малочных прадуктаў варта абмежаваць, зрабіўшы ўмеранае выключэньне для фэрмэнтаваных прадуктаў (цэльныя кефір, ёгурт) або сыроў, але ў~невялікай колькасьці. Так, кефір можна ня проста піць, а~запраўляць ім салату. А~ўжо пра кефір на ноч, як мы цяпер ведаем, і зусім варта забыць!

\paragraph{Соя і яе вытворныя.}
Соевы бялок часта выкарыстоўваецца ў~гатовай ежы і паўфабрыкатах. Ён мае высокае ўтрыманьне мэтыяніну (у параўнаньні зь іншымі бабовымі), высокую эстрагенную актыўнасьць і ня вельмі добра засвойваецца. Іншыя бабовыя больш карысныя, чым соевыя прадукты. Іх есьці можна, але ва ўмеранай колькасьці.

\paragraph{Нізкагістамінавая дыета.}
Вельмі часта прычынай непераноснасьці асобных прадуктаў зьяўляецца не харчовая алергія, а~непераноснасьць гістаміну (фальшывая харчовая алергія). У~гэтым выпадку добра дапамагае абмежаваньне прадуктаў, якія ўтрымліваюць гістамін ды іншыя біягенныя аміны. Паўфабрыкаты зь бялковых прадуктаў, працяглае іх захоўваньне павялічваюць рызыку іх непераноснасьці ў~адчувальных да гэтага людзей. У~нізкагістамінавую дыету ўваходзіць выключэньне мясных кансэрваў, сушанага, вяленага мяса, мяса працяглага захоўваньня, каўбасы, фаршу, мяса бяз даты расфасоўкі і г.~д. Шмат можа быць гістаміну ў~рыбе, асабліва ў~тунцы, сардзіне, скумбрыі, у~рыбных соўсах, сьвежай крамнай рыбе. Выбірайце рыбу глыбокай замарозкі і не размарожвайце яе працягла ў~лядоўні. У~вытрыманых сырах таксама можа зьмяшчацца шмат гістаміну ды іншых амінаў. Пры гэтым сьметанковае масла і маладыя сыры дапушчальныя.

\chapter{Водны балянс}

Слушны водны балянс важны для здаровага харчаваньня. Піцьцё дастатковай колькасьці вадкасьці дабратворна адбіваецца на рабоце мозгу, цягліцаў, кішачніка ды іншых органаў. Хранічны стрэс, узрост могуць прыводзіць да прытупленьня смагі, што можа выліцца ў спажываньне меншых колькасьцяў вадкасьці. А вось спажываньне «вадкіх калёрыяў» – любых вадкасьцяў, якія месьцяць калёрыі, – зьвязанае з нэгатыўным уплывам на здароўе.

У нашым мозгу ёсьць адмысловая сыстэма, якая кантралюе колькасьць вадкасьці ў арганізьме, нэўроны пітнога цэнтра выдатна кіруюць водным балянсам і смагаю. Пры недахопе вады ў нас узьнікае смага, і вада здаецца нам смачнейшай. А вось калі водны балянс у норме, мы ня хочам піць і вада здаецца нясмачнай. Гэты мэханізм дакладны, і яму можна давяраць. Калі людзі п'юць ваду па сыгналах са спэцыяльнай праграмы на смартфоне, а не калі іх смажыць, мне гэта здаецца крыху дзіўным і ненатуральным.

\section{Як зьявілася праблема?}

Здавён-даўна селішчы засноўваліся ля крыніцаў пітной вады. У апошнія дзесяцігоддзі пры актыўнай падтрымцы вытворцаў вады стаў укараняцца міт аб неабходнасьці піць вялікую яе колькасьць. Зьявіліся розныя формулы разьліку для здаровых людзей, якія змушаюць іх піць больш воды, чым хочацца, што ня вельмі правільна і карысна. Колькасьць неабходнай вадкасьці – гэта вельмі зьменлівая велічыня. Яна залежыць ад масы цела, тэмпэратуры, вільготнасьці, фізычнай актыўнасьці, дыеты і да т. п. Ёсьць агульная рэкамэндацыя: 30 мл на кіляграм масы цела, прычым гэтая колькасьць улічвае і ваду ў ежы (супы, садавіна і г.д.). Нягледзячы на тое, што часта чую пытаньні накшталт «Доктар, колькі мне напраўду піць вады ў мілілітрах?» – я не сьпяшаюся адказваць канкрэтнай і няслушнай лічбай.

Акрамя таго, людзі пачалі піць не ваду, а вялікую колькасьць напояў, якія ўтрымліваюць калёрыі: гэта і розныя салодкія газіроўкі, сокі, фрэшы, смузі і да т. п. Нават піцьцё гарбаты і кавы суправаджаецца даданьнем вялікіх колькасьцяў цукру, вяршкоў, малака, а дадаткова да такіх «напояў» спажываецца дэсэрт. Такім чынам, людзі пачынаюць піць менш вады, а больш спажываць напояў, што кепска спаталяюць голад.

\section{Як гэта ўплывае на здароўе?}

\subsection{Абязводжваньне.}
Ёсьць даныя, што нават страта вады ў 1-1,3% ад масы цела пагаршае настрой і канцэнтрацыю, павышае рызыку галаўнога болю. Калі страта вады перавышае 2% ад масы цела, зьніжаюцца нашыя фізычныя і кагнітыўныя здольнасьці, мы горш выглядаем. Страта 5% і больш вядзе да парушэньняў тэрмарэгуляцыі, што актуальна для тых, хто жыве ў гарачым клімаце, і спартоўцаў, недахоп больш чым 6-7% прывядзе да моцнай страты трываласьці. Абязводжваньне пагаршае стан скуры, можа справакаваць галаўны боль, парушыць працу страўнікава-кішачнага тракту, зьменшыць разумовыя здольнасьці, узмацніць апэтыт.

Страта вады ў 1-1,3% ад масы цела пагаршае настрой і канцэнтрацыю, павышае рызыку галаўнога болю. Калі страта вады перавышае 2% ад масы цела, зьніжаюцца нашы фізычныя і кагнітыўныя здольнасьці. Страта 5% і больш вядзе да парушэньняў тэрмарэгуляцыі, што актуальна для тых, хто жыве ў гарачым клімаце, і спартоўцаў, недахоп больш чым 6-7% прывядзе да моцнай страты трываласьці.

\subsection{Дадатковая вадкасьць.}
Так, спажываючы залішнюю колькасьць вадкасьці, можна нашкодзіць свайму здароўю. Часта людзі спрабуюць піць у запас, баючыся абязводжваньня, што можа быць небясьпечна. Дадатковае ўжываньне вадкасьці не ўплывае на зьніжэньне рызыкі захворваньняў (даныя аналізу назіраньняў на 120 тысячах людзей цягам 10 гадоў), залішняя вада нават не спрыяе ўвільгатненьню вашай скуры і не ўплывае на працягласьць жыцьця. Такім чынам, спажываньне вады пры абязводжваньні дапамагае, а вось пітво звыш нормы ані не ўплывае на здароўе, за выключэньнем шэрагу невялікіх рызык.

\subsection{Вадкія калёрыі.}
Дасьледаваньні паказваюць, што спажываньне фруктовых сокаў, квасу ды іншых салодкіх вадкасьцяў небясьпечнае для здароўя. У сярэднім у 1 шклянцы фруктовага соку бяз цукру зьмяшчаецца каля 20-23 грамаў цукраў. Хуткасьць усмоктваньня цукру з соку нашмат вышэйшая, чым з цэльнага фрукта: садавіна як такая не дае магчымасьці пераесьці. Арганічныя кіслоты ў соках маскуюць цукар, таму людзі недаацэньваюць рэальную небясьпеку сокаў для здароўя і іх калярыйнасьць. Вадкія калёрыі спрыяюць карыесу (цукар + кіслата), павялічваюць рызыку атлусьценьня, цукровага дыябэту. Салодкія напоі прыводзяць да пагаршэньня работы мозгу і ў маладых людзей, зьніжаюць аб'ём мозгу і павялічваюць рызыку хваробы Альцгеймэра. Шклянка апэльсінавага соку прыгнятае эфэктыўнасьць тлушчаспаленьня на 25%. Устаноўленая карысьць фрукта не пераносіцца аўтаматычна на сок – ужываньне граната зьніжае рызыку разьвіцьця сардэчных хваробаў, а гранатавы сок не валодае падобнымі эфэктамі.

\section{Асноўныя прынцыпы}

Піце чыстую ваду, не выкарыстоўвайце сокі, ваду зь мёдам, салодкія газіроўкі, смузі, гарбату, каву для здаволеньня смагі!

\subsection{Правіла тэставага глытка: трымайце каля сябе бутэльку.}
Калі вас раптам засмажыла (ці вам здалося, што гэта смага), то адпіце адзін глыток і спыніцеся. Калі вы адчулі задавальненьне і смага стала выразьнейшай, то піце больш. Самае галоўнае – адчуць эмоцыю пасьля першага глытка. Ключ да правільнага выкарыстаньня гэтага, так напраўду, відавочнага мэтаду – уменьне прыслухоўвацца да сыгналаў уласнага цела, усвядомленасьць.

Арганізм працуе па мэтадзе каліброўкі: спачатку ён папаўняе адразу 50-80% дэфіцыту вадкасьці, а потым павольна дабірае рэшту. Смага зьнікае яшчэ да таго, як вада ўсмоктваецца страўнікам. Такое паступовае аднаўленьне страты вадкасьці больш натуральнае для нас.

\subsection{Піць пакрысе.}
Калі вы п'яце, зьлёгку затрымліваючы ваду, невялікімі глыткамі і смакуючы, то адбываецца больш эфэктыўнае ўхіленьне смагі. Нават калі мы папілі да адсутнасьці смагі, то папоўнілі балянс ня цалкам. Арганізм працуе па мэтадзе каліброўкі: спачатку ён папаўняе адразу 50-80% дэфіцыту вадкасьці, а потым павольна дабірае астатняе. Смага зьнікае яшчэ да таго, як вада ўсмоктваецца страўнікам. Такое паступовае аднаўленьне страты вадкасьці больш натуральнае для нас. Таму не сьпяшайцеся выпіваць адразу велізарную колькасьць вадкасьці, лепей гэта рабіць пакрысе, памяншаючы аб'ёмы выпітага.

Вызначце недахоп вадкасьці. Навучыцеся ідэнтыфікаваць у сябе простыя прыкметы недахопу вады:
1. смага – самы надзейны крытэр;
2. зьмена колеру і колькасьці мачы (чым цямнейшая мача, чым яе менш і мацнейшы пах, тым мацнейшае абязводжваньне);
3. зьмены скуры: зьніжэньне тургору скуры, вялікая выяўленасьць зморшчынаў (іх можна праверыць шчыпковым тэстам: ушчыкніце сябе і адпусьціце – скура павінна хутка расправіцца, у адваротным выпадку ёсьць недахоп вады);
4. зьмена сьлізьніцаў (сухасьць у роце, перасыханьне сьлізьніцы носа, сухі язык і г.д.);
5. цяглічная слабасьць, павелічэньне частаты пульса;
6. вы ў стрэсе і не заўважаеце абязводжваньня, зрабіце тэставы глыток.

\section{Як трымацца правіла? Ідэі і парады}

\subsection{Фальшывая смага.}
Пры стрэсе павялічваецца актыўнасьць сымпацыйнай нэрвовай сыстэмы, што гняце работу сьлінных залозаў і выклікае адчуваньне перасыханьня ў роце. Пасьля аднаго глытка вады жаданьне піць зьнікае. Увільгатненьне ротавай поласьці прыбірае фальшывую смагу.

Пры стрэсе павялічваецца актыўнасьць сымпацыйнай нэрвовай сыстэмы, што гняце работу сьлінных залозаў і выклікае адчуваньне перасыханьня ў роце. Пасьля аднаго глытка вады жаданьне піць зьнікае. Увільгатненьне ротавай поласьці прыбірае фальшывую смагу.

\subsection{Прапала смага.}
Смага часта можа зьнікаць, асабліва ва ўмовах хранічнага стрэсу (калі, па лёгіцы арганізму, ужо лепш абязводжваньне, чым страта натрыю). Але тэставы глыток вады актывуе ўтоенае пачуцьцё смагі, і зьяўляецца пачуцьцё задавальненьня.

\subsection{Не адкладайце ваду далёка.}
Дасьледаваньні паказваюць, што пры фізычнай дзейнасьці ў гарачым клімаце спартоўцы выпіваюць толькі 50% ад страчанай вадкасьці. Таму трэба пэрыядычна рабіць тэставы глыток, а не адкладаць далёка бутэльку, калі вы адзін раз папілі. Не, адзін раз поўнасьцю дэфіцыт не папоўніць!

\subsection{Вільготнасьць паветра.}
Узімку, калі працуе ацяпленьне і вільготнасьць паветра нізкая, мы губляем з паветрам шмат вадкасьці. Гэта можа прывесьці да дадатковай страты да 500 мл вадкасьці на содні. Сухасьць сьлізьніцаў павялічвае рызыку вострых рэспіраторных захворваньняў. Усталюйце ўвільгатняльнік паветра, падтрымлівайце аптымальную вільготнасьць удома.

\subsection{Газаваная вада.}
Газаваная вада (насычаная вуглякіслым газам) мае рознаскіраваныя эфэкты на здароўе. Так, яна можа дапамагаць пры праблемах з кішачнікам, можа зьмяншаць пачуцьцё голаду за кошт павелічэньня аб'ёму ў страўніку. Аднак вялікія яе колькасьці могуць залішне стымуляваць страўнікава-кішачны тракт у чыстых прамежках, што непажадана. Кіслотнасьці газіроўкі не асьцерагайцеся, яна нязначная.

\subsection{Мінэральны склад вады.}
Існуюць комплексныя сыстэмы ацэнкі якасьці вады, якія ўлічваюць дзясяткі паказьнікаў. Вылучаюць небясьпечныя для здароўя паказьнікі, якія непажадана перавышаць, карысныя злучэньні, якія патрабуюцца для здароўя. Многія са злучэньняў, нават тыя, што перавышаюць норму, могуць быць карысныя. Напрыклад, больш высокія канцэнтрацыі літыю ў вадзе зьніжаюць рызыку дэпрэсій, суіцыдаў і хваробы Альцгеймэра. Такім чынам, невялікае перавышэньне горных водоў па ўтрыманьні літыю можа быць карысным.

\subsection{Жорсткасьць вады.}
Жорсткасьць вады зьвязаная з утрыманьнем у ёй соляў кальцыю і магнію, дыстыляваная вада цалкам ачышчаная ад соляў. Занадта мяккая вада нават небясьпечнейшая, чым жорсткая. Залішняе ўжываньне мяккай вады, у тым ліку дыстыляванай, можа павышаць рызыку сардэчна-сасудзістых захворваньняў і вымываць карысныя мінэралы з арганізму. У рэгіёнах з жорсткай вадой сьмяротнасьць ад сардэчных захворваньняў вышэйшая, а вось на камяні ў нырках і жоўцевым пухіры жорсткая вада ніяк не ўплывае (ёсьць даныя, што нават можа зьмяншаць іх рызыку).

\subsection{Аналіз вады.}
Зрабіце аналіз вады ў сябе дома. Памятайце, што на яе якасьць уплывае і сыстэма яе разьмеркаваньня (матэрыял трубаў, яго стан). Пры разбурэньні трубаў шкодныя рэчывы могуць пападаць у ваду.

\subsection{«Палепшаныя воды».}
Існуе вялікая колькасьць «карыснай» вады – «шчолачная», «структураваная», «талая», «зараджаная», «вадародная», але яны ня маюць навукова даведзенай карысьці і ў лепшым выпадку бескарысныя. «Цяжкая» і «лёгкая» воды маюць тэарэтычнае абгрунтаваньне магчымага эфэкту, але пакуль няма надзейных дасьледаваньняў іх магчымай эфэктыўнасьці.

Існуюць міты пра ўнікальную карысьць як цёплай, так і халоднай воды, але яны ня маюць навуковага пацверджаньня. Калі вам падабаецца цёплая вада – піце яе з задавальненьнем. Вада пакаёвай тэмпературы – гэта разумны кампраміс.

\subsection{Прыгожы посуд.}
Трымайце прыгожую шклянку і графін каля сябе. Піць са шклянога і прыгожага посуду бясьпечней і прыемней, чым з плястыкавай бутэлькі.

\subsection{Кафэінзьмяшчальныя вадкасьці.}
Зьвярніце ўвагу, што лішак кафэіну ў гарбаце, каве і г. д. можа мець вялікую колькасьць нэгатыўных аспэктаў, асабліва ў адчувальных да іх людзей – ад бессані да трывожнасьці. Пазьбягайце спажываньня кафэіну пасьля 15:00.

\subsection{Час дня.}
Разумна выпіць ваду раніцай (страта вадкасьці ноччу), пасьля фізычнай актыўнасьці, меней піць на ноч.

\subsection{Прадукты харчаваньня.}
Вада ў прадуктах харчаваньня таксама лічыцца. Чым больш вы ясьцё супоў, сакавітай садавіны і гародніны, тым менш вам трэба піць.

\subsection{Карысьць цёплай вады, карысьць халоднай вады.}
Існуюць міты пра ўнікальную карысьць як цёплай, так і халоднай воды, але яны ня маюць навуковага пацьверджаньня. Калі вам падабаецца цёплая вада – піце яе з задавальненьнем. Вада пакаёвай тэмпературы – гэта разумны кампраміс.

\subsection{Вада з дабаўкамі.}
Калі звычайная вада нясмачная вам, то можаце дадаць невялікую колькасьць лайму, цытрыны, імберца. Пазьбягайце даданьня калярыйных дабавак накшталт мёду.

\subsection{Можна піць ваду падчас яды.}
Вада сьцякае па складках малой крывізны страўніка, а цьвёрдая ежа ў большай ступені знаходзіцца на вялікай крывізьне, таму вада не пагаршае страваваньня. Але часта людзі п'юць, каб аблегчыць жаваньне або хутчэй праглынуць ежу, вось у гэтым выпадку вада відавочна супрацьпаказаная. Шклянка вады перад ежай можа зьменшыць колькасьць зьедзеных калёрыяў у час сталаваньня.

\chapter{Балянс натрый – калій}

Важным правілам харчаваньня зьяўляецца захаваньне правільных суадносінаў натрыю Na + і калію К + у прадуктах харчаваньня. Яно вызначае вельмі многія паказьнікі: ад агульнага тонусу да здароўя касьцей і, што важней за ўсё, артэрыяльнага ціску.

Праблему солевага дысбалянсу вырашыць дастаткова лёгка: для гэтага нам трэба паменшыць ужываньне натрыю (перастаўшы саліць і зьнізіўшы долю гатовых прадуктаў харчаваньня) і павялічыць паступленьне калію (есьці больш цэльнай расьліннай ежы). Гэта дазволіць вам ня толькі зьменшыць рызыку артэрыяльнай гіпэртэнзіі, але й лепей пачувацца і выглядаць.

\section{Як зьявілася праблема?}

У старажытныя часы наша асяроддзе было пазбаўленае крыніцаў солі, за выключэньнем узьбярэжжа, атрымаць натрый нашы продкі маглі толькі праз жывёльную ежу. Нашаму арганізму патрабуецца натрый, таму ва ўмовах яго дэфіцыту ў нас сфармаваліся магутныя сыстэмы яго ўтрыманьня. Цікава, што сыстэмы ўтрыманьня натрыю (праз гармон альдастэрон) актывуюцца ня толькі пры яго страце, але і пры стрэсе. Бо стрэс – гэта і страта крыві, і інтэнсіўныя фізычныя нагрузкі, у тым і іншым выпадку  гэта страта натрыю і вады. Таму стрэс можа запусьціць мэханізмы ўтрыманьня натрыю і вады ў арганізьме. Удалечыні ад мора паляўнічыя-зьбіральнікі маглі абыходзіцца бяз солі, але пераход да земляробства зь пераважна вугляводнай ежай патрабуе яе наяўнасьці (саляны голад), так яна стала дарагім таварам у мінулым.

А вось калій багата ўтрымліваецца ў расьліннай ежы – ад клубняў да садавіны і ягадаў. Паляўнічыя-зьбіральнікі елі расьлінную ежу штодня, таму ў нас у арганізьме няма мэханізмаў назапашваньня і ўтрыманьня калію! Мы ніколі не адчувалі патрэбы ўтрымліваць у арганізьме калій, больш за тое, пры стрэсе арганізм нават актыўна пазбаўляецца ад калію.

Нашаму арганізму патрабуецца натрый, таму ва ўмовах яго дэфіцыту сфармаваліся магутныя сыстэмы яго ўтрыманьня. Цікава, што сыстэмы ўтрыманьня натрыю (праз гармон альдастэрон) актывуюцца ня толькі пры яго страце, але і пры стрэсе. Бо стрэс – гэта і страта крыві, і інтэнсіўныя фізычныя нагрузкі, у тым і іншым выпадку гэта страта натрыю і вады.

Сучасная праблема балянсу натрый – калій узьнікла, бо соль стала шырока даступнай і дадаецца ва ўсе прадукты (схаваная соль), формула «цукар, тлушч, соль» – гэта класічная трыяда фастфуду. Чым больш мы ямо апрацаваных прадуктаў, тым больш ямо натрыю. А вось зьніжэньне спажываньня цэльных расьлінных прадуктаў прывяло да таго, што мы спажываем нашмат менш калію, чым трэба. Цяпер сярэдні чалавек мае сутачнае спажываньне натрыю ў 2-3 разы вышэй за норму, а калію, наадварот, у 2-3 разы менш за норму.

\section{Як гэта ўплывае на здароўе?}

\subsection{Пераяданьне.}
Соль заўважна стымулюе дафамінавую сыстэму мозгу, таму выклікае пераяданьне. Салёных прадуктаў мы зьядаем больш, а вось на жывёльных мадэлях лішак солі ў дыеце прыводзіў да ангеданіі (зьніжэньня здольнасьці атрымліваць задавальненьне), а поўнае выключэньне – да дэпрэсіі. Для нашага арганізму соль – гэта чыньнік выжываньня, каманда «еж яе колькі ўлезе і запасай на будучыню». У нашым арганізьме ёсьць спэцыяльныя сыстэмы ўтрыманьня і запасаньня натрыю. Цікава, што бацькі, якія злоўжываюць сольлю, могуць перадаць гэтую цягу дзецям праз эпігенэтычныя мэханізмы.

\subsection{Водны балянс.}
Калій і натрый аказваюць супрацьлеглае дзеяньне на водны балянс: калій валодае мочагнальным эфэктам, а натрый затрымлівае ваду. Калій патрэбны для вывядзеньня лішкаў натрыю і вады з арганізму, таму багатая на яго ежа памяншае азызласьць. А вось лішак натрыю можа прыводзіць да страты калію. Стрэс, які вядзе да затрымкі натрыю, таксама можа ўзмацніць ня толькі артэрыяльны ціск, але і азызласьць. А спалучэньне «стрэс, соль, недасыпаньне і салодкае» можа нанесьці моцны ўдар па вашай зьнешнасьці.

\subsection{Сардэчна-сасудзістыя захворваньні.}
У тых краінах, дзе ядуць больш расьліннай ежы, меней сардэчна-сасудзістых захворваньняў. Так гіпэртанія (падвышаны ціск) сустракаецца толькі ў 1% насельніцтва, а вось у заходніх краінах – у 30% насельніцтва. Навукоўцы высьветлілі, што людзі з дэфіцытам калію і лішкам натрыю маюць удвая большую рызыку сьмерці ад хваробаў сэрца і на 50% павышаную рызыку памерці ад іншых захворваньняў.

\subsection{Рак страўніка.}
Залішняе спажываньне солі зьвязанае з падвышанай рызыкай раку страўніка (у спалучэньні з Helicobacter pylori) і з ракам тоўстага кішачніка.

\subsection{Азызласьць.}
Вельмі частай праблемай зьяўляецца пастознасьць, друзласьць, азызласьць, млявасьць падскурнай клятчаткі ў цэлым або ў пэўных частках цела. Адкуль бярэцца друзласьць у здаровага чалавека бязь першаснай паталёгіі лімфатычнай сыстэмы? Вінаваты залішні натрый, які назапашваецца ў падскурна-тлушчавай клятчатцы, дзе стымулюе павялічаны рост (гіпэрплазію) лімфатычных сасудаў з затрымкай вадкасьці. Сетка лімфатычных сасудаў расьце, а іх тонус падае, і гэта ўсё прыводзіць да затрымкі вадкасьці. Памятайце, што ўвесь лішак натрыю і зьвязанай вады захоўваецца менавіта ў падскурна-тлушчавай клятчатцы!

Залішні натрый назапашваецца ў падскурна-тлушчавай клятчатцы, дзе стымулюе павялічаны рост (гіпэрплазію) лімфатычных сасудаў з затрымкай вадкасьці. Сетка лімфатычных сасудаў расьце, а іх тонус падае, і гэта ўсё прыводзіць да затрымкі вадкасьці.

\subsection{Імунітэт.}
Натрый стымулюе імунітэт, што павялічвае рызыку аўтаімунных захворваньняў. Больш высокая канцэнтрацыя натрыю зьяўляецца актыватарам імуннай сыстэмы, соль стымулюе актыўнасьць імуннай сыстэмы на ўсіх узроўнях – ад дыфэрэнцыяваньня Т-лімфацытаў да актыўнасьці макрафагаў. А вось дэфіцыт натрыю можа аслабляць імунную сыстэму. Навукова даведзена, што лішак солі павялічвае рызыку захварэць на расьсеяны склероз. Павелічэньне колькасьці алергій зьвязанае з тым, што высокасолевая дыета парушае імунны адказ.

\subsection{Соль і мэтабалізм.}
Лішак натрыю зьніжае адчувальнасьць да інсуліну і павялічвае рызыку атлусьценьня. Нават зусім здаровыя маладыя людзі, якія ўжываюць вялікую колькасьць натрыю, могуць выяўляць пагаршэньне свайго здароўя. А людзей з атлусьценьнем і тых, хто ўжывае вялікую колькасьць натрыю, гэта вядзе да паскарэньня працэсу старэньня.

\subsection{Іншыя праблемы.}
Лішак натрыю павялічвае рызыку захворваньняў нырак, астэапарозу, катаракты, пагаршае кішачную мікрафлёру. Дастатковая колькасьць калію запавольвае і перадухіляе ўтварэньне камянёў у нырках і жоўцевым пухіры.

\section{Асноўныя прынцыпы}

Неабходна захоўваць правільныя суадносіны натрыю і калію ў прадуктах харчаваньня (1:3-1:4). Балянс натрыю і калію нашмат важнейшы, чым захаваньне канкрэтнай колькасьці. Зьвярніце ўвагу, што наш арганізм недахоп натрыю пераносіць лягчэй, чым яго лішак (бо ёсьць сыстэмы захоўваньня і ўтрыманьня), а вось недахоп калію – нашмат горш, чым яго лішак. Таму ня страшна часам зьесьці калію крыху больш, а натрыю – крыху менш.

Звычайная соль, якой мы падсольваем ежу, – гэта ўсяго толькі вяршыня айсбэрга, 10–20% ад агульнай колькасьці спажыванага намі натрыю. Вялікая яго частка зьмяшчаецца ў гатовых прадуктах харчаваньня, якія і зусім могуць быць несалёнымі. Да прыкладу, 40-50% натрыю паступае з хлеба і выпечкі.

\subsection{Паменшыце колькасьць натрыю.}
Для гэтага спачатку выявіце асноўныя крыніцы натрыю ў вашым рацыёне. Як ні парадаксальна, звычайная соль, якой мы падсольваем ежу, – гэта ўсяго толькі вяршыня айсбэрга, 10–20% ад агульнай колькасьці натрыю. Вялікая яго частка зьмяшчаецца ў гатовых прадуктах харчаваньня, якія і зусім могуць быць несалёнымі. Так, 40-50% натрыю паступае з хлеба і выпечкі. Вельмі шмат солі ў сырах, каўбасе (да 3 грамаў на 100 грамаў каўбасы), вэнджаніне і да т. п. Зьвярніце ўвагу, што значная колькасьць натрыю знаходзіцца ў выглядзе шматлікіх харчовых дабавак: натрыю сорбат, натрыю бікарбанат, натрыю глутамат, натрыю бэнзаат. Гэта значыць, чым больш у вашым рацыёне гатовай ежы, тым больш вы ясьцё натрыю. У цэлым пажадана спажываць ня больш за 1500 мг натрыю ў дзень.

\subsection{Не саліце ежу.}
Адмоўцеся ад пастаяннага дадаваньня солі, бо калі вы хоць зрэдку ясьцё мяса, рыбу, яйкі і гатовую ежу, то вам солі хапае.

\subsection{Павялічце колькасьць калію.}
Цэльная расьлінная ежа – гэта ўнівэрсальная крыніца калію. Калій у садавіне і гародніне выдатна засвойваецца, так як спалучэньне глюкозы зь мінэраламі дапамагае лепшаму засваеньню калію клеткамі. Зьвярніце ўвагу, што содневая патрэба ў каліі ў сярэднім 4700мг на дзень, яна заўважна павялічваецца пры фізычных нагрузках, цяжарнасьці і пры харчаваньні зь вялікай колькасьцю солі. Пры недахопе калію назіраюцца слабасьць, стамляльнасьць, паскараецца ўтварэньне камянёў, парушаецца абмен рэчываў, затрымліваецца вадкасьць у арганізьме. Вырашыць пытаньне з каліем вельмі проста, бо цэльная гародніна і садавіна ўтрымліваюць мала натрыю і шмат калію. Выбірайце, што вам даспадобы: фасоля, сачавіца, капуста, бульба, банан, авакада, буракі, памідор, салата, рэпа, радыска, пятрушка і шмат іншага.

\section{Як трымацца правіла? Ідэі і парады}

\subsection{Паступовасьць.}
Бяз солі спачатку ўсё здасца нясмачным, але гэта нармальна. За месяц адчувальнасьць рэцэптараў мяняецца, і вы зможаце аднавіць смакавую адчувальнасьць і насалоджвацца ад смакамі і адценьнямі. Лішак солі зьніжае смакавую адчувальнасьць.

\subsection{Зрэдку можна салёнае.}
У тым, каб зрэдку зьесьці нешта салёнае, небясьпекі няма. Рэдкае ўжываньне нават вельмі салёных страў не нашкодзіць, арганізм эфэктыўна дасьць рады з гэтым. Дадавайце больш гародніны да салёных страў.

\subsection{Перапрацаваная расьлінная ежа.}
Прадукты, вырабленыя з расьлінаў, тым ня менш часьцей за ўсё ўтрымліваюць лішак натрыю. Так, падсолены таматавы сок зьмяшчае лішак натрыю, а таксама шмат солі і ў хлебе.

\subsection{Недахоп солі.}
Занадта нізкае ўтрыманьне натрыю небясьпечнае і шкоднае. Але дэфіцыт натрыю ў людзей, якія спажываюць дастаткова жывёльнай ежы і хоць зрэдку нешта з гатовай ежы, малаімаверны. У групе рызыкі могуць быць вэганы (якія не ўжываюць соль і гатовыя прадукты), людзі, якія жывуць у гарачым клімаце, занятыя цяжкай фізычнай працай, пасьля дыярэі і да т.п.

\subsection{Салодкае (і крухмалістае) і азызласьць.}
Інсулін дзейнічае на ныркі, стымулюючы затрымку натрыю і вадкасьці, выдзяленьне альдастэрону. Чым вышэйшая глікемічная нагрузка і чым меншая ваша адчувальнасьць да інсуліну, тым больш вы азызлыя і горай выглядаеце.

\subsection{Стрэс.}
Стрэс узмацняе цягу да салёнага і сам правакуе затрымку вадкасьці і натрыю. Пры стрэсе пазьбягайце салёнага і салодкага, сьпіце дастатковую колькасьць часу. Рэляксацыя палепшыць ваш стан і зьнізіць азызласьць. Пры хранічным стрэсе і паслабленьні функцыі наднырачнікаў можа ўзьнікаць дэфіцыт натрыю, бо арганізму яго складаней утрымліваць.

\subsection{Стомленасьць.}
Дэфіцыт калію і магнію часта праяўляецца ў выглядзе пастаяннай стомленасьці. Папаўненьне іх узроўняў дапамагае палепшыць тонус цягліцаў і трываласьць.

Правіла 24. Індывідуальныя асаблівасьці

У кнізе выкладзены найважнейшыя ўнівэрсальныя парады наконт рэжыму харчаваньня і выбару прадуктаў. Але пры гэтым трэба разумець, што канкрэтная фармулёўка харчовага правіла (колькі грамаў вугляводаў мне зьесьці на дзень, колькі гадзінаў і хвілінаў рабіць харчовае вакно і да т.п.) залежыць ад вашых індывідуальных асаблівасьцяў. У цэлым, чым больш вы пра сябе ведаеце, тым дакладней вы можаце фармуляваць сабе мэты. Таму мне хочацца, каб вы ня проста чакалі парады, колькі грамаў тлушчу есьці на содні, а занялі актыўную пазыцыю – дасьледчыка і назіральніка.

Зважайце на сваё здароўе, і ў гэтай сферы жыцьця адбудуцца станоўчыя зьмены. Навучыцеся назіраць за сабой, дасьледаваць, тэставаць і вывучаць сябе, ня толькі суб'ектыўна, але і з дапамогай тэстаў. Так вы зможаце колькасна вымераць уплыў таго ці іншага стылю харчаваньня ды іншых зьменаў.

На франтоне старажытнагрэцкага храма дэльфійскага аракула было высечана: «Спазнай самога сябе». Сапраўды, самавывучэньне, самадасьледаваньне зьяўляюцца важнымі практыкамі ў фармаваньні здароўя і неад'емнай умовай стварэньня насамрэч індывідуальнага падыходу. Калі вы зьвяртаеце ўвагу на пэўную сфэру жыцьця, у ёй непазьбежна адбываюцца станоўчыя зьмены. Навучыцеся назіраць за сабой, дасьледаваць, тэставаць і вывучаць сябе, ня толькі суб'ектыўна, але і з дапамогай тэстаў. Так вы зможаце напраўду заўважыць уплыў таго ці іншага стылю харчаваньня ды іншых зьменаў. Існуюць розныя тэсты, якія могуць выявіць стан вашага вугляводнага і тлушчавага абмену, асаблівасьці засваеньня розных рэчываў, стан мікрафлёры і шматлікае іншае.

Як зьявілася праблема?

Усе людзі вельмі падобныя між сабой, з аднаго боку, але маюць шэраг адрозненьняў, якія ўплываюць на іх рэакцыю на пэўны рэжым харчаваньня і прадукты. Гэтая рэакцыя залежыць ад генэтыкі, эпігенэтыкі, стану мікрафлёры, актыўнасьці імуннай сыстэмы, рэжыму фізычнай актыўнасьці і шмат чаго іншага. Таму для розных людзей розныя дыетычныя парады могуць мець розную ступень эфэктыўнасьці.
Многія з генэтычных асаблівасьцяў сфармаваліся пад уплывам чыньнікаў асяроддзя, у якім жылі нашы продкі, і ўяўляюць сабой адаптацыі. Напрыклад, у большасьці эўрапэйцаў сустракаюцца такія варыянты генаў FADS1 і FADS2, якія ня могуць сынтэзаваць жывёльныя формы амэга-3 тлустых кіслотаў з расьлінных, а ў Азіі «вэгетарыянскіх» формаў такіх генаў больш.
Зразумела, вялікае значэньне мае і актуальны стан здароўя чалавека, дастатковасьць усіх вітамінаў і мінэралаў. Як я ўжо згадваў, іх аптымальны ўзровень паляпшае мэтабалізм, розныя нутрыенты здольныя палепшыць сыстэму дэтаксыкацыі арганізму і тым самым паўплываць на абмен гармонаў унутры арганізму. Прапорцыі розных макранутриентов залежаць ад адчувальнасьці чалавека да інсуліну і да лептыну, узроўню яго фізычнай актыўнасьці. Чым ён вышэйшы, тым, напрыклад, больш бясьпечным будзе ўжываньне вялікіх колькасьцяў вугляводаў. На ступень засваеньня мінэралаў, прыкладам, моцна ўплывае кіслотнасьць страўніка: калі яна паніжаная, тое іх засваеньне будзе слабейшым.

Як гэта ўплывае на здароўе?

Веданьне сваіх індывідуальных асаблівасьцяў дапаможа найболей дакладна падабраць дыетычныя рэкамэндацыі. Напрыклад, рэкамэндацыя есьці больш аліўкавага алею лепей спрацуе для носьбітаў нуклеатыдаў G у rs1801282 (знаходзіцца ў гене PPARG). Такія людзі найбольш атрымаюць карысьці ад міжземнаморскай дыеты. Значнасьць генэтычных асаблівасьцяў таксама вар'іруецца ад умеранай схільнасьці да выяўленай непераноснасьці. Копія А ў БНіП rs4988235 гену LCT дае магчымасьць засвойваць ляктозу. Калі няма хоць адной копіі, то разьвіваецца непераноснасьць. Ёсьць цэлы шэраг генэтычна прыроджаных непераноснасьцяў (фруктозы, трэгалозы, глютэну і г. д.).

Многія з генэтычных асаблівасьцяў сфармаваліся пад уплывам чыньнікаў асяроддзя, у якім жылі нашы продкі, і ўяўляюць сабой адаптацыі. Напрыклад, у большасьці эўрапэйцаў сустракаюцца такія варыянты генаў FADS1 і FADS2, якія ня могуць сынтэзаваць жывёльныя формы амэга-3 тлустых кіслотаў з расьлінных, а ў Азіі «вэгетарыянскіх» формаў такіх генаў больш.

З дапамогай генэтычнага тэсту можна выявіць парушэньне мэтабалізму шэрагу вітамінаў, асабліва важныя парушэньні ў абмене фалатаў. Тэст дапаможа ацаніць дакладны ўплыў кафэіну, бо ёсьць такія асаблівасьці, калі нават невялікая порцыя кавы заўважна ўзмацняе трывогу і турботу. Што тычыцца насычаных тлушчаў, тут можна згадаць мутацыю ў гене FABP2, у гэтым выпадку носьбітам А-генатыпу лепш абмяжоўваць колькасьць насычаных тлушчаў, зьядаючы іх ня больш за 50 грамаў на содні. Што да вітаміну D, то і там ёсьць індывідуальныя асаблівасьці яго рэцэптараў. Пэўныя палімарфізмы гену рэцэптару вітаміну D (Bsm I, Fok I, Taq I і інш.) уплываюць на разьвіццё шматлікіх захворваньняў. Таму тэст дапаможа вам вызначыць аптымальны менавіта для вас узровень вітаміну D. Прадукты харчаваньня і гены ўзаемадзейнічаюць вельмі складана, уплываючы адзін на аднаго. Многія спалучэньні ў прадуктах могуць узмацняць або аслабляць актыўнасьць генаў, а гены, у сваю чаргу, уплываюць на рэакцыю арганізму пры ўжываньні прадуктаў.
Вядома, генэтыка ўплывае і на асаблівасьці харчовых паводзінаў. Для розных людзей працуюць розныя падыходы да кантролю псыхагеннага пераяданьня. Для прыкладу разгледзім тры гены, генэтычныя варыяцыі ў якіх павялічваюць рызыку кампульсіўнага пераяданьня. Гэта гены GAD2, Tag1A1 і FTO.

GAD2.
Гэты ген стымулюе пераяданьне, ён зьвязаны з кампульсіўнымі парушэньнямі харчовых паводзінаў. Ген кадуе фэрмэнт глутаматдэкарбаксілазу, падвышаная актыўнасьць якой вядзе да 6-разовага ўзмацненьня выпрацоўкі ГАМК (гама-амінамасьлянай кіслаты) у мозгу. Выдзяленьне лішку ГАМК у гіпаталамусе вядзе да павышэньня ўзроўню нэўрапэптыду Y ў аркуатным ядры. А гэта прыводзіць да моцнага голаду. Што рабіць? Заблякаваць залішнюю актыўнасьць нэўрапэптыду Y можна праз страўнікава-кішачны падыход. Там ёсьць адмысловыя L-клеткі, якія выпрацоўваюць пэптыд YY, што прыгнятае апэтыт. Галоўнымі стымулятарамі пэптыду YY зьяўляюцца тлушчы і жоўцевыя кіслоты. А яшчэ – харчовыя валокны, а таксама аэробныя фізычныя практыкаваньні.

Taq1A1.
Гэты ген зьвязаны з больш нізкай шчыльнасьцю дафамінавых D2 рэцэптараў у галаўным мозгу. Яго носьбіты атрымліваюць прыкметна менш задавальненьня ад жыцьця і ад ежы. Людзям зь нізкай шчыльнасьцю дафамінавых рэцэптараў для атрыманьня задавальненьня трэба зьесьці нашмат больш, што яны і робяць. Як толькі жыцьцё такіх людзей становіцца бязрадасным, яны пачынаюць вельмі шмат есьці, пераядаць для кампэнсацыі ўзроўню дафаміну. Што рабіць? Трэба больш радасьці, а ня ежы. Ёсьць і некалярыйныя спосабы атрыманьня задавальненьня ад ежы, і ня толькі ад яе. Гэта і прыгожы абрус, прыборы, сэрвіроўка, падача, размовы, час ежы, смакаваньне і ўсьвядомленая ежа. Таксама важна дадаць больш задавальненьня зь іншых сфэр жыцьця. Гэта падыме ўзровень дафаміну, і пераяданьне аслабне.

FTO.
Гэты ген служыць актыватарам выдзяленьня гармону голаду грэліну. У носьбітаў яго «тлустых» вэрсыяў узровень грэліну ня падае пасьля ежы, як павінна быць, а застаецца высокім. Таму такія людзі пераядаюць і аддаюць перавагу высокатлушчавыя прадукты. Узровень грэліну павялічваецца пры стрэсе, таму ў адказ на яго носьбіты зьядаюць шмат лішняга. Што рабіць? Грэлін – гэта ня так кепска, як здаецца, ён, напрыклад, зьніжае рызыку дэпрэсіі і стымулюе нэўрагенэз. Для носьбітаў AA вэрсыяў FTO важна прытрымлівацца рэжыму харчаваньня, не перакусваць, не прапускаць прыёмы ежы і не рабіць харчовых разгрузак на тле павышанага стрэсу. Кантроль стрэсу і рэжыму харчаваньня дазволіць эфэктыўна даць рады з кампульсіўным пераяданьнем у гэтым выпадку.

Генэтыка смаку.
У розных людзей генэтычна розны парог адрозьніваньня смакаў, таму сапраўды «дзеду міла, а ўнуку гніла». Памятайце, што ваша харчаваньне павінна прыносіць задавальненьне, таму з шырокага спэктру прадуктаў важна выбіраць, кіруючыся сваім смакам!

Мікрафлёра.
Асаблівасьці засваеньня прадуктаў і індывідуальная рэакцыя людзей залежаць ад стану мікрафлёры іх кішачніка. Таму, напрыклад, уздым узроўню глюкозы ў крыві ад аднаго і таго ж прадукту будзе ва ўсіх розным. Вывучэньне мікрафлёры кішачніка дапаможа ацаніць яе разнастайнасьць і выявіць, якія віды бактэрыяў адсутнічаюць, а таксама вызначыць аптымальныя нутрыенты для падкорму бактэрыяў таго ці іншага віду.

Асноўныя прынцыпы

Правіла 80/20.
Назіраньне за сабой, выкарыстаньне розных дыягнастычных тэстаў дапаможа вам выявіць індывідуальныя рэакцыі і асаблівасьці страваваньня ды ўлічваць іх для складаньня свайго пэрсанальнага рацыёну. Пры рабоце над сваім харчаваньнем прыслухоўвайцеся ня толькі да навуковых парадаў, але і да сваёй індывідуальнай рэакцыі. Бо гэта ўсярэдненыя парады, а вам трэба тое, што пасуе асабіста вам. Далёка ня ўсе парады будуць працаваць ідэальна, вам сярод усіх магчымых харчовых інструмэнтаў варта адшукаць тыя 20%, якія дадуць вам 80% выніку. Напрыклад, падбярыце, што мацней за ўсё вас насычае. Для гэтага адзначайце ў харчовым дзёньніку, колькі гадзінаў трымалася сытасьць пасьля тых ці іншых прадуктаў і наколькі ў балах яна была выяўленая. Гэтак жа можна вывучыць і свой голад: ацэньвайце, калі ён узьнікае і якой інтэнсіўнасьці.

Харчовы дзёньнік.
Харчовы дзёньнік – гэта запіс рэжыму харчаваньня і прадуктаў, пры гэтым варта ўлічваць таксама іншыя фактары ладу жыцьця, каб усталяваць паміж імі дакладныя прычынна-выніковыя сувязі. Улічвайце стрэс, сон, фізычную актыўнасьць, дзень цыклю і іншыя фактары. Харчовы дзёньнік можна весьці ў розных формах: фатаграфуючы стравы і затым складаючы справаздачу і ведучы дзёньнік за тыдзень з адзнакамі памылак, дзе фіксаваць свае адступленьні ад рацыёну і аналізаваць іх прычыны. Вы можаце выявіць асноўныя прычыны зрываў і пераяданьня, вельмі часта яны знаходзяцца нават ня ў сфэры ежы, а зьвязаныя з дэфіцытам сну, недахопам фізычнай актыўнасьці ды іншымі прычынамі.

Сярод усіх магчымых харчовых інструмэнтаў варта адшукаць тыя 20%, якія дадуць вам 80% выніку. Напрыклад, падбярыце, што мацней за ўсё вас насычае. Для гэтага адзначайце ў харчовым дзёньніку, колькі гадзінаў трымалася сытасьць пасьля тых ці іншых прадуктаў і наколькі ў балах яна была выяўленая. Гэтак жа можна вывучыць і свой голад: ацэньвайце, калі ён узьнікае і якой інтэнсіўнасьці.

Элімінацыйная дыета. 
Вось спосаб сапраўды выявіць прадукты, якія выклікаюць непераноснасьць. Для гэтага спачатку вы выключаеце большасьць падазроных прадуктаў на працягу тыдня, а затым кожныя 2-3 дні дадаяце па адным падазроным прадукце, пры гэтым ведучы дзёньнік харчаваньня і адсочваючы сымптомы.

Глікемічны кантроль.
Вы можаце з дапамогай глюкомэтра і дыягнастычных палосак адсочваць уздым глюкозы праз гадзіну і дзьве гадзіны пасьля яды і абраць стравы ці прадукты, якія даюць мінімальны ўздым канкрэтна ў вашым выпадку. Ёсьць і сыстэмы сталага маніторынгу глюкозы ў крыві.

Генэтыка.
Цяпер ёсьць вялікая колькасьць розных кампаній, якія робяць генэтычныя тэсты, кошт на іх увесь час зьніжаецца. Зьвярніце ўвагу, што далёка ня ўсе ДНК-парады маюць валіднасьць. На шматлікія паказьнікі ўплываюць адначасова некалькі генаў, а прадказаць іх сумарны ўплыў сапраўды пакуль яшчэ цяжка. Таму абавязкова правярайце валіднасьць кожнай рэкамэндацыі і стаўцеся да яе як да парады, якую можна праверыць, а не як да ісьціны ў апошняй інстанцыі.

Мікрафлёра.
Шмат кампаніяў робіць 16S рРНК – мэтагеномнае сэквэнаваньне мікрабіёмы, гэта найлепшы тэст. Ён дазваляе сапраўды вызначыць склад і суадносіны розных відаў бактэрыяў, ацаніць разнастайнасьць мікрабіёмы, убачыць, якіх штамаў вам бракуе, што трэба зьмяніць у рацыёне для паляпшэньня стану мікрафлёры. Робячы гэты тэст паўторна, можна напраўду ацаніць, як вашыя харчовыя зьмены паўплывалі на мікрабіем.

Харчовая непераноснасьць. 
У розных людзей шэраг прадуктаў выклікае паталягічныя рэакцыі, якія зьнікаюць пры выключэньні пэўных прадуктаў. Праявы бываюць самыя розныя: ад цяжару ў жываце, высыпаньняў на скуры, сьлёзацёку да ацёку Квінке, які пагражае жыцьцю. У цэлым да 20% людзей лічыць, што ў іх ёсьць алергія на пэўныя прадукты. Але ў рэальнасьці гэтая лічба нашмат меншая. У структуры харчовай непераноснасьці вылучаюць розныя станы, якія патрабуюць розных падыходаў.

Асобна мы можам гаварыць пра сапраўдную харчовую алергію, фэрмэнтапатыях, псэўдаалергіях (выдзяленьне гістаміну неімунным мэханізмам), парушэньні засваеньня асобных прадуктаў пры пэўных захворваньнях страўнікава-кішачнага тракту (прычына непераноснасьці – хвароба, яе лячэньне ліквідуе непераноснасьць), вылучаюць таксама і псыхагенныя рэакцыі на ежу, калі людзі самі або пад уздзеяньнем зьнешняга аўтарытэту могуць намаўляць сябе, што ім блага ад пэўных прадуктаў.

Сапраўдная алергія.
Сапраўдная алергія выклікаецца нават мінімальнай колькасьцю алергену, ёсьць станоўчы скурны тэст, як правіла падвышаны ўзровень IgE ў крыві. Варта адзначыць, што існуе і ўтоеная алергія, калі імунаглабуліны IgE ў крыві да алергену павышаныя, але рэакцыі пры гэтым няма. Зьвярніце ўвагу, што папулярныя ў нашы дні тэсты на вызначэньне IgG для фармаваньня сьпісу забароненых прадуктаў ненавуковыя і прыводзяць адно да звужэньня разнастайнасьці рацыёну. IgG утвараецца пры спажываньні ежы і не зьвязаны зь непераноснасьцю. Асноўныя прадукты – прычыны алергіі (вялікая васьмёрка): каровіна малако, яйкі (часьцей за ўсё курыныя), рыба (бывае асобна на прэснаводную і марскую), пшаніца, арахіс, гарэхі, соя, ракападобныя. Многія з гэтых прадуктаў уваходзяць у склад рознай гатовай ежы (соевы парашок, сухое малако і да т. п.). У выпадку сапраўднай алергіі неабходнае поўнае выключэньне прадукту з рацыёну.

Псэўдаалергія (фальшывая алергія).
Пры фальшывай алергіі шэраг рэчываў выклікае выкід гістаміну (і рэакцыю, падобную да алергічнай), але без наўпроставага ўдзелу імунных клетак. Пры фальшывай алергіі колькасьць зьедзенага прадукту павінна быць вялікая, пры гэтым ёсьць прамая залежнасьць паміж выяўленасьцю сымптомаў і колькасьцю зьедзенага, скурныя тэсты пры гэтым адмоўныя, а ўзровень IgE ў крыві не падвышаны. Пры фальшывай алергіі дзейнічаюць так званыя рэчывы – лібэратары гістаміну. Часьцей за ўсё гэта рыба, яйкі, чакаляда, арэхі. Часам рэакцыі няма на сьвежыя прадукты, а на тыя, што доўга захоўваліся, – ёсьць. Гэта зьвязана з павелічэньнем ўзроўню гістаміну пры захоўваньні. Часта прычынай псэўдаалергіі зьяўляюцца ня самі прадукты, а некаторыя харчовыя дабаўкі. Пры фальшывай алергіі эфэктыўная нізкагістамінавая дыета.

Фэрмэнтапатыя.
У кожнага чалавека розная актыўнасьць фэрмэнтаў, якія расшчапляюць ежу, так званая «біяхімічная індывідуальнасьць». Бывае прыроджаны недахоп фэрмэнтаў, тады чалавек ня можа засвойваць ежу. Бывае непераноснасьць ляктозы (тады трэба выключыць малочныя цэльныя прадукты), трэгалозы (выключыць грыбы), глютэну (выключыць збажыну і да т. п.). Часта фэрмэнтапатыі разьвіваюцца на тле захворваньняў страўнікава-кішачнага тракту.

Асноўныя прадукты – прычыны алергіі (вялікая васьмёрка): каровіна малако, яйкі (часьцей за ўсё курыныя), рыба (бывае асобна на прэснаводную і марскую), пшаніца, арахіс, гарэхі, соя, ракападобныя.

Хваробы страўнікава-кішачнага тракту.
Пры шматлікіх захворваньнях патрабуюцца пэўныя дыеты, бо парушаецца звычайнае засваеньне прадуктаў. Аднак пры лячэньні непераноснасьць зьнікае. Часьцей за ўсё прычынамі бываюць зьніжэньне кіслотнасьці страўнікавага соку, аслабленая функцыя падстраўніцы, павышэньне пранікальнасьці кішачніка (дзіравы кішачнік), парушэньні мікрафлёры, сындром раздражнёнага кішачніка.

Як трымацца правіла? Ідэі і парады

Рызыка алергічных і аўтаімунных захворваньняў.
Зьнізіць рызыку алергічных і аўтаімунных захворваньняў, асабліва ў дзяцей, можна ўлічваючы чатыры важныя чыньнікі: дастатковы ўзровень вітаміну D і знаходжаньне на сонцы, дастатковы ўзровень мікробнай нагрузкі (кантакт з прыродай, бываць на ферме, кантакт зь іншымі дзецьмі і да т.п.), нізкая колькасьць солі ў дыеце, выкарыстаньне ашчаднага харчаваньня нават для дзяцей, якія яшчэ растуць, бо пастаянная стымуляцыя mTORС павялічвае рызыку алергічных захворваньняў.

Структура цела.
Многія людзі пры зьменах харчаваньня крытэрам эфэктыўнасьці лічаць вагу цела. Але адна і тая ж вага можа выглядаць зусім па-рознаму, таму правільней ацэньваць эфэктыўнасьць сваіх дзеяньняў на аснове вымярэньня структуры цела. Структура цела – гэта суадносіны і разьмеркаваньне тлушчавай і цяглічнай тканкі. Так, падскурны тлушч ня вельмі небясьпечны, а самую сур'ёзную пагрозу ўяўляюць менавіта ўнутраны тлушч (на такіх органах, як печань, сэрца, кішачнік, падстраўніца) і страта цяглічнай масы (саркапэнія). Унутраны (эктапічны) тлушч выдзяляе адмысловыя рэчывы, якія правакуюць запаленьне, парушаюць гарманальны балянс. Аб'ектыўнае вымярэньне вісцэральнага тлушчу – гэта вельмі важна. Бо ні ваша вага, ні адсотак тлушчу ня важныя так, як колькасьць вісцэральнага тлушчу. Чым яго больш, тым вышэйшая рызыка захворваньняў. Чалавек можа быць худым і пры гэтым мець высокі ўзровень унутранага тлушчу і, адпаведна, высокую рызыку захворваньняў.

Структура цела – гэта суадносіны і разьмеркаваньне тлушчавай і цяглічнай тканкі. Падскурны тлушч ня вельмі небясьпечны, а самую сур'ёзную пагрозу ўяўляюць менавіта ўнутраны тлушч (на такіх органах, як печань, сэрца, кішачнік, падстраўніца) і страта цяглічнай масы (саркапэнія). Унутраны (эктапічны) тлушч выдзяляе адмысловыя рэчывы, якія правакуюць запаленьне, парушаюць гарманальны балянс. Чалавек можа быць худым і пры гэтым мець высокі ўзровень унутранага тлушчу і, адпаведна, высокую рызыку захворваньняў.

Антрапамэтрыя.
Гэта замеры стужкай, якія паказваюць асаблівасьці разьмеркаваньня тлушчу ў целе і рызыку захворваньняў. Часьцей за ўсё выкарыстоўваюцца наступныя паказчыкі: абхоп таліі ў жанчынаў складае да 75 (80) сантымэтраў, ад 80 да 88 сантыметраў – перавышэньне нармальнай вагі, звыш 88 – атлусьценьне, у мужчынаў нармальныя парамэтры складаюць да 94 сантыметраў. Суадносіны талія-сьцёгны ў норме менш за 0,85 для жанчынаў і менш за 1,0 для мужчынаў (аптымальна 0,7 (0,65-0, 78) для жанчынаў і ня больш за 0,9 для мужчынаў), акружнасьць шыі ў самым вузкім месцы ў жанчынаў ня больш за 34,5 см (больш строгая норма – 32 см), у мужчынаў акружнасьць шыі ня больш за 38,8 см (больш строгая норма – 35,5 гл). Суадносіны талія-сьцягно – менш за 1,5 для жанчынаў і менш за 1,7 для мужчынаў, ABSI – гэта комплексны індэкс формы цела, які ўлічвае суадносіны паміж аб'ёмам таліі, ростам і вагой ды разлічвае індывідуальную рызыку.

Цяглічная маса.
З узростам зьніжэньне цяглічнай масы толькі ўзмацняецца. Захаваньне дастатковай колькасьці цягліцаў зьяўляецца ўмовай захаваньня мэтабалічнага здароўя. Памятайце, што зьніжэньне сілы – гэта страта цягліцаў. Страту цягліцаў цяжка заўважыць, бо яны замяшчаюцца тлушчам, і аб'ём канцавіны можа заставацца ранейшым. Важна захоўваць высокі ўзровень фізычнай актыўнасьці ў любым узросьце.

Вымярэньне вісцэральнага тлушчу.
Самыя надзейныя вынікі дае DEXA-сканаваньне цела і тамаграфія. Але можна ацаніць узровень вісцэральнага тлушчу і па УГД. Спачатку робім дасьледаваньне печані: лінейныя памеры, прыкметы тлушчавага гепатозу, стан жоўцевага пухіра. Затым вымяраем таўшчыню эпікардыяльнага тлушчу (рызыкі растуць пры лічбе больш за 5 мм), колькасьць якога карэлюе з узроўнем вісцэральнага тлушчу, і таўшчыню пэрыкардыяльнага тлушчу (сардэчная рызыка). Пасьля чаго вымяраем адлегласьць паміж белай лініяй жывата і пярэдняй сьценкай аорты (больш за 100 мм – вісцэральнае атлусьценьне). Дадаткова можна разлічыць індэкс тлушчу брушной сьценкі (ІТБС) – гэта стасунак максымальнай таўшчыні перадбрушнога тлушчу да мінімуму таўшчыні падскурнага тлушчу. Гэтыя паказьнікі проста вымяраць у дынаміцы (лепей на адным апараце ў аднаго спэцыяліста).

Зьнізіць рызыку алергічных і аўтаімунных захворваньняў можна ўлічваючы чатыры важныя чыньнікі: дастатковы ўзровень вітаміну D, мікробная нагрузка, зьніжэньне колькасьці солі ў дыеце, умеранае харчаваньне.

Наведваньне стаматоляга.
Здаровыя зубы і здаровая ротавая поласьць – гэта найважнейшыя ўмовы добрага страваваньня. Дбайна чысьціце зубы, палашчыце рот пасьля кожнага прыёму ежы, выкарыстоўвайце ірыгатары, зубныя ніткі, своечасова выдаляйце зубны камень. Больш жуйце для здароўя зубоў.

Мікрафлёра рота.
Мікрафлёра ротавай поласьці важная для нашага здароўя. Пазьбягайце лішку мучнога (глютэн прыляпляе часьціцы крухмалу да зубоў), бактэрыцыдных ополаскивателей для рота, выкарыстоўвайце аральныя прабіётыкі (S. Salivarius і інш.).

Індывідуальная праграма абсьледаваньня.
Таксама на ваш стан уплывае і здароўе. Рэгулярна правярайце яго, здавайце неабходныя, а лепей – пашыраныя дыягнастычныя панэлі. Прыклад падобнай лябараторнай панэлі: глюкоза, глікаваны гемаглабін, інсулін, індэкс інсулінарэзыстэнтнасьці, Алат, Асат, крэатынін, мачавая кіслата, гомацыстэін, ультраадчувальны тэст на С-рэактыўны бялок, гармоны шчытападобнай залозы (ТТГ, св. Т3), палавыя гармоны, ІФР-1, гармон росту, цынк, магній, жалеза, фэрытын, вітамін D, вітамін В12. Да іх можна дадаць інструмэнтальныя тэсты, напрыклад УГД сонных артэрыяў з вызначэньнем таўшчыні комплексу інтым-мэдыя. Пры неабходнасьці можна дапаўняць іншымі абсьледаваньнямі, у залежнасьці ад узросту і індывідуальнай рызыкі (сямейная гісторыя захворваньняў, генэтычныя рызыкі шэрагу захворваньняў і дэфіцытаў і інш.): анкамаркеры, каланаскапія, фібрагастрадуадэнаскапія і іншыя паказьнікі.


\end{document}
