
\chapter[Калярыйная і нутрыентная шчыльнасьць ежы][Шчыльнасьць ежы]{Калярыйная і нутрыентная шчыльнасьць ежы}

Правіла шчыльнасьці заключаецца ў~тым, каб выбіраць сабе ежу зь нізкай удзельнай калярыйнай шчыльнасьцю і з~высокай шчыльнасьцю нутрыентаў (клятчатка\index{клятчатка}, вітаміны, антыаксыданты\index{антыаксыданты}, макра- і мікраэлемэнты). Калярыйная шчыльнасьць (энэргетычная шчыльнасьць)~--- гэта колькасьць калёрыяў на грам прадукту, чым вышэйшая шчыльнасьць, тым меншы аб'ём прадукту і вышэйшая колькасьць калёрыяў у~ім. Так, напрыклад, гародніна ўтрымоўвае 30\,ккал на 100 грамаў прадукту, а~вось чакалядка~--- да 550\,ккал на 100 грамаў прадукту.

Можна ўявіць сабе сытуацыю і па-іншаму: так, 400\,ккал~--- гэта 100 грамаў чакаляднага батончыка, амаль 400 грамаў мяса і амаль паўтара кіляграмы брокалі. Менавіта таму сьцьвярджэньні аб тым, што калёрыі цалкам ідэнтычныя, няслушныя: бо 400\,ккал батончыка і 400\,ккал брокалі будуць па-рознаму ўплываць на пачуцьцё голаду і сытасьці, а~гэта шмат у~чым вызначае вашы харчовыя паводзіны. Чым вышэйшая калярыйная шчыльнасьць, тым вышэйшая ступень ачысткі прадукту і тым меншая яго нутрыентная шчыльнасьць. Ежце як кароль~--- толькі багатыя прадукты. «Бедныя» прадукты ўтрымліваюць шмат калёрыяў і мала карысных рэчываў. «Багатыя» прадукты ўтрымліваюць мала калёрыяў, але вельмі шмат біялягічна актыўных рэчываў.

\subsection{Як зьявілася праблема?}

З даўніх часоў у~прыродзе чалавек рэдка сустракаўся з~багацьцем ежы высокай шчыльнасьці, яна была эпізадычнай ці сэзоннай. Зь цягам часу калярыйная шчыльнасьць ежы пакрысе павялічваецца. Так, узровень тлушчу ў~дзікіх жывёл~--- 5--7\,\% і дасягае 30\,\% у~сельскагаспадарчых, тое ж тычыцца многіх расьлінных культураў. Зьяўленьне рафінаваных прадуктаў прывяло да таго, што ежа стала неймаверна калярыйнай, ачыстка прадуктаў прывяла да зьніжэньня іх біялягічнай вартасьці.

Павелічэньне калярыйнай шчыльнасьці часьцей за ўсё суправаджаецца зьніжэньнем нутрыентнай шчыльнасьці ежы (гэта значыць зьмесьціва ў~ёй розных некалярыйных, але важных для здароўя прадуктаў, уключаючы клятчатку\index{клятчатка}, вітаміны, антыаксыданты\index{антыаксыданты}, рэчывы, якія адказваюць за смак і пах, і многае іншае). Таму ежа з~высокай калярыйнай шчыльнасьцю, як правіла, бедная на біялягічна актыўныя спалучэньні. Акрамя гэтага, працяглае вырошчваньне ежы на адных і тых жа землях зьнясільвае глебу, зьніжае ўтрыманьне ў~ёй шэрагу карысных кампанэнтаў; працяглае захоўваньне і транспарціроўка таксама ўзмацняюць гэты працэс.

\tipbox{У старажытнасьці некаторымі хваробамі ганарыліся, лічылі іх прыкметай вытанчанасьці. Так, падагру называлі хваробай каралёў, бо толькі багатым людзям была даступная такая колькасьць бялковай ежы і алькаголю на працягу доўгага часу ў~спалучэньні з~маларухомым ладам жыцьця.}

Многія «хваробы цывілізацыі», распаўсюджаныя сёньня (алергіі, атлусьценьне, дыябэт\index{дыябэт}, дэпрэсія), у~старажытнасьці былі толькі ў~некаторых людзей з~высокім сацыяльным статусам і даходам, а~дакладней~--- зь няправільным харчаваньнем, гіпадынаміяй\index{гіпадынамія} і нястрымнасьцю. Больш за тое, некаторымі хваробамі нават ганарыліся, лічачы іх прыкметай вытанчанасьці. Так, падагру называлі хваробай каралёў, паколькі толькі багатым людзям была даступная такая колькасьць бялковай ежы і алькаголю на працягу доўгага часу ў~спалучэньні з~маларухомым ладам жыцьця. Гэта ж датычыцца і цукровага дыябэту\index{дыябэт} другога тыпу, які сустракаўся, як правіла, толькі ў~багатых. Алергія першапачаткова таксама была прыкметай заможнасьці. Сянны катар быў «моднай» хваробай эліты. Лекары адразу заўважылі, што хвароба сустракаецца ў~багатых, а~ў сялян яе не назіралася. Чаму? Багатыя больш пераядалі, елі больш тлустага, мучнога, салодкага, менш рухаліся, таму і часьцей хварэлі. Цяпер сытуацыя поўнасьцю перакулілася. Калярыйная бедная ежа (мучное, тлустае, цукар) стала таннай, а~вось ``багатыя'' прадукты (морапрадукты, зеляніна, якаснае мяса, ягады і гародніна) сталі даражэйшымі.

\subsection{Як гэта ўплывае на здароўе?}

\paragraph{Пераяданьне.}
Дасьледаваньні паказваюць, што чым вышэйшая калярыйная шчыльнасьць ежы, тым больш ежы чалавек зьядае. Зьніжэньне шчыльнасьці ежы прыводзіць да таго, што чалавек аўтаматычна пачынае есьці менш на 300--400 ккал. Навукоўцы давялі, што тыя, хто есьць больш ежы зь нізкай шчыльнасьцю, маюць меншую вагу і абхоп таліі, павелічэньне калярыйнай шчыльнасьці можа да 56\,\% павялічыць спажываньне калёрыяў, у~экспэрымэнтах людзі зьядалі ў~сярэднім на 425 ккал больш пры пераходзе на шчыльную сытную ежу.

\paragraph{Кантроль апэтыту.}
Чым вышэйшая ўдзельная калярыйная шчыльнасьць, тым менш месца займае гэтая ежа ў~страўніку, такім чынам, менш стымулюе мэханарэцэптары страўніка, якія ўспрымаюць ціск на яго сьценкі. Акрамя гэтага, сама шчыльнасьць ежы стымулюе апэтыт. З~эвалюцыйнага пункту гледжаньня прадукты з~высокай калярыйнай шчыльнасьцю былі рэдкія, таму іх важна было зьесьці і пераўтварыць у~тлушчавыя запасы, якія могуць дапамагчы ў~будучыні. Расьлінныя прадукты зь нізкай удзельнай калярыйнасьцю былі шырока распаўсюджаныя, таму яны не выклікаюць такога павышэньня апэтыту. Многія сумуюць па выпечцы, але мала хто сумуе па брокалі.

\paragraph{Забясьпечанасьць вітамінамі і мінэраламі.}
Многія калярыйныя прадукты бедныя вітамінамі і мікраэлемэнтамі. Сэлекцыя, якая прыводзіць да павелічэньня ўтрыманьня крухмалу і тлушчу, паскарэньня росту, адначасова вядзе і да зьніжэньня ўтрыманьня карысных рэчываў. Чым больш чалавек зьядае шчыльнай ежы, тым менш у~яго рацыёне багатых вітамінамі і мінэраламі прадуктаў, што можа прыводзіць да шэрагу дэфіцытных станаў.

\tipbox{Калярыйныя прадукты бедныя вітамінамі і мікраэлемэнтамі. Сэлекцыя, якая прыводзіць да павелічэньня ўтрыманьня крухмалу і тлушчу, паскарэньня росту, адначасова вядзе і да зьніжэньня ўтрыманьня карысных рэчываў.}

\paragraph{Удзельная шчыльнасьць вугляводаў і кішачнік.}
Рэч у~тым, што навуковыя дасьледаваньні паказалі: менавіта ўдзельная шчыльнасьць вугляводаў, а~не глікемічны індэкс, уплывае на іх дзеяньне. Чым вышэйшая шчыльнасьць, тым мацнейшы яе ўплыў на мікрафлёру і тым мацнейшая запаленчая рэакцыя, незалежна ад агульнай колькасьці вугляводаў. Пасьля прыёму канцэнтраваных вугляводаў узьнікае «мэтабалічная эндатаксэмія» (павелічэньне ўзроўню таксынаў), якая трымаецца да 5 гадзінаў пасьля яды. Гэта выклікае хранічнае запаленьне, што прыводзіць да лептынарэзыстэнтнасьці\index{лептынарэзыстэнтнасьць} і парушэньня працы вагусу\index{вагус (блукальны нэрв)}, зьніжае насычэньне, гнетучы актыўнасьць такіх мэдыятараў насычэньня, як халецыстакінін\index{халецыстакінін}, пэптыд YY. Даведзена, што харчаваньне ``старажытнымі'' вугляводамі прыводзіць да зьмены мікрафлёры, зьніжэньня пранікальнасьці кішачніка і памяншэньня выяўнасьці запаленчых зьяваў. Аналягічная колькасьць ``сучасных'' вугляводаў не аказвае падобнага эфэкту. Чым больш «сучасных» вугляводаў, тым больш неспрыяльнай мікрафлёры як у~ротавай поласьці, так і ў~тонкім кішачніку, і тым мацнейшы запаленчы статут арганізму.

Чым вышэйшая ўдзельная калярыйная шчыльнасьць, тым ніжэйшая разнастайнасьць мікрафлёры. Так, узровень Bac\-te\-roidetes вышэйшы на дыеце ў~2\,Ккал/грамаў, а~ўзровень Firmicutes вышэйшы на дыеце ў~4\,Ккал/грамаў. Узровень Parabacteroides вышэйшы на дыеце ў~2\,Ккал/грамаў, а~ўзровень Barnesiella вышэйшы на дыеце ў~4\,Ккал/грамаў.

\subsection{Асноўныя прынцыпы}

\paragraph{Павялічце долю прадуктаў зь нізкай калярыйнай шчыльнасьцю,} то-бок «багатых» прадуктаў, дзе мала калёрыяў. Вы можаце есьці без абмежаваньняў зеляніну, нізкакрухмалістую гародніну, ягады, багавіньне і г.~д. Форма прадукту таксама мае значэньне: так, калярыйная шчыльнасьць вінаграду ў~пяць разоў меншая, чым разынак. Усе вугляводныя прадукты мы можам умоўна падзяліць на «старажытныя клеткавыя», такія як банан, батат, бурак і морква, гарбуз, капуста і інш. І~«сучасныя пазаклеткавыя» вугляводы, куды адносяцца крупы і іх перапрацаваныя вытворныя. Важна ня проста памяншаць вугляводы, а~радыкальна памяншаць колькасьць «сучасных пазаклеткавых» (мучное любога паходжаньня, перапрацаваныя крупы) вугляводаў. А~вось «старажытныя клеткавыя» вугляводы можна спажываць без істотных асьцярогаў. Правіла калярыйнай шчыльнасьці датычыцца ня толькі расьлінных, але і жывёльных прадуктаў. Параўнайце тлустую марскую рыбу і тлустую сьвініну: тлусты селядзец 248 ккал на 100 грамаў, а~вось сьвініна~--- 491 ккал, розьніца вельмі істотная!

\paragraph{Паменшыце долю прадуктаў з~высокай калярыйнай шчыльнасьцю,} то-бок «бедных» прадуктаў, у~якіх шмат калёрыяў і мала нутрыентаў. Нават проста крупы, якія зьядаюцца асобна, ня будуць здаровым выбарам, абавязкова дадавайце ў~прыём ежы больш зеляніны і гародніны, каб калярыйныя прадукты займалі меншую частку талеркі. У~першую чаргу гэта мучныя вырабы, кандытарскія, якія зьмяшчаюць шмат цукру і тлушчу.

\tipbox{Усе вугляводныя прадукты мы можам умоўна падзяліць на «старажытныя клеткавыя», такія як банан, батат, бурак і морква, гарбуз, капуста і інш., і «сучасныя пазаклеткавыя» вугляводы, куды адносяцца крупы і іх перапрацаваныя вытворныя.}

\paragraph{Павялічце долю прадуктаў з~высокай нутрыентнай шчыльнасьцю,} дзе шмат вітамінаў, антыаксыдантаў\index{антыаксыданты}, мінэралаў і да т.~п. Такія прадукты часта называюць «супэрфудамі». Як мы з~вамі ведаем, рабіць дыету на некалькіх, нават вельмі карысных, прадуктах не зусім правільна, але долю такіх прадуктаў варта павялічыць. Для дакладнага вызначэньня дэфіцыту вітамінаў і мінэралаў вы можаце здаць біяхімічныя аналізы крыві, самыя распаўсюджаныя дэфіцыты коратка адлюстраваныя ніжэй. Пад супэрфудамі маюць на ўвазе прадукты, якія ядуцца ў~малых колькасьцях і валодаюць высокай удзельнай біялягічнай актыўнасьцю (канцэнтрацыя карыснага на грам прадукту). Пад азначэньне супэрфудаў трапляюць авакада, шпінат, марское багавіньне, гранат, чарніцы, буякі, брокалі, кале і ўсе крыжакветныя, міндаль, какосавы алей, дзікі ласось, аліўкавы алей, зялёная гарбата, какава-бабы, грэцкія гарэхі. Многія з~супэрфудаў уяўляюць зь сябе маркетынгавы ход: так, абляпіха нічым не саступае годжы, а~шматклеткавае багавіньне карысьнейшае і бясьпечнейшае за сьпіруліну (можа ўтрымоўваць шэраг таксынаў).

Адносна дэфіцыту мікранутрыентаў\index{мікранутрыенты} можна сустрэць розныя вэрсіі, пачынаючы ад той, што «ўся ежа сёньня ўжо ня тая» да «ў ежы ёсьць усё, што трэба». Ісьціна заключаецца ў~тым, што нават калі ў~ежы ўсё ёсьць, то бываюць індывідуальныя праблемы з~засваеньнем неабходных рэчываў. Акрамя гэтага, важна разумець, што ў~выпадку ўмеранага дэфіцыту арганізм накіроўвае дэфіцытныя рэчывы на рашэньне галоўных пытаньняў выжываньня, пры гэтым пакутуюць іншыя важныя функцыі. Так, напрыклад, пры ўтоенай жалезадэфіцытнай анэміі эрытрацыты будуць у~норме, але вось валасы і энэргічнасьць~--- не. Аналягічная гісторыя адбываецца і пры дэфіцыце вітаміна D, сэлену\index{сэлен}, цынку і шэрагу іншых элемэнтаў. Таму важнае харчаваньне багатай нутрыентамі ежай, але, акрамя гэтага, аптымальнае падтрыманьне патрэбных узроўняў розных мікранутрыентаў\index{мікранутрыенты}.

\paragraph{Прапорцыі і талерка.}
Калі ў~вас мэта схуднець, то зьмяняйце прапорцыі розных харчовых групаў, павялічваючы ў~рацыёне долю прадуктаў зь нізкай удзельнай калярыйнай шчыльнасьцю, а~калі набраць вагу~--- то павялічвайце з~высокай шчыльнасьцю.

У прынцыпе, мы можам есьці любыя прадукты, але важна памятаць, што чым вышэйшая яго ўдзельная калярыйная шчыльнасьць, тым меншую колькасьць гэтага прадукту нам варта зьесьці. Пра тое, як есьці менш і атрымліваць больш задавальненьня, я расказваю ў~разьдзеле «Смачная ежа». Дадаючы ў~свой рацыён больш такіх прадуктаў, мы зможам лягчэй спаталяць пачуцьцё голаду. У~сярэднім лічыцца, што агульная вага ежы для дасягненьня сытасьці павінна складаць 1,5--2 кіляграмы пры адэкватнай шчыльнасьці. Таму вам важна зьядаць вялікія, аб'ёмныя порцыі ежы зь нізкай шчыльнасьцю.

\tipbox{У катэгорыю супэрфудаў трапляюць авакада, шпінат, марское багавіньне, гранат, чарніцы, буякі, брокалі, кале і ўсе крыжакветныя, мігдал, какосавы алей, дзікі ласось, аліўкавы алей, зялёная гарбата, какава-бабы, грэцкія гарэхі. Многія з~супэрфудаў~--- толькі маркетынгавы ход: так, абляпіха нічым не саступае годжы, а~шматклеткавае багавіньне карысьнейшае і бясьпечнейшае за сьпіруліну (можа ўтрымліваць шэраг таксынаў).}

\paragraph{Пасьлядоўнасьць прыёму прадуктаў.}
Варта пачынаць сталаваньне з~прадуктаў нізкай удзельнай калярыйнасьці, зьядаючы іх больш (зялёная ліставая гародніна). Затым пераходзіце да прадуктаў сярэдняй калярыйнай шчыльнасьці (мяса, рыба, крухмалістая гародніна) і толькі ў~канцы можна (але не абавязкова) зьесьці дэсэрт (гарэхі, ягады і да т.~п.).

\subsection{Як трымацца правіла? Ідэі і парады}

\paragraph{Стоеная зеляніна і гародніна.}
Вы можаце прыкметна зьменшыць удзельную калярыйную шчыльнасьць страў, калі будзеце больш дадаваць у~іх здробненай зеляніны і гародніны, зьмешваючы з~асноўнай стравай і паступова павялічваючы прапорцыю. Гэта зьменшыць колькасьць калёрыяў і павялічыць аб'ём стравы, што добра для сытасьці.

\paragraph{Прадукты з~адмоўнай калярыйнасьцю.}
Існуе міт пра прадукты з~адмоўнай калярыйнасьцю, нібыта на засваеньне якіх арганізм выдаткуе больш энэргіі, чым яны даюць. У~цэлым на засваеньне (тэрмічны эфэкт ежы) выдаткуецца ня больш за 5--10\,\%.

\paragraph{Заправіць салату.}
Заправіць салату можна ня толькі калярыйнымі алеямі, але і выкарыстоўваць бальзамічны воцат, кефір, цытрынавы сок, здробненыя гарэхі і да т.~п.

\paragraph{Спартовае харчаваньне.}
Спартхарч, па сутнасьці, можа быць аднесены да «бедных» калёрыяў, бо там шмат чыстых рэчываў, напрыклад бялку, але мала іншых важных злучэньняў. Калі вы не элітны атлет з~ужываньнем неймавернай колькасьці энэргіі, вам спартовае харчаваньне ня трэба. Ежце цэльныя прадукты~--- так вы атрымаеце нашмат больш карысьці.

\paragraph{Багавіньне.}
Багавіньне мае высокія канцэнтрацыі мінэралаў, нізкую колькасьць калёрыяў, яно карыснае для мікрафлёры і зручнае тым, што яго можна купляць празапас сушанае і выкарыстоўваць па меры неабходнасьці. Асноўныя віды: лямінарыі (комбу, келп), вакамэ, ірляндзкі мох, фукус, хідзікі і многія іншыя.

\paragraph{Ягады.}
Ягады маюць высокую шчыльнасьць і багацьце біялягічна актыўных рэчываў. Магчымасьць замарозкі забясьпечвае іх даступнасьць у~любую паравіну года. Выбірайце ягады мясцовыя, сакавітыя, у~іх, як правіла, вышэйшая канцэнтрацыя антыаксыдантаў\index{антыаксыданты}: чарніцы, ажыны, шаўкоўніца, вішня, чарнаплодная рабіна, буякі і многія іншыя. Кававы кубак ягадаў~--- неабходны мінімум.

\paragraph{Грыбы.}
Розныя віды грыбоў прыкметна адрозьніваюцца па складзе, але ў~цэлым усе маюць высокую біялягічную вартасьць. Грыбы станоўча ўплываюць на мікрафлёру кішачніка, утрымліваюць шэраг мінэралаў, паляпшаюць ліпідны спэктар крыві, зьмяншаюць запаленьне. Многія грыбы (кардцыцэпс, рэйшы і інш.) валодаюць унікальнымі здольнасьцямі ўплываць на імунны адказ.

\paragraph{Гарэхі.}
Гарэхі, нягледзячы на сваю калярыйнасьць, вельмі станоўча ўплываюць на здароўе, зьмяншаюць рызыку сардэчна-сасудзістых захворваньняў, памяншаюць узровень запаленьня. Можна сьмела іх есьці на дэсэрт, але пры гэтым таксама памятайце пра ўмеранасьць. Большасьць гарэхаў маюць даведзенае карыснае дзеяньне: грэцкі гарэх, мігдал, фісташкі, кеш'ю, бразільскі гарэх. Калі гарэхі пашкоджаныя ці здробненыя, іх якасьць падае~--- купляйце арэхі ня дробненыя, а~з цэлай плёнкай ці шкарлупінай.

\paragraph{Спэцыі.}
Спэцыі ўтрымоўваюць нікчэмна малую колькасьць калёрыяў, але пры гэтым неймаверна шмат карысных спалучэньняў. Яны зьмяншаюць запаленьне, паляпшаюць насычэньне і адчувальнасьць да інсуліну, стымулююць спаленьне тлушчу і валодаюць яшчэ цэлым шэрагам карысных эфэктаў. Выбірайце розныя спэцыі і дадавайце іх у~ежу ў~дастатковых колькасьцях: чырвоны перац, размарын, куркума, лаўровы ліст, аніс, кардамон, шафран і многія іншыя. Таксама дадавайце ў~ежу і такія прадукты, як імберац, цыбуля, часнык, хрэн, гарчыца: яны ня толькі смачныя, але й~вельмі карысныя.

\paragraph{Зеляніна.}
Зеляніна даступная ня толькі ў~сьвежым выглядзе, што ідэальна, але можа актыўна ўжывацца і ў~сушаным і замарожаным выглядзе. Вы можаце на свой густ выкарыстоўваць шырокі набор: кроп, шпінат, базілік, кінза, пятрушка, лук, мангольд, кале і інш.

\paragraph{Мультывітаміны.}
Лягічным рашэньнем зьнізіць рызыку дэфіцыту вітамінаў зьяўляецца прыём дабавак. Аднак дасьледаваньні паказваюць, што мультывітаміны не аказваюць прыкметнага станоўчага ўзьдзеяньня, а~ў шматлікіх выпадках могуць толькі нашкодзіць. Таму важна выбіральна падыходзіць да прызначэньня розных дабавак, кантралюючы ўзровень дэфіцыту вітамінаў лябараторнымі аналізамі.

\tipbox{Спэцыі ўтрымоўваюць нікчэмна малую колькасьць калёрыяў, пры гэтым неймаверна шмат карысных спалучэньняў. Дадавайце ў~ежу чырвоны перац, размарын, куркуму, ляўровы ліст, аніс, кардамон, шафран.}

\paragraph{Найбольш частыя дэфіцыты.}
Найбольш частымі дэфіцытамі зьяўляюцца дэфіцыт амэга-3 тлустых кіслот (дадаткі або тлустая марская рыба два разы на тыдзень, кантроль па амэга-індэксе), дэфіцыт сэлену\index{сэлен} (лепш хелатные формы дабавак або морапрадукты, часнык, бразільскі гарэх), дэфіцыт вітаміну D (бываць часьцей апоўдні на сонцы, есьці больш рыбы, яек або дабаўкі ў~высокіх дозах пад лябараторным кантролем), дэфіцыт жалеза (мясныя прадукты або хелатные дабаўкі, кантроль па профілі жалеза ў~крыві: жалеза, ферытын\index{ферытын}, трансфэрын), дэфіцыт цынку (больш мяса і рыбы або дабаўкі), дэфіцыт магнію, дэфіцыт вітамінаў В9 і В12 (лепш прымаць у~выглядзе мэтыльных формаў, мэтылфалат і мэтылкабаламін, кантроль па ўзроўні В9, В12, гомацыстэіну), дэфіцыт ёду (цяпер сустракаецца ўсё радзей).

