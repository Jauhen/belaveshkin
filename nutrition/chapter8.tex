\chapter{Паўза перад ядой}

Паўза перад ядой – гэта прамежак часу паміж момантам, калі вы селі за стол зь ежай, і да моманту пачатку яды. Паўза перад ядой мае вялікае значэньне для наладкі на ежу, зьніжэньня ўзроўню стрэсу, павелічэньня апэтыту, стымуляцыі мазгавой фазы сакрэцыі інсуліну, паляпшэньня смаку і насычэньня, стымуляцыі выдзяленьня стрававальных фэрмэнтаў. Усе гэтыя чыньнікі ў сукупнасьці вельмі станоўча ўплываюць на харчовыя паводзіны і страваваньне. Я часта бачу, як людзі ядуць нецярпліва і похапкам накідваюцца на ежу, сам быў такі. Але гэта не здаровыя харчовыя паводзіны. Для таго каб складаныя працэсы працякалі правільна і з задавальненьнем, да іх трэба падрыхтавацца. Бо мы ведаем, што сакрэт добрага сэксу – у прэлюдыі. Гэтак жа сама і зь ежай: накідвацца прагна на ежу, толькі сеўшы за стол, – гэта шмат чаго сябе пазбаўляць. Зрабіце прэлюдыю перад ядой – вы атрымаеце нашмат больш задавальненьня і карысьці!

\section{Як зьявілася праблема?}

Праблема «накідваньня на ежу» зьяўляецца, калі гэтаму папярэднічае доўгі пэрыяд голаду, высокі ўзровень стрэсу. У спробе суняць пачуцьцё голаду і стрэсу людзі выкарыстоўваюць ежу, каб як мага хутчэй пазбавіцца ад непрыемных адчуваньняў, «набіць» страўнік, каб адчуць расслабленьне і спакой. Такія дзеяньні могуць прывесьці да пераяданьня і да горшага засваеньня ежы.

Час прыёму ежы зьмяншаецца, ежу ўсё часьцей мы купляем. Гэта прыводзіць да таго, што мы пачынаем есьці практычна бяз паўзы перад ядой. Але такая дробная на першы погляд дэталь істотна ўзьдзейнінчае на наш працэс страваваньня.

\section{Як гэта ўплывае на здароўе?}

\subsection{Мазгавая фаза страваваньня.}
Выгляд смачнай ежы, яе водар стымулююць выдзяленьне інсуліну ў невялікіх колькасьцях, што рыхтуе нашу сыстэму страваваньня да прыёму ежы. Такая падрыхтоўка дапамагае лепей кантраляваць апэтыт і меней пераядаць.

\subsection{Стрэс.}
Рэзкае пераключэньне з працы на яду непажаданае. Паўза перад ядой дапамагае настроіцца на прыняцьце ежы. Бо стрэсавая сымпацыйная сыстэма прыгнятае выдзяленьне стрававальных фэрмэнтаў і зьніжае маторыку кішачніка. А вось расслабленьне зьвязанае з актывацыяй парасымпацыйнай антыстрэсавай сыстэмы, якая стымулюе выдзяленьне сьліны ды іншых сакрэтаў, стымулюе маторыку кішачніка, паляпшае крывацёк у страўнікава-кішачным тракце.

\subsection{Аўтаматычнае пераяданьне і ўсьвядомленасьць.}
Паўза перад ядой зьніжае сьпех і аўтаматычнае рэагаваньне, дапамагае пакінуць нэгатыўныя эмоцыі не за сталом, атрымаць больш задавальненьня і зрабіць больш здаровы выбар стравы.

Выгляд смачнай ежы, яе пах стымулююць вылучэньне інсуліну ў невялікіх колькасьцях, што рыхтуе нашу сыстэму страваваньня да прыёму ежы. Такая падрыхтоўка дапамагае лепей кантраляваць апэтыт і меней пераядаць.

\subsection{Апэтыт.}
Як гаворыць народная мудрасьць, 2апэтыт прыходзіць падчас яды». Узровень грэліну ўплывае на задавальненьне ад ежы, на колькасьць зьедзенага. Чым вышэйшы апэтыт перад прыёмам ежы, тым лягчэй вам своечасова заўважыць сытасьць. Зьніжэньне пачуцьця голаду – гэта і ёсьць зьяўленьне сытасьці. Калі вы пачынаеце есьці без пачуцьця голаду, то можаце аўтаматычна пераесьці.

\subsection{Паляпшэньне смаку ежы і задавальненьня ад ежы.}
У адным дасьледаваньні навукоўцы паказалі, што паўза перад ядой, запоўненая фатаграфаваньнем ежы і публікацыяй фота, палепшыла смак і павысіла задавальненьне ў 90% удзельнікаў экспэрымэнту. Аўтары дасьледаваньня лічаць, што паўза перад ядой павялічвае асалоду. Таму карыстайцеся невялікім чаканьнем, каб узмацніць смак.

\subsection{Сытасьць.}
Калі вы ясьцё зьлёту, ігнаруеце тое, што вы ясьцё, нічога ня ведаеце і ня думаеце аб карысьці ежы і харчовай вартасьці, то такая ежа горай спаталяе голад. Так, у адным дасьледаваньні падыспытныя атрымлівалі малочны кактэйль аднолькавай калярыйнасьці, але з рознымі этыкеткамі (на адной указвалася істотна меншая калярыйнасьць, на другой – вялікая). Аказалася, што падзеньне гармону голаду пасьля кактэйлю залежала ад этыкеткі, а не ад рэальнай калярыйнасьці. Такім чынам, псыхалягічная ўстаноўка на сытасьць і дастатковасьць ежы рэальна ўплывае на ўзровень гармону голаду ў крыві чалавека.

\section{Асноўныя прынцыпы}

Зрабіце паўзу перад ядой ад 2 да 5 хвілінаў, разгледзьце ежу, адчуйце яе пах, падумайце аб тым, для чаго вам энэргія ежы, адкуль зьявілася гэтая ежа і што ў ёй карыснага і дзейснага.

\subsection{Час.}
Дасьледаваньні паказваюць, што ў першыя 30 сэкундаў пах ежы распальвае апэтыт і жаданьне зьесьці менавіта больш высокакалярыйных страваў. Але вось дзьве і больш хвіліны нюху ежы зьніжалі цягу да высокакалярыйнай ежы і дазвалялі рабіць больш здаровы харчовы выбар.

\subsection{Усьвядомленасьць.}
З пункту гледжаньня ўсьвядомленасьці самая эфэктыўная яе форма, якая зьніжае стрэс, – гэта падзяка. Мы можам падзякаваць тым людзям, хто лавіў, вырошчваў, захоўваў, перавозіў, гатаваў нашую сёньняшнюю ежу. Мы можам падзякаваць нашым абставінам, што можам спакойна есьці нашую ежу сёньня і з намі ўсё ў парадку.

\subsection{Вывучэньне ежы.}
Разгледзьце ежу ўважліва. Сфакусуйцеся на той энэргіі, якую яна вам дорыць. Як вы ёй карыстаецеся? Назавіце ўсе візуальныя інгрэдыенты і водары. Што карыснага ёсьць у кожным зь іх? Як гэтая ежа можа дапамагчы вам у дасягненьні вашых мэтаў? Такі падыход дапаможа вам зрабіць найболей здаровы выбар у ежы.

\subsection{Камунікацыя.}
Пагаварыце пра ежу. Падзякуйце таму, хто яе прыгатаваў. Вы можаце сказаць і тост (і без алькаголю): «Няхай гэтая ежа дасьць сілы нам скончыць сёньняшні праект!» Чаканьне і размовы, як і ў рэстаране, робяць ежу смачнейшай.

\section{Як трымацца правіла? Ідэі і парады}

\subsection{Ці гатовыя вы да ежы?}
Ці зьнізіўся стрэс? Гэта лёгка праверыць, бо пры стрэсе прыгнятаецца выпрацоўка сьліны і ротавая поласьць сухаватая. Калі пры выглядзе ежы ў вас сухі рот, то трэба яшчэ адпачыць, калі сьліна зьяўляецца – можна есьці!

\subsection{Гарачая ежа.}
Занадта гарачая ежа і напоі пры працяглым выкарыстаньні шкодзяць сьлізьніцы і павялічваюць рызыку апёку і раку рота, гартані і стрававода. А рэзкая зьмена гарачай і халоднай ежы можа сапсаваць зубную эмаль. Таму паўза перад ядой добрая, каб даць астыць ежы, бясьпечная тэмпэратура – 40 ° C і ніжэй.

\subsection{Халодная вада.}
Вымыйце рукі, можна і твар, перад дой. Халодная вада зьніжае стрэс, стымулюе тонус вагусу (блукальнае нэрвы). Высокі тонус вагусу, у сваю чаргу, спрыяе лепшаму страваваньню.

\subsection{Дыхайце.}
Дыхальныя практыкаваньні дапамагаюць супакоіцца і паслабіцца. Некалькі глыбокіх выдыхаў зь невялікай затрымкай на выдыху дапамогуць вам. Можна выкарыстоўваць адмысловы дадатак на смартфоне.

\subsection{Паважайце ежу.}
Важна ставіцца да ежы з павагай! Калі мы садзімся за стол з думкай «не памерці б з голаду ад гэтай травы», то ані задавальненьня, ані сытасьці мы не атрымаем. Калі вы ўспрымаеце ежу як крыніцу энэргіі, з удзячнасьцю і падзякай, задавальненьнем, гэта будзе карысна для здаровых харчовых паводзінаў. А калі ведаеце, колькі там вітамінаў, мінэралаў і карысных рэчываў, колькі энэргіі – то гэтыя веды і павага трансфармуюцца ў сытасьць і задавальненьне. Думайце пра ежу як пра крыніцу энэргіі. Бо гэта сапраўды так: лічаныя месяцы таму энэргія, зьмешчаная ў хімічных сувязях вашай брокалі, была фатонамі, якія нарадзіліся ў тэрмаядзерных сонечных рэакцыях! Ведайце і паважайце ежу – і яна адкажа вам узаемнасьцю!

Занадта гарачая ежа і напоі пры працяглым выкарыстаньні шкодзяць сьлізьніцы і павялічваюць рызыку апёку і раку рота, гартані і стрававода. Рэзкая зьмена гарачай і халоднай ежы можа пашкодзіць эмаль зубоў. Бясьпечная тэмпэратура ежы – 40 ° C і ніжэй.

\subsection{Задайце сабе правільныя пытаньні.}
Ці ўзбагачае гэтая ежа мой рацыён? Настолькі я галодны? Ці можна зьесьці больш здаровую альтэрнатыву? Якую порцыю я зьбіраюся зьесьці? Дасьледаваньні паказалі, што візуалізацыя меншай порцыі падчас ежы прыводзіць да зьніжэньня колькасьці зьедзеных калёрыяў.

\subsection{Ці можна піць перад ежай?}
У цэлым можна, калі адчуваеце смагу. Пасьля шклянкі вады няхай пройдзе яшчэ пара хвілінаў да прыёму ежы. Водна-солевы балянс разабраны ў разьдзелах «Водны балянс», «Балянс натрый – калій».

Думайце пра ежу як пра крыніцу энэргіі. Бо гэта сапраўды так: лічаныя месяцы таму энэргія, зьмешчаная ў хімічных сувязях вашай брокалі, была фатонамі, якія нарадзіліся ў тэрмаядзерных сонечных рэакцыях! Ведайце і паважайце ежу – і яна адкажа вам узаемнасьцю!

\subsection{Малітва.}
Вы можаце прыдумаць (ці ўзяць існую) малітву, якую будзеце чытаць перад ядой. Стаўшы часткай вашага рытуалу, яна выдатна супакоіць і цудоўна настроіць на сталаваньне.
