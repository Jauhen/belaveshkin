\chapter{Сьняданак}

Сняданак~--- гэта найважнейшы прыём ежы. У~ранішні час арганізм заводзіць наш унутраны гадзіньнік і настройвае біярытмы. Тое, як мы пасьнедаем, будзе шмат у~чым вызначаць нашае самаадчуваньне і энэргічнасьць днём, голад і сытасьць, а~таксама наш настрой і нават сон. Для многіх людзей раніца~--- гэта час без апэтыту і радасьці, але гэта няправільна. Наладзіўшы сваю раніцу, вы будзеце прачынацца са смакам да жыцьця і добрым апэтытам~--- і гэта значыць, што вы на слушным шляху да здароўя! Калі мае кліенты сустракаюць раніцу з~задавальненьнем і вяртаюць апэтыт, яны хутчэй здаравеюць.

Добры сьняданак дапамагае добра пачаць дзень і захаваць энэргічнасьць да самага вечара! Успомніце, калі вы адпраўляецеся ў~падарожжа і бэнзіну мала, вы адразу запраўляе поўны бак, каб затым ехаць спакойна і не баяцца таго, што паліва скончыцца. Пачынаючы дзень, важна папоўніць свае энэргетычныя запасы, каб цягам дня ваш арганізм спраўна функцыянаваў.

\tipbox{Для многіх людзей раніца~--- гэта час без апэтыту і радасьці, але гэта няправільна. Наладзіўшы сваю раніцу, вы будзеце прачынацца са смакам да жыцьця і добрым апэтытам~--- і гэта значыць, што вы на слушным шляху да здароўя!}

\section{Як зьявілася праблема?}
Павелічэньне тэмпу жыцьця і адсутнасьць рэжыму дня зрушваюць наш графік, мы вячэраем і кладзёмся спаць пазьней. Позьняя вячэра і ўздым прыводзяць да адсутнасьці часу для гатаваньня сьняданку, і ў~нас да таго ж няма апэтыту раніцай. Усё гэта фармуе заганнае кола, і мы ссоўваем гадзіны прыёмаў ежы на ўсё пазьнейшы час. Цяпер увогуле ў~сярэднім 30\,\% людзей ня сьнедаюць, многія пераходзяць на позьні сьняданак, а~колькасьць ужытых ранкам калёрыяў імкліва зьніжаецца.

\section{Як гэта ўплывае на здароўе?}

\subsection{Сардэчна-сасудзістыя захворваньні.}
Навуковыя дасьледаваньні пацьвердзілі, што рэгулярны сьняданак станоўча ўплывае на здароўе сардэчна-сасудзістай сыстэмы і зьніжае рызыку многіх захворваньняў. Адсутнасьць сьняданку прыводзіць да парушэньня ліпіднага профілю крыві, павелічэньня атэрасклератычнай паразы сасудаў. Агульная колькасьць атэрасклератычных бляшак у~паўтара разы вышэйшая ў~тых, хто прапускае сьняданак. Акрамя таго, павялічваецца рызыка сардэчнага прыступу ды ішэміі сэрца са сьмяротным вынікам.

\subsection{Атлусьценьне і цукровы дыябэт.}
Сьняданак зьніжае рызыку мэтабалічнага сындрому (атлусьценьне, дыябэт, гіпэртэнзія і да т.~п.). Большасьць людзей, якія пакутуюць на атлусьценьне, ня маюць звычкі сьнедаць або зьядаюць за сьняданкам вельмі мала. Тыя, хто рэгулярна сьнедае, маюць на 30\,\% меншую імавернасьць атлусьценьня і ў~два разы зьніжаюць рызыку цукровага дыябэту.

\subsection{Фэртыльнасьць.}
Сьняданак уплывае нават на разьвіцьцё захворваньняў, якія на першы погляд складана зьвязаць з~харчаваньнем. Напрыклад, на полікістоз яечнікаў, што ўзьнікае ў~сувязі з~інсулінарэзыстэнтнасьцю. Сытны сьняданак і нізкакалярыйная вячэра дапамагаюць павысіць адчувальнасьць да інсуліну і палепшыць гарманальны профіль, падвышаючы шанцы на зачацьце.

\subsection{Сыстэма «голад~--- сытасьць».}
Адсутнасьць сьняданку прыводзіць да парушэньня насычэньня і паніжанага кантролю голаду на працягу ўсяго дня. Навукоўцы высьветлілі, што маладыя людзі, якія прапускаюць сьняданак, зьядаюць на 40\,\% больш прысмакаў цягам дня.

\tipbox{Спалучэньне сьняданку зь яркім сонечным сьвятлом, фізычнай актыўнасьцю і прыемным абуджэньнем~--- гарантыя добрага пачатку дня.}

\subsection{Прадуктыўнасьць, стрэс.}
Рэгулярны сьняданак спрыяе паляпшэньню канцэнтрацыі ўвагі, стабілізуе ўзровень цукру ў~крыві. IQ дзяцей, якія рэгулярна сьнедаюць, вышэйшае. Пры адсутнасьці сьняданку зьмяншаецца наша стрэсаўстойлівасьць, бо раніцай самы высокі ўзровень картызолу, што на тле адсутнасьці ежы павялічвае рызыку разбурэньня цягліц і назапашваньня тлушчу.

\section{Асноўныя прынцыпы}

Здаровы сьняданак~--- гэта рэгулярны, сытны, высокабялковы раньні сьняданак пры дастатковым узроўні асьвятленьня.
Такі сьняданак спрыяе спаленьню тлушчу, пры гэтым мацней выдаткоўваецца менавіта вісцэральны тлушч, а~не падскурны (інтэнсіўней зьніжаецца ахоп таліі). Сьняданак дапамагае кантраляваць узровень грэліну, таму вы пачуваецеся сытым на працягу ўсяго дня. Сьняданак стабілізуе ваганьні цукру ў~крыві, паляпшае настрой і прадуктыўнасьць. Спалучэньне сьняданку зь яркім сьвятлом, фізычнай актыўнасьцю і прыемным абуджэньнем зьяўляецца гарантыяй добрага пачатку дня. Бо менавіта раніца крытычна важная для нашага здароўя: яна дазваляе перазагрузіць, наладзіць і сынхранізаваць працу цыркадных рытмаў.

\subsection{Высокабялковы сьняданак.}
Высокабялковыя прадукты на сьняданак засвойваюцца і насычаюць лепей, чым у~любы іншы прыём ежы. Так, сньяданак з~двух яйкаў паказаў сябе лепш, чым аналягічная па калёрыях колькасьць вугляводаў. Пасьля сьняданку з~падвышанай колькасьцю бялку моладзь ела на вячэру на 200\,ккал менш, а~на працягу дня не адчувала жаданьня чым-небудзь перакусіць. Сьняданак павінен утрымоўваць 50--80 грамаў якаснага бялку і дастатковую колькасьць тлушчаў. Вугляводы на сьняданак павінны складаць меншую долю калёрыяў, кашу варта замяніць гароднінай.

\subsection{Сытны па аб'ёме (breakfast diet).}
Сьняданак павінен складаць больш за 30\,\% аб'ёму сутачнага спажываньня калёрыяў. Ухапіць кавалачак чаго-небудзь з~кавай ня лічыцца паўнавартасным сьняданкам. Важнасьць сытнага сьняданку~--- гэта факт, які знайшоў адлюстраваньне ня толькі ў~навуковых дасьледаваньнях, але і ў~традыцыйнай мудрасьці: «Сьняданак~--- золата, абед~--- срэбра, вячэра~--- медзь», «Сьняданак зьеж сам, абедам падзяліся зь сябрам, а~вячэру аддай ворагу», «Сьнедай як кароль, вячэрай як жабрак» і шматлікія іншыя.

\subsection{Раньні сьняданак.}
Сьнедаць трэба ў~першую гадзіну пасьля абуджэньня. Каб пасьпяваць, вы мусіце прыгатаваць усё зь вечара. Прачнуліся~--- уключайце пліту і займайцеся ранішнімі справамі.

\tipbox{Паўгадзіны сонечнага сьвятла раніцай~--- гэта навукова абгрунтаваная, эфэктыўная працэдура для здаровай працы цыркадных рытмаў. Зімой можна сьнедаць з~уключанымі яркімі лямпамі для фотатэрапіі.}

\subsection{Яркі сьняданак.}
З самае раніцы трэба ўключаць яркае сьвятло, якое зьнізіць узровень мэлятаніну і картызолу. Ранішняе сьвятло павялічвае адчувальнасьць да інсуліну і, як вынік, спрыяе пахудзеньню. Сонечнае сьвятло палепшыць настрой і павялічыць выпрацоўку дафаміну, бо дафамінавыя нэўроны ёсьць і ў~сятчатцы вока, дзе ўтвараюць асаблівую рэтына-дафамінавую сыстэму. Паўгадзіны сонечнага сьвятла раніцай~--- гэта навукова абгрунтаваная, эфэктыўная працэдура для здаровай працы цыркадных рытмаў. Зімой можна сьнедаць з~уключанымі яркімі лямпамі для фотатэрапіі.

\section{Як трымацца правіла? Ідэі і парады}

\subsection{Хуткі сьняданак.}
У ідэале плянуйце такі сьняданак, каб прыгатаваць яго за 10--15 хвілінаў. Складаныя стравы, прадукты з~працяглым часам гатаваньня не пасуюць. Актыўна выкарыстоўвайце спэцыі і травы, каб надаць яркасьці звыклым прадуктам.

\subsection{Прачынайцеся лёгка.}
Лёгкае і прыемнае ранішняе абуджэньне~--- гэта важная прыкмета моцнага здароўя. Усталюйце на будзільнік прыемную мэлёдыю з~паступовым павелічэньнем гучнасьці. Увечары візуалізуйце час абуджэньня, няхай лічбы застануцца ў~вас у~памяці. Прайграйце некалькі разоў у~галаве працэс абуджэньня, як вы лёгка прачынаецеся за пяць хвілінаў да сыгналу будзільніка і адчуваеце сябе выдатна. Сфакусуйцеся на прадчуваньні ад прыемнага абуджэньня.

\subsection{Няма апэтыту.}
Пры адсутнасьці апэтыту раніцай рабіце лёгкую вячэру ці зусім прапускайце яе. Часам пры парушэньні сыстэмы «голад~--- сытасьць» раніцай апэтыту няма, але па меры нармалізацыі вашага стану апэтыт будзе прыходзіць.

\subsection{Прачынайцеся заўсёды ў~адзін час.}
І ў~працоўны, і ў~выходны дзень прачынайцеся ў~аднолькавы час. Ваша цела абвыкне абуджацца ў~правільнай фазе сну, і вам будзе лягчэй уставаць. Можна паэкспэрымэнтаваць, ссоўваючы час абуджэньня на 10--20 хвілінаў у~абодва бакі. А~што рабіць, калі трэба выспацца? Усё проста: кладзіцеся раней спаць на наступны дзень. Гэтае проста правіла: прачынаецеся ў~адзін час, засынаеце, калі захацелася.

\subsection{Выкарыстоўвайце сьветлавы будзільнік.}
Сьветлавы будзільнік плыўна павышае яркасьць сьвятла, і вы прачынаецеся лёгка і хутка. Гэта добры і карысны спосаб, не такі траўматычны, як гучны будзільнік.

\subsection{Выйдзіце на сонца.}
Кароткі шпацыр або прабежка на вуліцы пад сонцам выдатна наладзяць вас і вернуць апэтыт. Цудоўна, калі ў~вас ёсьць сабака! Памятайце, што сонечнае сьвятло~--- гэта дзівосная крыніца здароўя і даўгалецьця, выкарыстоўвайце любую магчымасьць пабываць пад сонцам, асабліва раніцай!

\tipbox{Сьветлавы будзільнік плыўна павышае яркасьць сьвятла, і вы прачынаецеся лёгка і хутка. Гэта добры і карысны спосаб, не такі траўматычны, як гучны будзільнік.}

\subsection{Выкарыстоўвайце тэмпэратурны будзільнік.}
Тэмпэратура таксама зьяўляецца сыгналам для нашых унутраных гадзінаў. Уначы тэмпэратура падае, і сасуды скуры звужаюцца, у~мозг трапляе больш крыві, і мы лягчэй прачынаемся. Зрабіце меншым ацяпленьне на ноч~--- і ранкам прачняцеся ў~прыемнай і бадзёрай прахалодзе. Падвышэньне тэмпэратуры з~дапамогай водных працэдур і зарадкі дапаможа вам зарадзіцца энэргіяй і лягчэй прачнуцца. Разнасьцежце вокны, выйдзіце на балькон, абліцеся вадой.

\subsection{Нагуляць апэтыт.}
Зрабіце інтэнсіўную зарадку, але не даўжэй за 20 хвілінаў. Халодны душ са старанным расьціраньнем цела, заняткі спортам на сьвежым паветры добра падвышаюць апэтыт.

\subsection{Шклянка вады.}
Выпіце шклянку вады пасьля абуджэньня. За ноч вы страцілі крыху вадкасьці, зараз самы час аднавіць водны балянс.

\subsection{Пазьбягайце канцэнтраваных вугляводаў.}
Дапускаюцца нізкавугляводныя гародніна і зеляніна: капуста, салера, шпінат, цыбуля і г.~д. У~звычайным варыянце можна разгледзець сярэднекрухмалістую гародніну. Непажадана рабіць сьняданак з~адной толькі кашы, няхай і карыснай, ня кажучы ўжо пра макарону.

\subsection{Прадоўжыце сытасьць да абеду.}
Для прадаўжэньня сытасьці выкарыстоўвайце тлушчы. У~ідэале ваш сьняданак павінен быць такім, каб думкі аб ежы не прыходзілі вам у~галаву да абеду. Для прадаўжэньня сытасьці дадавайце тлушчы: сьметанковае масла, какосавы, аліўкавы алей, ялавічны тлушч.

\subsection{Дэсэрт на сьняданак.}
На дэсэрт разгледзьце авакада, гарэхі, кавалачкі какосу, ягады, садавіну. Гарэхі выдатна насычаюць і добра пасуюць да гарбаты і кавы.

\subsection{Аўсянка, сэр.}
Насуперак мітам, ангельскі сьняданак~--- гэта не аўсянка, а~яйкі, бекон і фасолю, т.~е. высокабялковыя прадукты. Многія вытворцы шматкоў актыўна іх прасоўваюць, але кашы на сьняданак~--- гэта не самае лепшае рашэньне, так як іх не хапае для кантролю апэтыту і энэргіі да самага абеду.