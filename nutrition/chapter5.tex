\chapter{Вячэра}

Вечар~--- гэта важная частка нашых цыркадных рытмаў\index{цыркадныя рытмы}. Добрая вячэра і якасны сон забясьпечваюць паўнавартаснае аднаўленьне арганізму ўначы. Вячэрнія пераяданьні, начныя перакусы, празьмерны стрэс, лішак сьвятла і стымуляцыі парушаюць вячэрні спакой, структуру сну і яго працягласьць. Чалавек часам сьпіць столькі ж, але аднаўляецца горай. Чым горшы сон, тым горшая работа мэтабалізму, сыстэмы «голад~--- сытасьць», слабейшы самакантроль. Вячэрайце лёгка, рана і правільна~--- гэта падмурак паўнавартаснага сну і здароўя!

На прыкладзе сваіх пацыентаў я бачу, што вечар~--- гэта ўразьлівае месца для многіх людзей. Чаму? Яны ня ўлічваюць стрэс і парушэньні сьветлавога рэжыму. Прывяду прыклад. Ёсьць такая глыбакаводная і вельмі страшная на выгляд рыба~--- марскі д'ябал. У~яе ёсьць нарост на галаве~--- «вуда»,~--- які сьвеціцца ў~цёмнай глыбокай вадзе. Прывабленыя сьвятлом рыбы падплываюць і трапляюць проста ў~пашчу марскога д'ябла. Так і ў~нас залішняе сьвятло і стымуляцыя ўвечары зьбіваюць цыркадныя рытмы\index{цыркадныя рытмы}, прымушаюць думаць, што цяпер дзень, прыліпаць да экранаў ды шмат і сытна есьці ўвечары. Не трапляйце ў~сьветлавую пастку!

\subsection{Як зьявілася праблема?}

Розныя краіны маюць розныя традыцыі. У~адных краінах вячэра раньняя і лёгкая, у~іншых~--- позьняя і досыць сытная, што зьвязана з~працай удзень і гарачым дзённым кліматам. Але ў~цэлым усюды мы бачым стаўленьне да вячэры як да лягчэйшага прыёму ежы («вячэру аддай ворагу», «вячэрай як жабрак» і да т.~п.). А~вось сучасныя людзі пры адсутнасьці культуры харчаваньня сутыкаюцца зь вечаровай спакусай ежай і часта паддаюцца ёй.

\tipbox{Велізарная колькасьць сьвятлодыёдных выпраменьваньняў з~высокаю доляй сьвятла сіняга спэктру (тэлефоны, кампутары, пляншэты, тэлевізары, сьвятлодыёдныя лямпы і інш.) зьбіваюць нашы ўнутраныя гадзіньнікі. Сьвятло такога спэктру дае сыгнал аб тым, што цяпер дзень, і абвастрае пачуцьцё голаду.}

\paragraph{Цыркадная гіпэрфагія (пераяданьне)\index{цыркадная гіперфагія}.}
Пытаньне вячэрняга харчаваньня і вячэры шчыльна зьвязана з~работай нашых цыркадных біярытмаў\index{цыркадныя рытмы}. Велізарная колькасьць сьвятлодыёдных выпраменьваньняў з~высокаю доляй сьвятла сіняга спэктру (тэлефоны, кампутары, пляншэты, тэлевізары, сьвятлодыёдныя лямпы і інш.) зьбіваюць нашы ўнутраныя гадзіньнікі. Сьвятло такога спэктру дае сыгнал аб тым, што цяпер дзень, і абвастрае пачуцьцё голаду.

\paragraph{Гіперфагія\index{гіперфагія} і эпідэмія стрэсу.}
Хранічны стрэс становіцца распаўсюджанай зьявай. Калі востры стрэс часьцей прытупляе апэтыт, то хранічны~--- наадварот, узмацняе з~прычыны падвышанай выпрацоўкі грэліну\index{грэлін}. Стомленасьць і голад на глебе стрэсу павялічваюцца ўвечары, што прыводзіць да пераяданьня. Больш за тое, грэлін\index{грэлін} узбуджае нэрвовую сыстэму, таму людзі вымушаныя есьці для таго, каб заснуць, выкарыстоўваючы ежу як снатворнае. Злоўжываньне кафэінам\index{кафэін} і алькаголем\index{алькаголь} таксама пагаршае разлады прыёму ежы.

\paragraph{Геданічная гіпэрфагія} (імкненьне есьці дзеля атрыманьня задавальненьня пры адсутнасьці дэфіцыту энэргіі).
Акрамя стрэсу, існуе і сындром дэфіцыту задаволенасьці, калі назіраецца зьніжэньне здольнасьці атрымліваць задавальненьне ад жыцьця. Такія людзі ядуць, бо ім сумна, тужліва, самотна, мала радасьцяў, і выбіраюць пры гэтым максымальна калярыйныя прадукты, нездаровыя спалучэньні мучнога, тлустага, салодкага, смажанага, салёнага і інш.

\tipbox{У людзей, якія пакутуюць на начныя перакусы, парушаная работа біялягічных рытмаў, фізыялягічная работа шматлікіх гармонаў, нармальны абмен рэчываў. Чым больш і чым часьцей чалавек есьць позна, тым больш выяўленымі могуць быць яго сымптомы і парушэньні сну.}

\paragraph{Працоўная нагрузка.}
Інтэнсіўная працоўная нагрузка, адсутнасьць паўнавартаснага сьняданку і часу на абед прыводзяць да зьяўленьня фізыялягічнага інтэнсіўнага голаду ўвечары і пераяданьня. Працяглыя прамежкі бязь ежы раніцай і днём прыводзяць да падвышанай выпрацоўкі картызолу, што дадаткова ўзмацняе стрэс.

У людзей, якія пакутуюць на начныя перакусы, парушаная работа біялягічных рытмаў, фізыялягічная работа шматлікіх гармонаў, нармальны абмен рэчываў. Чым больш і чым часьцей чалавек есьць позна, тым больш выяўленымі могуць быць яго сымптомы і парушэньні сну. Позьнія вячэры і начныя перакусы прыводзяць да атлусьценьня, падвышанай рызыкі цукровага дыябэту 2-га тыпу, а~таксама псыхалягічных праблемаў. Дасьледаваньні пацьвердзілі, што падвышаны апэтыт у~начны час зьвязаны з~прыгнечанасьцю, схаванай дэпрэсіяй, эмацыйнай напругай, трывожнасьцю.

\subsection{Як гэта ўплывае на здароўе?}

Шэраг нэгатыўных узьдзеяньняў позьніх прыёмаў ежы разабраны ў~папярэднім разьдзеле. Мы адзначым, што існыя дасьледаваньні даюць далёка не заўсёды адназначны адказ. Па сутнасьці, калі лічыць колькасьць спажываных калёрыяў (што далёка не заўсёды рэальна ў~доўгатэрміновай пэрспэктыве), то ўплыў вячэры на структуру цела не такі вялікі. Так, ва ўмовах лёгкага дэфіцыту калёрыяў, харчаваньне раніцай і ўвечары на працягу 12 тыдняў не паўплывала на ўзровень тлушчу. Але большая колькасьць дасьледаваньняў знаходзіць сувязь паміж позьняй сытнай вячэрай і атлусьценьнем.

Рэжым харчаваньня зьвязаны з~запаленьнем у~тлушчавай тканцы. Прыём ежы ўначы актывуе імунныя клеткі макрафагаў і выклікае большы запаленчы адказ у~тлушчавай тканцы ў~параўнаньні зь ядой удзень. Яда на ноч больш чым у~7 разоў павялічвае імавернасьць пякоткі ў~параўнаньні зь ядой за 3 гадзіны да сну. Існуюць дасьледаваньні, якія паказваюць, што позьнія прыёмы ежы зьніжаюць навучальнасьць і прыводзяць да парушэньняў работы гіпакампу\index{гіпакамп}.

Ёсьць і іншыя праблемы. Так, рызыка разьвіцьця ішэмічнай хваробы сэрца на 55\,\% вышэйшая ў~тых, хто есьць позна ноччу. Цікава, што рызыка разьвіцьця гарманальных парушэньняў, у~тым ліку раку грудзей і раку падкарэньнічнай залозы, таксама зьвязаная з~позьнімі прыёмамі ежы. Рызыка разьвіцьця раку малочнай залозы зьніжаецца на 16\,\%, а~раку падкарэньніцы~--- на 26\,\% у~тых, хто вячэрае мінімум за дзьве гадзіны да сну.

\paragraph{Сындром начной яды.}
Скрайнім парушэньнем харчовых біярытмаў зьяўляецца сындром начной яды. Яго асноўныя сымптомы~--- гэта частыя (больш за 2 разы на тыдзень) эпізоды прыёму ежы позна ўвечары ці ноччу, за якія паглынаецца больш за 25\,\% сутачнай дозы калёрыяў, пэрыядычная (больш за 4 разы на тыдзень) адсутнасьць апэтыту раніцай, пачуцьцё віны і сораму праз харчовыя паводзіны, зьніжэньне якасьці жыцьця. Падобны разлад патрабуе кансультацыі псыхатэрапэўта і работы над нармалізацыяй біярытмаў, перамагчы яго адной сілай волі цяжка. Сындром начнога апэтыту небясьпечны і для людзей са звычайнай вагой, бо пагаршае якасьць сну. Чым большую долю дзённых калёрыяў вы зьядаеце ўвечары і ўначы, тым вышэйшая рызыка атлусьценьня для вас у~будучыні.

\subsection{Асноўныя прынцыпы}

\paragraph{Вячэрайце рана, умерана і карысна.}
Памятайце, што ня толькі сьняданак уплывае на сон, але й~вячэра закладае падмурак для сьняданку. Лёгкая раньняя вячэра~--- гэта гарантыя лёгкага абуджэньня і наяўнасьці апэтыту раніцай. Напрыклад, навукоўцы высьветлілі, што прадукты зь нізкім глікемічным індэксам на вячэру стабілізуюць цукар у~крыві на ўвесь наступны дзень!

Неабходна вячэраць як мінімум за 3--4 гадзіны да сну. Калярыйнасьць вячэры павінна складаць ад 20 да 30\,\% ад содневай калярыйнасьці. Аптымальныя прадукты~--- сярэдне-і нізкакрухмалістыя гародніна і зеляніна, карысныя тлушчы (сьметанковае масла, аліўкавы, какосавы алеі). Порцыя можа быць вялікай, і не забывайце ўжываць і сырую гародніну. Абмяжуйце выкарыстаньне крупаў, мучных вырабаў, паўфабрыкатаў і вялікай колькасьці бялковых прадуктаў.

\tipbox{Аддайце перавагу таму, што ўтрымлівае больш клятчаткі\index{клятчатка}: гародніна і зеляніна. Рафінаваныя вугляводы павялічваюць рызыку бессані, а~вось клятчатка\index{клятчатка} на вячэру можа падоўжыць фазу павольнага сну!}

\paragraph{Раньняя вячэра.}
Раньняя вячэра дапамагае павялічыць харчовую паўзу, а~акрамя таго, неўзаметку скарачае колькасьць калёрыяў. Навукоўцы давялі, што поўнае абмежаваньне яды з~7-й вечара да 6-й раніцы ў~здаровых людзей без абмежаваьньня каляражу прыводзіць да таго, што яны ў~сярэднім пачынаюць есьці менш на 240\,ккал у~содні! А~гэта ў~доўгатэрміновай пэрспэктыве дае дастаткова прыкметнае паніжэньне вагі! Раньняя вячэра спрыяе выпрацоўцы гармону росту, больш за 70\,\% колькасьці якога выдзяляецца ноччу. Гармон росту стымулюе рост цяглічнай тканкі і спальваньне тлушчу. Яда на ноч, у~сваю чаргу, спрыяе выдзяленьню інсуліну, які зьмяншае ўзровень гармону росту.

\paragraph{Вугляводы.}
Нягледзячы на тое, што вугляводы могуць даць прыемную дрымоту, ня варта на вячэру есьці крухмалістыя вугляводы, варта абмежаваць мучное і салодкае. Аддайце перавагу таму, што зьмяшчае больш клятчаткі\index{клятчатка},~--- гародніне і зеляніне. Рафінаваныя вугляводы павялічваюць рызыку бессані, а~вось клятчатка\index{клятчатка} на вячэру можа падоўжыць фазу павольнага сну! Таму складаная салата~--- гэта выдатнае рашэньне! Навуковыя дасьледаваньні паказваюць, што цукар на вячэру выклікае павелічэньне эпізодаў мікраабуджэньня цягам ночы, што заўважна зьніжае якасьць сну.

\paragraph{Тлушчы.}
Тлушчы таксама зьяўляюцца важным кампанэнтам вячэры, але толькі ў~спалучэньні з~клятчаткай\index{клятчатка}. Тлушчы не выклікаюць уздыму інсуліну, стымулююць выдзяленьне гармону халецістакініну, які дапамагае адчуваць сытасьць. Какосавы, аліўкавы алей, сьметанковае масла і да т.~п. будуць найлепшыя. Нават кавалачак сала пойдзе на карысьць. Таму сьмела запраўляйце гародніну тлушчамі, але захоўвайце пачуцьцё меры.

\tipbox{Багацьце бялковай ежы не пасавацьме трывожным людзям з~праблемным засынаньнем. Але для тых, хто трэніруецца і моцна сьпіць, працуе дапазна, бялок на вячэру таксама можа быць прыдатным варыянтам.}

\paragraph{Бялкі.}
Пытаньне бялковых прадуктаў на вячэру не зусім адназначнае і павінна вырашацца індывідуальна. З~аднаго боку, бялкі ўзмацняюць актыўнасьць і стымулююць нэрвовую сыстэму, што ня вельмі добра для сну. Багацьце бялковай ежы не пасавацьме трывожным людзям з~праблемным засынаньнем. Зь іншага боку, бялкі выдатна спаталяюць голад і доўга падтрымліваюць апэтыт. Таму для тых, хто трэніруецца і моцна сьпіць, працуе дапазна, бялок на вячэру таксама можа быць прыдатным варыянтам.

\subsection{Як трымацца правіла? Ідэі і парады}

\paragraph{Вячэрняе асьвятленьне.}
Для правільнай работы цыркадных біярытмаў сьвятло ўдоме павінна па спэктры супадаць са сьвятлом на захадзе сонца, быць жоўтым, расьсеяным, няяркім. Для гэтага ўвечары трэба ўключаць не сьвятлодыёдныя лямпы, а~лямпы напальваньня, якія дадуць патрэбны спэктар. Але большую частку сьвятлодыёднага сьвятла даюць смартфоны, ноўтбукі, тэлевізары~--- для гэтага таксама ёсьць варыянты. Можна ўсталяваць праграмы накшталт f.lux, twilight\footnote{f.lux, twilight~--- гэта праграмы, што рэгулююць колеравую тэмпэратуру кампутарнага манітора ў~адпаведнасьці з~геаграфічным месцівам і часам сутак карыстальніка.}, скарыстацца функцыяй night shift. Але найлепшым варыянтам будзе купіць спэцыяльныя акуляры, якія блякуюць сінюю частку спэктру.

\paragraph{Тэмпэратура ўвечары.}
Тэмпэратура таксама зьяўляецца важным сыгналам для нашага цела. Тэмпэратура нашага асяродку павінна зьніжацца прыкладна зь 19-й. Усё, што замінае нам «астыць», будзе замінаць расслабіцца і заснуць. Таму такія парады, як пагуляць перад сном, праветрыць спальню, паменшыць інтэнсіўнасьць апалу,~--- выдатна працуюць.

\paragraph{Стрэс увечары.}
Вячэрні стрэс~--- частая прычына страты кантролю над харчовымі паводзінамі і пераяданьня. Пры стрэсе нас цягне на тлустае, салодкае і салёнае. Гэта частка ахоўнай рэакцыі, бо наш арганізм успрымае стрэс як фізычную актыўнасьць («нападаю~--- уцякаю»), таму стымулюе нас папоўніць каляраж. Антыстрэсавыя тэхнікі выдатна памяншаюць апэтыт і спрыяюць глыбейшаму сну: цягліцавае паслабленьне, вядзеньне дзёньніка, глыбокае раўнамернае дыханьне, масажны кілімок з~іголкамі, фрырайтынг\index{фрырайтынг}\footnote{Фрырайтынг (ці вольнае пісьмо)~--- тэхніка і методыка пісьма, якая дапамагае знайсьці неардынарныя рашэньні і ідэі, падобная з~мэтадам мазгавога штурму.}, ёга дапамагаюць палепшыць самаадчуваньне і зьнізіць узровень стрэсу, а~адпаведна і апэтыту. Усе працэдуры, накіраваныя на цела, павялічваюць выпрацоўку антыстрэсавага гармону аксытацыну\index{аксытацын}: водныя працэдуры, масаж, расьціраньні, саўна, самамасаж і да т.~п.

\paragraph{Кафэін\index{кафэін}.}
Кафэін\index{кафэін} утрымліваецца ня толькі ў~каве, але і ў~гарбаце, шматлікіх напоях, батончыках і леках. Яго сярэдні пэрыяд паўраспаду складае 4--5 гадзінаў, але дзеяньне можа адчувацца і ўсе восем гадзінаў. Таму калі вы хочаце пазьбегнуць узьдзеяньня кафэіну на сон, то ня варта ўжываць каву, гарбату ды іншыя кафэінавыя напоі пасьля 12:00–14:00 гадзінаў. Навукоўцы высьветлілі, што кафэін ссоўвае біярытмы на больш позьні час, скарачае час павольнага сну. Залішняе ўжываньне кафэіну прыводзіць да парушэньня сну і зьяўленьня ранішняй стомленасьці. Ёсьць людзі, генэтычна вельмі адчувальныя да кафэіну,~--- у~іх ён выклікае трывогу і ўзмацняе нэўроз. Такім людзям абмежаваньне кафэіну\index{кафэін} вельмі станоўча адаб'ецца на здароўе.

\tipbox{Сярэдні пэрыяд кафэіну\index{кафэін} паўраспаду складае 4--5 гадзінаў, але дзеяньне можа адчувацца і ўсе восем гадзінаў. Таму калі вы хочаце пазьбегнуць узьдзеяньня кафэіну\index{кафэін} на сон, то ня варта ўжываць каву, гарбату ды іншыя кафэінавыя напоі пасьля 12:00–14:00 гадзінаў.}

\paragraph{Алькаголь\index{алькаголь}.}
Алькаголь\index{алькаголь}~--- вельмі калярыйны прадукт. Ён хоць і паскарае засынаньне, але зьмяншае якасьць сну, асабліва ў~другой палове ночы, узмацняе храп, выклікае абязводжваньне\index{абязводжваньне}, пры працяглым ужываньні можа выклікаць бессань. Кладзіцеся спаць цьвярозымі.

\paragraph{Ніякіх паўфабрыкатаў і гатовай ежы.}
Яны ўтрымліваюць вялікую колькасьць цукру, солі і ўзмацняльнікаў смаку. Салёныя прадукты на вячэру прывядуць да затрымкі вадкасьці і да таго, што вы будзеце выглядаць раніцай горш.

\paragraph{Менш тыраміну\index{тырамін}.}
На вячэру ня варта есьці прадукты, якія ўтрымліваюць шмат біягенных амінаў\index{біягенныя аміны}, напрыклад тыраміну. Яны ўзмацняюць выдзяленьне адрэналіну і пагаршаюць сон. Тырамін\index{тырамін} ёсьць у~чакалядзе, віне, соўсах, сырах і шэрагу бялковых і малочных паўфабрыкатаў.

\paragraph{Задавальненьне ўвечары.}
Плянуйце прыемнае на вечар загадзя і рабіце гэта, нават калі ня хочацца. Такі прынцып ляжыць у~аснове мэтаду паводзіннай актывацыі, вельмі эфэктыўнай пры дэпрэсіі. Тут пасуе хобі, камунікацыя з~роднымі і сябрамі, любое некалярыйнае задавальненьне. Калі мы стамляемся, то можам адмаўляцца ад зносінаў, значных для нас справаў. Але ж калі мозг пазбаўляецца эмацыйнага падсілкоўваньня, ён шукае спосабы папоўніць яго дэфіцыт~--- і знаходзіць у~ежы! Са свайго досьведу я бачу, што вячэрні голад вельмі часта мае эмацыйныя карані.

\paragraph{Вымушаная вячэра.}
Часта людзі вымушаныя вячэраць позна, калі няма магчымасьці есьці паўнавартасна і спакойна на працы. Гэта не ідэальны рэжым, але ня варта палохаць сябе жахалкамі, што ежа будзе «гніць уначы ў~кішачніку» і да т.~п. Паступова адбываецца адаптацыя арганізму~--- і ежа засвойваецца нармальна. Жахалкі, што ўначы страўнік і кішачнік не працуюць, няслушныя: іх функцыя крыху зьніжаная, але не крытычна. А~ў выпадку адаптацыі яны могуць прыстасавацца да вашага рытму жыцьця.