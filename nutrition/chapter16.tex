\chapter{Хуткая і павольная ежа}

Розныя прадукты па-рознаму ўплываюць на ўмоўную хуткасьць мэтабалізму, без увагі на іх калярыйнасьць. Некаторыя прадукты валодаюць большай здольнасьцю стымуляваць паскарэньне працэсаў у клетцы. У аснове гэтага паскарэньня ляжыць малекулярны комплекс mTORC (mammalian target of rapamycin complex, то-бок «мішэнь рапаміцыну ў млекакормячых»), які рэгулюе хуткасьць росту і ачышчэньня клетак.

mTORC – гэта своеасаблівая пэдаль газу для нашага мэтабалізму, падвышаная актыўнасьць якога паскарае рост і размнажэньне клетак, сынтэз новых бялкоў, спрыяе большаму выжываньню клетак. Актыўнасьць mTORС таксама важная для здароўя: ён стымулюе рост цягліцаў, стварае новыя мітахондрыі ды паляпшае сувязі паміж нэрвовымі клеткамі.

Гэты малекулярны комплекс рэагуе на сыгналы ад розных нутрыентаў і зьмяняе актыўнасьць розных фактараў росту клеткі. Нізкая актыўнасьць mTORC запавольвае рост клетак, спрыяе павелічэньню ўзроўню рэпарацый і аўтафагіі (ачысткі клетак), гібелі пашкоджаных і непатрэбных клетак. Залішняя актыўнасьць mTORC ляжыць у аснове разьвіцьця шматлікіх захворваньняў. Для аптымальнага здароўя важна чаргаваць пэрыяды спажываньня прадуктаў, якія падвышаюць mTORC, з больш працяглымі пэрыядамі яды прадуктаў зь нізкай стымуляцыяй актыўнасьці mTORC або з харчовым устрыманьнем.

Уявіце сабе, што нашае жыцьцё – гэта язда на аўтамабілі. Для доўгай і бясьпечнай язды трэба прытармажваць, спыняцца, прапускаць іншыя машыны. Калі ціснуць на пэдаль аўтамабіля вельмі моцна, ды яшчэ і на нізкіх перадачах, то ён, вядома, будзе эфэктна раўці, але гэта можа прывесьці да заўчаснага зносу рухавіка. Плыўны націск, правільнае выкарыстаньне перадач падоўжаць жыцьцё машыне і дазволяць зэканоміць паліва. Так і з нашым арганізмам – пажадана не злаўжываць хуткай ежай, а знайсьці правільны балянс!

\section{Як зьявілася праблема?}

Калі б нашага продка спыталі, вэган ён ці мясаед, то яго адказ здзівіў бы нас. У дні вялікіх і малых постаў людзі мяса ня елі, часта былі посты з выразным абмежаваньнем калёрыяў, і спажываньне мяса было абмежаванае такім рэжымам харчаваньня. Так і нашы ўсяедныя продкі паляўнічыя-зьбіральнікі мелі цыклічны рэжым харчаваньня: падчас паляваньня яны пераходзілі на карнівор-дыету (толькі жывёльная ежа), бо мяса захоўваць складана, а ў часы паходаў або няўдалага паляваньня вялі лад жыцьця вэганаў (клубні, карэньні і да т. п.). Павелічэньне спажываньня мяса і тлушчу традыцыйна было зьвязанае зь зімовай парой, калі недахоп корму і зьніжэньне запасаў ежы прыводзілі да забою хатніх жывёлаў.

Нашы ўсяедныя продкі паляўнічыя-зьбіральнікі мелі цыклічны рэжым харчаваньня, падчас паляваньня яны пераходзілі на карнівор-дыету (толькі мяса), бо мяса захоўваць складана, а ў час паходаў або няўдалага паляваньня вялі лад жыцьця вэганаў (клубні, карэньні і г.д.).

Калі ў нас ёсьць шмат ежы (вугляводаў або бялку) і шмат калёрыяў, то гэта падвышае актыўнасьць mTORC, і гэты сыгнал кажа арганізму, што цяпер часы багацьця, значыць, можна расьці ды размнажацца. Уласна, стары жарт пра тое, што мясаеды «раздражняльныя празь мяса» мае навуковае абгрунтаваньне, бо залішняя стымуляцыя mTORС павялічвае ўзровень адрэналіну і актыўнасьць стрэсавай сымпацыйнай сыстэмы. Таму зьніжэньне стымуляцыі прыводзіць да зьніжэньня энэргічнасьці, і мясаеды маюць рацыю, што «слабыя бязь мяса». Пагаворым пра гэта падрабязьней.

\section{Як гэта ўплывае на здароўе?}

У аснове шматлікіх хваробаў і старэньня ляжыць залішняя актыўнасьць клетак. Гэта разрастаньне атэрасклератычных бляшак, паскораны рост ракавых клетак, гіпэртрафія міякарду і да т. п. Такія хваробы часта называюць mTORС-залежнымі, бо яны часьцей сустракаюцца ў краінах з высокім узроўнем спажываньня «хуткіх» прадуктаў.

\subsection{Сыгнальны шлях інсуліну і IGF-1.}
Інсулін і IGF-1 зьяўляюцца гармонамі, якія непасрэдна павялічваюць стымуляцыю mTORС і запавольваюць працэсы аўтафагіі. Павелічэньне іх узроўню павышае рызыку ўзьнікненьня многіх тыпаў раку, у тым ліку раку грудзей, прастаты, тоўстай кішкі. Павелічэньне даволі заўважнае ў выпадку раку грудзей, высокі ўзровень IGF-1 на 40% павялічвае рызыку раку грудзей.

\subsection{Пухліны.}
Залішняя актывацыя работы mTORС прыводзіць да бескантрольнага дзяленьня клетак, што азначае ператварэньне іх у ракавыя. mTORС актыўна ўплывае на разьвіцьцё злаякасных пухлінаў, павялічваючы ангіягенэз (рост новых крывяносных сасудаў у самой пухліне і вакол яе), што дапамагае раку расьці.

\subsection{Акнэ.}
Акнэ зьяўляецца вонкавай праявай падвышанай актыўнасьці mTORC і частым прадвесьнікам аддаленай рызыкі для здароўя. Нізкавугляводная дыета, якая ўключае прадукты зь нізкім глікемічным індэксам і выключае малочныя прадукты, прыкметна паляпшае стан скуры, а заадно і вонкавы выгляд. Цікава, што ў карэнных народаў у розных кутках сьвету зьява акнэ ў падлеткаў і ў дарослых цалкам адсутнічае!

\subsection{Імунітэт.}
Залішняя актыўнасьць mTORC спрыяе выжываньню аўтарэактыўных клонаў клетак, што павялічвае рызыку аўтаімунных захворваньняў. Увесьчасна высокая актыўнасьць mTORС павялічвае запаленьне.

\subsection{Мэтабалізм.}
Паніжэньне актыўнасьці mTORС паляпшае адчувальнасьць цяглічных клетак да інсуліну, што абараняе ад разьвіцьця дыябэту ці запавольвае яго разьвіцьцё.

У аснове шматлікіх хвароб і старэньні ляжыць залішняя актыўнасьць клетак. Гэта разрастаньне атэрасклератычных бляшак, паскораны рост ракавых клетак, гіпэртрафія міякарду і г. д. Такія хваробы часта называюць mTORС-залежнымі, бо яны часьцей сустракаюцца ў краінах з высокім узроўнем спажываньня «хуткіх» прадуктаў.

\subsection{Сардэчна-сасудзістыя захворваньні.}
У норме актывацыя mTORC зьмяншае апэтыт і прыводзіць да зьніжэньня вагі праз актывацыю тлушчаспаленьня. Чым вышэйшая актыўнасьць mTORС, тым больш павялічваецца выкід адрэналіну і актыўнасьць стрэсавай сымпацыйнай вэгетатыўнай сыстэмы. Менавіта таму інсулагенныя прадукты і дробавае харчаваньне могуць паляпшаць самаадчуваньне. Але ўвесь час падвышаны тонус сымпацыйнай сыстэмы вядзе да падвышэньня артэрыяльнага ціску і рызыкі гіпэртэнзіі.

\subsection{Трывога, стрэс і выгараньне.}
У выпадку з гіперактывацыяй mTORС нарастае актыўнасьць сымпацыйнай сыстэмы (стрэс), а парасымпацыйнай (расслабленьне) – прыгнятаецца. Працяглае павышэньне сымпацыйнай актыўнасьці ў людзей са схільнасьцю павялічвае ўзровень турботы, трывогі і раздражняльнасьці. Працяглая гіпэрактывацыя сымпацыйнай сыстэмы (рэжым хранічнага стрэсу) прыводзіць да выгараньня (ці заяданьня праблемы).

\subsection{Аднаўленьне.}
Залішняя актывацыя mTORС – гэта ня толькі празьмерны рост клетак, але й прыгнятаньне рэпарацый і аўтафагіі. Пры парушэньні аднаўленьня ўзмацняецца назапашваньне пашкоджваньняў у клетках, павялічваецца колькасьць клеткавага сьмецьця, што зьяўляецца асновай разьвіцьця мноства розных захворваньняў.

\subsection{Нэўрадэгенэратыўныя захворваньні.}
Больш высокі ўзровень IGF-1 назіраецца ў пацыентаў з хваробай Альцгеймэра, і яго ўзровень карэлюе з выяўленасьцю сымптомаў.

\subsection{Старэньне.}
Важна разумець, што старэньне і дзяленьне клетак – гэта два бакі аднаго медаля. Тое, што стымулюе рост колькасьці клетак, у выпадку завяршэньня росту арганізму можа ўзмацняць працэсы старэньня. Чым мацней мы стымулюем клеткі дарослага арганізму надмерам пажыўных рэчываў, тым хутчэй яны старэюць. Мутацыі ў генах інсуліну і IGF-1 відавочна падаўжалі жыцьцё экспэрымэнтальным жывёлам.

\section{Асноўныя прынцыпы}

\subsection{Важна чаргаваць пэрыяды высокай і нізкай mTORС-актыўнасьці.}
Пэрыяды неактыўнага mTORС важныя для аднаўленьня клетак. Пастаянная стымуляцыя прыводзіць да таго, што нашы клеткі становяцца «засьмечанымі» і страчваюць адчувальнасьць да сыгналаў арганізму. Зразумела, актыўнасьць mTORС будзе падаць, калі мы не ямо зусім, але паўплываць на актыўнасьць можна ня толькі голадам, але й выбарам прадуктаў.

Прадукты харчаваньня маюць розны ўплыў на актыўнасьць mTORС. Ёсьць нэўтральныя прадукты, якія стымулююць mTORС прапарцыйна колькасьці калёрыяў, а ёсьць «хуткія» прадукты, якія стымулююць mTORС нашмат мацней. Калі чалавек расьце або фізычна актыўны значную частку дня, то асаблівай шкоды для яго няма. Але калі чалавек мае меншую фізычную актыўнасьць, то гэтыя прадукты будуць прыводзіць да росту mTORС-залежных хваробаў, пра якія я казаў раней. Калі гаварыць у цэлым, то для чалавека важна, каб mTORС утрымліваўся на нізкім узроўні з кароткімі часавымі адрэзкамі яго актывацыі. Таму чалавеку важна трымаць здаровы балянс паміж двума станамі: рост / непажаданае сьмецьце і адпачынак / ачыстка.

Старэньне і дзяленьне клетак – гэта два бакі аднаго медаля. Тое, што стымулюе рост колькасьці клетак, у выпадку завяршэньня росту арганізму можа ўзмацняць працэсы старэньня. Чым мацней мы стымулюем клеткі дарослага арганізму лішкам пажыўных рэчываў, тым хутчэй яны старэюць.

Важна датрымлівацца рэжыму харчаваньня, рабіць дні з харчовым устрыманьнем і «павольныя» дні. Стымуляцыя mTORC ежай таксама важная для абнаўленьня клетак і іх рэгенэрацыі. Таму пастаяннае павольнае mTORC-дэфіцытнае харчаваньне можа прыводзіць да дыстрафічных зьяў. Больш «хуткай» ежы могуць бясьпечна дазволіць сабе тыя, хто «расьце» ці аднаўляецца – дзеці, людзі пасьля хваробы, спартоўцы, але людзям, старэйшым за 30, важна абмежаваць «хуткую» ежу.

\subsection{«Хуткія» прадукты.}
Сярод хуткіх прадуктаў трэба вылучыць наступныя групы.

\subsection{Бялкі.}
Мацней за ўсё стымулююць mTORС амінакіслоты з разгалінаваным ланцугом (лейцын, ізалейцын і валін, у першую чаргу лейцын). Лейцын стымулюе mTORС наўпрост, а таксама павялічвае выдзяленьне інсуліну. Больш за ўсё амінакіслотаў з разгалінаваным ланцугом знаходзіцца ў сыроватачным пратэіне, малочных прадуктах, яйках, мясе. Шмат іх і ў некаторых расьлінных бялках, уключна з пшаніцай і сояй. Падобныя дабаўкі амінакіслотаў з разгалінаваным ланцугом – гэта папулярныя дабаўкі для росту цягліцаў. Падобным дзеяньнем валодае і мэтыянін.

\subsection{Вугляводы.}
Чым вышэйшы ўздым інсуліну, тым мацней стымулюецца mTORС. Высокакрухмалістыя прадукты з высокай глікемічнай нагрузкай даюць больш высокі ўзровень стымуляцыі. Сюды ж адносіцца высокае спажываньне цукру. Зьвярніце ўвагу, што на глікемічны індэкс узьдзейнічаюць такія пераменныя, як тэмпэратура, ступень драбненьня і гатоўля прадукту (гл. разьдзел «Гатоўля»).

\subsection{Агульная колькасьць калёрыяў.}
Чым больш калёрыяў, тым вышэйшы ўзровень актыўнасьці mTORС. Зьвярніце ўвагу, што многія з mTORС-хваробаў могуць узьнікаць і ў чалавека нармальнай масы цела, бо ў некаторых людзей ёсьць вялікая ўстойлівасьць да набору вагі.

\subsection{Рэжым харчаваньня.}
Чым часьцей чалавек есьць, тым большую частку часу стымулюецца mTORС.

\subsection{Камбінацыі.}
Праблема становіцца яшчэ горшай, калі розныя прадукты, якія павялічваюць актыўнасьць mTORС, спалучаюцца разам: вугляводы + мяса, салодкія малочныя стравы і да т. п. Спалучэньне: мяса + мучное выклікае большую актывацыю mTORС, чым сыроватачны бялок. Цяпер сухое малако часта дадаецца ў розныя прадукты для ўзмацненьня іх смаку.

\subsection{«Павольныя» прадукты.}
Да іх адносяцца вугляводы зь нізкім інсулінавым і глікемічным індэксам і нізкай глікемічнай нагрузкай, практычна ўсе тлушчы, некаторыя расьлінныя бялкі.

\subsection{Балянс «хуткія» і «павольныя» прадукты.}
Зьмена рацыёну прыводзіць да зьмены актыўнасьці mTORC. У сучасным харчаваньні ёсьць вялікі лішак «хуткіх» прадуктаў, таму важна абмежаваць іх узровень (нашмат менш малочных прадуктаў, есьці мяса, але ня кожны дзень і да т.п.). Можна павялічыць долю «хуткіх» прадуктаў у тыя дні, калі вам патрабуецца большы ўзровень энэргіі і актыўнасьці. Важна зьменшыць долю «хуткіх» прадуктаў у дні адпачынку і пэрыядычна рабіць разгрузкі, асабліва пры нарастаньні сымптомаў стрэсу. Не камбінуйце розныя «хуткія» прадукты ў адзін прыём ежы (калі ясьцё мяса, то з гароднінай, а ня з пастай, калі ясьцё тварог, то дадавайце зеляніну, а не бутэрброд з сочывам, і да т.п.).

Пры памяншэньні колькасьці mTORC-стымулятараў кшталту амінакіслотаў і цукру ў чалавека зьніжаецца артэрыяльны ціск, раздражняльнасьць, ён пачуваецца больш залагоджаным, сьвядомым і спакойным. Таму людзі на расьліннай дыеце відавочна больш спакойныя і запаволеныя, а вось тыя, хто ўжывае малако, мяса, мучныя вырабы, – залішне актыўныя, раздражняльныя і схільныя да аўтаматызмаў.

\subsection{Знайдзіце свой уласны балянс, ня кідаючыся ў скрайнасьці.}
Пры памяншэньні колькасьці mTORC-стымулятараў кшталту амінакіслотаў і цукру ў чалавека зьніжаецца артэрыяльны ціск, раздражняльнасьць, ён адчувае сябе больш залагоджаным, сьвядомым і спакойным. Людзі на расьліннай дыеце відавочна больш спакойныя і запаволеныя, а вось тыя, хто ўжывае малако, мяса, мучныя вырабы, – залішне актыўныя, з падвышаным ціскам, раздражняльныя і схільныя да аўтаматызмаў. Пэрыядычная адмова ад «хуткай» ежы (ці пост) спрыяе расслабленьню і падвышэньню ўсьвядомленасьці. Пры гэтым на кета-дыеце такога эфэкту не назіраецца, бо пры ёй таксама расьце сымпацыйная актыўнасьць, злучаная зь неабходнасьцю ўзмацніць спаленьне тлушчу.

\section{Як трымацца правіла? Ідэі і парады}

\subsection{Амаладжэньне хваробаў.}
Чым большая стымуляцыя mTORС, пачынаючы з разьвіцьця ва ўлоньні маці, тым хутчэй захворваньні праходзяць пэрыяд утоенага разьвіцьця. Гэта абумоўлівае амаладжэньне хваробаў, што мы цяпер назіраем. Тыя захворваньні, якія мы лічылі тыповымі для аднаго ўзросту, сёньня сустракаюцца ў маладзейшым узросьце.

\subsection{Фізычная актыўнасьць.}
Фізычная актыўнасьць вядзе да актывацыі кіназы АМРК, што зьмяншае актыўнасьць бялку. Быць у руху і займацца спортам – гэта добрая ідэя.

\subsection{Інгібітары mTORС.}
Існуюць пэўныя лекі і злучэньні, якія памяншаюць актыўнасьць mTORС. Да іх адносяць зялёную гарбату, рэсвэратрол, фісэтын, квэрцэтын, геністэін, сілімарын, элагавая кіслата і некаторыя іншыя, якія маюцца ў нашых прадуктах харчаваньня.

\subsection{Адпачынак без стымулятараў.}
Цалкам выключыце «хуткую» ежу, кафэін і смартфон у дні адпачынку. Гэта дазволіць адпачыць ня толькі мозгу, але й вашаму абмену рэчываў.

\subsection{Пазьбягайце прадуктаў mTORС-бомбаў.}
Часта ў розных батончыках, нават для здаровага харчаваньня, сустракаецца камбінацыя сухога малака, крухмалу, цукру і да т. п. Па магчымасьці пазьбягайце такіх прадуктаў, замяняючы іх паўнавартаснай ежай.

\subsection{MTORС-дыеты.}
Дыета з абмежаваньнем перапрацаваных вугляводаў, зьніжэньнем долі вугляводаў, павелічэньнем долі тлушчаў, умераным ужываньнем мяса. То-бок, па сутнасьці, мы гаворым пра высокатлушчавую сярэднебялковую нізкавугляводную дыету.

\subsection{Дзеці растуць.}
Існуе міт, што калі дзеці растуць, то яны могуць бескантрольна ўжываць шмат мяса, малака ці кандытарскіх вырабаў. Насамрэч гэта ня так: такая ежа не заўсёды прывядзе да атлусьценьня, але можа выклікаць амаладжэньне хваробаў і павелічэньне іх рызыкі ўжо ў дарослых людзей. Важна думаць пра здароўе зь дзіцячага ўзросту.
