\chapter{Рэфіды і «салодкія дні»}

Рэфіды і «салодкія дні»~--- гэта пэрыядычнае павелічэньне каляражу з~мэтай псыхалягічнай разгрузкі («салодкія дні») і з~мэтай павышэньня ўзроўню лептыну ды пераадоленьня мэтабалічнай адаптацыі да дыеты, загрузкі цягліцаў глікагенам (рэфід). Пастаяннае працяглае абмежаваньне калёрыяў запускае ў~нашым целе шэраг мэханізмаў, закліканых захоўваць колькасьць тлушчу, таму хуткасьць страты вагі зьніжаецца. Таксама харчовыя абмежаваньні і забароны павялічваюць узровень стрэсу, абмяжоўваюць сацыяльную актыўнасьць, павялічваюць рызыку парушэньняў харчовых паводзін, зрываў, таму іх доўгатэрміновае выкарыстаньне не зусім здаровае рашэньне. Плянавае ўвядзеньне пэрыядычных павелічэньняў каляражу дапаможа захаваць псыхічнае і мэтабалічнае здароўе і, як ні парадаксальна, паскорыць пахуданьне.

Багата хто схільны да крайнасьцяў і забаронаў, але яны не зьяўляюцца карыснымі. Памятаеце гісторыю «ня думай пра белага мядзьведзя»? Гэтак жа і з~харчовымі забаронамі: забараняючы, мы толькі ўзмацняем значнасьць гэтых прадуктаў, яны здаюцца нам яшчэ больш прывабнымі і жаданымі. Забарона як бумеранг~--- чым мацней забароніш, тым мацней ударыць. Факусуйцеся на парадкаваньні дыеты, адводзячы ў~ёй месца любой ежы, якая здаецца вам асабліва апэтытнай.

\subsection{Як зьявілася праблема?}

Пры кантролі свайго харчаваньня чалавек сутыкаецца з~шэрагам пабочных эфэктаў. Яны спалучаныя зь неабходнасьцю стварэньня дэфіцыту калёрыяў цягам доўгага пэрыяду часу, адмовай ад пэўных прадуктаў і зьменай ладу жыцьця. Так як новае харчаваньне ня стала яшчэ звычкай, то для яго падтрыманьня выдаткуецца вялікая колькасьць псыхічных рэсурсаў, і гэта патрабуе валявога намаганьня. Праблема яшчэ ўскладняецца тым, што зьмяненьне складу прадуктаў прыводзіць да таго, што тыя стравы, якія давалі задавальненьне і дафамінавы выкід (салодкія і тлустыя), выключаныя з~рацыёну, і чалавек сутыкаецца з~дэфіцытам задавальненьня.

\paragraph{Мэтабалічная адаптацыя да дыеты.}
Пры працяглым абмежаваньні калярыйнасьці ўзровень лептыну паступова зьніжаецца. Зьніжэньне лептыну зьяўляецца сыгналам да пераходу ў~рэжым эканоміі калёрыяў, што прыводзіць да нэгатыўнага ўплыву на ўзровень палавых гармонаў і гармонаў шчытавіцы, падзеньня энэргічнасьці, настрою і запаволеньня тлушчаспаленьня.

\paragraph{Вычарпаньне глікагенавых дэпо.}
Зьніжэньне колькасьці цяглічнага глікагену (запас глюкозы ў~цягліцах) пры трэніроўцы зьяўляецца эфэктыўным спосабам стымуляваць спаленьне тлушчу. Аднак моцнае яго зьніжэньне ў~спалучэньні з~абмежаваньнем вугляводаў можа прывесьці да пагаршэньня самаадчуваньня і павышэньня ўзроўню картызолу. У~такім выпадку рэфід дапаможа аднавіць энэргетычныя запасы ў~цягліцах (гэта больш актуальна для тых, хто займаецца сілавымі відамі спорту). Папярэднічаць рэфіду павінна інтэнсіўная трэніроўка на ўсё цела.

\paragraph{Хуткасьць мэтабалізму ў~спакоі.}
Посты і абмежаваньне калярыйнасьці на працягу доўгага часу (больш за 20 дзён) прыкметна зьніжаюць хуткасьць мэтабалізму. Пэрыядычныя рэфіды дапамагаюць часткова прадухіліць яго зьніжэньне і павялічыць эфэктыўнасьць дыеты.

\paragraph{Небясьпека забаронаў.}
Працяглая забарона ўжываньня пэўнай ежы стварае рызыку парушэньняў харчовых паводзінаў. «Забаронены плод салодкі», і забарона ўзмацняе фіксацыю на ежы, выклікае навязьлівыя думкі, павышае ўзровень стрэсу. Таму правільней будзе дзейнічаць, зыходзячы з~пазыцыі кароткатэрміновых абмежаваньняў, а~не забаронаў, успрымаючы абмежаваньне як структураваньне, парадкаваньне свайго харчаваньня, а~не татальную забарону. Жорсткія забароны павялічваюць рызыку стрэсу і зрываў, якія могуць зьвесьці на нішто ўсе нашы намаганьні нармалізацыі рэжыму.

\tipbox{Забарона ўзмацняе фіксацыю на ежы, выклікае дакучлівыя думкі, павялічвае ўзровень стрэсу. Таму правільней будзе дзейнічаць, зыходзячы з~пазыцыі кароткатэрміновых абмежаваньняў, а~не забаронаў, успрымаючы абмежаваньне як структураваньне, упарадкаваньне свайго харчаваньня, а~не татальную забарону.}

\paragraph{Сацыяльная ізаляцыя.}
Пастаяннае жорсткае датрыманьне пэўнай дыеты часта можа ствараць пэўныя сацыяльныя нязручнасьці, калі вы ў~нейкі час устрымліваецеся ад ежы на сяброўскіх сустрэчах, афіцыйных мерапрыемствах і да т.~п. Гэта можа прывесьці да непаразуменьня і адчужэньня. Мая парада~--- сацыяльная актыўнасьць, у~тым ліку зьвязаная зь ежай, павінна захоўвацца, няхай і ў~кампрамісным выглядзе.

\subsection{Як гэта ўплывае на здароўе?}

\paragraph{Павышэньне эфэктыўнасьці пахудзеньня.}
Аптымальны ўзровень гармонаў шчытавіцы і ўзровень мэтабалізму ў~спакоі ды шэраг іншых паказчыкаў важныя для падтрыманьня прагрэсу. Рэфіды дапамагаюць падтрымліваць посьпех пахудзеньня. Лептын напрамую стымулюе актыўнасьць сымпатаадреналавай сыстэмы, узмацняючы спаленьне тлушчу. З~усіх нутрыентаў мацней за ўсё ўзровень лептыну паднімаюць вугляводы, таму менавіта яны часьцей выкарыстоўваюцца ў~рэфідах. Паводле дасьледаваньняў, рэфідная група пахудзеньня пасьля завяршэньня праграмы практычна ня мела эфэкту адскоку (рыкашэту) вагі~--- і гэта вельмі важна для захаваньня дасягнутых посьпехаў.

Асіметрыя лептынавага цыклу заключаецца ў~наступным. Калі вы ясьцё мала некалькі дзён, то ўзровень лептыну падае і спаленьне тлушчу запавольваецца. Але калі вы зьядаеце больш вугляводаў у~1--2 прыёмы ежы, узровень лептыну ўзрастае і тлушчаспаленьне працягваецца. Такім чынам, узровень лептыну ўзрастае хутка, а~зьніжаецца павольна. Таму кароткі абмежаваны рэфід на тле дыеты дапамагае падтрымаць дастатковы ўзровень лептыну.

\paragraph{Памяншэньне псыхалягічнага стрэсу.}
Рэфіды і «салодкія дні» дапамагаюць зьменшыць стрэс і зьнізіць узровень картызолу, захаваць сацыяльную актыўнасьць.

\paragraph{Ежа як узнагарода.}
Ежа, асабліва смачная, выклікае выкід нэўрамэдыятара дафаміну, што спрыяе фармаваньню ўмоўнага рэфлексу з~папярэднімі паводзінамі. Менавіта з~гэтай прычыны маленькімі кавалачкамі ежы дрэсіроўнік можа прымусіць жывёлу рабіць розныя фокусы бяз гвалту. Наш мозг таксама лёгка стварае прычынна-выніковыя сувязі, таму задумайцеся, што адбываецца, калі вы ясьце салодкаё, будучы ў~стрэсе. Сувязь наступная: «Я няўдачнік, таму вось мне ўзнагарода~--- ежа. Каб наступным разам атрымаць смачную ежу, трэба быць у~стрэсе». Гэта вельмі небясьпечная і нездаровая стратэгія, таму я ня раю есьці «сьвяточную» ежу ў~стрэсе. Аднак з~пункту гледжаньня паводзінаў адзначыць свае посьпехі ў~коле блізкіх сяброў смачнай ежай, узнагародзіць сябе за дасягненьні~--- гэта цалкам працоўная стратэгія, закліканая ў~будучыні ўзмацніць вашую матывацыю на далейшыя перамогі.

\subsection{Асноўныя прынцыпы}

Ёсьць шмат назоваў для розных схэмаў рэфідаў. Гэта і дыета для пахудзеньня з~павышэньнем калёрыяў (CSD), калі вы 11 дзён на дыеце, а~затым ідуць 3 дні рэфіду. Таксама маюцца й~іншыя схэмы, накшталт 3+1, калі на тры дні дыеты прыпадае адзін зь вялікай колькасьцю калёрыяў.

\paragraph{Рэфід} (ад анг. re-feed, feed~--- харчаваньне, ежа)~--- гэта мэтавае павышэньне калярыйнасьці і вугляводаў (вугляводная загрузка). Рэфід карысны для тых, хто знаходзіцца на выразнай дыеце, і для тых, хто займаецца спортам. Рэфід аптымальна рабіць вугляводнымі прадуктамі, зь невялікай колькасьцю бялку, але бяз тлушчу і фруктозы (батат, бульба, рыс і да т.~п.). Колькасьць дадатковых калёрыяў таксама мусіць быць умераная, каб не перакрэсьліць усе вашыя вынікі.

Для спартоўцаў рэфіды шчыльна зьвязаныя з~графікам трэніровак (EOD refeeds), чаргаваньне дзён з~розным харчаваньнем, калі чаргуюцца фазы абмежаваньня калярыйнасьці (мала калёрыяў і вугляводаў + больш тлушчаў для спальваньня тлушчу) і фаза нарошчваньня цягліц (шмат калёрыяў і вугляводаў + мала тлушчаў для нарошчваньня цягліцаў).

\tipbox{Ежа, асабліва смачная, выклікае выкід дафаміна, што спрыяе фармаваньню ўмоўнага рэфлексу з~папярэднімі паводзінамі. Менавіта з~гэтай прычыны маленькімі кавалачкамі ежы дрэсіроўнік можа прымусіць жывёлу рабіць розныя фокусы бяз гвалту.}

\paragraph{Зрабіце салодкі дзень, або чытміл.}
Cheat meal перакладаецца як яда-падман, яда-парушэньне. Гэта адзін прыём ежы раз на тыдзень, калі вы ясьцё ўсё, што захочаце. Зьвярніце ўвагу, што «салодкі дзень» пераносіць нельга, ён павінен быць рэгулярным. «Салодкі дзень» зьніжае ціск забаронаў, дае псыхалягічную разгрузку, дапамагае захаваць сацыяльную актыўнасьць, зьніжае рызыку зрываў. Пры гэтым неабходна датрымлівацца агульнага рэжыму харчаваньня ў~свой «салодкі дзень». Калі ў~вас моцная залежнасьць ад ежы, то вы можаце спачатку рабіць «дыету празь дзень», чаргуючы прадукты, а~затым паступова кожны тыдзень-два зьніжаць колькасьць «салодкіх дзён».

\paragraph{Частасьць і працягласьць.}
Адзін дзень, яшчэ лепей~--- адзін прыём ежы. Для рэфіду можна зрабіць 1--2 прыёмы ежы, а~вось у~«салодкі дзень» (чытміл) лепш есьці ўсё, што вы любіце, але ў~межках аднаго прыёму ежы. Людзі з~высокім адсоткам тлушчу могуць рабіць рэфід не часьцей за адзін раз на два тыдні, з~сярэднім~--- раз на тыдзень, а~зь нізкім узроўнем (ды інтэнсіўнымі трэніроўкамі)~--- два рэфіды на тыдзень.

\subsection{Як трымацца правіла? Ідэі і парады}

\paragraph{Дух свабоды.}
Памятайце, што ежа~--- цудоўная, гэта крыніца энэргіі, і вы заўсёды можаце зьесьці ўсё, што захочаце. «Усё можна, але ня вам і ня сёньня». Рабіце забароны кароткатэрміновымі~--- на тыдзень, на месяц, з~магчымасьцю падаўжэньня. Думкі аб тым, што да канца жыцьця вы ня будзеце есьці цукар, сапраўды не дададуць вам здароўя і ўпэўненасьці.

\paragraph{Ня дома.}
Рабіце «салодкі дзень» ня дома. Наведайце ўлюбёную кавярню, каб прыём такой ежы ўдома не ўваходзіў у~звычку. Старанна і густоўна выбірайце прадукты для «салодкага дня».

\paragraph{Структуруйце ежу.}
Памятайце, што «сьвяточная ежа»~--- толькі на сьвяты (але не падманвайце сябе). Падумайце, што асаблівага вы можаце прыгатаваць да традыцыйных ці сямейных сьвятаў, каб падкрэсьліць іх атмасфэру. Многія стравы традыцыйна выкарыстоўваюць толькі на сьвяты. Ня ежце сьвяточную ежу ў~будні.

\paragraph{Абарона часам.}
Калі цяга моцная, дазвольце сабе гэта зьесьці заўтра раніцай. Калі вы высьпіцеся і сілы дададуцца, магчыма, вы зьменіце сваё рашэньне.

\tipbox{Памятайце, што ежа~--- цудоўная, гэта крыніца энэргіі, і вы заўсёды можаце зьесьці ўсё, што захочаце. Рабіце забароны кароткатэрміновымі~--- на тыдзень, на месяц, з~магчымасьцю працягу. Думкі аб тым, што да канца жыцьця вы ня будзеце есьці цукар, сапраўды не дададуць вам здароўя і ўпэўненасьці.}

\paragraph{Абарона адлегласьцю.}
Чым далей вы ад ежы, тым лягчэй трымацца. Таму ня майце гатовай ежы дома: так меней рызыкі яе зьесьці.

\paragraph{Ня ежце «сьвяточную» ежу падчас стрэсу.}
Ніколі ня ладзьце «салодкія дні» ў~стане стрэсу. Спачатку антыстрэсавыя працэдуры і рэлаксацыя, заспакаеньне~--- і толькі тады яда.

\needspace{3\baselineskip}
\paragraph{Калі можна прызначыць «салодкі дзень».}
\begin{enumerate}[itemindent=3em,labelwidth=1.5em,leftmargin=0pt,nosep]
  \item У~вас дастаткова часу паесьці і атрымаць асалоду ад ежы.
  \item Няма фізыялягічных і псыхалягічных прыкметаў стрэсу.
  \item Вы здаровыя.
  \item Усьвядомленасьць і фокус на працэсе: вы знаходзіцеся «тут і цяпер», факусуецеся на працэсе «як я буду есьці».
  \item Рытуалы і правілы «салодкага дня»: у~вас ёсьць фіксаваны рытуал спажываньня: «салодкі дзень», пэўнае месца, іншыя ўмовы.
  \item Узнагарода: вы дамагліся посьпеху ў~складанай справе, якую доўга адкладалі. Абгрунтаваная ўзнагарода замацоўвае атрыманы посьпех на ўзроўні рэфлексу.
  \item Балянс «жаданьне» і «асалода»: вы атрымліваеце шмат задавальненьня ў~працэсе, ня менш чым ад самога чаканьня і прадчуваньня.
  \item Добры, спакойны посмак, расслабленьне.
\end{enumerate}

\paragraph{Калі нельга прызначаць «салодкі дзень».}
\begin{enumerate}[itemindent=3em,labelwidth=1.5em,leftmargin=0pt,nosep]
  \item Вы ясьцё ў~сьпеху, на хаду, час абмежаваны.
  \item Адчуваеце стрэс, злосьць і нэгатыўныя эмоцыі, цягліцавую напругу, падвышаны ціск, пачашчанае дыханьне.
  \item Вы хварэеце.
  \item Усьвядомленасьць і фокус на працэсе: вы занураныя ў~праблемы мінулага ці будучыя справы, сфакусаваныя на мэце «трэба зьесьці, і мне стане лепш».
  \item Рытуалы і правілы «салодкага дня»: вы ня ведаеце ці не выконваеце.
  \item Узнагарода: вы нічога не дамагліся або вашыя дасягненьні звычайныя. Пустая ўзнагарода толькі зьніжае ўзровень матывацыі.
  \item Балянс «жаданьне» і «асалода»: задавальненьне ад чаканьня мацнейшае, чым ад працэсу насалоджваньня. Так-так, у~гэтым выпадку нельга!
  \item Нядобры посмак, шкадаваньне, трывога.
\end{enumerate}