\chapter{Балянс натрый~--- калій}

Важным правілам харчаваньня зьяўляецца захаваньне правільных суадносінаў натрыю \ce{Na+} і калію \ce{K+} у~прадуктах харчаваньня. Яно вызначае вельмі многія паказьнікі: ад агульнага тонусу да здароўя касьцей і, што важней за ўсё, артэрыяльнага ціску.

Праблему солевага дысбалянсу вырашыць дастаткова лёгка: для гэтага нам трэба паменшыць ужываньне натрыю (перастаўшы саліць і зьнізіўшы долю гатовых прадуктаў харчаваньня) і павялічыць паступленьне калію (есьці больш цэльнай расьліннай ежы). Гэта дазволіць вам ня толькі зьменшыць рызыку артэрыяльнай гіпэртэнзіі, але й~лепей пачувацца і выглядаць.

\section{Як зьявілася праблема?}

У старажытныя часы наша асяродзьдзе было пазбаўленае крыніцаў солі, за выключэньнем узьбярэжжа, атрымаць натрый нашы продкі маглі толькі праз жывёльную ежу. Нашаму арганізму патрабуецца натрый, таму ва ўмовах яго дэфіцыту ў~нас сфармаваліся магутныя сыстэмы яго ўтрыманьня. Цікава, што сыстэмы ўтрыманьня натрыю (праз гармон альдастэрон) актывуюцца ня толькі пры яго страце, але і пры стрэсе. Бо стрэс~--- гэта і страта крыві, і інтэнсіўныя фізычныя нагрузкі, у~тым і іншым выпадку  гэта страта натрыю і вады. Таму стрэс можа запусьціць мэханізмы ўтрыманьня натрыю і вады ў~арганізьме. Удалечыні ад мора паляўнічыя-зьбіральнікі маглі абыходзіцца бяз солі, але пераход да земляробства зь пераважна вугляводнай ежай патрабуе яе наяўнасьці (саляны голад), так яна стала дарагім таварам у~мінулым.

А вось калій багата ўтрымліваецца ў~расьліннай ежы~--- ад клубняў да садавіны і ягадаў. Паляўнічыя-зьбіральнікі елі расьлінную ежу штодня, таму ў~нас у~арганізьме няма мэханізмаў назапашваньня і ўтрыманьня калію! Мы ніколі не адчувалі патрэбы ўтрымліваць у~арганізьме калій, больш за тое, пры стрэсе арганізм нават актыўна пазбаўляецца ад калію.

\tipbox{Нашаму арганізму патрабуецца натрый, таму ва ўмовах яго дэфіцыту сфармаваліся магутныя сыстэмы яго ўтрыманьня. Цікава, што сыстэмы ўтрыманьня натрыю (праз гармон альдастэрон) актывуюцца ня толькі пры яго страце, але і пры стрэсе. Бо стрэс~--- гэта і страта крыві, і інтэнсіўныя фізычныя нагрузкі, у~тым і іншым выпадку гэта страта натрыю і вады.}

Сучасная праблема балянсу натрый~--- калій узьнікла, бо соль стала шырока даступнай і дадаецца ва ўсе прадукты (схаваная соль), формула «цукар, тлушч, соль»~--- гэта класічная трыяда фастфуду. Чым больш мы ямо апрацаваных прадуктаў, тым больш ямо натрыю. А~вось зьніжэньне спажываньня цэльных расьлінных прадуктаў прывяло да таго, што мы спажываем нашмат менш калію, чым трэба. Цяпер сярэдні чалавек мае сутачнае спажываньне натрыю ў~2--3 разы вышэй за норму, а~калію, наадварот, у~2--3 разы менш за норму.

\section{Як гэта ўплывае на здароўе?}

\subsection{Пераяданьне.}
Соль заўважна стымулюе дафамінавую сыстэму мозгу, таму выклікае пераяданьне. Салёных прадуктаў мы зьядаем больш, а~вось на жывёльных мадэлях лішак солі ў~дыеце прыводзіў да ангеданіі (зьніжэньня здольнасьці атрымліваць задавальненьне), а~поўнае выключэньне~--- да дэпрэсіі. Для нашага арганізму соль~--- гэта чыньнік выжываньня, каманда «еж яе колькі ўлезе і запасай на будучыню». У~нашым арганізьме ёсьць спэцыяльныя сыстэмы ўтрыманьня і запасаньня натрыю. Цікава, што бацькі, якія злоўжываюць сольлю, могуць перадаць гэтую цягу дзецям праз эпігенэтычныя мэханізмы.

\subsection{Водны балянс.}
Калій і натрый аказваюць супрацьлеглае дзеяньне на водны балянс: калій валодае мочагнальным эфэктам, а~натрый затрымлівае ваду. Калій патрэбны для вывядзеньня лішкаў натрыю і вады з~арганізму, таму багатая на яго ежа памяншае азызласьць. А~вось лішак натрыю можа прыводзіць да страты калію. Стрэс, які вядзе да затрымкі натрыю, таксама можа ўзмацніць ня толькі артэрыяльны ціск, але і азызласьць. А~спалучэньне «стрэс, соль, недасыпаньне і салодкае» можа нанесьці моцны ўдар па вашай зьнешнасьці.

\subsection{Сардэчна-сасудзістыя захворваньні.}
У тых краінах, дзе ядуць больш расьліннай ежы, меней сардэчна-сасудзістых захворваньняў. Так гіпэртанія (падвышаны ціск) сустракаецца толькі ў~1\,\% насельніцтва, а~вось у~заходніх краінах~--- у~30\,\% насельніцтва. Навукоўцы высьветлілі, што людзі з~дэфіцытам калію і лішкам натрыю маюць удвая большую рызыку сьмерці ад хваробаў сэрца і на 50\,\% павышаную рызыку памерці ад іншых захворваньняў.

\subsection{Рак страўніка.}
Залішняе спажываньне солі зьвязанае з~падвышанай рызыкай раку страўніка (у спалучэньні з~Helicobacter pylori) і з~ракам тоўстага кішачніка.

\subsection{Азызласьць.}
Вельмі частай праблемай зьяўляецца пастознасьць, друзласьць, азызласьць, млявасьць падскурнай клятчаткі ў~цэлым або ў~пэўных частках цела. Адкуль бярэцца друзласьць у~здаровага чалавека бязь першаснай паталёгіі лімфатычнай сыстэмы? Вінаваты залішні натрый, які назапашваецца ў~падскурна-тлушчавай клятчатцы, дзе стымулюе павялічаны рост (гіпэрплазію) лімфатычных сасудаў з~затрымкай вадкасьці. Сетка лімфатычных сасудаў расьце, а~іх тонус падае, і гэта ўсё прыводзіць да затрымкі вадкасьці. Памятайце, што ўвесь лішак натрыю і зьвязанай вады захоўваецца менавіта ў~падскурна-тлушчавай клятчатцы!

\tipbox{Залішні натрый назапашваецца ў~падскурна-тлушчавай клятчатцы, дзе стымулюе павялічаны рост (гіпэрплазію) лімфатычных сасудаў з~затрымкай вадкасьці. Сетка лімфатычных сасудаў расьце, а~іх тонус падае, і гэта ўсё прыводзіць да затрымкі вадкасьці.}

\subsection{Імунітэт.}
Натрый стымулюе імунітэт, што павялічвае рызыку аўтаімунных захворваньняў. Больш высокая канцэнтрацыя натрыю зьяўляецца актыватарам імуннай сыстэмы, соль стымулюе актыўнасьць імуннай сыстэмы на ўсіх узроўнях~--- ад дыфэрэнцыяваньня Т-лімфацытаў да актыўнасьці макрафагаў. А~вось дэфіцыт натрыю можа аслабляць імунную сыстэму. Навукова даведзена, што лішак солі павялічвае рызыку захварэць на расьсеяны склероз. Павелічэньне колькасьці алергій зьвязанае з~тым, што высокасолевая дыета парушае імунны адказ.

\subsection{Соль і мэтабалізм.}
Лішак натрыю зьніжае адчувальнасьць да інсуліну і павялічвае рызыку атлусьценьня. Нават зусім здаровыя маладыя людзі, якія ўжываюць вялікую колькасьць натрыю, могуць выяўляць пагаршэньне свайго здароўя. А~людзей з~атлусьценьнем і тых, хто ўжывае вялікую колькасьць натрыю, гэта вядзе да паскарэньня працэсу старэньня.

\subsection{Іншыя праблемы.}
Лішак натрыю павялічвае рызыку захворваньняў нырак, астэапарозу, катаракты, пагаршае кішачную мікрафлёру. Дастатковая колькасьць калію запавольвае і перадухіляе ўтварэньне камянёў у~нырках і жоўцевым пухіры.

\section{Асноўныя прынцыпы}

Неабходна захоўваць правільныя суадносіны натрыю і калію ў~прадуктах харчаваньня (1:3--1:4). Балянс натрыю і калію нашмат важнейшы, чым захаваньне канкрэтнай колькасьці. Зьвярніце ўвагу, што наш арганізм недахоп натрыю пераносіць лягчэй, чым яго лішак (бо ёсьць сыстэмы захоўваньня і ўтрыманьня), а~вось недахоп калію~--- нашмат горш, чым яго лішак. Таму ня страшна часам зьесьці калію крыху больш, а~натрыю~--- крыху менш.

\tipbox{Звычайная соль, якой мы падсольваем ежу,~--- гэта ўсяго толькі вяршыня айсбэрга, 10–20\,\% ад агульнай колькасьці спажыванага намі натрыю. Вялікая яго частка зьмяшчаецца ў~гатовых прадуктах харчаваньня, якія і зусім могуць быць несалёнымі. Да прыкладу, 40--50\,\% натрыю паступае з~хлеба і выпечкі.}

\subsection{Паменшыце колькасьць натрыю.}
Для гэтага спачатку выявіце асноўныя крыніцы натрыю ў~вашым рацыёне. Як ні парадаксальна, звычайная соль, якой мы падсольваем ежу,~--- гэта ўсяго толькі вяршыня айсбэрга, 10–20\,\% ад агульнай колькасьці натрыю. Вялікая яго частка зьмяшчаецца ў~гатовых прадуктах харчаваньня, якія і зусім могуць быць несалёнымі. Так, 40--50\,\% натрыю паступае з~хлеба і выпечкі. Вельмі шмат солі ў~сырах, каўбасе (да 3 грамаў на 100 грамаў каўбасы), вэнджаніне і да т.~п. Зьвярніце ўвагу, што значная колькасьць натрыю знаходзіцца ў~выглядзе шматлікіх харчовых дабавак: натрыю сорбат, натрыю бікарбанат, натрыю глутамат, натрыю бэнзаат. Гэта значыць, чым больш у~вашым рацыёне гатовай ежы, тым больш вы ясьцё натрыю. У~цэлым пажадана спажываць ня больш за 1500 мг натрыю ў~дзень.

\subsection{Не саліце ежу.}
Адмоўцеся ад пастаяннага дадаваньня солі, бо калі вы хоць зрэдку ясьцё мяса, рыбу, яйкі і гатовую ежу, то вам солі хапае.

\subsection{Павялічце колькасьць калію.}
Цэльная расьлінная ежа~--- гэта ўнівэрсальная крыніца калію. Калій у~садавіне і гародніне выдатна засвойваецца, так як спалучэньне глюкозы зь мінэраламі дапамагае лепшаму засваеньню калію клеткамі. Зьвярніце ўвагу, што содневая патрэба ў~каліі ў~сярэднім 4700мг на дзень, яна заўважна павялічваецца пры фізычных нагрузках, цяжарнасьці і пры харчаваньні зь вялікай колькасьцю солі. Пры недахопе калію назіраюцца слабасьць, стамляльнасьць, паскараецца ўтварэньне камянёў, парушаецца абмен рэчываў, затрымліваецца вадкасьць у~арганізьме. Вырашыць пытаньне з~каліем вельмі проста, бо цэльная гародніна і садавіна ўтрымліваюць мала натрыю і шмат калію. Выбірайце, што вам даспадобы: фасоля, сачавіца, капуста, бульба, банан, авакада, буракі, памідор, салата, рэпа, радыска, пятрушка і шмат іншага.

\section{Як трымацца правіла? Ідэі і парады}

\subsection{Паступовасьць.}
Бяз солі спачатку ўсё здасца нясмачным, але гэта нармальна. За месяц адчувальнасьць рэцэптараў мяняецца, і вы зможаце аднавіць смакавую адчувальнасьць і насалоджвацца ад смакамі і адценьнямі. Лішак солі зьніжае смакавую адчувальнасьць.

\subsection{Зрэдку можна салёнае.}
У тым, каб зрэдку зьесьці нешта салёнае, небясьпекі няма. Рэдкае ўжываньне нават вельмі салёных страў не нашкодзіць, арганізм эфэктыўна дасьць рады з~гэтым. Дадавайце больш гародніны да салёных страў.

\subsection{Перапрацаваная расьлінная ежа.}
Прадукты, вырабленыя з~расьлінаў, тым ня менш часьцей за ўсё ўтрымліваюць лішак натрыю. Так, падсолены таматавы сок зьмяшчае лішак натрыю, а~таксама шмат солі і ў~хлебе.

\subsection{Недахоп солі.}
Занадта нізкае ўтрыманьне натрыю небясьпечнае і шкоднае. Але дэфіцыт натрыю ў~людзей, якія спажываюць дастаткова жывёльнай ежы і хоць зрэдку нешта з~гатовай ежы, малаімаверны. У~групе рызыкі могуць быць вэганы (якія не ўжываюць соль і гатовыя прадукты), людзі, якія жывуць у~гарачым клімаце, занятыя цяжкай фізычнай працай, пасьля дыярэі і да т.~п.

\subsection{Салодкае (і крухмалістае) і азызласьць.}
Інсулін дзейнічае на ныркі, стымулюючы затрымку натрыю і вадкасьці, выдзяленьне альдастэрону. Чым вышэйшая глікемічная нагрузка і чым меншая ваша адчувальнасьць да інсуліну, тым больш вы азызлыя і горай выглядаеце.

\subsection{Стрэс.}
Стрэс узмацняе цягу да салёнага і сам правакуе затрымку вадкасьці і натрыю. Пры стрэсе пазьбягайце салёнага і салодкага, сьпіце дастатковую колькасьць часу. Рэляксацыя палепшыць ваш стан і зьнізіць азызласьць. Пры хранічным стрэсе і паслабленьні функцыі наднырачнікаў можа ўзьнікаць дэфіцыт натрыю, бо арганізму яго складаней утрымліваць.

\subsection{Стомленасьць.}
Дэфіцыт калію і магнію часта праяўляецца ў~выглядзе пастаяннай стомленасьці. Папаўненьне іх узроўняў дапамагае палепшыць тонус цягліцаў і трываласьць.
