\chapter{Бялкі}

Бялкі~--- гэта найважнейшая частка рацыёну, амінакіслоты, якія ўваходзяць у~іх склад, зьяўляюцца істотным будаўнічым матэрыялам, а~таксама крыніцай энэргіі. Важна падтрымліваць аптымальную колькасьць якаснага бялку, каб кантраляваць насычэньне і пры гэтым не падвышаць рызыкі шэрагу захворваньняў. У~дачыненьні да бялкоў многія дыетолягі займаюць процілеглыя пазыцыі, ад пастуляваньня поўнай шкоды жывёльных бялкоў (вэган) да прызнаньня іх абсалютнай карысьці ў~любых колькасьцях (карнівор\index{карнівор-дыета}). Акрамя колькасьці бялку, вельмі важным зьяўляецца яго якасьць, што зьвяза са зьменамі ў~яго апрацоўцы і захоўваньні. Давайце разьбяромся, як нам навучыцца атрымліваць ад бялковых прадуктаў максымум карысьці і пазьбягаць іх неспрыяльных уздзеяньняў на наша здароўе.

\subsection{Як зьявілася праблема?}

У працэсе эвалюцыі нашыя продкі ў~свой час з~расьліннай ежы перайшлі на ўсяеднасьць і павялічылі долю мясной ежы. Гэта дазволіла павялічыць таксама колькасьць больш даступных калёрыяў і паскорыць эвалюцыю. Акрамя таго, бялковая ежа ўтрымлівала шмат важных для разьвіцьця мозгу рэчываў. Першасныя міграцыі людзей адбываліся ўздоўж берагоў акіянаў і мораў, багатых даступнай бялковай ежай ад малюскаў да рыбы. Ёд, амэга-3 тлустыя кіслоты ды іншыя рэчывы з~марской бялковай ежы аказваюць стымулюючае ўздзеяньне на разьвіцьцё мозгу, некаторыя палеаантраполягі бачаць у~гэтым ключ да эвалюцыі мозгу. Расьсяляючыся па зямным шары, нашы продкі пакідалі к'ёкенмэдынгі~--- агромністыя шматмэтровыя горы з~ракавінак зьедзеных малюскаў.

Паляўнічыя-зьбіральнікі елі ня толькі мяса жывёлаў, але й~косткі, храсткі, іншыя субпрадукты. І~гэта правільнае рашэньне, бо ў~печані, напрыклад, утрымліваецца больш вітамінаў і мінэралаў, чым у~чырвоным мясе. Аднак пераход да земляробства прывёў да скарачэньня колькасьці жывёльнай ежы, зьбядненьня рацыёну, павелічэньня спажываньня крупаў, што адбілася на здароўі і вонкавым выглядзе нашых продкаў і нават прывяло да памяншэньня сярэдняга росту.

У шматлікіх традыцыйных культурах спажываньне бялковых, у~першую чаргу мясных, прадуктаў было збалянсаваным. Ад мяса практычна нідзе не адмаўляліся (нават у~будызьме), але ў~цэлым яно абмяжоўвалася. Так, у~хрысьціянстве ў~дні даволі шматлікіх постаў мяса не было. Такі рэжым ужываньня мяса імітуе рэжым харчаваньня паляўнічых-зьбіральнікаў: шмат мяса падчас пасьпяховага паляваньня і шмат расьліннай ежы падчас зьбіральніцтва (няўдалае паляваньне).

\tipbox{Ёд, амэга-3 тлустыя кіслоты ды іншыя рэчывы, якія ўтрымліваюцца ў~марской ежы, аказваюць стымулюючае ўздзеяньне на разьвіццё мозгу, некаторыя палеаантраполягі бачаць у~гэтым ключ да эвалюцыі мозгу.}

У сучасным сьвеце колькасьць мяса ў~рацыёне павялічваецца, пры гэтым часта яно ўжываецца ў~нездаровым харчовым кантэксьце: не зь зелянінай і гароднінай, а~з салодкім і мучным. Бялковыя прадукты прысутнічаюць практычна ў~кожным прыёме ежы. У~структуры бялковых прадуктаў таксама назіраецца перакос у~бок перапрацаванага чырвонага мяса і малочных прадуктаў з~памяншэньнем долі расьліннага бялку, яек і рыбы. Такое высокае спажываньне бялку павялічвае рызыкі шматлікіх «захворваньняў цывілізацыі» (атлусьценьня, дэпрэсіі, аўтаімунных захворваньняў, раку і да т.~п.).

\subsection{Як гэта ўплывае на здароўе?}

\paragraph{Голад і сытасьць.}
Бялкі добра засвойваюцца, сытасьць захоўваецца надоўга, чым і тлумачыцца наступнае зьніжэньне колькасьці спажываных калёрыяў. Высокабялковыя дыеты спрыяюць пахудзеньню, пры гэтым дапамагаючы лепш кантраляваць апэтыт, чым нізкабялковыя ці нізкатлушчавыя дыеты. Тым ня менш уплыў доўгатэрміновых дыетаў розных відаў на вагу не адрозьніваецца.

\paragraph{Агульны тонус.}
Вялікая колькасьць бялку ў~харчаваньні аказвае стымулюючае ўзьдзеяньне, узмацняе сымпацыйны тонус, павялічвае энэргічнасьць і настрой.

\paragraph{Мэтабалізм.}
Надмер шэрагу амінакіслотаў, у~прыватнасьці амінакіслотаў з~разгалінаваным ланцугом (BCAA)\index{амінакіслота з~разгалінаваным ланцугом (BCAA)}, пагаршае адчувальнасьць да інсуліну. Дыеты з~больш высокім утрыманьнем бялку хоць раўназначныя ў~лічбах скінутай вагі, але пры гэтым паказваюць горшую адчувальнасьць да інсуліну ў~параўнаньні з~дыетамі зь нізкім утрыманьнем бялку. Пры гэтым высокае ўтрыманьне бялку можа зьмяншаць узровень трыгліцэрыдаў і колькасьць тлушчу ў~печані.

\tipbox{Бялкі добра засвойваюцца, сытасьць захоўваецца надоўга, чым і тлумачыцца наступнае зьніжэньне колькасьці спажываных калёрыяў. Высокабялковыя дыеты спрыяюць пахудзеньню, пры гэтым дапамагаючы лепш кантраляваць апэтыт.}

\paragraph{Стымулятары mTORС\index{ген!mTORC}.}
ВСАА амінакіслоты\index{амінакіслота з~разгалінаваным ланцугом (BCAA)} (лейцын\index{лейцын}, ізалейцын\index{ізалейцын} і валін\index{валін}) і метыянін\index{метыянін} стымулююць актыўнасьць mTORС\index{ген!mTORC}. Само па сабе гэта карысна пэрыядычна, але залішняя стымуляцыя прыводзіць да павелічэньня рызыкі шэрагу захворваньняў: ад атлусьценьня і аўтаімунных захворваньняў да заўчаснага старэньня, зьніжэньня хуткасьці аўтафагіі\index{аўтафагія} і рызыкі пухлінавых захворваньняў (гл. разьдзел «Хуткая і павольная ежа»). Ёсьць дадзеныя, што высокае спажываньне BCAA\index{амінакіслота з~разгалінаваным ланцугом (BCAA)} прыводзіць да зьніжэньня працягласьці жыцьця.

Больш за ўсё гэтых амінакіслотаў у~малочных прадуктах, асабліва ў~сыроватачным бялку і малацэ, менш~--- у~расьлінным бялку (за выключэньнем соевага бялку). Часта BCAA\index{амінакіслота з~разгалінаваным ланцугом (BCAA)} неўзаметку прысутнічае ў~выглядзе соевага парашка, сухога малака і г.~д. Важная ўласьцівасьць BCAA\index{амінакіслота з~разгалінаваным ланцугом (BCAA)}~--- магутная стымуляцыя інсуліну і актывацыя mTORС\index{ген!mTORC}, што ўзмацняецца ў~спалучэньні з~высокаглікемічнымі вугляводамі. Залішняя ўвесьчасная актывацыя mTORС\index{ген!mTORC} прыводзіць да залішняй сімпатаадрэналавай актыўнасьці і ўзмацняе стрэс, трывогу, нэўратычнасьць, павялічвае артэрыяльны ціск і пагаршае рэляксацыю і сон.

\paragraph{Рызыка захворваньняў.}
Высокабялковыя дыеты, асабліва на тле высокакалярыйнага харчаваньня, павялічваюць рызыку сардэчна-сасудзістых захворваньняў, нагрузку на ныркі. У~людзей з~утрыманьнем бялку ў~дыеце больш за 20\,\% ад каляражу рызыка захворваньняў вышэйшая, чым у~людзей з~утрыманьнем бялку ў~дыеце 10\,\% і менш. Высокабялковая дыета павялічвае рызыку захворваньня на цукровы дыябэт\index{дыябэт} другога тыпу і рызыку ракавых захворваньняў. Дыета зь нізкай колькасьцю мэтыяніну паляпшае хаду некаторых анкалягічных захворваньняў, зьніжае ўзровень сыстэмнага запаленьня.

\paragraph{Старэньне.}
Лішак бялку ў~ежы, асабліва жывёльнага, скарачае працягласьць жыцьця і паскарае старэньне. Пры гэтым расьлінныя бялкі ня маюць такога ўплыву на захворваньне і працягласьць жыцьця, як жывёльныя (празь меншае ўтрыманьне мэтыяніну і ВСАА). Дыеты зь нізкім утрыманьнем мэтыяніну памяншаюць рызыку разьвіцьця шэрагу захворваньняў і павялічваюць працягласьць жыцьця. Абмежаваньне прадуктаў, багатых лейцынам\index{лейцын} (уваходзіць у~ВСАА), узьдзейнічае амаль эквівалентна нізкакалярыйнаму харчаваньню і падаўжае жыцьцё.

\paragraph{Засваеньне мінэралаў.}
Мінэралы, якія атрымліваюцца з~жывёльных прадуктаў, засвойваюцца нашмат лепш, чым з~расьліннай ежы. Антынутрыенты\index{антынутрыенты} запавольваюць усмоктваньне цынку, жалеза і сэлену\index{сэлен}, таму мяса лепш есьці з~гароднінай, а~ня з~крупамі (збажына, бабовыя). Аднак лішак жалеза (часьцей у~мужчын), назапашваючыся з~узростам, вельмі нэгатыўна ўплывае на здароўе. Жалеза (як і медзь)~--- гэта мэтал зь пераходнай валентнасьцю, таму яго надмер паскарае перакіснае акісьленьне тлушчаў, што павялічвае рызыку шматлікіх захворваньняў~--- ад атлусьценьня і раку да нэўрадэгенэратыўных захворваньняў. Дэфіцыт жалеза (часьцей у~жанчын)~--- таксама сур'ёзная праблема, якая пагаршае самаадчуваньне, вонкавы выгляд і здароўе.

\tipbox{Ёд, цынк, селен, бор і дзясяткі іншых мінэралаў сустракаюцца ў~добрай канцэнтрацыі ў~морапрадуктах. Рыбны бялок заўважна адрозьніваецца ад бялку чырвонага і белага мяса, ён валодае даведзеным антыгіпэртэнзіўным эфэктам, стымулюе фібрыноліз\index{фібрыноліз}, спрыяе зьніжэньню вагі і зьніжае ўзровень запаленчага маркера С-рэактыўнага бялку, паляпшае адчувальнасьць да інсуліну.}

\paragraph{Рыба і морапрадукты.}
У сучасным рацыёне пераважае ўжываньне бялкоў з~малочных прадуктаў і чырвонага мяса. У~сваю чаргу, рыба і морапрадукты ўтрымліваюць мноства неабходных для здароўя злучэньняў: таўрын\index{таўрын}, астаксанцын\index{астаксанцын}, сэлен\index{сэлен}, цынк, вітамін D, ёд, амэга-3 тлустыя кіслоты і г.~д. Таўрын\index{таўрын}~--- гэта важная амінакіслата, якая ахоўна ўзьдзейнічае на шматлікія паказьнікі здароўя і спрыяе даўгалецьцю. Рыбны бялок выгодна адрозьніваецца ад бялку чырвонага і белага мяса, ён валодае даведзеным антыгіпэртэнізіўным эфэктам, стымулюе фібрыноліз\index{фібрыноліз}, спрыяе зьніжэньню вагі і ўзроўню С-рэактыўнага бялку, паляпшае адчувальнасьць да інсуліну. Большасьць эфэктыўных дыетаў утрымліваюць морапрадукты і рыбу ў~сваім складзе.

Дасьледаваньні паказваюць, што морапрадукты перадухіляюць разьвіццё хваробы Альцгаймэра\index{хвароба!Альцгаймэра}, страту слыху, павялічваюць інтэлект, захоўваюць жаночае і мужчынскае здароўе, спрыяюць зачацьцю і да т.~п. Для дасягненьня эфэкту ня трэба есьці, як эскімос, дастаткова ўмеранага даданьня ў~рацыён морапрадуктаў і рыбы. Цікава, што асобныя дадаткі амэга-3 ня могуць прыраўняцца да эфэктыўнасьці цэльнага прадукту для прафіляктыкі хваробы Альцгаймэра\index{хвароба!Альцгаймэра}. Нават адна порцыя рыбы на тыдзень, з'едзеная ў~пажылым узросьце, можа аддаліць наступленьне дэмэнцыі і хваробы Альцгаймэра\index{хвароба!Альцгаймэра}, у~тым ліку генэтычна дэтэрмінаваных.

\subsection{Асноўныя прынцыпы}

\paragraph{Колькасьць.}
Колькасьць спажыванага бялку можна вар'іраваць у~залежнасьці ад узроўню фізычнай актыўнасьці і ўзросту. Менш бялку могуць ужываць людзі ва ўзросьце 40--55 гадоў. Для людзей, старэйшых за 65 гадоў, разумна павялічыць долю бялку ў~рацыёне. Мінімальная рэкамэндаваная колькасьць бялку на содні складае 0,8 г / кг масы цела. Дыета з~высокім утрыманьнем бялку ўключае больш за 1,5 г / кг, умераным 0,8--1,3 г, менш за 0,8~--- нізкім. Болей бялку вы можаце ўжываць пры больш высокім узроўні фізычнай актыўнасьці.

\paragraph{Рэжым.}
Высокабялковая дыета ў~доўгатэрміновай пэрспэктыве нясе патэнцыйны нэгатыўны ўплыў на здароўе. Аднак для падвышэньня тонусу і энэргічнасьці мы можам ладзіць сабе высокабялковыя дні, ураўнаважваючы іх вэганскімі днямі сярод тыдня. Большую колькасьць бялку ў~рацыёне варта збалянсаваць нізкакалярыйнай зялёнай гароднінай і ўмеранай колькасьцю спажываных калёрыяў, высокабялковыя дні можна рабіць падчас інтэнсіўных трэніровак.

\paragraph{Выбар аптымальных крыніц бялку.}
Найлепшымі крыніцамі бялку зьяўляюцца морапрадукты, яйкі, мяса хатняй жывёлы пашавага выпасу. Расьлінны бялок~--- таксама крыніца важных амінакіслотаў, пры гэтым ён пазбаўлены многіх нэгатыўных аспэктаў жывёльных бялкоў. Чырвонае мяса рэкамэндуецца есьці не часьцей за 2--3 разы на тыдзень.

\paragraph{Марскі бялок.}
Найлепшай крыніцай бялку і мноства іншых незаменных злучэньняў зьяўляецца «марская дыета», якая мяркуе ўжываньне рыбы дзікай лоўлі, ракападобных, малюскаў і да т.~п. Аптымальна есьці марскую рыбу 2--3 разы на тыдзень (але ня смажаную, вэнджаную, а~вараную).

\tipbox{Навукова даведзена, што самым шкодным сярод мясных прадуктаў зьяўляецца перапрацаванае мяса (бэкон, сасіскі, каўбасы, паўфабрыкаты і да т.~п.): яно ўваходзіць у~сьпіс канцэрагенаў. Усяго 50 грамаў такога мяса на содніі дастаткова, каб заўважна павысіць рызыку сардэчна-сасудзістых захворваньняў (на 42\,\%) і дыябэту\index{дыябэт} (на 19\,\%).}

\paragraph{Мяса жывёлаў пашавага выпасу, мяса і яйкі хатняй птушкі.}
Пры гэтым ня трэба факусавацца толькі на мясе, важна есьці й~субпрадукты. Белае і чырвонае мяса можна есьці некалькі разоў на тыдзень у~цэльным выглядзе. Што тычыцца расьліннага бялку, ён менш збалянсаваны па амінакіслотах, таму трэба камбінаваць розныя яго віды для атрыманьня поўнай нормы незаменных амінакіслотаў.

\paragraph{Ідэальная якасьць.}
Якасьць~--- гэта найважнейшае патрабаваньне да бялковых прадуктаў, якія хутка псуюцца, вельмі адчувальныя да награваньня, сьвятла, забруджваньня (у іх размнажаюцца бактэрыі), працяглых тэрмінаў захоўваньня, тэрмаапрацоўкі. Выбірайце і купляйце цэльныя фрагмэнты ці тушы мяса, на костцы і са скурай у~правераных пастаўнікоў і гатуйце іх здаровымі спосабамі. Навукова даведзена, што самым шкодным сярод мясных прадуктаў зьяўляецца перапрацаванае мяса (бэкон, сасіскі, каўбасы, паўфабрыкаты і да т.~п.), аднесенае да сьпісу канцэрагенаў. Усяго 50 грамаў такога мяса на содні дастаткова, каб прыкметна павысіць рызыку сардэчна-сасудзістых захворваньняў (на 42\,\%) і дыябэту\index{дыябэт} (на 19\,\%). Пры смажаньні мяса адбываецца назапашваньне ў~ім канчатковых прадуктаў глікацыі (гл. разьдзел «Захоўваньне і гатоўля»), а~таксама зьніжэньне колькасьці важных злучэньняў.

\subsection{Як трымацца правіла? Ідэі і парады}

\paragraph{Кантроль апэтыту.}
Бялок выдатна насычае, таму, калі вы губляеце кантроль над апэтытам, зрабіце некалькі высокабялковых дзён, каб яго аднавіць.

\paragraph{Бялковыя дні.}
У дні высокай фізычнай ці разумовай нагрузкі вы можаце кароткачасова перайсьці на «рэжым паляўнічага», прыкметна павялічыўшы частку бялку ў~рацыёне. Гэта дазволіць вам выкарыстоўваць яго стымулюючае дзеяньне.

\paragraph{У першую чаргу~--- бялок.}
Бялок варта есьці першым падчас сталаваньня~--- так ён лепш насычае.

\paragraph{Інсулінарэзыстэнтнасьць\index{інсулінарэзыстэнтнасьць}.}
Высокабялковая дыета вядзе да зьніжэньня адчувальнасьць да інсуліну, нават пры пахудзеньні. Гэта вядзе да захаваньня падвышанай рызыкі інсулінарэзыстэнтнасьці\index{інсулінарэзыстэнтнасьць}, таму пазьбягайце высокабялковых дыетаў працяглы час.

\paragraph{ВСАА\index{амінакіслота з~разгалінаваным ланцугом (BCAA)}.}
Надмер амінакіслотаў з~разгалінаваным ланцугом у~рацыёне (малочныя прадукты, яйкі, мяса і да т.~п.) або ў~выглядзе спартовых дабавак можа запаволіць цяглічны рост у~атлетаў-пачаткоўцаў (за кошт хутчэйшага аднаўленьня), зьнізіць адчувальнасьць да інсуліну і зрабіць яшчэ шэраг адмоўных узьдзеяньняў. Але мы можам выкарыстоўваць гэтыя дадаткі ў~выпадках экстранагрузак (маратон і г.~д.). Трыптафан зьяўляецца папярэднікам такіх важных малекулаў, як сэратанін\index{сэратанін} і мэлятанін\index{мэлятанін}. Рэч у~тым, што трыптафан слаба пранікае праз гематаэнцэфалічны бар'ер, яго перанос залежыць ад суадносінаў у~крыві трыптафан / ВСАА. Таму пасьля бялковай ежы ўзровень трыптафану вырастае ў~плазьме крыві, але ня ў~мозгу. Калі вы зьядаеце расьлінную ежу, то інсулін зьніжае ўзровень ВСАА (але не трыптафану) у~плазьме крыві, узмацняючы іх паглынаньне цягліцамі. Зьніжэньне ўзроўню ВСАА вядзе да зьмены суадносінаў трыптафан / ВСАА, што ўзмацняе перанос трыптафану ў~мозг і паляпшае самаадчуваньне. Даданьне нават невялікіх колькасьцяў бялку да вугляводнай ежы прыводзіць да павелічэньня ўзроўню ВСАА, што блякуе гэты мэханізм. Дастаткова 4\,\% бялку, каб прадухіліць зьмяненьне суадносінаў трыптафан / ВСАА. Таму ня варта дадаваць да кожнага прыёму ежы багатыя ВСАА\index{амінакіслота з~разгалінаваным ланцугом (BCAA)} прадукты!

\tipbox{Зьнізіць узровень жалеза ў~крыві дапаможа донарства. Здавайце 1--3 разы на год кроў і рэгулярна правярайце фэрытын\index{фэрытын}, асабліва калі ў~вас ёсьць генэтычная схільнасьць да назапашваньня жалеза.}

\paragraph{Бялковыя кактэйлі.}
Бялковыя парашкі~--- гэта «бедныя» калёрыі, бо ў~іх шмат калёрыяў і мала дадатковых нутрыентаў (мінэралы, вітаміны і да т.~п.) у~параўнаньні з~цэльнымі прадуктамі. Магчыма, яны будуць карыснымі для прафэсійных атлетаў, але не для тых, хто не займаецца фізычнымі нагрузкамі.

\paragraph{Вэганства і карнівор\index{карнівор-дыета} (zero carb diet).}
Як поўная адмова ад мяса, так і высокабялковая дыета на адным мясе не зьяўляюцца здаровымі варыянтамі харчаваньня, гэтыя скрайнасьці шкодзяць здароўю.

\paragraph{Лішак і недахоп жалеза.}
Вялікая колькасьць чырвонага мяса (асабліва ў~рацыёне мужчынаў) прыводзіць да назапашваньня залішняга жалеза. Праверыць яго варта, здаўшы аналіз на ўзровень фэрытыну\index{фэрытын}. Калі ён высокі, то лепшы спосаб зьніжэньня~--- стаць донарам. Здавайце 1--3 разы на год кроў, рэгулярна правярайце ўзровень жалеза, асабліва калі ў~вас ёсьць генэтычная схільнасьць да яго назапашваньня.

\paragraph{Мэтыянін і кантэкст дыеты.}
Істотна ўплывае і кантэкст дыеты. Напрыклад, у~чалавека, які есьць шмат мяса і мучнога, мала зеляніны і гародніны, спажываньне мэтыяніну можа прыводзіць да павелічэньня ў~крыві гомацыстэіну, што вельмі шкодна для сасудаў. А~вось калі зь мясам есьці больш гародніны, то мэтыянін будзе лепш засвойвацца з~наяўнасьцю вітамінаў В12, В9, прадуктаў-донараў мэтыльных групаў (халін, бэтаін у~капустах і да т.~п.). Людзям з~мутацыямі фолатнага цыклю (MTHFR, MTR, MTRR) і з~падвышаным гомацыстэінам варта таксама больш уважліва ставіцца да спажываньня чырвонага мяса.

\paragraph{Гатоўля.}
Бялковыя прадукты вельмі адчувальныя да гатоўлі (гл. разьдзел па гатаваньні). Аптымальныя ашчадныя спосабы гатоўлі, марынаваньне мяса кіслым (воцат, цытрына) і спэцыямі, варыць, а~ня смажыць. Гэта скарачае тэрмін гатаваньня і прадухіляе ўтварэньне шкодных рэчываў. Цікава, што вараная рыба павялічвае працягласьць жыцьця і паляпшае настрой (зьніжэньне рызыкі дэпрэсіі), а~вось смажаная~--- не.

\paragraph{Мяса, зеляніна і гародніна.}
Карысныя кампанэнты зеляніны і гародніны нават у~страўніку абясшкоджваюць тыя шкодныя рэчывы, якія ўтворацца пры смажаньні мяса. Таму найлепшы гарнір да мяса~--- гэта зеляніна (базілік, рукала, шпінат), закрасы (часнык, лук), крыжакветныя (кале, брокалі, пэкінская капуста і г.~д.).

\paragraph{Бялок і цяглічны рост.}
Вядома, дастатковы ўзровень бялку важны для цяглічнага росту. Але яго ўплыў часта пераацэньваецца. Так, сілавыя трэніроўкі павышаюць сынтэз бялку на 40\,\%, а~выкарыстаньне бялку дадае толькі 10\,\%, то-бок ключавым момантам зьяўляюцца трэніроўкі, а~не бялок. Залішні бялок не павялічвае інтэнсіўнасьці росту цягліцаў.

\paragraph{Малочныя прадукты.}
Малако~--- гэта не звычайны структурны бялок (як мяса), а~сыгнальны, які актывуе рост маладых арганізмаў. У~ім шмат лейцыну (да 14\,\%), ён мае высокі інсулінавы індэкс і пры ўжываньні ў~вялікай колькасьці можа выклікаць шэраг адмоўных эфэктаў. У~малацэ таксама ёсьць цукар галактоза, які паскарае старэньне, і ляктоза, непераноснасьць якой (як прыроджаная, так і набытая) нэгатыўна ўплывае на работу кішачніка. Залішняе спажываньне малака можа павялічваць рызыку разьвіцьця некаторых відаў раку. Асабліва неспрыяльна спалучэньне малочнага, салодкага і мучнога разам. Таму ўжываньне цэльных малочных прадуктаў варта абмежаваць, зрабіўшы ўмеранае выключэньне для фэрмэнтаваных прадуктаў (цэльныя кефір, ёгурт) або сыроў, але ў~невялікай колькасьці. Так, кефір можна ня проста піць, а~запраўляць ім салату. А~ўжо пра кефір на ноч, як мы цяпер ведаем, і зусім варта забыць!

\paragraph{Соя і яе вытворныя.}
Соевы бялок часта выкарыстоўваецца ў~гатовай ежы і паўфабрыкатах. Ён мае высокае ўтрыманьне мэтыяніну (у параўнаньні зь іншымі бабовымі), высокую эстрагенную актыўнасьць і ня вельмі добра засвойваецца. Іншыя бабовыя больш карысныя, чым соевыя прадукты. Іх есьці можна, але ва ўмеранай колькасьці.

\paragraph{Нізкагістамінавая дыета.}
Вельмі часта прычынай непераноснасьці асобных прадуктаў зьяўляецца не харчовая алергія, а~непераноснасьць гістаміну (фальшывая харчовая алергія). У~гэтым выпадку добра дапамагае абмежаваньне прадуктаў, якія ўтрымліваюць гістамін ды іншыя біягенныя аміны\index{біягенныя аміны}. Паўфабрыкаты зь бялковых прадуктаў, працяглае іх захоўваньне павялічваюць рызыку іх непераноснасьці ў~адчувальных да гэтага людзей. У~нізкагістамінавую дыету ўваходзіць выключэньне мясных кансэрваў, сушанага, вяленага мяса, мяса працяглага захоўваньня, каўбасы, фаршу, мяса бяз даты расфасоўкі і г.~д. Шмат можа быць гістаміну ў~рыбе, асабліва ў~тунцы, сардзіне, скумбрыі, у~рыбных соўсах, сьвежай крамнай рыбе. Выбірайце рыбу глыбокай замарозкі і не размарожвайце яе працягла ў~лядоўні. У~вытрыманых сырах таксама можа зьмяшчацца шмат гістаміну ды іншых амінаў. Пры гэтым сьметанковае масла і маладыя сыры дапушчальныя.
