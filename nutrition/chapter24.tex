Правіла 24. Індывідуальныя асаблівасьці

У кнізе выкладзены найважнейшыя ўнівэрсальныя парады наконт рэжыму харчаваньня і выбару прадуктаў. Але пры гэтым трэба разумець, што канкрэтная фармулёўка харчовага правіла (колькі грамаў вугляводаў мне зьесьці на дзень, колькі гадзінаў і хвілінаў рабіць харчовае вакно і да т.п.) залежыць ад вашых індывідуальных асаблівасьцяў. У цэлым, чым больш вы пра сябе ведаеце, тым дакладней вы можаце фармуляваць сабе мэты. Таму мне хочацца, каб вы ня проста чакалі парады, колькі грамаў тлушчу есьці на содні, а занялі актыўную пазыцыю – дасьледчыка і назіральніка.

Зважайце на сваё здароўе, і ў гэтай сферы жыцьця адбудуцца станоўчыя зьмены. Навучыцеся назіраць за сабой, дасьледаваць, тэставаць і вывучаць сябе, ня толькі суб'ектыўна, але і з дапамогай тэстаў. Так вы зможаце колькасна вымераць уплыў таго ці іншага стылю харчаваньня ды іншых зьменаў.

На франтоне старажытнагрэцкага храма дэльфійскага аракула было высечана: «Спазнай самога сябе». Сапраўды, самавывучэньне, самадасьледаваньне зьяўляюцца важнымі практыкамі ў фармаваньні здароўя і неад'емнай умовай стварэньня насамрэч індывідуальнага падыходу. Калі вы зьвяртаеце ўвагу на пэўную сфэру жыцьця, у ёй непазьбежна адбываюцца станоўчыя зьмены. Навучыцеся назіраць за сабой, дасьледаваць, тэставаць і вывучаць сябе, ня толькі суб'ектыўна, але і з дапамогай тэстаў. Так вы зможаце напраўду заўважыць уплыў таго ці іншага стылю харчаваньня ды іншых зьменаў. Існуюць розныя тэсты, якія могуць выявіць стан вашага вугляводнага і тлушчавага абмену, асаблівасьці засваеньня розных рэчываў, стан мікрафлёры і шматлікае іншае.

Як зьявілася праблема?

Усе людзі вельмі падобныя між сабой, з аднаго боку, але маюць шэраг адрозненьняў, якія ўплываюць на іх рэакцыю на пэўны рэжым харчаваньня і прадукты. Гэтая рэакцыя залежыць ад генэтыкі, эпігенэтыкі, стану мікрафлёры, актыўнасьці імуннай сыстэмы, рэжыму фізычнай актыўнасьці і шмат чаго іншага. Таму для розных людзей розныя дыетычныя парады могуць мець розную ступень эфэктыўнасьці.
Многія з генэтычных асаблівасьцяў сфармаваліся пад уплывам чыньнікаў асяроддзя, у якім жылі нашы продкі, і ўяўляюць сабой адаптацыі. Напрыклад, у большасьці эўрапэйцаў сустракаюцца такія варыянты генаў FADS1 і FADS2, якія ня могуць сынтэзаваць жывёльныя формы амэга-3 тлустых кіслотаў з расьлінных, а ў Азіі «вэгетарыянскіх» формаў такіх генаў больш.
Зразумела, вялікае значэньне мае і актуальны стан здароўя чалавека, дастатковасьць усіх вітамінаў і мінэралаў. Як я ўжо згадваў, іх аптымальны ўзровень паляпшае мэтабалізм, розныя нутрыенты здольныя палепшыць сыстэму дэтаксыкацыі арганізму і тым самым паўплываць на абмен гармонаў унутры арганізму. Прапорцыі розных макранутриентов залежаць ад адчувальнасьці чалавека да інсуліну і да лептыну, узроўню яго фізычнай актыўнасьці. Чым ён вышэйшы, тым, напрыклад, больш бясьпечным будзе ўжываньне вялікіх колькасьцяў вугляводаў. На ступень засваеньня мінэралаў, прыкладам, моцна ўплывае кіслотнасьць страўніка: калі яна паніжаная, тое іх засваеньне будзе слабейшым.

Як гэта ўплывае на здароўе?

Веданьне сваіх індывідуальных асаблівасьцяў дапаможа найболей дакладна падабраць дыетычныя рэкамэндацыі. Напрыклад, рэкамэндацыя есьці больш аліўкавага алею лепей спрацуе для носьбітаў нуклеатыдаў G у rs1801282 (знаходзіцца ў гене PPARG). Такія людзі найбольш атрымаюць карысьці ад міжземнаморскай дыеты. Значнасьць генэтычных асаблівасьцяў таксама вар'іруецца ад умеранай схільнасьці да выяўленай непераноснасьці. Копія А ў БНіП rs4988235 гену LCT дае магчымасьць засвойваць ляктозу. Калі няма хоць адной копіі, то разьвіваецца непераноснасьць. Ёсьць цэлы шэраг генэтычна прыроджаных непераноснасьцяў (фруктозы, трэгалозы, глютэну і г. д.).

Многія з генэтычных асаблівасьцяў сфармаваліся пад уплывам чыньнікаў асяроддзя, у якім жылі нашы продкі, і ўяўляюць сабой адаптацыі. Напрыклад, у большасьці эўрапэйцаў сустракаюцца такія варыянты генаў FADS1 і FADS2, якія ня могуць сынтэзаваць жывёльныя формы амэга-3 тлустых кіслотаў з расьлінных, а ў Азіі «вэгетарыянскіх» формаў такіх генаў больш.

З дапамогай генэтычнага тэсту можна выявіць парушэньне мэтабалізму шэрагу вітамінаў, асабліва важныя парушэньні ў абмене фалатаў. Тэст дапаможа ацаніць дакладны ўплыў кафэіну, бо ёсьць такія асаблівасьці, калі нават невялікая порцыя кавы заўважна ўзмацняе трывогу і турботу. Што тычыцца насычаных тлушчаў, тут можна згадаць мутацыю ў гене FABP2, у гэтым выпадку носьбітам А-генатыпу лепш абмяжоўваць колькасьць насычаных тлушчаў, зьядаючы іх ня больш за 50 грамаў на содні. Што да вітаміну D, то і там ёсьць індывідуальныя асаблівасьці яго рэцэптараў. Пэўныя палімарфізмы гену рэцэптару вітаміну D (Bsm I, Fok I, Taq I і інш.) уплываюць на разьвіццё шматлікіх захворваньняў. Таму тэст дапаможа вам вызначыць аптымальны менавіта для вас узровень вітаміну D. Прадукты харчаваньня і гены ўзаемадзейнічаюць вельмі складана, уплываючы адзін на аднаго. Многія спалучэньні ў прадуктах могуць узмацняць або аслабляць актыўнасьць генаў, а гены, у сваю чаргу, уплываюць на рэакцыю арганізму пры ўжываньні прадуктаў.
Вядома, генэтыка ўплывае і на асаблівасьці харчовых паводзінаў. Для розных людзей працуюць розныя падыходы да кантролю псыхагеннага пераяданьня. Для прыкладу разгледзім тры гены, генэтычныя варыяцыі ў якіх павялічваюць рызыку кампульсіўнага пераяданьня. Гэта гены GAD2, Tag1A1 і FTO.

GAD2.
Гэты ген стымулюе пераяданьне, ён зьвязаны з кампульсіўнымі парушэньнямі харчовых паводзінаў. Ген кадуе фэрмэнт глутаматдэкарбаксілазу, падвышаная актыўнасьць якой вядзе да 6-разовага ўзмацненьня выпрацоўкі ГАМК (гама-амінамасьлянай кіслаты) у мозгу. Выдзяленьне лішку ГАМК у гіпаталамусе вядзе да павышэньня ўзроўню нэўрапэптыду Y ў аркуатным ядры. А гэта прыводзіць да моцнага голаду. Што рабіць? Заблякаваць залішнюю актыўнасьць нэўрапэптыду Y можна праз страўнікава-кішачны падыход. Там ёсьць адмысловыя L-клеткі, якія выпрацоўваюць пэптыд YY, што прыгнятае апэтыт. Галоўнымі стымулятарамі пэптыду YY зьяўляюцца тлушчы і жоўцевыя кіслоты. А яшчэ – харчовыя валокны, а таксама аэробныя фізычныя практыкаваньні.

Taq1A1.
Гэты ген зьвязаны з больш нізкай шчыльнасьцю дафамінавых D2 рэцэптараў у галаўным мозгу. Яго носьбіты атрымліваюць прыкметна менш задавальненьня ад жыцьця і ад ежы. Людзям зь нізкай шчыльнасьцю дафамінавых рэцэптараў для атрыманьня задавальненьня трэба зьесьці нашмат больш, што яны і робяць. Як толькі жыцьцё такіх людзей становіцца бязрадасным, яны пачынаюць вельмі шмат есьці, пераядаць для кампэнсацыі ўзроўню дафаміну. Што рабіць? Трэба больш радасьці, а ня ежы. Ёсьць і некалярыйныя спосабы атрыманьня задавальненьня ад ежы, і ня толькі ад яе. Гэта і прыгожы абрус, прыборы, сэрвіроўка, падача, размовы, час ежы, смакаваньне і ўсьвядомленая ежа. Таксама важна дадаць больш задавальненьня зь іншых сфэр жыцьця. Гэта падыме ўзровень дафаміну, і пераяданьне аслабне.

FTO.
Гэты ген служыць актыватарам выдзяленьня гармону голаду грэліну. У носьбітаў яго «тлустых» вэрсыяў узровень грэліну ня падае пасьля ежы, як павінна быць, а застаецца высокім. Таму такія людзі пераядаюць і аддаюць перавагу высокатлушчавыя прадукты. Узровень грэліну павялічваецца пры стрэсе, таму ў адказ на яго носьбіты зьядаюць шмат лішняга. Што рабіць? Грэлін – гэта ня так кепска, як здаецца, ён, напрыклад, зьніжае рызыку дэпрэсіі і стымулюе нэўрагенэз. Для носьбітаў AA вэрсыяў FTO важна прытрымлівацца рэжыму харчаваньня, не перакусваць, не прапускаць прыёмы ежы і не рабіць харчовых разгрузак на тле павышанага стрэсу. Кантроль стрэсу і рэжыму харчаваньня дазволіць эфэктыўна даць рады з кампульсіўным пераяданьнем у гэтым выпадку.

Генэтыка смаку.
У розных людзей генэтычна розны парог адрозьніваньня смакаў, таму сапраўды «дзеду міла, а ўнуку гніла». Памятайце, што ваша харчаваньне павінна прыносіць задавальненьне, таму з шырокага спэктру прадуктаў важна выбіраць, кіруючыся сваім смакам!

Мікрафлёра.
Асаблівасьці засваеньня прадуктаў і індывідуальная рэакцыя людзей залежаць ад стану мікрафлёры іх кішачніка. Таму, напрыклад, уздым узроўню глюкозы ў крыві ад аднаго і таго ж прадукту будзе ва ўсіх розным. Вывучэньне мікрафлёры кішачніка дапаможа ацаніць яе разнастайнасьць і выявіць, якія віды бактэрыяў адсутнічаюць, а таксама вызначыць аптымальныя нутрыенты для падкорму бактэрыяў таго ці іншага віду.

Асноўныя прынцыпы

Правіла 80/20.
Назіраньне за сабой, выкарыстаньне розных дыягнастычных тэстаў дапаможа вам выявіць індывідуальныя рэакцыі і асаблівасьці страваваньня ды ўлічваць іх для складаньня свайго пэрсанальнага рацыёну. Пры рабоце над сваім харчаваньнем прыслухоўвайцеся ня толькі да навуковых парадаў, але і да сваёй індывідуальнай рэакцыі. Бо гэта ўсярэдненыя парады, а вам трэба тое, што пасуе асабіста вам. Далёка ня ўсе парады будуць працаваць ідэальна, вам сярод усіх магчымых харчовых інструмэнтаў варта адшукаць тыя 20%, якія дадуць вам 80% выніку. Напрыклад, падбярыце, што мацней за ўсё вас насычае. Для гэтага адзначайце ў харчовым дзёньніку, колькі гадзінаў трымалася сытасьць пасьля тых ці іншых прадуктаў і наколькі ў балах яна была выяўленая. Гэтак жа можна вывучыць і свой голад: ацэньвайце, калі ён узьнікае і якой інтэнсіўнасьці.

Харчовы дзёньнік.
Харчовы дзёньнік – гэта запіс рэжыму харчаваньня і прадуктаў, пры гэтым варта ўлічваць таксама іншыя фактары ладу жыцьця, каб усталяваць паміж імі дакладныя прычынна-выніковыя сувязі. Улічвайце стрэс, сон, фізычную актыўнасьць, дзень цыклю і іншыя фактары. Харчовы дзёньнік можна весьці ў розных формах: фатаграфуючы стравы і затым складаючы справаздачу і ведучы дзёньнік за тыдзень з адзнакамі памылак, дзе фіксаваць свае адступленьні ад рацыёну і аналізаваць іх прычыны. Вы можаце выявіць асноўныя прычыны зрываў і пераяданьня, вельмі часта яны знаходзяцца нават ня ў сфэры ежы, а зьвязаныя з дэфіцытам сну, недахопам фізычнай актыўнасьці ды іншымі прычынамі.

Сярод усіх магчымых харчовых інструмэнтаў варта адшукаць тыя 20%, якія дадуць вам 80% выніку. Напрыклад, падбярыце, што мацней за ўсё вас насычае. Для гэтага адзначайце ў харчовым дзёньніку, колькі гадзінаў трымалася сытасьць пасьля тых ці іншых прадуктаў і наколькі ў балах яна была выяўленая. Гэтак жа можна вывучыць і свой голад: ацэньвайце, калі ён узьнікае і якой інтэнсіўнасьці.

Элімінацыйная дыета. 
Вось спосаб сапраўды выявіць прадукты, якія выклікаюць непераноснасьць. Для гэтага спачатку вы выключаеце большасьць падазроных прадуктаў на працягу тыдня, а затым кожныя 2-3 дні дадаяце па адным падазроным прадукце, пры гэтым ведучы дзёньнік харчаваньня і адсочваючы сымптомы.

Глікемічны кантроль.
Вы можаце з дапамогай глюкомэтра і дыягнастычных палосак адсочваць уздым глюкозы праз гадзіну і дзьве гадзіны пасьля яды і абраць стравы ці прадукты, якія даюць мінімальны ўздым канкрэтна ў вашым выпадку. Ёсьць і сыстэмы сталага маніторынгу глюкозы ў крыві.

Генэтыка.
Цяпер ёсьць вялікая колькасьць розных кампаній, якія робяць генэтычныя тэсты, кошт на іх увесь час зьніжаецца. Зьвярніце ўвагу, што далёка ня ўсе ДНК-парады маюць валіднасьць. На шматлікія паказьнікі ўплываюць адначасова некалькі генаў, а прадказаць іх сумарны ўплыў сапраўды пакуль яшчэ цяжка. Таму абавязкова правярайце валіднасьць кожнай рэкамэндацыі і стаўцеся да яе як да парады, якую можна праверыць, а не як да ісьціны ў апошняй інстанцыі.

Мікрафлёра.
Шмат кампаніяў робіць 16S рРНК – мэтагеномнае сэквэнаваньне мікрабіёмы, гэта найлепшы тэст. Ён дазваляе сапраўды вызначыць склад і суадносіны розных відаў бактэрыяў, ацаніць разнастайнасьць мікрабіёмы, убачыць, якіх штамаў вам бракуе, што трэба зьмяніць у рацыёне для паляпшэньня стану мікрафлёры. Робячы гэты тэст паўторна, можна напраўду ацаніць, як вашыя харчовыя зьмены паўплывалі на мікрабіем.

Харчовая непераноснасьць. 
У розных людзей шэраг прадуктаў выклікае паталягічныя рэакцыі, якія зьнікаюць пры выключэньні пэўных прадуктаў. Праявы бываюць самыя розныя: ад цяжару ў жываце, высыпаньняў на скуры, сьлёзацёку да ацёку Квінке, які пагражае жыцьцю. У цэлым да 20% людзей лічыць, што ў іх ёсьць алергія на пэўныя прадукты. Але ў рэальнасьці гэтая лічба нашмат меншая. У структуры харчовай непераноснасьці вылучаюць розныя станы, якія патрабуюць розных падыходаў.

Асобна мы можам гаварыць пра сапраўдную харчовую алергію, фэрмэнтапатыях, псэўдаалергіях (выдзяленьне гістаміну неімунным мэханізмам), парушэньні засваеньня асобных прадуктаў пры пэўных захворваньнях страўнікава-кішачнага тракту (прычына непераноснасьці – хвароба, яе лячэньне ліквідуе непераноснасьць), вылучаюць таксама і псыхагенныя рэакцыі на ежу, калі людзі самі або пад уздзеяньнем зьнешняга аўтарытэту могуць намаўляць сябе, што ім блага ад пэўных прадуктаў.

Сапраўдная алергія.
Сапраўдная алергія выклікаецца нават мінімальнай колькасьцю алергену, ёсьць станоўчы скурны тэст, як правіла падвышаны ўзровень IgE ў крыві. Варта адзначыць, што існуе і ўтоеная алергія, калі імунаглабуліны IgE ў крыві да алергену павышаныя, але рэакцыі пры гэтым няма. Зьвярніце ўвагу, што папулярныя ў нашы дні тэсты на вызначэньне IgG для фармаваньня сьпісу забароненых прадуктаў ненавуковыя і прыводзяць адно да звужэньня разнастайнасьці рацыёну. IgG утвараецца пры спажываньні ежы і не зьвязаны зь непераноснасьцю. Асноўныя прадукты – прычыны алергіі (вялікая васьмёрка): каровіна малако, яйкі (часьцей за ўсё курыныя), рыба (бывае асобна на прэснаводную і марскую), пшаніца, арахіс, гарэхі, соя, ракападобныя. Многія з гэтых прадуктаў уваходзяць у склад рознай гатовай ежы (соевы парашок, сухое малако і да т. п.). У выпадку сапраўднай алергіі неабходнае поўнае выключэньне прадукту з рацыёну.

Псэўдаалергія (фальшывая алергія).
Пры фальшывай алергіі шэраг рэчываў выклікае выкід гістаміну (і рэакцыю, падобную да алергічнай), але без наўпроставага ўдзелу імунных клетак. Пры фальшывай алергіі колькасьць зьедзенага прадукту павінна быць вялікая, пры гэтым ёсьць прамая залежнасьць паміж выяўленасьцю сымптомаў і колькасьцю зьедзенага, скурныя тэсты пры гэтым адмоўныя, а ўзровень IgE ў крыві не падвышаны. Пры фальшывай алергіі дзейнічаюць так званыя рэчывы – лібэратары гістаміну. Часьцей за ўсё гэта рыба, яйкі, чакаляда, арэхі. Часам рэакцыі няма на сьвежыя прадукты, а на тыя, што доўга захоўваліся, – ёсьць. Гэта зьвязана з павелічэньнем ўзроўню гістаміну пры захоўваньні. Часта прычынай псэўдаалергіі зьяўляюцца ня самі прадукты, а некаторыя харчовыя дабаўкі. Пры фальшывай алергіі эфэктыўная нізкагістамінавая дыета.

Фэрмэнтапатыя.
У кожнага чалавека розная актыўнасьць фэрмэнтаў, якія расшчапляюць ежу, так званая «біяхімічная індывідуальнасьць». Бывае прыроджаны недахоп фэрмэнтаў, тады чалавек ня можа засвойваць ежу. Бывае непераноснасьць ляктозы (тады трэба выключыць малочныя цэльныя прадукты), трэгалозы (выключыць грыбы), глютэну (выключыць збажыну і да т. п.). Часта фэрмэнтапатыі разьвіваюцца на тле захворваньняў страўнікава-кішачнага тракту.

Асноўныя прадукты – прычыны алергіі (вялікая васьмёрка): каровіна малако, яйкі (часьцей за ўсё курыныя), рыба (бывае асобна на прэснаводную і марскую), пшаніца, арахіс, гарэхі, соя, ракападобныя.

Хваробы страўнікава-кішачнага тракту.
Пры шматлікіх захворваньнях патрабуюцца пэўныя дыеты, бо парушаецца звычайнае засваеньне прадуктаў. Аднак пры лячэньні непераноснасьць зьнікае. Часьцей за ўсё прычынамі бываюць зьніжэньне кіслотнасьці страўнікавага соку, аслабленая функцыя падстраўніцы, павышэньне пранікальнасьці кішачніка (дзіравы кішачнік), парушэньні мікрафлёры, сындром раздражнёнага кішачніка.

Як трымацца правіла? Ідэі і парады

Рызыка алергічных і аўтаімунных захворваньняў.
Зьнізіць рызыку алергічных і аўтаімунных захворваньняў, асабліва ў дзяцей, можна ўлічваючы чатыры важныя чыньнікі: дастатковы ўзровень вітаміну D і знаходжаньне на сонцы, дастатковы ўзровень мікробнай нагрузкі (кантакт з прыродай, бываць на ферме, кантакт зь іншымі дзецьмі і да т.п.), нізкая колькасьць солі ў дыеце, выкарыстаньне ашчаднага харчаваньня нават для дзяцей, якія яшчэ растуць, бо пастаянная стымуляцыя mTORС павялічвае рызыку алергічных захворваньняў.

Структура цела.
Многія людзі пры зьменах харчаваньня крытэрам эфэктыўнасьці лічаць вагу цела. Але адна і тая ж вага можа выглядаць зусім па-рознаму, таму правільней ацэньваць эфэктыўнасьць сваіх дзеяньняў на аснове вымярэньня структуры цела. Структура цела – гэта суадносіны і разьмеркаваньне тлушчавай і цяглічнай тканкі. Так, падскурны тлушч ня вельмі небясьпечны, а самую сур'ёзную пагрозу ўяўляюць менавіта ўнутраны тлушч (на такіх органах, як печань, сэрца, кішачнік, падстраўніца) і страта цяглічнай масы (саркапэнія). Унутраны (эктапічны) тлушч выдзяляе адмысловыя рэчывы, якія правакуюць запаленьне, парушаюць гарманальны балянс. Аб'ектыўнае вымярэньне вісцэральнага тлушчу – гэта вельмі важна. Бо ні ваша вага, ні адсотак тлушчу ня важныя так, як колькасьць вісцэральнага тлушчу. Чым яго больш, тым вышэйшая рызыка захворваньняў. Чалавек можа быць худым і пры гэтым мець высокі ўзровень унутранага тлушчу і, адпаведна, высокую рызыку захворваньняў.

Структура цела – гэта суадносіны і разьмеркаваньне тлушчавай і цяглічнай тканкі. Падскурны тлушч ня вельмі небясьпечны, а самую сур'ёзную пагрозу ўяўляюць менавіта ўнутраны тлушч (на такіх органах, як печань, сэрца, кішачнік, падстраўніца) і страта цяглічнай масы (саркапэнія). Унутраны (эктапічны) тлушч выдзяляе адмысловыя рэчывы, якія правакуюць запаленьне, парушаюць гарманальны балянс. Чалавек можа быць худым і пры гэтым мець высокі ўзровень унутранага тлушчу і, адпаведна, высокую рызыку захворваньняў.

Антрапамэтрыя.
Гэта замеры стужкай, якія паказваюць асаблівасьці разьмеркаваньня тлушчу ў целе і рызыку захворваньняў. Часьцей за ўсё выкарыстоўваюцца наступныя паказчыкі: абхоп таліі ў жанчынаў складае да 75 (80) сантымэтраў, ад 80 да 88 сантыметраў – перавышэньне нармальнай вагі, звыш 88 – атлусьценьне, у мужчынаў нармальныя парамэтры складаюць да 94 сантыметраў. Суадносіны талія-сьцёгны ў норме менш за 0,85 для жанчынаў і менш за 1,0 для мужчынаў (аптымальна 0,7 (0,65-0, 78) для жанчынаў і ня больш за 0,9 для мужчынаў), акружнасьць шыі ў самым вузкім месцы ў жанчынаў ня больш за 34,5 см (больш строгая норма – 32 см), у мужчынаў акружнасьць шыі ня больш за 38,8 см (больш строгая норма – 35,5 гл). Суадносіны талія-сьцягно – менш за 1,5 для жанчынаў і менш за 1,7 для мужчынаў, ABSI – гэта комплексны індэкс формы цела, які ўлічвае суадносіны паміж аб'ёмам таліі, ростам і вагой ды разлічвае індывідуальную рызыку.

Цяглічная маса.
З узростам зьніжэньне цяглічнай масы толькі ўзмацняецца. Захаваньне дастатковай колькасьці цягліцаў зьяўляецца ўмовай захаваньня мэтабалічнага здароўя. Памятайце, што зьніжэньне сілы – гэта страта цягліцаў. Страту цягліцаў цяжка заўважыць, бо яны замяшчаюцца тлушчам, і аб'ём канцавіны можа заставацца ранейшым. Важна захоўваць высокі ўзровень фізычнай актыўнасьці ў любым узросьце.

Вымярэньне вісцэральнага тлушчу.
Самыя надзейныя вынікі дае DEXA-сканаваньне цела і тамаграфія. Але можна ацаніць узровень вісцэральнага тлушчу і па УГД. Спачатку робім дасьледаваньне печані: лінейныя памеры, прыкметы тлушчавага гепатозу, стан жоўцевага пухіра. Затым вымяраем таўшчыню эпікардыяльнага тлушчу (рызыкі растуць пры лічбе больш за 5 мм), колькасьць якога карэлюе з узроўнем вісцэральнага тлушчу, і таўшчыню пэрыкардыяльнага тлушчу (сардэчная рызыка). Пасьля чаго вымяраем адлегласьць паміж белай лініяй жывата і пярэдняй сьценкай аорты (больш за 100 мм – вісцэральнае атлусьценьне). Дадаткова можна разлічыць індэкс тлушчу брушной сьценкі (ІТБС) – гэта стасунак максымальнай таўшчыні перадбрушнога тлушчу да мінімуму таўшчыні падскурнага тлушчу. Гэтыя паказьнікі проста вымяраць у дынаміцы (лепей на адным апараце ў аднаго спэцыяліста).

Зьнізіць рызыку алергічных і аўтаімунных захворваньняў можна ўлічваючы чатыры важныя чыньнікі: дастатковы ўзровень вітаміну D, мікробная нагрузка, зьніжэньне колькасьці солі ў дыеце, умеранае харчаваньне.

Наведваньне стаматоляга.
Здаровыя зубы і здаровая ротавая поласьць – гэта найважнейшыя ўмовы добрага страваваньня. Дбайна чысьціце зубы, палашчыце рот пасьля кожнага прыёму ежы, выкарыстоўвайце ірыгатары, зубныя ніткі, своечасова выдаляйце зубны камень. Больш жуйце для здароўя зубоў.

Мікрафлёра рота.
Мікрафлёра ротавай поласьці важная для нашага здароўя. Пазьбягайце лішку мучнога (глютэн прыляпляе часьціцы крухмалу да зубоў), бактэрыцыдных ополаскивателей для рота, выкарыстоўвайце аральныя прабіётыкі (S. Salivarius і інш.).

Індывідуальная праграма абсьледаваньня.
Таксама на ваш стан уплывае і здароўе. Рэгулярна правярайце яго, здавайце неабходныя, а лепей – пашыраныя дыягнастычныя панэлі. Прыклад падобнай лябараторнай панэлі: глюкоза, глікаваны гемаглабін, інсулін, індэкс інсулінарэзыстэнтнасьці, Алат, Асат, крэатынін, мачавая кіслата, гомацыстэін, ультраадчувальны тэст на С-рэактыўны бялок, гармоны шчытападобнай залозы (ТТГ, св. Т3), палавыя гармоны, ІФР-1, гармон росту, цынк, магній, жалеза, фэрытын, вітамін D, вітамін В12. Да іх можна дадаць інструмэнтальныя тэсты, напрыклад УГД сонных артэрыяў з вызначэньнем таўшчыні комплексу інтым-мэдыя. Пры неабходнасьці можна дапаўняць іншымі абсьледаваньнямі, у залежнасьці ад узросту і індывідуальнай рызыкі (сямейная гісторыя захворваньняў, генэтычныя рызыкі шэрагу захворваньняў і дэфіцытаў і інш.): анкамаркеры, каланаскапія, фібрагастрадуадэнаскапія і іншыя паказьнікі.
